
\section{Exp.\ 1: Reproduction of Original Work}

Our first experiment covers the reproduction of the original results~\citep{koppel_determining_2014}.
It is noteworthy, that we do not expect exact replication since we could only reimplement the approach to our best knowledge.
Both Mr. Koppel and Mr. Winter were very forthcoming when we contacted them regarding implementation details that could help to improve our implementation.
Unfortunately, the code was no longer traceable and neither author could recall the preprocessing steps.
After consultation with Mr. Winter, he assured that our preprocessing steps seem reasonable.
This, however, certainly is one of the reasons why our implementation diverges from the original work.

\paragraph{Exp.\ 1(a): Varying number of \imps{}.}

For this experiment, we generate \imps{} using the fixed method while varying only the number of \imps{}.
\citet{koppel_determining_2014}\ report results of this experiment exclusively on the \dataBlog{} dataset.
While we compare their findings with ours, it is important to note that direct comparability is limited due to differences in the underlying datasets, as our results are based on the \dataStudent{} corpus.
In contrast to the original study, our \impAppr{} recall–precision curves intersect, suggesting that the results reported by \citet{koppel_determining_2014}\ exhibit stronger separation.
In our experiments, the number of \imps{} exerts minimal influence on precision and recall.
Nevertheless, similar to the original findings, we observe that using 50 \imps{} yields marginally superior performance.


\begin{figure}[htbp]
    \centering
    \includesvg[width=\textwidth]{images/imposter/reproduction_koppel_figures/fig2/student_essays/student_roc_prec_recall_curve_r100_top100000_dif_n_imp.svg}
    \caption[Recall-precision curves for the various sized \imp{} set sizes.]{Recall-precision curves for the various sized \imp{} set sizes on the \dataStudent{} dataset.}
    \label{fig:student_essays_dif_n}
\end{figure}

% \begin{figure}[htbp]
%   \centering
%   \begin{subfigure}[b]{0.48\textwidth}
%     \centering
%     \includesvg[width=\linewidth]{images/imposter/reproduction_koppel_figures/fig2/student_essays/student_roc_prec_recall_curve_r100_top100000_dif_n_imp.svg}
%     \caption{\dataBlog{} \textcolor{red}{TODO: needs to run (15.09.2025)}}
%     \label{fig:blog_dif_n}
%   \end{subfigure}
%   \hfill
%   \begin{subfigure}[b]{0.48\textwidth}
%     \centering
%     \includesvg[width=\linewidth]{images/imposter/reproduction_koppel_figures/fig2/student_essays/student_roc_prec_recall_curve_r100_top100000_dif_n_imp.svg}
%     \caption{\dataStudent{}}
%     \label{fig:student_essays_dif_n}
%   \end{subfigure}
%   \caption{Recall-precision curves for the various sized \imp{} set sizes.}
%   \label{fig:repr_diff_n_imps_fixed}
% \end{figure}


\paragraph{Exp.\ 1(b): Varying \imp{} generation.}

Adhering to \citet{koppel_determining_2014}, we compare multiple \ac{av} methods with variants of the \impAppr{} that employ different \imp{} generation strategies, evaluating them in the \ac{av} scenario with human-authored text pairs.
The \imp{} generation strategies include sampling from a fixed set of potential \imp{} candidates, and on-the-fly \imp{} generation. 
The baselines introduced by \citet{koppel_determining_2014} are \ac{svc}, and unsupervised similarity based approaches.
Each class is treated as the positive (reference) class once, with the other class serving as the negative class. 
This choice affects how the evaluation scores are computed, but not how the thresholds are fit during the training.
Although not explicitly stated, it is likely that \citet{koppel_determining_2014} have trained each approach for both reference class scenarios once.
The original work reports the results for the \dataBlog{} dataset only.

% original results
The original recall-precision curves by \citet{koppel_determining_2014}\ suggest that fixed \imp{} generation is superior to on-the-fly \imp{} generation.
In the original work, minimal \imp{} generation recall value were higher for the different-author reference class (around $0.625$ for fixed and $0.58$ for on-the-fly) than for the same-author reference class counterpart (around $0.38$ for fixed and $0.22$ for on-the-fly).
\Imp{} generation precision values behaved oppositional, since all approaches perform better in the same-author reference setting than in the different-author reference scenario.
These findings are likely due to the fact that \ac{av} is an open-set one-class problem.
Hence, the different-author class is ill-defined due to the lack of representatives.
Consequently, both \imp{} generation methods cause more different author class predictions than same author class predictions resulting in high different-author class recall and high same-author class precision.
We visualized our interpretation of the results reported by \citet{koppel_determining_2014} in \autoref{fig:findings_original_work}.

\begin{figure}[htbp]
    \centering
    \includesvg[width=\textwidth]{images/imposter/reproduction_koppel_figures/fig2/student_essays/fig2_original_findings.svg}
    \caption[Results from the original \impAppr{} experiments.]{Findings from the original \impAppr{} experiments by \citet{koppel_determining_2014} suggest that the different-author class is more difficult to model, leading to more different-author class predictions and thus, a higher different-author class recall.}
    \label{fig:findings_original_work}
\end{figure}

% our results
Both \citet{koppel_determining_2014}\ and us find that fixed \imp{} generation outperforms other approaches for the same-author reference scenario.
Other than that, our recall-precision curves differ from the original.
In the different-author reference scenario, all but on-the-fly generation have low precision values with fixed \imps{} performing the worst. 
Low precision means many \acp{fp}, or in this case, many same author pairs which were predicted as different authors.
This suggests that different-author reference scenario is difficult and that most \imps{} are too hard. 

Opposed to the original work of \citet{koppel_determining_2014}, our on-the-fly \imps{} lack difficulty explaining the low precision in the same-author and the high precision in the different-author scenario.
Since we had to reduce the number of test samples for on-the-fly \imp{} generation due to API call limits, we suggest analysing this approaches' results with a grain of salt.
However, Mr. Winter noted that our on-the-fly \imp{} generation implementation seems insufficient, due to the rise of bot prevention blocking web scraping and the very limited number of API calls per month.

\textcolor{red}{TODO: figures SVC to SVM}
\begin{figure}[htbp]
  \centering
  \begin{subfigure}[b]{0.49\textwidth}
    \centering
    \includesvg[width=\linewidth]{images/imposter/reproduction_koppel_figures/fig4/blog/blog_roc_prec_recall_curve_r100_top100000_Same_Author_dif_imp_gen.svg}
    \caption{Same author reference class. }
    \label{fig:blog_same_author}
  \end{subfigure}
  \hfill
  \begin{subfigure}[b]{0.49\textwidth}
    \centering
    \includesvg[width=\linewidth]{images/imposter/reproduction_koppel_figures/fig4/blog/blog_roc_prec_recall_curve_r100_top100000_Different_Author_dif_imp_gen.svg}
    \caption{Different author reference class.}
    \label{fig:blog_diff_author}
  \end{subfigure}
  \caption[Recall-precision curves for the \dataBlog{} dataset. ]{Recall-precision curves for the \dataBlog{} dataset. 
  (B) indicates the original baseline approaches from~\citep{koppel_determining_2014}.
  Due to API limit restrictions, the test set for on-the-fly was smaller which is visible in the respective curves.}
  \label{fig:diff_imp_gen_blog}
\end{figure}
