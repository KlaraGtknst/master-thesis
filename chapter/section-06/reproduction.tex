
\section{Exp.\ 1: Reproduction of Original Work}

Our first experiment covers the reproduction of the original results~\citep{koppel_determining_2014}.
It is noteworthy, that we do not expect exact replication since we could only reimplement the approach to our best knowledge.
Both Mr. Koppel and Mr. Winter were very forthcoming when we contacted them regarding implementation details that could help to improve our implementation.
Unfortunately, the code was no longer traceable and neither author could recall the preprocessing steps.
After consultation with Mr. Winter, he assured that our preprocessing steps seem reasonable.
This, however, certainly is one of the reasons why our implementation diverges from the original work.

\paragraph{Exp.\ 1(a): Varying number of \imps{}.}

For this experiment, we generate \imps{} using the \texttt{fixed} method while varying only the number of \imps{}.
\citet{koppel_determining_2014} report results of this experiment exclusively on the \dataBlog{} dataset.
Although we draw comparisons between their findings and ours, direct comparability remains limited, since our experiments were constrained to a maximum of \num{1000} \imps{} rather than the \num{5000} examined in their study.

The original \impAppr{} recall–precision curves demonstrate a clearer distinction between the evaluated approaches.
However, we were unable to reproduce exactly the values reported in the original work by \citet{koppel_determining_2014}.
On the \dataBlog{} dataset, as illustrated in \Cref{fig:blog_dif_n}, and on the \dataStudent{} dataset, detailed in \Cref{app:exp_student_dif_n}, increasing the number of \imps{} from 500 to \num{1000} has only a negligible effect on precision and recall.
In line with the original findings, we observe that employing 50 \imps{} yields marginally superior performance.
The experiment was repeated multiple times on subsets of 20 samples per class, with the best-performing number of \imps{} varying across runs, thereby underscoring the sample-dependent nature of the effect.

% By contrast, our experiments on the \dataStudent{} dataset  support the original conclusion that 50 \imps{} achieve superior performance, thereby highlighting a dataset-dependent effect.



\begin{figure}[htbp]
    \centering
    \includesvg[width=0.93\textwidth]{images/imposter/reproduction_koppel_figures/fig2/blog/roc_prec_recall_curve_fixed_r100_top100000_dif_n_imp_cleansed.svg}
    \caption[Recall-precision curves for the various sized \imp{} set sizes.]{Recall-precision curves for the various sized \texttt{fixed} \imp{} set sizes on the \dataBlog{} dataset.
    }
    \label{fig:blog_dif_n}
\end{figure}



% \begin{figure}[htbp]
%   \centering
%   \begin{subfigure}[b]{0.48\textwidth}
%     \centering
%     \includesvg[width=\linewidth]{images/imposter/reproduction_koppel_figures/fig2/student_essays/student_roc_prec_recall_curve_r100_top100000_dif_n_imp.svg}
%     \caption{\dataBlog{} \textcolor{red}{TODO: needs to run (15.09.2025)}}
%     \label{fig:blog_dif_n}
%   \end{subfigure}
%   \hfill
%   \begin{subfigure}[b]{0.48\textwidth}
%     \centering
%     \includesvg[width=\linewidth]{images/imposter/reproduction_koppel_figures/fig2/student_essays/student_roc_prec_recall_curve_r100_top100000_dif_n_imp.svg}
%     \caption{\dataStudent{}}
%     \label{fig:student_essays_dif_n}
%   \end{subfigure}
%   \caption{Recall-precision curves for the various sized \imp{} set sizes.}
%   \label{fig:repr_diff_n_imps_fixed}
% \end{figure}


\paragraph{Exp.\ 1(b): Varying \imp{} generation.}

Adhering to \citet{koppel_determining_2014}, we compare multiple \ac{av} methods with variants of the \impAppr{} that employ different \imp{} generation strategies, evaluating them in the \ac{av} scenario with human-authored text pairs.
The \imp{} generation strategies include sampling from a \texttt{fixed} set of potential \imp{} candidates, and \texttt{on-the-fly} \imp{} generation. 
The baselines introduced by \citet{koppel_determining_2014} are \ac{svm}, and unsupervised similarity based approaches.
Each class is treated as the positive (i.e. reference) class once, with the other class serving as the negative class. 
This choice affects how the evaluation scores are computed, but not how the thresholds are fit during the training.
Although not explicitly stated, it is likely that \citet{koppel_determining_2014} have trained each approach for both reference class scenarios once.
The original work reports the results for the \dataBlog{} dataset only.

% original results
The recall–precision curves reported by \citet{koppel_determining_2014}\ indicate that \texttt{fixed} \imp{} generation outperforms \texttt{on-the-fly} \imp{} generation.
In their study, the minimum recall values for the different-author class were consistently higher (approximately $0.625$ for \texttt{fixed} and $0.58$ for \texttt{on-the-fly}) than for the same-author class (approximately $0.38$ for \texttt{fixed} and $0.22$ for \texttt{on-the-fly}).
In contrast, precision values exhibited the opposite trend where all approaches achieved higher precision in the same-author setting compared to the different-author setting.

Our interpretation of the results reported by \citet{koppel_determining_2014}\ is presented in \Cref{fig:findings_original_work}.
We argue that these findings are a consequence of the nature of \ac{av} as an open-set, one-class classification problem.
Because the different-author class lacks representative samples, it remains inherently ill-defined.
As a consequence, both \imp{} generation methods tend to yield more different-author predictions than same-author predictions, producing higher recall for the different-author class and higher precision for the same-author class.

\begin{figure}[htbp]
    \centering
    \includesvg[width=\textwidth]{images/imposter/reproduction_koppel_figures/fig2/student_essays/fig2_original_findings.svg}
    \caption[Aggregating original \impAppr{} experiment results.]{Results reported by \citet{koppel_determining_2014} suggest that the different\-author class is more difficult to model, leading to more different-author predictions and consequently higher recall for that class.}
    \label{fig:findings_original_work}
\end{figure}

% our results
Both \citet{koppel_determining_2014}\ and our results indicate that \texttt{fixed} \imp{} generation performs best in the same-author scenario.
Beyond this agreement, however, our recall–precision curves deviate from the original findings.
In the different-author scenario, all methods except \texttt{on-the-fly} generation exhibit low precision, with \texttt{fixed} \imps{} performing worst.
Low precision corresponds to a high number of \acp{fp}, i.e. many same-author pairs incorrectly classified as different authors.
This outcome suggests that training on the same-author class produces systematically poor performance in a different-author class test scenario, as the model is not optimized for it.

Opposed to the original work of \citet{koppel_determining_2014}, our \texttt{on-the-fly} \imps{} lack difficulty explaining the low precision in the same-author and the high precision in the different-author scenario.
Since we had to reduce the number of test samples for \texttt{on-the-fly} \imp{} generation due to API call limits, we suggest analysing this approaches' results with a grain of salt.
However, Mr. Winter noted that our \texttt{on-the-fly} \imp{} generation implementation seems insufficient, due to the rise of bot prevention blocking web scraping and the very limited number of API calls per month.

% \textcolor{red}{TODO: figures SVC to SVM}
\begin{figure}[htbp]
  \centering
  \begin{subfigure}[b]{0.49\textwidth}
    \centering
    \includesvg[width=\linewidth]{images/imposter/reproduction_koppel_figures/fig4/blog/blog_roc_prec_recall_curve_r100_top100000_Same_Author_dif_imp_gen.svg}
    \caption{Same-author reference class. }
    \label{fig:blog_same_author}
  \end{subfigure}
  \hfill
  \begin{subfigure}[b]{0.49\textwidth}
    \centering
    \includesvg[width=\linewidth]{images/imposter/reproduction_koppel_figures/fig4/blog/blog_roc_prec_recall_curve_r100_top100000_Different_Author_dif_imp_gen.svg}
    \caption{Different-author reference class.}
    \label{fig:blog_diff_author}
  \end{subfigure}
  \caption[Recall-precision curves for the \dataBlog{} dataset. ]{Recall-precision curves for the \dataBlog{} dataset. 
  (B) indicates the original baseline approaches from~\citep{koppel_determining_2014}.
  Due to API limit restrictions, the test set for \texttt{on-the-fly} was smaller which is visible in the respective curves.
  Classifiers were not retrained for the different-author reference class scenario explaining the poor results.}
  \label{fig:diff_imp_gen_blog}
\end{figure}
