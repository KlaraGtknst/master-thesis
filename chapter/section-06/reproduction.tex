
\section{Exp. 1: Reproduction of Original Work}

Our first experiment covers the reproduction of the original results attached in \citet{koppel_determining_2014}'s paper.
It is noteworthy, that we do not expect exact replication since we could only reimplement the approach to our best knowledge.
Both Mr. Koppel and Mr. Winter were very forthcoming when we contacted them regarding implementation details that could help to improve our implementation.
Unfortunately, the code was no longer traceable and neither author could recall the preprocessing steps.
After consultation with Mr. Winter, he assured that our preprocessing steps seem reasonable.
This, however, certainly is one of the reasons why our implementation diverges from the original work.

\paragraph{Exp. 1(a): Varying number of \imps{}.}

For this experiment, we set the \imp{} generation method to \texttt{fix} and vary only the number of \imps{}.
We only have reference results of this experiment on the \dataBlog{} dataset from the original paper.
Unlike the original curves, our \impAppr{} recall-precision curves intersect.
This suggests that \citet{koppel_determining_2014}'s results were better separated.
In our case, the number of \imps{} has no big impact on the precision and recall scores.
Although the form of the original precision-recall plot differs, we find that 50 \imps{} achieve the best results in terms of recall.

\begin{figure}[htbp]
  \centering
  \begin{subfigure}[b]{0.48\textwidth}
    \centering
    \includesvg[width=\linewidth]{images/imposter/reproduction_koppel_figures/fig2/student_essays/student_roc_prec_recall_curve_r100_top100000_dif_n_imp.svg}
    \caption{\dataBlog{} \textcolor{red}{TODO: runs (22.08.2025)}}
    \label{fig:blog_dif_n}
  \end{subfigure}
  \hfill
  \begin{subfigure}[b]{0.48\textwidth}
    \centering
    \includesvg[width=\linewidth]{images/imposter/reproduction_koppel_figures/fig2/student_essays/student_roc_prec_recall_curve_r100_top100000_dif_n_imp.svg}
    \caption{\dataStudent{}}
    \label{fig:student_essays_dif_n}
  \end{subfigure}
  \caption{Recall-precision curves for the various sized \imp{} set sizes.}
  \label{fig:repr_diff_n_imps_fixed}
\end{figure}


\paragraph{Exp. 1(b): Varying \imp{} generation.}

Adhereing to \citet{koppel_determining_2014}, we compare the \impAppr{} for \texttt{fixed}, and \texttt{on-th-fly} \imp{} generation with \ac{svm}, and similarity based baselines.
They consider both the same-author and the different-authors class the reference class once.
The original work included only the results for the \dataBlog{} dataset.
% original results
\citet{koppel_determining_2014}'s recall-precision curves suggest that fixed is superior to on-the-fly \imp{} generation.
Moreover, both minimal fixed and minimal on-the-fly \imp{} generation recall values in the different author reference scenario are better than in the same author reference scenario.
In terms of precision, both fixed and minimal on-the-fly \imp{} generation recall values in the different author reference scenario are worse than in the same author reference scenario. % true for min and max vals
This is no surprise, since \ac{av} is an open-set one-class problem, i.e. the different author class is ill-defined due to the lack of (exhaustive) representatives.
Consequently, there is less information about the different-author class which decreases the portion of samples correct different author predictions and increases the portion of different author pairs correctly predicted.
In other words, both fixed and minimal on-the-fly \imp{} generation lead to many different author predictions resulting in high different author recall and high same author precision.

% our results
Our recall-precision curves' form differs from the original.
While the same author reference results resemble \citet{koppel_determining_2014} in that fixed \imp{} generation is superior, different author reference results paint an entirely different picture than the original.
In the different author reference scenario, fix \imps{} perform the worst. 
Low precision means many \acp{fp}, or in this case, many same author pairs which were predicted as different authors.
This indicated that all but the on-the-fly \imps{} were too hard for correct classification.
Generally, all approaches have low precision in this case.
This suggests, that different author pairs are difficult to identify with high certainty across all approaches. 

For different authors as reference class, our results greatly differ from the original work.
Even though, \citet{koppel_determining_2014}'s precision scores for fixed and on-the-fly \imp{} generation are generally lower than those for the same-author reference class, our scores are way lower.
% Notably, in our case, the on-the-fly method seems to perform best in this open-set scenario.
Since we had to reduce the number of test samples for on-the-fly \imp{} generation due to API call limits, we suggest analysing this approaches' results with a grain of salt.
On-the-fly \imp{} generation performs worse for same-author and best for different author reference class.
We conclude that on-the-fly based detectors predict same author very often since the generated \imps{} are too easy and therefore, produce many \acp{fp}.
This would explain the low precision in the same-author and the high precision in the different-author scenario.
Moreover, on-the-fly \imps{} from the original work seem to have been more difficult since this problem was not apparent in their plots.
Mr. Winter noted that our on-the-fly \imp{} generation implementation seems insufficient, due to the rise of bot prevention blocking web scraping and the very limited number of API calls per month.




\textcolor{red}{TODO: report AUROC values for av approaches}
\begin{figure}[htbp]
  \centering
  \begin{subfigure}[b]{0.595\textwidth}
    \centering
    \includesvg[width=\linewidth]{images/imposter/reproduction_koppel_figures/fig4/blog/blog_roc_prec_recall_curve_r100_top100000_Same_Author_dif_imp_gen.svg}
    \caption{Same author reference class. }
    \label{fig:blog_same_author}
  \end{subfigure}
  \hfill
  \begin{subfigure}[b]{0.595\textwidth}
    \centering
    \includesvg[width=\linewidth]{images/imposter/reproduction_koppel_figures/fig4/blog/blog_roc_prec_recall_curve_r100_top100000_Different_Author_dif_imp_gen.svg}
    \caption{Different author reference class.}
    \label{fig:blog_diff_author}
  \end{subfigure}
  \caption{Recall-precision curves for the \dataBlog{} dataset. 
  (B) indicates \citet{koppel_determining_2014}'s baseline approaches.
  Due to API limit restrictions, the test set for on-the-fly was smaller which is visible in the respective curves.}
  \label{fig:diff_imp_gen_blog}
\end{figure}

% For the same-author reference class, both \citet{koppel_determining_2014} and us find that the fixed \imp{} generation approach generally performs best across different threshold.
% While in the original work, the \ac{svm} baseline performed best from the three baselines, we find the opposite to be true.
% Generally, our results perform worse than the original. 
% While \citet{koppel_determining_2014} obtain precision higher than $0.9$ for recall values up to $0.7$, we drop to sub $0.8$ precision fairly quickly.
% Except for the on-the-fly results, the overall form resembles the original work roughly.

% OUR FINDINGS
We find that on the \dataBlog{} dataset, the on-the-fly generation method produces too simple \imps{}, leading to low precision for the same author reference class and higher precision in the different author reference class.
The fixed \imp{} generation technique seems to retrieve too difficult \imps{} since the precision is quite high compared to the baselines and many \acp{fn} lead to a low recall.
The fact that all baselines perform poorly with the different author reference class indicates that this scenario is difficult.


% Student dataset
All approaches have poor precision in the same author reference class.
This indicates that there are many \acp{fp} and thus, the \imps{} are too simple.
Since most pairs are wrongly attributed to same author, those few that are attributed to different author are correct most of the time.
In other words: There is a higher precision for the different author reference class.


