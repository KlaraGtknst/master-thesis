
\section{Exp. 1: Reproduction of Original Work}

Our first experiment covers the reproduction of the original results attached in \citet{koppel_determining_2014}'s paper.
It is noteworthy, that we do not expect exact replication since we could only reimplement the approach to our best knowledge.
Both Mr. Koppel and Mr. Winter were very forthcoming when we contacted them regarding implementation details that could help to improve our implementation.
Unfortunately, the code was no longer traceable and neither author could recall the preprocessing steps.
After consultation with Mr. Winter, he assured that our preprocessing steps seem reasonable.
This, however, certainly is one of the reasons why our implementation diverges from the original work.

\paragraph{Exp. 1(a): Varying number of \imps{}.}

For this experiment, we set the \imp{} generation method to \texttt{fix} and vary only the number of \imps{}.
We only have reference results of this experiment on the \dataBlog{} dataset from the original paper.
Unlike the original curves, our \impAppr{} recall-precision curves intersect.
This suggests that \citet{koppel_determining_2014}'s results were better separated.
In our case, the number of \imps{} has no big impact on the precision and recall scores.
Although the form of the original precision-recall plot differs, we find that 50 \imps{} achieve the best results in terms of recall.

\begin{figure}[htbp]
  \centering
  \begin{subfigure}[b]{0.48\textwidth}
    \centering
    \includesvg[width=\linewidth]{images/imposter/reproduction_koppel_figures/fig2/student_essays/student_roc_prec_recall_curve_r100_top100000_dif_n_imp.svg}
    \caption{\dataBlog{} \textcolor{red}{TODO: runs (22.08.2025)}}
    \label{fig:blog_dif_n}
  \end{subfigure}
  \hfill
  \begin{subfigure}[b]{0.48\textwidth}
    \centering
    \includesvg[width=\linewidth]{images/imposter/reproduction_koppel_figures/fig2/student_essays/student_roc_prec_recall_curve_r100_top100000_dif_n_imp.svg}
    \caption{\dataStudent{}}
    \label{fig:student_essays_dif_n}
  \end{subfigure}
  \caption{Recall-precision curves for the various sized \imp{} set sizes.}
  \label{fig:repr_diff_n_imps_fixed}
\end{figure}


\paragraph{Exp. 1(b): Varying \imp{} generation.}

Adhereing to \citet{koppel_determining_2014}, we compare the \impAppr{} for \texttt{fixed}, and \texttt{on-th-fly} \imp{} generation with \ac{svm}, and similarity based baselines.
They consider both the same-author and the different-authors class the reference class once.
The original work included only the results for the \dataBlog{} dataset.

\begin{figure}[htbp]
  \centering
  \begin{subfigure}[b]{0.495\textwidth}
    \centering
    \includesvg[width=\linewidth]{images/imposter/reproduction_koppel_figures/fig4/blog/blog_roc_prec_recall_curve_r100_top100000_Same_Author_dif_imp_gen.svg}
    \caption{\dataBlog{}}
    \label{fig:blog_same_author}
  \end{subfigure}
  \hfill
  \begin{subfigure}[b]{0.495\textwidth}
    \centering
    \includesvg[width=\linewidth]{images/imposter/reproduction_koppel_figures/fig4/student_essays/student_roc_prec_recall_curve_r100_top100000_Same_Author_dif_imp_gen.svg}
    \caption{\dataStudent{}}
    \label{fig:student_essays_same_author}
  \end{subfigure}
  \caption{Recall-precision curves for the class same-author. Due to API limit restrictions, the test set for on-the-fly was smaller which is visible in the respective curves.}
  \label{fig:same_authors}
\end{figure}

For the same-author reference class, both \citet{koppel_determining_2014} and us find that the fixed \imp{} generation approach generally performs best across different threshold.
While in the original work, the \ac{svm} baseline performed best from the three baselines, we find the opposite to be true.
Generally, our results perform worse than the original. 
While \citet{koppel_determining_2014} obtain precision higher than $0.9$ for recall values up to $0.7$, we drop to sub $0.8$ precision fairly quickly.
Except for the on-the-fly results, the overall form resembles the original work roughly.

\begin{figure}[htbp]
  \centering
  \begin{subfigure}[b]{0.495\textwidth}
    \centering
    \includesvg[width=\linewidth]{images/imposter/reproduction_koppel_figures/fig4/blog/blog_roc_prec_recall_curve_r100_top100000_Different_Author_dif_imp_gen.svg}
    \caption{\dataBlog{}}
    \label{fig:blog_different_author}
  \end{subfigure}
  \hfill
  \begin{subfigure}[b]{0.495\textwidth}
    \centering
    \includesvg[width=\linewidth]{images/imposter/reproduction_koppel_figures/fig4/student_essays/student_roc_prec_recall_curve_r100_top100000_Different_Author_dif_imp_gen.svg}
    \caption{\dataStudent{}}
    \label{fig:student_essays_different_author}
  \end{subfigure}
  \caption{Recall-precision curves for the class different-author. Due to API limit restrictions, the test set for on-the-fly was smaller which is visible in the respective curves.}
  \label{fig:different_authors}
\end{figure}

For different authors as reference class, our results greatly differ from the original work.
Even though, \citet{koppel_determining_2014}'s precision scores for fixed and on-the-fly \imp{} generation are generally lower than those for the same-author reference class, our scores are way lower.
Notably, in our case, the on-the-fly method seems to perform best in this open-set scenario.
