\section{Exp. 3: Paraphrasing Chunks}

Initially, we hypothesized that smaller chunks of text would lead to better paraphrases, since smaller chunks are easer to process and control in terms of topic, suggesting that the separation of text into smaller chunks would be beneficial for the paraphrasing process.
We therefore designed an experiment to test this hypothesis.
We computed several paraphrasing measurements for the same input texts averaged over the number of chunks.

\begin{table}[]
\centering
\caption{Effect of increasing the number of chunks on different paraphrasers on different datasets. The asterisk * indicates that \texttt{qwen3-32b}'s syntactic measures have a peak at three chunks.}
\label{tab:impact_chunks_dataset_paraphraser}
\resizebox{\textwidth}{!}{%
\begin{tabular}{llll}
    \toprule
\textbf{Dataset} & \textbf{One-step} & \multicolumn{2}{l}{\textbf{Two-step}}                                                   \\
                 &                   & \textbf{Extraction-Generation} & \textbf{Translation}                                   \\
\midrule
\dataBlog{}           & --* & $\downarrow \ \text{Syntactic} $ & $\downarrow \ \text{Syntactic} $ \\
\dataGutenberg{}      & -- & --                                   & $\downarrow \ \text{Semantic}$ \\
\dataPan{}          & -- & \text{Syntactic Peak at two chunks}  & \text{Syntactic Peak at two chunks}   \\
\dataStudent{} & -- & $\downarrow \ \text{Syntactic}$      & $\downarrow \ \text{Syntactic}$       \\
\bottomrule
\end{tabular}%
}
\end{table}

The essence of our findings is summarized in \autoref{tab:impact_chunks_dataset_paraphraser}.
For two-step paraphrasers, the Gohsen $\Delta$ increases with increasing number of chunks due to decreasing average Syntactic similarity for the \dataBlog{} dataset and the \dataStudent{} dataset.
The impact of the number of chunks on the two-step translation based approach is viusalized in \autoref{fig:abl_chunks_blog_translation}.
On the \dataGutenberg{} dataset, chunking had no effect but on the translation based paraphraser where the semantic similarity and therefore, the Gohsen $\Delta$ decrease with increasing number of chunks.
Two-step approaches had a peak in syntactic measures at two chunks on the \dataPan{} dataset.
As visualized in \autoref{fig:abl_chunks_student_essays_llama}, chunking had no effect on one-step paraphrasing approaches.
Find more visualizations in \autoref{sec:app_chunks}.

In conclusion, multiple chunks can improve paraphrases in terms of syntactic diversity for some datasets using the two-step approach.
Since applying the two-step approach to $n$ chunk requires $2n$ API calls in the best case scenario and chunking has no effect on one-step paraphrases, we decided to forgo chunking our texts as part of our paraphrasing pipeline.

 
\begin{figure}[htbp]
    \centering
    \includesvg[width=\textwidth]{images/paraphrasing/experiments/chunks/setup/results/Blog/Translation_metrics_plot_category_Blog.svg}
    \caption{Average score over different temperature (standard deviation shaded) for different paraphrasing scores for the translation paraphraser.
    With increase of the number of chunks the syntactic scores decrease.}
    \label{fig:abl_chunks_blog_translation}
\end{figure}



\begin{figure}[htbp]
    \centering
    \includesvg[width=\textwidth]{images/paraphrasing/experiments/chunks/setup/results/Student_Essays/meta-llama-3.1-8b-instruct_metrics_plot_category_Student Essays.svg}
    \caption{Average score over different prompts (standard deviation shaded) for different paraphrasing scores for a LLAMA model. 
    This model is not affected by the number of chunks.}
    \label{fig:abl_chunks_student_essays_llama}
\end{figure}