\section{Exp.\ 5: Comparing \acs{av} Methods in Traditional Human-Human Scenario}
\label{sec:results_trad_av}


This experiment investigates how our \ac{llm}-based \imp{} generation performs relative to (a) traditional \imp{} generation within the \impAppr{}, and (b) \acl{sota} \ac{av} methods in a conventional human-human \ac{av} scenario. 
To this end, we construct \textcolor{orange}{10} same-author and \textcolor{orange}{10} different-author pairs from the \dataStudent{} dataset and evaluate the approaches across varying thresholds.

The \mirrorMinds{} approach assigns nearly every input pair to the same-author class, yielding a precision of approximately $0.5$. 
This behaviour is due to the nature of its paraphrases, which consist of single words. 
The discriminator interprets any candidate text as stylistically more similar than \mirrorMinds{}' minimal \imps{}, leading to a bias toward same-author predictions. 
In contrast, naive \ac{llm}-based \imp{} generation achieves the highest $F_1$ scores, although its precision remains between $0.6$ and $0.7$. 
This suggests that the generated \imps{} are still relatively easy, producing \acp{fp} in some cases.

\begin{figure}[h]
\centering
    \includesvg[width=\linewidth]{images/AV_comparison/detection_scenarios/f1/student_essays_Human-Human_threshold_f1s_curves_all_incl_baselines.svg}
  \caption[Traditional \ac{av} $F_1$ scores.]{$F_1$ scores for the same-author class across different thresholds on the \dataStudent{} dataset. 
\ac{llm}-based approaches achieve slightly better effectiveness than baselines across thresholds.}
  \label{fig:human-human_f1}
\end{figure}

$F_1$ scores generally decline for most approaches as the threshold increases. 
\ac{llm}-based methods maintain relatively stable effectiveness across thresholds due to consistently high recall values. 

Among the original baselines, \citet{koppel_determining_2014}'s fixed \imp{} generation performs best, with its text-length-informed variant showing slightly improved results. 
Both approaches exhibit high precision but low recall. 
The length-based \imp{} generation consistently achieves higher recall than the fixed variant. 
Incorporating content similarity into \imp{} selection does not significantly affect effectiveness, as both content-based and fixed \imp{} approaches yield similar outcomes. 
Recall curves across thresholds are provided in the Appendix (cf.~\autoref{sec:app_detection_scenarios}).
