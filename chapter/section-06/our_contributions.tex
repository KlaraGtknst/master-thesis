\subsection{Exp.\ 5: Comparing \acs{av} Methods in Traditional Human-Human Scenario}
\label{subsec:imp_gen_res}

We compared precision recall values for different thresholds for the baseline approaches and the fixed approach proposed by \citet{koppel_determining_2014}.
We find that the supervised baseline and our one-step paraphrasing approach (here: Naive \ac{llm}) exhibit the exact same values.
Further investigation of the actual values supports this perception, proving the same value pairs for different threshold.
We suppose that this is due to the relatively small sample size of \textcolor{red}{10} text pairs.

We find that in terms of optimization of the precision-recall trade-off, both the supervised baseline and our \ac{llm} extension to the \impAppr{} perform best for a threshold of $0.32$.
If we aim for high precision while considering the recall of secondary importance, we find that all but the supervised baseline and our \ac{llm} extension to the \impAppr{} perform better because their recall values are higher for the same maximal precision.
Though the criteria for optimal detection depends on the application at hand, we find that even recall of $0.4$ for precision of $1.0$ generally not to be more important than optimizing both recall and precision at the same time.

\begin{figure}[htbp]
    \centering
    \includesvg[width=\textwidth]{images/imposter/our_contribution/roc_prec_recall_curve_r100_top100000_Same_Author_dif_imp_gen.svg}
    \caption[Recall-precision curves for the \dataStudent{}.]{Recall-precision curves for the \textcolor{red}{10} samples of the \dataStudent{}.}
    \label{fig:sem_syn_blog}
\end{figure}

It is noteworthy that these findings do not align with our expectation of the \ac{llm} based \imp{} generation producing high precision results.