% results
\section{Exp. 6: Comparing \ac{av} Methods in different scenarios}

\begin{figure}[htbp]
\centering
    \includesvg[width=\linewidth]{images/AV_comparison/detection_scenarios/accuracy/student_essays_Human-Human_threshold_accs_curves_all_incl_baselines.svg}
  \caption{Accuracy curves for the class same-author across different threshold on \dataStudent{} dataset without \ac{llm} candidates. We find that \ac{llm} based approaches perform slightly above chance and better than the rest. For high thresholds, recall is not as high and thus, this approach seems not to be as bad as in other scenarios. 
  }
  \label{fig:human-human_acc}
\end{figure}

Baseline Mirror minds performs reasonable well compared to the other approaches due to its high recall.
You may find the recall curves across different threshold attached in the Appendix in \autoref{sec:app_detection_scenarios}.
Paraphrases are single words, which prompt the discriminator to always predict same-author since any candidate text contains more stylistic similarity than the single-word \imps{}.
We find this approach produces many \acp{fp}.


\paragraph{Human-\ac{llm}}
\begin{figure}[htbp]
  \centering
  \begin{subfigure}[b]{0.48\textwidth}
    \centering
    \includesvg[width=\linewidth]{images/AV_comparison/detection_scenarios/f1/student_essays_Human-LLM_threshold_f1s_curves_all_incl_baselines.svg}
    \caption{F1}
    \label{fig:human-lllm_f1}
  \end{subfigure}
  \hfill
  \begin{subfigure}[b]{0.48\textwidth}
    \centering
    \includesvg[width=\linewidth]{images/AV_comparison/detection_scenarios/precision/student_essays_Human-LLM_threshold_precisions_curves_all_incl_baselines.svg}
    \caption{Precision}
    \label{fig:student_essays_same_author}
  \end{subfigure}
  \caption{Different score curves for the class same-author. 
  We find that in the Human-\ac{llm} scenario, non-\ac{llm} \ac{av} approaches perform best. 
  Both Mirror minds and naive \ac{llm}-based have low precision values.
  Their high recall suggests, that \imps{} do not serve as hard negatives yet.}
  \label{fig:detec_scen_human-llm}
\end{figure}