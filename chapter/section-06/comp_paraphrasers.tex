\section{Exp.\ 2: Comparison of Paraphrasers}
\label{sec:comp_paraphrases}

To evaluate the quality of generated paraphrases, we conducted two experiments. 
In Exp.\ 2(a), we assessed paraphrasing using standard quantitative metrics, while in Exp.\ 2(b), we compared the extracted information and text lengths of the generated paraphrases to the ground truth metadata and original texts, for different paraphrasers respectively.

\paragraph{Exp.\ 2(a): Quantitative evaluation.}

Paraphrasing scores were computed separately for the \dataBlog{}, \dataGutenberg{}, and \dataStudent{} datasets, with each score derived from a single sample text per dataset and two parameter configurations per paraphraser.
\Cref{fig:sem_syn_blog} presents aggregated semantic and syntactic measurement scores for the \dataBlog{} dataset, while results for \dataGutenberg{} are provided in the Appendix in \Cref{sec:app_paraphrases}.

It is important to note that for our purpose, syntactic diversity is desirable.
High syntactic similarity values may reflect near-identical paraphrases due to n-gram overlap. 
Semantic similarity measures the content overlap between the paraphrase and the original text, often via vector-based cosine similarity. 
% The precise interpretation of these metrics remains somewhat unclear.

\begin{figure}[htbp]
    \centering
    \includesvg[width=\textwidth]{images/paraphrasing/experiments/sem_syn_scatter/Blog_sem_syn_scatter_grouped_by_Paraphraser_altered.svg}
    \caption[Comparison of paraphrasers on the \dataBlog{} dataset]{Average semantic $\diameter_{sem}$ and syntactic similarity $\diameter_{syn}$ for different prompts grouped by paraphraser on the \dataBlog{}.}
    \label{fig:sem_syn_blog}
\end{figure}

Analysis reveals two distinct clusters corresponding to one-step and two-step paraphrasers. 
Most one-step paraphrasers achieve lower syntactic and semantic similarity than two-step paraphrasers. 
The exception is the one-step paraphraser using \texttt{qwen3-32b}, which exhibits higher syntactic similarity than most two-step paraphrasers. 
The translation-based paraphraser emerges as an outlier in terms of both syntactic and semantic similarity. 
Overall, most paraphrases fall within the desired quadrant of high semantic similarity with low syntactic similarity.

It is noteworthy, however, that both prompts used to obtain these results proved inadequate upon manual evaluation of the generated paraphrases. 
The paraphrases were often too short and generally yielded poor performance when employed as \imps{} in the \impAppr{}. 
This underscores how reliance on automatic evaluation measures can lead to misleadingly favourable assessments of paraphrase quality prior to manual inspection.


\paragraph{Exp.\ 2(b): Evaluation of prompt adherence.}

To evaluate the \pextractor{}'s extraction quality, we compared the extracted topic, genre, and century against ground truth metadata for five samples each from the \dataBlog{}, \dataGutenberg{}, and \dataStudent{} datasets.
Genre and topic were evaluated in terms of semantic similarity, while century was assessed via the percentage deviation from the ground truth.

We observed that instructions for the \pextractor{} must follow the input text, as \acp{llm} tend to focus attention towards the end of the input. 
Otherwise, the \pextractor{} failed to produce the requested JSON format for long texts from the \dataGutenberg{} dataset. 


We also obtained the relative length difference, where values greater zero indicate that the paraphrase is longer than the reference, whereas values smaller than zero indicates the opposite.

Results across datasets are summarised in \Cref{tab:extraction_eval_stats}, with $\diameter$ and $\sigma$ denoting the mean and standard deviation across the five selected samples. 
The \dataBlog{} dataset proved most challenging for genre and topic extraction. 
Although the \dataGutenberg{} dataset exhibited the largest differences in text length between reference and paraphrase, the \pextractor{} achieved its best effectivity on this dataset.


\begin{table}[h]
\centering
\caption[Extraction effectiveness and length deviation for different datasets]{Extraction effectiveness and length deviation for different datasets. The scores were computed based on five text samples per dataset.
}
\label{tab:extraction_eval_stats}
\begin{tabular}{@{}lrrrrrrrr@{}} % numbers should be right aligned, text left aligned
\toprule
 &
  \multicolumn{2}{l}{\textbf{Genre}} &
  \multicolumn{2}{l}{\textbf{Century}} &
  \multicolumn{2}{l}{\textbf{Topic}} &
  \multicolumn{2}{l}{\textbf{Length}} \\
  \textbf{Dataset}
 &
  \textbf{\diameter} &
  \textbf{$\sigma$} &
  \textbf{\diameter} &
  \textbf{$\sigma$} &
  \textbf{\diameter} &
  \textbf{$\sigma$} &
  \textbf{\diameter} &
  \textbf{$\sigma$} \\
  \midrule
\dataBlog{}            & 0.38 & 0.06  & 0.99 & 0.02 & 0.04  & 0.05  & -0.10 & 0.73 \\
\dataGutenberg{}       & 0.58 & 0.14  & 1.00 & 0.04 & 0.30 & 0.15 & -1.00 & 0.00  \\
\dataStudent{} & 0.53 & 0.26 & 0.60 & 0.55 & 0.25 & 0.05  & 0.34 & 0.20 \\
  \bottomrule
\end{tabular}%
\end{table}
