\chapter{Discussion}
\label{chap:discussion}
In comparing the extended \impAppr{} with existing \ac{av} methods, our findings address the research questions and highlight inherent limitations of using \acp{llm} for \imp{} generation. 
\textcolor{red}{Note: Due to resource constraints, no experiments were conducted for the two-step paraphrase-based \imp{} generation or on the \dataPan{} dataset, which constitutes an important limitation.}

% Interpretation of results
% What do the findings mean in relation to your research questions or hypotheses?
Regarding the first research question (\autoref{enum:rq1}), the results suggest that controlled \ac{llm}-generated \imps{} can, in principle, serve as more effective hard negatives than traditional methods in \ac{av} scenarios restricted to human-authored texts. 
Our one-step \imp{} generation achieved slightly better results than the baselines, but overall performance remained unsatisfactory. 
We identify three potential reasons for this outcome.
(1) By design, hard negatives reduce same-author predictions. 
This can increase precision but imposes an upper limit on recall, accuracy, and $F_1$ scores. 
(2) Low semantic similarity between candidate texts and paraphrases (approximately $0.55$ to $0.65$) reduces stylistic resemblance due to the interdependence of topic and style, making the paraphrases overly simple as negatives. 
(3) The \impAppr{} relies on \ac{tfidf} vectors of space-free character 4-grams. 
Low syntactic similarity scores (around $0.2$) indicate minimal n-gram overlap between paraphrase and candidate, resulting in sparse vector representations that fail to capture the intended distinctions.
Consequently, \imps{} are inherently dissimilar and act as simple negatives. 

\textcolor{red}{TODO: Discuss the precise impact of syntactic similarity.}

Interestingly, \ac{av} experiments in exclusive \ac{llm}-\ac{llm} scenarios outperforms \ac{av} experiments in scenarios including human candidates, suggesting that human-written text complicates the task. 
One explanation is that \acp{llm} paraphrase human-authored candidates more aggressively than machine-gen\-erated candidates, producing \imps{} that are easier to distinguish from the candidate. 
Overall, our findings indicate that \acp{llm}-based \imp{} generation may not justify the computational overhead unless paraphrase quality can be substantially improved.

Our baseline and reproduction experiments yielded results inferior to prior reports, limiting our ability to fully validate the original \impAppr{}. 
While Mr.~Winter deemed our preprocessing strategy reasonable, he noted that our on-the-fly \imp{} generation was insufficient due to (1) bot-prevention mechanisms limiting web scraping, and (2) severe restrictions on API calls. 
Thus, differences in code, data selection, or preprocessing likely contributed significantly to reproducibility issues.

Concerning the second research question (\autoref{enum:rq2}), we observed that the multitude of syntactic and semantic paraphrase metrics commonly applied in literature complicates interpretation. 
Following \citet{gohsen_captions_2023}, we adopt aggregated syntactic and semantic scores for clearer two-dimensional comparison. 
In our implementation, \ac{rouge}-Lsum proved redundant, producing results identical to \ac{rouge}-L.

Low syntactic similarity emerged as an unintuitive predictor of paraphrase quality.
A measure of syntactic dissimilarity would better align with human judgment. 
Additionally, we introduced metrics for \pextractor{} performance and text length preservation. 
Extraction of metadata such as genre and topic performed poorly, while century extraction was reliable. 
Paraphrases of long \dataGutenberg{} texts diverged considerably from the original text in length.

Addressing the third research question (\autoref{enum:rq3}), we compared \ac{av} methods across human-human, mixed human-\ac{llm}, and \ac{llm}-\ac{llm} scenarios, as well as for \ac{llm} detection. 
\unmasking{} and \ac{ppmd} served as baselines for evaluating the extended \impAppr{}. 
Although computationally demanding, our method was intended as a task-specific solution rather than a general-purpose detector. 

Performance was generally poor across all methods, indicating inherently difficult text pairs. 
Most models failed to achieve acceptable precision–recall trade-offs, leading to low $F_1$ scores. 
While the extended \impAppr{} outperformed baselines in some cases, it is not yet practical due to three main issues. 
(1) Generated \imps{} are often too simplistic, resulting in excessive \acp{fp}, (2) substantial computational overhead, and (3) reliability problems due to \pgenerator{} failures or API rate limits.

We think that the limited quality of the extended \impAppr{} largely stems from insufficiently robust \imps{}. 
Performance was highly sensitive to prompt design, underscoring a central challenge of working with \acp{llm}. 
Prompt engineering remains ad-hoc, diverting focus from framework development toward instruction crafting. 
While larger \acp{llm} can leverage intrinsic knowledge under looser prompts, smaller models often require carefully tailored instructions~\citep{schmidt_llm_av_latin_24}. 

Theoretically, extending the \impAppr{} should increase precision at the cost of recall and $F_1$, as harder negatives reduce same-author predictions. 
However, this trade-off did not materialize as expected.

This work assumes that the traditional \impAppr{} was implemented correctly, and that observed discrepancies in reproducing original results arise from differences in input pair selection rather than implementation errors. 
All experiments were conducted on texts exceeding 700 words and performance on shorter texts remains untested, representing an open question for future work.
