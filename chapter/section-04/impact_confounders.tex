\section{Contextual factors/ Confounder impact on Authorial Style}
\label{sec:contextual_factors}
\textcolor{red}{TODO}

Authorial style is heavily influenced by contextual factors, or so-called confounders.
Confounders compose of the author's personal traits, customs an author adopts due to genre, register, type, and topic.
Since these concepts are too vague and ill-separable to be efficiently operationalized, stylometry research has focused on finding features invariant to domain variables~\citep{bischoff_importance_2020}.
As studies show, this undertaking has not yet borne measurable fruit. 
In the context of \ac{llm} detection, \ac{sota} models including DetectGPT~\citep{mitchell_detectgpt_2023} have proven vulnerable to \ac{ood} scenarios.
When faced with texts originating from a domain, language, or \ac{llm} other than those they were trained on, performance declines rapidly~\citep{Wu_ODD_challenges_2025}.




They claim that features frequent function words and word length have a high correlation with topic. \todo{????}
Hence, traditional author style models are highly susceptible to learning domain-specific features and 
prone to pick up domain artefacts unless domains are controlled, 
which imposes severe practical limitations.

% char 3-gram
\citet{bischoff_importance_2020} analyse the robustness of character trigrams as a feature for \ac{aa}.
They find that the character trigrams feature set is not robust in a cross-topic setting, but across two genres.



% error
\citet{altakrori_topic_2021} propose topic confusion.
It is a task specifically designed to evaluate \ac{aa} models' effectiveness.
The task splits the error into two parts:
\begin{itemize}
    \item \textbf{Models' confusion}: The model is confused about the topic of the text.
    \item \textbf{Features' inability}: The features are unable to capture the authors' writing style.
\end{itemize}

% dependency of author and topic
\citet{altakrori_topic_2021} present an example to explain the dependency of author and topic:
Consider a topic defined by a unique word distribution.
An author selects a subset of words from the topic due to the limited length of documents.
Since the authors choose wordings, i.e. synonyms, the dependency of the topic on the author varies from one author to another.

% features
% stylometric features
Based on empirical evidence (one paper, one dataset), stylometric features are rather topic-invariant \citep{altakrori_topic_2021}.
% POS
According to \citet{altakrori_topic_2021}, \ac{pos} tags capture stylistic variations in language grammar between authors.
% n-grams
Character-level n-grams are favourable over word-level n-grams in terms of cross-group error.
Hence, character-level n-grams are more topic-invariant than word-level n-grams.
