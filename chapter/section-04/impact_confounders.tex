\section{Impact of Confounders on Authorial Style}
\label{sec:contextual_factors}

State-of-the-art models for authorship analysis exhibit strong sensitivity to domain shifts. 
Effectivity often deteriorates sharply when models are applied in out-of-distribution scenarios, i.e.\ in cross-domain settings, a phenomenon largely attributable to the influence of confounders such as topic, genre, and register~\citep{Sundararajan_style_18,bischoff_importance_2020}

Confounders are problematic because they influence the very features used to characterise authorial style. 
Topical vocabulary, for example, can dominate lexical distributions, while genre-specific conventions shape syntax. 
As these factors cannot be cleanly separated from genuine stylistic markers~\citep{bischoff_importance_2020}, they obscure the style markers.

If topics can be represented by characteristic word distributions, then a document can be seen as a subset of words selected by the author, reflecting individual preferences in synonym choice~\citep{altakrori_topic_2021}. 
Consequently, texts by the same author on different topics may appear unrelated, whereas texts by different authors on the same topic may seem deceptively similar.

Empirical evidence confirms the severity of this problem.
Both contemporary \acp{av} methods~\citep{Thomas_cross_topic_24}, and \ac{llm} detection approaches, such as DetectGPT~\citep{mitchell_detectgpt_2023,Wu_ODD_challenges_2025} suffer significant effectiveness degradation in out-of-distribution scenarios.

As a result, authorship research has diverged into two main directions. 
One line of work aims to identify domain-invariant features, a challenge that remains largely unresolved~\citep{bischoff_importance_2020}. 
The other focuses on in-domain scenarios, in which confounding factors are deliberately controlled. % \ac{id}
While restricting tasks to a single topic or genre does not fully eliminate the entanglement between content and style, it reduces its impact sufficiently to produce stable and interpretable results. 
Moreover, some studies indicate that simply using domain or topic labels is insufficient to control for topic similarity in corpora, as this approach ignores semantic relationships between topics~\citep{sawatphol_cross_topic_av_24}.
