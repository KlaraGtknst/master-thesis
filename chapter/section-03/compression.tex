\subsection{Compression-based}
% compression models, e.g. RAR or GZIP 
This category of \ac{aa} approaches is based on general-purpose compression models such as RAR or PPMD. %(i.e. a variant of \ac{ppm}~\citep{tyo_state_2022}), LZW, GZIP, BZIP2 and 7ZIP.
Such models capture textual characteristics by exploiting repeated character sequences~\citep{stamatatos_survey_2009,neal_surveying_2018}. 
Natural language, in particular, allows for high compression ratios due to its strong predictability (English has an entropy of at most 1.75 bits per character). 
For example, PPMD employs finite-order Markov language models for compression, which are highly effective in predicting character sequences in natural text but are also sensitive to increased entropy caused by text obfuscation~\citep{bevendorff_divergence_based_2020}.
Accordingly, compression-based \ac{aa} methods are considered character-based approaches.

They are further classified as profile-based methods. In this framework, an author profile is first constructed for each candidate author by concatenating all texts attributed to them and then compressing the resulting sequence. 
The disputed text is subsequently concatenated with each author profile and compressed as well. 
The difference in compression size between (i) the concatenated profile with the disputed text and (ii) the profile alone is then calculated~\citep{stamatatos_survey_2009,elmanarelbouanani_authorship_2014,neal_surveying_2018}. 
The author whose profile yields the smallest difference is selected as the most likely author~\citep{stamatatos_survey_2009,elmanarelbouanani_authorship_2014}.

The rationale behind this approach is that texts written by the same author can typically be compressed more efficiently than texts produced by different authors~\citep{stamatatos_survey_2009,elmanarelbouanani_authorship_2014}.

% RAR is the most accurate one \citep{elmanarelbouanani_authorship_2014}.
% \citet{elmanarelbouanani_authorship_2014} include the Normalized Compressor Distance (NCD) as a distance measure for compression-based methods. % Chap. 4.2
% \citet{stamatatos_survey_2009} claim that probabilistic approaches are faster in comparison to compression models.
% \citet{neal_surveying_2018} state that LZ77 is a lossless data compression algorithm that is used to compress data by detecting duplicates.

% Compression-based models can also be considered similarity-based measures which are slow 
% since the compression algorithm is called for each training text \citep{stamatatos_survey_2009,neal_surveying_2018}.