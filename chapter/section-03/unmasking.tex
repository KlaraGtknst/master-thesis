\subsection{\unmasking{} Method}
\label{subsec:unmasking}

The \unmasking{} algorithm was first introduced by \citet{koppel_authorship_2004}\ as a meta-classification approach to \ac{av}. 
First, both the disputed text $d \in U$ and the candidate text $k_a \in K_a$ are split into non-overlapping chunks, which serve as samples from the unknown author and the candidate author $a \in A$ respectively. 
Second, the chunks are represented using word-frequency features of the $n=250$ most frequent words across the combined texts.
Third, a linear \ac{svm} is trained to discriminate between the two sets of chunks and the 10-fold cross-validation accuracy is recorded.
Fourth, the most discriminative features (e.g., the top three positive and top three negative weights) are removed.  
The third and fourth step are repeated until no features remain, yielding an accuracy degradation curve.  
Finally, a meta-classifier is trained on the degradation curve (and optionally on its first- and second-order gradients) to decide whether the two texts were written by the same author.  

The intuition is that, for same-author comparisons, the classifier’s accuracy quickly drops to chance level as the most informative features are eliminated. 
For different-author comparisons, classification remains feasible for longer, resulting in a slower degradation curve~\citep{stein_intrinsic_2011,tyo_state_2022,bevendorff_divergence_based_2020,stamatatos_survey_2009}. 

To address the applicability restricted to long texts, \citet{bevendorff_generalizing_2019,bevendorff_divergence_based_2020}\ proposed generalized \unmasking{}. 
Instead of splitting texts into contiguous chunks, they construct chunks by randomly sampling words in a bootstrap-aggregated manner. 
Words are drawn without replacement from a pool until the target chunk size is reached.
If the pool is exhausted, it is replenished. 
Because random sampling introduces variance, \unmasking{} is repeated multiple times and the resulting curves are averaged. 
The generalized \unmasking{} workflow is visualized in \autoref{fig:generalized_unmasking}. 
Except for the modified sampling strategy and the adjusted number of features removed per iteration (six in the original method~\citep{koppel_authorship_2004}, 10 in the generalized variant~\citep{bevendorff_generalizing_2019}), the procedure remains identical to the original \unmasking{} framework.

\begin{figure}[htbp]
    \centering
    \includesvg[width=\textwidth]{images/unmasking/generalized_unmasking.svg}
    \caption[Generalized \unmasking{} workflow.]{Workflow of generalized \unmasking{}~\citep{bevendorff_generalizing_2019}: 
    (1) Generate chunks by oversampling words from the disputed text $d$ and the candidate text $k_a$. 
    (2) Represent the chunks using the most frequent words as features. 
    (3) Train a linear \ac{svm} and record the 10-fold cross-validation accuracy. 
    (4) Remove the 10 most discriminative features. 
    (5) Repeat steps (3)–(4) until no features remain.}
    \label{fig:generalized_unmasking}
\end{figure}
