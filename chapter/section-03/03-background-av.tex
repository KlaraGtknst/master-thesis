\chapter{Background on Authorship Identification}
\label{chap:authorship_identification}
   AI
    - Woher kommt es?
        - initial statistische analyse
        - Dylo hypethese???
        - stylometry
        - Use case 1/ original use case: Literatur Forschung
        - Use case 2: Digital text forensics
    - AA
    - AV als Kernproblem von allem
        - Fokus hier, und LLM detection istr ein Plus, eher im Anhang
    - technischer Hintergrund: open/ closed set, one-class classification
    - state of the art models
    - \imp{} method

This section gives background on ...

\textcolor{red}{TODO}






\section{\acl{av}}

\subsection{Compression-based???}
\textcolor{red}{TODO}


\subsection{Unmasking Method}
\textcolor{red}{TODO}


\begin{figure}[htbp]
    \centering
    \includesvg[width=\textwidth]{images/unmasking/unmasking.svg}
    \caption{Unmasking.}
    \label{fig:unmasking}
\end{figure}
Our method is an extension of the original \impAppr{} by \citet{koppel_determining_2014}.
By varying the seed and temperature, we can generate as many texts as we want.
  
\subsection{\imp{} Method}
\label{sec:impostor_method_theory}

The \impAppr{} leverages random projections to lower dimensional spaces (i.e. random set of features set to zero is a projection).
\begin{figure}[htbp]
    \centering
    \includesvg[width=\textwidth]{images/imposter/imposter.svg}
    \caption{\imp{}.}
    \label{fig:impostor}
\end{figure}


