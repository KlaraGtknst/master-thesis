\chapter{Background on \ai{}}
\label{chap:authorship_identification}
%    AI
%     - Woher kommt es?
%         - initial statistische analyse
%         - Dylo hypethese???
%         - stylometry
%         - Use case 1/ original use case: Literatur Forschung
%         - Use case 2: Digital text forensics
%     - AA
%     - AV als Kernproblem von allem
%         - Fokus hier, und LLM detection istr ein Plus, eher im Anhang
%     - technischer Hintergrund: open/ closed set, one-class classification
%     - state of the art models
%     - \imp{} method

This section offers background information on \ai{}.
Starting with the origin of \ai{} as solving disputes over authorship of historical texts, we outline the initial statistical approaches and touch upon stylometry.
Next, we highlight why \ai{} is still relevant today, and break down some essential tasks of \ai{}.
We proceed with the technical background and \ac{sota} approaches.
We conclude with an outlook on the impact of \acp{llm} on \ai{}.

% \textcolor{red}{TODO}

\section{Historical Background and Early Approaches}
\label{sec:origin}

Research on authorship originated in the 19th century as an attempt to resolve disputes over the authorship of literary works. 
Augustus de Morgan was among the first to propose a quantitative approach in 1851, using word-length frequencies. 
Building on this idea, in 1887 Thomas C. Mendenhall carried out the first systematic analysis of word-length distributions in the works of Bacon, Marlowe, and Shakespeare, aiming to shed light on the authorship of Shakespearean plays~\citep{neal_surveying_2018,stamatatos_survey_2009}.
Word-length distributions quantify the number of words on the vertical axis by their length in characters on the horizontal axis. 
\citet{wordlengths_mendenhall_1887} concluded that differences between the curves of two disputed texts may indicate different authorship, whereas similarity in the curves is less reliable as evidence of shared authorship.

% Popular examples of \ai{} include a collection of Hebrew-Aramaic letters supposedly by a rabbinic scholar in Baghdad in the late 19th century~\citep{koppel_authorship_2004}.
% The latter denied having authored the text collection, which was contested by \citet{koppel_authorship_2004}.

% \section{First Approaches}

% Approaches of the \nth{19} century were limited to the analysis of word lengths.
In the early \nth{20}, statistical measures such as Zipf's law (1932) and Yule's characteristic (1944) were introduced. 
These measures captured word frequency distributions and vocabulary richness, laying the foundation for more formalized analyses of writing style~\citep{neal_surveying_2018,stamatatos_survey_2009}.

A decisive milestone came with the work of Mosteller and Wallace (1964), who applied Bayesian statistical analysis to the "The Federalist Papers".
Their computer-assisted approach, based on the frequencies of a subset of common function words (e.g., "and", "to"), demonstrated the potential of rigorous quantitative methods for \ac{aa}. 
This work is often regarded as the beginning of modern stylometry, the systematic study of quantifiable features of writing style.

Between the 1960s and the late 1990s, stylometric research flourished.
Around \num{1000} features were proposed, including sentence and word lengths, character and word frequencies, and vocabulary richness. 
However, the field faced two major limitations: the absence of benchmark datasets prevented systematic comparison of methods, and evaluation often relied on subjective inspection of visualizations (e.g., scatterplots) rather than standardized metrics~\citep{stamatatos_survey_2009}.

The late 1990s marked the transition into a new era. 
The growing availability of electronic text collections enabled the construction of benchmark datasets and more thorough evaluation. 
At the same time, \ac{ml} algorithms facilitated more expressive text representations, moving beyond simple frequency counts toward feature-rich models. 
As a result, the scope of \ai{} expanded from resolving historical literary questions to addressing practical, real-world applications such as \ac{av} in digital forensics~\citep{stamatatos_survey_2009}. 
Nevertheless, as \citet{abbasi_writeprints_2008} note, even by 2008 stylometric methods struggled with scalability across large author sets, diverse genres, and open-world scenarios.

% \subsection{Stylometry}

\begin{definition}
    [Stylometry]
    Stylometry refers to a linguistic research area, where literary style is quantified by statistical features.
\end{definition}
% In other words, stylometry is the statistical analysis of literary style between one writer or genre and another \citep{tyo_state_2022}.
Researchers working on stylometric features believe that subconscious authorial syntactic idiosyncrasies are quantifiably measurable and sufficient to define an author's unique style~\citep{neal_surveying_2018}. 
These stylometric features are also referred to as style markers, or stylistic features if they are the most effective discriminators of authorship. 
Static features are context-free such as function words, word-length distributions, vocabulary richness measures, and dynamic features context-dependent attributes and include $n$-grams and misspelled words~\citep{abbasi_writeprints_2008}.
% Stylometric features include lexical, syntactic and structural features \citep{stein_intrinsic_2011}.
% Research includes five subtasks \citep{neal_surveying_2018}:
% \begin{itemize}
%     \item \ac{aa}
%     \item \ac{av}
%     \item Author profiling
%     \item Stylochronometry
%     \item adversarial stylometry
% \end{itemize}
We outline prominent style marker taxonomy categories in \autoref{tab:stylometric_features} in the following. 
Generally speaking, the more detailed the style marker extraction process, the more noisy are the produced measures~\citep{stamatatos_survey_2009}.

% lexical
Lexical features treat text as a mere sequence tokens~\citep{stamatatos_survey_2009}.
Token units include character, word, and sentence.
We \autoref{tab:comp_lexical} we present characteristics of the character and word feature units.
Character (n-gram) counts, average word~\citep{stein_intrinsic_2011}, sentence length~\citep{stein_intrinsic_2011,abbasi_writeprints_2008}, are examples for character, sentence, and sentence unit-based lexical features.
% line length~\citep{abbasi_writeprints_2008}, word length distribution~\citep{abbasi_writeprints_2008}, 
% vocabulary richness~\citep{abbasi_writeprints_2008,neal_surveying_2018} ...) 
Errors in \autoref{tab:stylometric_features} refers to idiosyncratic features include misspellings, grammatical mistakes, and other usage anomalies~\citep{abbasi_writeprints_2008,neal_surveying_2018}.


\begin{table}[]
\centering
\caption{Comparison of a subset of lexical features~\citep{stamatatos_survey_2009}. Requirements refer to computational requirements such as a tokenizer.}
\label{tab:comp_lexical}
\resizebox{\textwidth}{!}{%
\begin{tabular}{lllll}
    \toprule
    \textbf{unit} & \textbf{complexity} & \textbf{language-independent} & \textbf{affected by noise} & \textbf{requirements} \\
    \midrule
    character & low & \checkmark & low & \xmark  \\
    word & high & \xmark  & high  & \checkmark    \\
    \bottomrule
    \end{tabular}%
}
\end{table}

% syntactic
Syntax or syntactic structure is the structural organization of sentences \citep{kurt_pehlivanoglu_comparative_2024}.
Given robust \ac{nlp} tools, syntactic features are considered more reliable than lexical features.
Well-established syntactic features built on function words, or on syntactic errors such as mismatched tense or sentence fragments, or use morpho-syntactic \ac{pos} tags for each word token for \ac{pos} tag (n-gram) frequencies~\citep{abbasi_writeprints_2008,stamatatos_survey_2009}.
It is evident, that every syntactic feature requires parsing or processing of natural language and is thus, language-dependent~\citep{neal_surveying_2018,stamatatos_survey_2009}.

% semantic
Semantic features capture meaning behind words, phrases, and sentences, such as through analysis of synonyms and semantic dependencies \citep{neal_surveying_2018}.
Semantic similarities between words, synonym or hyponym relationships are derived using WordNet, any thesaurus or latent semantic analysis.
Semantic dependencies such as the specification of a person with location, can boost classification performance when combined with lexical and syntactic information~\cite{stamatatos_survey_2009}.


% application specific (structural + content)
Structural features include text organization, layout, file extensions, font, sizes, colours, 
use of braces and comments (for analysing computer programs)~\citep{abbasi_writeprints_2008,neal_surveying_2018}.
Content-specific features include important keywords and phrases on certain topics such as word $n$-grams~\citep{abbasi_writeprints_2008}.
Domain-specific features include ratios of quoted words and external links, number of paragraphs, and paragraphs average length for the news article domain~\citep{potthast_stylometric_2018}


\begin{table}[]
    \centering
    \caption{Incomplete taxonomy of style markers from \citep{stamatatos_survey_2009}.}
    \label{tab:stylometric_features}
 
    \begin{tabular}{@{}ll@{}} % numbers should be right aligned, text left aligned
    \toprule
    \textbf{Category} & \textbf{Features} \\ 
    \midrule
    Lexical & Token-based \\ %(word/ sentence length, ...) \\
     & Vocabulary richness  \\
     & Word frequencies  \\
     & Word n-grams  \\
     & Errors \\
    %  \midrule
    % Character & Character types (letters, digits, ...)\\
    %  & Character n-grams (fixed length)  \\
    %  & Character n-grams (variable length) \\
    %  & Compression methods \\
    %  \midrule
    Syntactic & Part-of-Speech (POS)  \\
    %  & Chunks \\
    %  & Sentence and phrase structure  \\
    %  & Rewrite rule frequencies \\
     & Errors  \\
    %  \midrule
    Semantic & Synonyms \\
     & Semantic dependencies \\
    %  & Functional  \\
    %  \midrule
    Application-specific & Structural  \\
     & Content-specific\\
    %  & Language-specific \\
     \bottomrule
    \end{tabular}%

\end{table}
\section{Contemporary Applications} % Contemporary Relevance/ Perspective

While early work in \ai{} focused on resolving historical literary disputes, its present-day relevance lies primarily in digital forensics. % forensic linguistics: sapkota_cross_topic_2014
\ac{aa} provides critical tools for addressing cybercrime, where anonymity enables identity deception, harassment, and financial fraud~\citep{abbasi_writeprints_2008,chendu_authorship_2020,bhattacharjee_fighting_2024}. 
Concrete applications include the detection of fraudulent e-mails, impersonation on social media, and the identification of fake product reviews that undermine trust in e-commerce platforms. 

Beyond forensic contexts, plagiarism detection constitutes another significant application area. 
Here, authorship analysis assists in identifying unacknowledged reuse of material across domains ranging from literature to academia~\citep{neal_surveying_2018}. 
Together, these examples highlight the contemporary relevance of \ai{} in safeguarding digital integrity, supporting legal investigation, and upholding standards of authorship in both professional and academic contexts.

\section{Technical Background}

This section outlines the \ac{ml} principles and paradigms that underpin modern authorship analysis. 
We first introduce the main classification concepts relevant to \ac{av}. 
We then discuss training and evaluation practices, including domain shift scenarios. 
Finally, we present the principal categories of authorship analysis methods.

\subsection{\acl{ml} Principles}

\ac{aa} tasks are conventionally formulated as classification problems, where the objective is to attribute an anonymous text to one of a set of candidate authors. 
Unlike regression, which estimates a continuous value, classification yields a discrete label corresponding to the author’s identity.

\paragraph{Closed- vs. open-set classification.} 
In a closed-set scenario, the true author is guaranteed to be among the candidate set~\citep{stamatatos_survey_2009,koppel_authorship_2011,barlas_cross_domain_2020,boenninghoff_o2d2_2021,neal_surveying_2018}. 
In contrast, open-set classification acknowledges that the author of a disputed document may not belong to the candidate set~\citep{stamatatos_survey_2009,barlas_cross_domain_2020,neal_surveying_2018}. 

\paragraph{One-class classification.} 
In some cases, training data is available for only a single class, and the task is to decide whether a new sample belongs to this class.
If counterexamples, i.e. so-called outliers, are available, they are usually not considered to be representative of non-target class. 
This is formalized as one-class classification, where the model learns the characteristics of the target class without reliable counterexamples~\citep{stein_intrinsic_2011,koppel_authorship_2004}.

\paragraph{Multi-class classification.} 
The standard formulation involves multiple authors, each represented by several training texts. 
Here, the challenge is to discriminate among a large and often imbalanced set of classes~\citep{stamatatos_survey_2009,koppel_authorship_2004,elmanarelbouanani_authorship_2014} .


\subsection{Training and Testing}

Models are typically trained on one portion of the data (training set), tuned on another (validation set), and evaluated on a disjoint partition (test set). 
Any overlap between these partitions constitutes data leakage and invalidates the results~\citep{bischoff_importance_2020,altakrori_topic_2021,boenninghoff_o2d2_2021}. 

A major challenge in stylometry is covariate shift, i.e., a mismatch between the distribution of training and test data. 
This often arises from topic variability~\citep{boenninghoff_o2d2_2021}. 
Two common evaluation settings are:
\begin{itemize}
    \item \textbf{Cross-topic attribution}, where models trained on one set of topics are tested on previously unseen topics~\citep{altakrori_topic_2021}.  
    \item \textbf{Cross-domain attribution}, where training and test texts differ in topic or genre~\citep{barlas_cross_domain_2020}.  
\end{itemize}

\paragraph{Supervised vs. unsupervised learning.}  
Supervised methods require labelled training data and include classifiers such as \acp{svm}, decision trees, \acp{nn}, and linear discriminant analysis. 
\acp{svm} are particularly common in authorship analysis due to their robustness. 
Unsupervised methods do not rely on labels.
Clustering techniques or \ac{pca} have been used to uncover latent stylistic patterns or to reduce feature dimensionality~\citep{abbasi_writeprints_2008}.


\subsection{Authorship Analysis Methods}
\label{subsec:attribution_methods}

Approaches to authorship analysis can be grouped into three families~\citep{stamatatos_survey_2009}:

\paragraph{Profile-based methods.} 
All training texts of an author are concatenated into a single profile, from which a cumulative feature representation is extracted. 
This approach is effective when only short texts are available.
Profile-based methods ignore intra-author variation~\citep{stamatatos_survey_2009,elmanarelbouanani_authorship_2014,neal_surveying_2018}.  

\paragraph{Instance-based methods.} 
Here, each training text is treated as a separate instance of the author's style. 
This allows models to capture intra-author variability~\citep{stamatatos_survey_2009,altakrori_topic_2021,elmanarelbouanani_authorship_2014,neal_surveying_2018}.  

\paragraph{Hybrid methods.} 
Hybrid approaches combine both paradigms by representing texts individually while aggregating author profiles through feature-wise averages computed over an author’s texts~\citep{stamatatos_survey_2009}. 

\section{Tasks}

In this section we outline the main tasks of authorship analysis. 
We begin with the foundational problems of \ac{aa} and \ac{av}, before turning to derivative tasks such as author profiling,  author obfuscation, and plagiarism detection.

\subsection{\Acl{aa}}
\ac{aa} is the classical multiclass, single-label text classification task in which, given a set of candidate authors with known writings, the goal is to determine which author wrote a disputed text~\citep{koppel_authorship_2004}. 
Formally, let $A$ be the set of authors, $K=\bigcup_{a\in A} K_a$ the set of known texts, and $U$ the set of unknown texts. 
In the closed-set setting, each $d \in U$ must be attributed to exactly one $a \in A$. 
Variants such as cross-topic or cross-genre \ac{aa} introduce distributional shifts between training and test data, complicating the task~\citep{barlas_cross_domain_2020}.

\subsection{\Acl{av}}
\ac{av} addresses the problem of establishing whether a given text $t$ was written by a candidate author $a$, using a set of the author’s known writings $K_a$ as reference~\citep{koppel_authorship_2004}.
Disputed-candidate pair denotes the input of the text of unknown authorship and the candidate author's text in the following.
In contrast to \ac{aa}, \ac{av} does not have reliable negative examples, since assembling a representative sample of all texts not authored by 
$a$ is impossible. 
This limitation makes \ac{av} a more challenging classification problem than \ac{aa}~\citep{llm_detection_av_2025,neal_surveying_2018,koppel_authorship_2004}.
\ac{av} is framed as a one-class, a binary, or a similarity detection task depending on the methodological perspective~\citep{neal_surveying_2018,koppel_authorship_2004}.  
\ac{aa} can be reduced to a series of \ac{av} problems, where the other direction is typically not true~\citep{barlas_cross_domain_2020,tyo_state_2022}.
% Gespräch Martin Potthast 19.05.2025: problem formulation 2 is less common and in the context of very sparse (metadata) information:
% This task can also be formulated as whether two texts $t_1$ and $t_2$ are written by the same author 
% \citep{bevendorff_generalizing_2019,bevendorff_divergence_based_2020,embarcadero_ruiz_graph_based_2022,rivera_soto_learning_2021,ordonez_will_2020,futrzynski_pairwise_2021,weerasinghe_feature_vector_difference_2021,llm_detection_av_2025}.


% \subsection{Author Profiling}
% Author profiling infers sociolinguistic attributes of an author, such as age, gender, education, or mental health, from a set of texts. 
% It is grounded in the assumption that subconscious idiosyncrasies of writing style encode personal traits~\citep{emmery_adversarial_2021,stamatatos_survey_2009,elmanarelbouanani_authorship_2014}. 
% Profiling raises substantial ethical and privacy concerns due to the sensitivity of the inferred attributes.

% \subsection{Author Obfuscation}
% Author obfuscation is an adversarial task in which a text is deliberately modified to conceal the author's identity while preserving its semantic content. 
% It directly opposes \ac{aa} and \ac{av} by aiming to neutralize stylometric features~\citep{bischoff_importance_2020,bevendorff_divergence_based_2020,gohsen_task_oriented_2024}. 

% \subsection{Intrinsic Plagiarism Detection}   % style change detection
% Plagiarism detection is the task of identifying reused or unattributed content in texts~\citep{stein_intrinsic_2011,gohsen_task_oriented_2024}. 
% Unlike \ac{aa}, the goal is not to establish authorship but to uncover overlap between documents, regardless of author identity~\citep{elmanarelbouanani_authorship_2014}.
% Hence, plagiarism detection is an application of \ac{av}~\citep{rivera_soto_learning_2021}.


\subsection{\acs{llm} detection}
Given the ability of \acp{llm} to generate text closely resembling human writing, we can conceptualize \acp{llm} as one or multiple distinct authors.
This perspective allows us to frame \ac{llm} detection as an \ac{av} task, where given two texts (i.e. one of unknown authorship and one known to be generated by an \ac{llm} author) the objective is to determine whether they share the same author~\citep{llm_detection_av_2025}.



\section{Canonical Methods}

The following sections introduce \ac{sota} approaches to \ac{av}.
We begin with compression-based methods, continue with traditional and generalized Unmasking and end with the traditional \impAppr{}.

\subsubsection{Compression-based}
% compression models, e.g. RAR or GZIP 
This category of \ac{aa} approaches is based on general-purpose compression models such as RAR or PPMD. %(i.e. a variant of \ac{ppm}~\citep{tyo_state_2022}), LZW, GZIP, BZIP2 and 7ZIP.
Such models capture textual characteristics by exploiting repeated character sequences~\citep{stamatatos_survey_2009,neal_surveying_2018}. 
Natural language, in particular, allows for high compression ratios due to its strong predictability (English has an entropy of at most 1.75 bits per character). 
For example, PPMD employs finite-order Markov language models for compression, which are highly effective in predicting character sequences in natural text but are also sensitive to increased entropy caused by text obfuscation~\citep{bevendorff_divergence_based_2020}.
Accordingly, compression-based \ac{aa} methods are considered character-based approaches.

They are further classified as profile-based methods. In this framework, an author profile is first constructed for each candidate author by concatenating all texts attributed to them and then compressing the resulting sequence. 
The disputed text is subsequently concatenated with each author profile and compressed as well. 
The difference in compression size between (i) the concatenated profile with the disputed text and (ii) the profile alone is then calculated~\citep{stamatatos_survey_2009,elmanarelbouanani_authorship_2014,neal_surveying_2018}. 
The author whose profile yields the smallest difference is selected as the most likely author~\citep{stamatatos_survey_2009,elmanarelbouanani_authorship_2014}.

The rationale behind this approach is that texts written by the same author can typically be compressed more efficiently than texts produced by different authors~\citep{stamatatos_survey_2009,elmanarelbouanani_authorship_2014}.

% RAR is the most accurate one \citep{elmanarelbouanani_authorship_2014}.
% \citet{elmanarelbouanani_authorship_2014} include the Normalized Compressor Distance (NCD) as a distance measure for compression-based methods. % Chap. 4.2
% \citet{stamatatos_survey_2009} claim that probabilistic approaches are faster in comparison to compression models.
% \citet{neal_surveying_2018} state that LZ77 is a lossless data compression algorithm that is used to compress data by detecting duplicates.

% Compression-based models can also be considered similarity-based measures which are slow 
% since the compression algorithm is called for each training text \citep{stamatatos_survey_2009,neal_surveying_2018}.



\subsubsection{Unmasking Method}
\label{subsec:unmasking}

The meta-learning approach Unmasking algorithm was first proposed by \citet{koppel_authorship_2004}.
Meta learning is a technique where the system learns to learn based on learning successes and failures.
It is based on the idea that omitting discriminant features and the consequent drop in accuracy of the classifier can be used for inference of the author of the unseen text.
For Unmasking, (1) an unseen text is chunked, such that the non-overlapping chunks compose multiple samples belonging either to the author or to a different author.
Next, one \ac{svm} is trained for each candidate author to discriminate the disputed texts' chunks from the candidate author's texts.
The \acp{svm}' features are usually frequencies over the $n=250$ highest average frequency words.
(2) The 10-fold cross validation accuracy for the trained model are obtained.
(3) For the next iteration, omit the most discriminating features among those left.
(4) Repeat steps (3) and (4).
(5) Another linear \ac{svm} classifier is trained on the accuracy curve, its central-difference gradients (first- and second order), 
and its gradients sorted by magnitude.
This classifier is used to predict the whether the texts originate from the same author.

After a few iterations, the classifier is no longer able to discriminate between the unseen text and the texts of the true author~\citep{stein_intrinsic_2011,tyo_state_2022,bevendorff_divergence_based_2020,koppel_authorship_2004,stamatatos_survey_2009} 
Two texts are probably written by different authors if the differences between are robust to changes in the underlying feature set used to represent the documents.

To operationalize this idea, differences are measured using classification via cross-validation accuracy~\citep{koppel_authorship_2011,bevendorff_generalizing_2019,bevendorff_divergence_based_2020,potthast_stylometric_2018,koppel_authorship_2004}, 
creating a performance degradation curve~\citep{tyo_state_2022,koppel_authorship_2004}.
An \ac{svm} is trained to classify the degradation curve to determine whether two text originated from the same author~\citep{tyo_state_2022,bevendorff_generalizing_2019,koppel_authorship_2004}.
Steep decrease in the curve indicates that the two texts are similar, and thus, written by the same authors~\citep{potthast_stylometric_2018,koppel_authorship_2004}.
% Provided that the unseen text is very large, this method can handle small open candidate sets \citep{koppel_authorship_2011}.
% koppel_determining_2014, pg. 1 + bevendorff_generalizing_2019 chap. 3.1 incl. algo: based on text chunks of length >= 500 words each
% \citet{koppel_determining_2014,bevendorff_generalizing_2019} claim that effective unmasking requires input documents to be large 
% (i.e. > 10000 words~\citep{koppel_determining_2014}, book-length~\citep{bevendorff_generalizing_2019}, 
% $\geq$ 5000 words (500 words per chunk) \citep{bevendorff_divergence_based_2020}).
% Otherwise the training set becomes too sparse and no descriptive curves can be generated 
% \citep{bevendorff_generalizing_2019,bevendorff_divergence_based_2020}.

% generalized unmasking
\citet{bevendorff_generalizing_2019,bevendorff_divergence_based_2020} propose creating chunks by oversampling words in a bootstrap aggregating manner. 
Each text is a pool of words, from which words are sampled without replacement.
The pool is replenished if it is exhausted before the chunk has sufficiently many words.
Since the random sampling of unmasking features introduces variance, unmasking is performed multiple times and the curves are averaged.
The algorithm is displayed in \autoref{alg:generalized_unmasking}.
The content of the while loop is, except the number of removed features (\citep{koppel_authorship_2004}: 6 total), similar to the original unmasking algorithm \citep{koppel_authorship_2004}.

\begin{algorithm}
    \caption{Generalized Unmasking Algorithm \citep{bevendorff_generalizing_2019,bevendorff_divergence_based_2020}}
    \label{alg:generalized_unmasking}
    \begin{algorithmic}[1]
    \Procedure{Unmasking}{$A$, $B$}
        \Comment{$A$, $B$: input documents}
    
        \State $\mathcal{C}_A \gets \text{RandomChunks}(A, 30, 700)$ \Comment{30 chunks, 700 words each}
        \State $\mathcal{C}_B \gets \text{RandomChunks}(B, 30, 700)$
        \State $\mathcal{F} \gets \text{TopFreqWords}(A, B, 250)$
        \State $\mathcal{C} \gets \mathcal{C}_A \cup \mathcal{C}_B$

        
        \While{$|\mathcal{F}| \geq 0$}
        \State $a \gets \text{CVAcc}(\mathcal{C}_A, \mathcal{C}_B, \mathcal{F}, linSVM)$ \Comment{Append $10$-fold cross-validation accuracy}
        \State $\mathcal{F} \gets \mathcal{F} \setminus \mathcal{F}_{\text{top}}^{\pm}$ \Comment{Remove top $5$ most significant positive and negative features}
    
        \EndWhile
    
        \State \Return List of recorded accuracies $a$
    \EndProcedure
    \end{algorithmic}
\end{algorithm}

% hyperparameters
% The most important hyperparameters are the number of chunks, the number of words per chunk, the size of feature vectors, 
% the number of word removals per round, and the number of averaged unmasking runs.
% More chunks result in generally shallower curves while shorter features vectors or more removals produce steep curves.
% Ideally, curves are not too steep and granular enough to allow distinguishing between different same and different author pairs.
% \citet{bevendorff_bias_2019} recommend 25 to 50 chunks, vector sizes of 250 to 400 features, not fewer than 5, yet not more than 20 removals per round, 
% between 500 and 700 words per chunk and about 10 runs to average for a curve.
% They increase the minimal distance between the \ac{svm} hyperplane and the decision boundary, i.e. their confidence parameter $c$, to increase precision.
% In a medium- to high-assurance scenario (where \acp{fp} should be avoided, but are not entirely critical), they recommend $c \geq 0.6$.
% If \acp{fp} should be avoided at all costs, they recommend $c \geq 0.7$.
% \citet{bevendorff_bias_2019} claim that, for this approach, hyperparameter tuning is simpler than for black box approaches.

% % metric results
% \citet{bevendorff_bias_2019} report that the generalized unmasking approach heavily prioritizes precision 
% opposed to compression-based approaches that balance precision and recall.
    






\begin{figure}[htbp]
    \centering
    \includesvg[width=\textwidth]{images/unmasking/unmasking.svg}
    \caption{Workflow of Generalized Unmasking~\citep{bevendorff_generalizing_2019}: (1) Create chunks by oversampling words of disputed and candidate texts and represent them using word frequencies. (2) Obtain \ac{svm} accuracy. (3) Eliminate most discriminative features. (4) Repeat from (3)-(4).}
    \label{fig:unmasking}
\end{figure}



  
\subsubsection{\imp{} Method}
\label{sec:impostor_method_theory}


\begin{definition}
    [\imp{} method]
    This method extends the ngram-unmasking method, i.e. iteratively omitting most influencely features (repeated feature subsampling \citep{koppel_determining_2014})
    from a trained classifier and classifying the accuracy drop.
    It takes score of how often an author is predicted after each feature-elimination step.
    The final prediction is made based on this score \citep{tyo_state_2022}.
\end{definition}


\begin{definition}
    [Hard Negative Mining]
    This method updates the model during training only with the most difficult examples in each batch.
    In the \ac{aa} context, difficult is defined as the most similar two texts from different authors, 
    which makes the decision the most difficult.
    \citet{tyo_state_2022} claim that the \ac{av} setting is strictly easier since 
    it most compare to only a single text.
    Due to the fact, that the most difficult example is model-dependent, \ac{av} problems can be made harder 
    but they can not exist of exactly the hardest negatives.
\end{definition}


\begin{definition}
    [Domain]
    The domain include topic, genre, register, idiolect, time period etc. \citep{bischoff_importance_2020}.
\end{definition}
  
\begin{definition}
    [Domain variables]
    These include topic, genre and language \citep{bischoff_importance_2020}.
\end{definition}

\begin{definition}
    [within-domain]
    Experiments with P=Q.
    Hence, it is necessary to ensure all texts are mutually from the same domain \citep{bischoff_importance_2020}.
    \begin{table}[tbp]
        \centering
        \caption{Typical scheme $S_1$ for \ac{aa} problem instances, where A, B, are authors and P, Q domains and 
        the vertical mapping denotes which author has written in which domain. 
        For training, texts from A and B take turn; for testing, previously unseen texts from A and B are used \citep{bischoff_importance_2020}.}
        \label{tab:within_domain_aa}
        \begin{tabular}{|l|ll|ll|}
        \hline
        \textbf{Scheme $S_1$} & \multicolumn{2}{l|}{\textbf{training}} & \multicolumn{2}{l|}{\textbf{testing}} \\ \hline
        \textbf{authors} & \multicolumn{1}{l|}{A} & B & \multicolumn{1}{l|}{A} & B \\ \hline
        \textbf{domains} & \multicolumn{1}{l|}{P} & Q & \multicolumn{1}{l|}{P} & Q \\ \hline
        \end{tabular}%
    \end{table}
\end{definition}


The \impAppr{} leverages random projections to lower dimensional spaces (i.e. random set of features set to zero is a projection).
\begin{figure}[htbp]
    \centering
    \includesvg[width=\textwidth]{images/imposter/imposter.svg}
    \caption{\imp{}.}
    \label{fig:impostor}
\end{figure}

% AV -> open-set
\ac{av} is an open-set problem, meaning that the author of an anonymous document 
may or may be not be part of the set of candidate authors.

% AA -> closed-set
\ac{aa} is a closed-set problem, meaning that the author of an anonymous document
is part of the set of candidate authors.
For each candidate author, writing samples are available.
The task is to determine the author of the anonymous document from the set of candidate authors.

% reduction: closed-set AA -> open-set AV
\citet{koppel_determining_2014} state that all closed-set \ac{aa} problems are reducible to the \ac{av} problem.
The reverse is not true.
To reduce the \ac{aa} problem to the \ac{av} problem, we solve a \ac{av} problem, i.e. if text was written by a candidate author, 
for each of the respective candidates.
Ideally, we receive one positive answer for the correct candidate author and negative answers for all other candidates.

% complexity
\citet{koppel_determining_2014} explain that the \ac{av} problem is more complex than the \ac{aa} problem.
They claim that the ability to solve a closed-set \ac{aa} problem does not imply the ability to solve an open-set \ac{av} problem.

% open-set identification/ AA = many candidates problem
\citet{koppel_determining_2014} define the many-candidates problem, or the so-called open-set identification problem:
Given a large set of candidate authors, determine which, if any, of them wrote a given anonymous document.
According to \citet{koppel_determining_2014}, the many-candidates problem can be solved reasonably well: \autoref{lst:many_candidate_algo}.
\section{Impact of \acs{llm} on Authorship Verification}

With the advances in \ac{gai} come risks and opportunities.
Similarly, we can frame \acp{llm} as authors weakening our trust in the authencity of digital texts, but also see them as valuable tools to enhance our regulatory methods.

\subsection{\acsp{llm} as authors}
There is no general definition of when a text is \ac{llm} generated rather than co-created by humans with \ac{llm} assistance.
% Obviously, fully generated texts should be marked as \ac{llm} generated.
Minor human edits of \ac{llm} generated texts do not change the fact that the core content was \ac{llm} generated.
If \acp{llm} are used for grammar checking, polishing, and editing suggestion the primary substantial contribution was human.
One could denote these texts "\ac{ai}-revised Human-Written Text"~\citep{wang_stumbling_2024}.

With advances of generative models with regard to mimicking human writing, we have to face the fact that \acp{llm} will play a crucial role in any authorship analysis related tasks from now on.
\citet{llm_detection_av_2025} claim that \ac{llm} detection is not an \ac{aa} task, i.e., a closed-set binary classification where both classes are sufficiently discriminative, but an \ac{av} task, i.e., an open-set one-class classification problem. 

% differences: also more in ~\citep{wang_stumbling_2024}
Despite the advances, there are still some statistical differences on \ac{llm} generated and human authored texts.
\ac{llm} generated texts lack lexical diversity, overuses certain adjectives (e.g. "innovative") and produces longer, more complex sentences.
Moreover, \acp{llm} possess stylistic fingerprints and memorize patterns from the training data.
% lengths
Furthermore, word length averages and distributions across genre differ for \acp{llm} and humans~\citep{llm_detection_av_2025}.

% future of LLMs
However, as \acp{llm} progress, basic heuristics applied by human detectors no longer suffice.
\acp{llm} will become more human-like and thus, \ac{llm} detection will increasingly resemble a human authorship classification task~\citep{llm_detection_av_2025}.


\subsection{\acsp{llm} as Discriminator}
\label{sec:llm_discriminator}

While \acp{llm} can generate coherent text, their performance as direct discriminators in \ac{aa} tasks varies. 
Words with similar meaning, such as "color" and "colour", are mapped to similar vector representations~\citep{altakrori_topic_2021}.
This reduces sensitivity to subtle language system differences, which are often critical for identifying an author. 
Consequently, relying on \acp{llm} as the primary discriminator in \ac{aa} or \ac{av} tasks may be suboptimal.

Rather than serving as standalone discriminators, \acp{llm} are more appropriately employed as supporting tools. 
For instance, they can generate cross-domain training data to improve model robustness. 
Moreover, prior methods assessed the extent to which text changes under \ac{llm}-based paraphrasing, since machine generated text tends to undergo minimal alteration, whereas human-authored text exhibits greater variation~\citep{mao_raidar_2024}.
Conversely, in contexts where privacy is a concern, \acp{llm} can facilitate author obfuscation through controlled paraphrasing. 

% \section{\ac{av} as One-Class categorization}
\label{sec:av_one_class}

% \citet{koppel_authorship_2004} claim research had shown that linear separators work well for text categorization.
% Linear models include Naive Bayes (linear separator for two classes), 
% Window and Exponential Gradient and linear \acp{svm} \citep{koppel_authorship_2004}. 

% \subsection{\ac{aa} framework}
% \citet{koppel_authorship_2004} state that the following framework solve a number of real world \ac{aa} problems:
% \begin{enumerate}
%     \item Construction of appropriate feature vectors
%     \item Construction of a distinguishing model via a learning algorithm
%     \item Assessment of effectiveness of methods using k-fold cross-validation or bootstrapping
% \end{enumerate}

\subsection{Chunks}
\citet{koppel_authorship_2004} propose chunking texts such that each chunk is of approximately equal length, 
and at least 500 words without breaking paragraphs. 

\subsection{Definitions}
For author $A$ and book $X$, \citet{koppel_authorship_2004} define the following:
If $A$ is not the author of $X$, $A_X$ is the set of all works by author $A$.
If $A$ is the author of $X$, $A_X$ is the set of all works by author $A$ except $X$.
A pair of $A_X$ and $X$ is called \emph{same-author} if X was authored by $A$.
A pair of $A_X$ and $X$ is called \emph{different-author} if $X$ was not authored by $A$.

\subsection{Initial feature set}
The initial feature set consists of the 250 words with the highest average (over $X$ and $X_A$) frequency \citep{koppel_authorship_2004}.

\subsection{Features for meta-classifier}
\citet{koppel_authorship_2004} propose the following features for the meta-classifier 
(where $i$ is the number of elimination steps):
\begin{itemize}
    \item accuracy after $i$ elimination steps
    \item accuracy difference between round $i$ and round $i+1$
    \item accuracy difference between round $i$ and round $i+2$
    \item $i^{th}$ highest accuracy drop in one iteration
    \item $i^{th}$ highest accuracy drop in two iterations
\end{itemize}

The vectors are grouped by \emph{same-author} and \emph{different-author} pairs and thus, 
used to train a meta-learning scheme.

\subsection{Negative examples for Elimination method}
\citet{koppel_authorship_2004} state that negative examples are neither exhaustive nor representative.
They propose using words of several authors $A_1, ..., A_n$ roughly filling the same profile as candidate $A$ 
in terms of geography, chronology, culture and genre.
$A_1, ..., A_n$ are said to collectively represent class not-$A$.

\subsection{Elimination method}

The elimination method is only used to overrule positive predictions.
Hence, it can eliminate \acp{fp}.
One can frame it as a filter which is applied after or before unmasking.

% training
\citet{koppel_authorship_2004} learn a model for $A$ and against not-$A$, 
and multiple models for $A_i$ and against not-$A_i$.
% inference
Then, $X$ is tested against all of these models.
$A(X)$ is the percentage of examples of $X$ classed as $A$ rather than not-$A$ 
(i.e., $A_i(X)$ analogously).
If $A(X)$ is not larger than all $A_i(X)$, $A$ is not the author of $X$.
If $A(X)$ is larger than all $A_i(X)$, conclude nothing.




