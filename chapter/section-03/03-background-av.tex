\chapter{Background on \ai{}}
\label{chap:authorship_identification}
%    AI
%     - Woher kommt es?
%         - initial statistische analyse
%         - Dylo hypethese???
%         - stylometry
%         - Use case 1/ original use case: Literatur Forschung
%         - Use case 2: Digital text forensics
%     - AA
%     - AV als Kernproblem von allem
%         - Fokus hier, und LLM detection istr ein Plus, eher im Anhang
%     - technischer Hintergrund: open/ closed set, one-class classification
%     - state of the art models
%     - \imp{} method

This section provides background on \ai{}. 
We begin with its historical roots, tracing early efforts in \ac{aa} and the emergence of statistical methods, with particular attention to stylometry. 
We then discuss the enduring relevance of \ai{} in contemporary research and applications, outlining the key tasks that define the field. 
Building on this, we present the technical background necessary to understand current methodologies, followed by a review of \ac{sota} approaches. 
Finally, we consider how recent advances in \ac{llm} development are reshaping the landscape of \ai{} and its future prospects.

\section{Historical Background and Early Approaches}

Research on authorship originated in the \nth{19} century as an attempt to resolve disputes over the authorship of literary works. 
Augustus de Morgan (1851) was among the first to propose a quantitative approach, using word-length frequencies. 
Building on this idea, Thomas C. Mendenhall (1887) carried out the first systematic analysis of word-length distributions in the works of Bacon, Marlowe, and Shakespeare, aiming to shed light on the authorship of Shakespearean plays~\citep{neal_surveying_2018,stamatatos_survey_2009}.

% Popular examples of \ai{} include a collection of Hebrew-Aramaic letters supposedly by a rabbinic scholar in Baghdad in the late 19th century~\citep{koppel_authorship_2004}.
% The latter denied having authored the text collection, which was contested by \citet{koppel_authorship_2004}.

% \section{First Approaches}

% Approaches of the \nth{19} century were limited to the analysis of word lengths.
The \nth{20} century brought new measures, such as Zipf's law (1932) and Yule's Characteristic (1944).
Zipf's law describes the relationship between rank and frequency of words.
Yule's Characteristic was proposed by George Yule as a first attempt to measure vocabulary richness via word frequencies~\citep{neal_surveying_2018,stamatatos_survey_2009}.

A massive milestone was proposed in 1964 by Mosteller and Wallace with their computer-assisted stylometry on "The Federalist Papers" from the \nth{18} century.
Their discriminations were built on a Bayesian statistical analysis of the frequencies of a small set of common words such as "and" or "to".
This approach instigated stylometry, the research of finding features that quantify writing style.
In the peak years of stylometry between 1964 and the late 1990s, around \num{1000} different measurements were proposed.
Such measures include sentence or words lengths, word or character frequencies, and vocabulary richness.
The main shortcomings of stylometry in the \nth{20} century included the lack of method comparison due to the absence of suitable benchmark data and the mostly subjective evaluation of individual methods via visual inspection of scatterplots~\citep{stamatatos_survey_2009}.

The late 1990s initiated a new era of \ai{}.
The growing amount of electronic texts and benchmark data marked significant progress towards objective comparative evaluation of methods.
Moreover, \ac{ml} algorithms allowed for more expressive text representations than before. 
The focus shifted from answering disputed literary questions to solving real-world problems and creating practical applications~\citep{stamatatos_survey_2009}.
According to \citet{abbasi_writeprints_2008}, until 2008 stylometry not only lacked of scalability in terms of number of authors and across genres, but also was applied exclusively to closed-world scenarios.


% % potential introduction to style chapter
% \citet{bischoff_importance_2020} assume that each author has a unique style, unconsciously encoded in their writing.
% This style depends on the author's personal traits, customs an author adopts due to genre, register, type, and topic.
% These concepts are too vague and ill-separable to be efficiently operationalized.
% The goal is to discover a set of style markers more likely to be determined by the author's personality than by domain customs.

% They claim that features frequent function words and word length have a high correlation with topic. \todo{????}
% Hence, traditional author style models are highly susceptible to learning domain-specific features and 
% prone to pick up domain artefacts unless domains are controlled, 
% which imposes severe practical limitations.

% % char 3-gram
% \citet{bischoff_importance_2020} analyse the robustness of character trigrams as a feature for \ac{aa}.
% They find that the character trigrams feature set is not robust in a cross-topic setting, but across two genres.



\section{Stylometry}
\label{sec:stylometry}

Stylometry is the quantitative analysis of linguistic style through measurable textual features, with the aim of distinguishing between authors.
The central assumption of stylometry is that authors unconsciously leave behind consistent idiosyncratic syntactic traces, so-called style markers, that can be exploited for authorship analysis tasks~\citep{neal_surveying_2018,bischoff_importance_2020}. 
These markers are thought to reflect relatively stable personal traits and habitual stylistic choices, rather than superficial properties imposed by domain factors such as genre, register, or topic.
Register refers to systematic variations in language arising from situational context, including lexical and grammatical characteristics.
For example, a register can be characterised by the situation and purpose of the text~\citep{register_2025,register_2019}. 
The principal challenge in stylometry is to identify markers that reliably reflect authorial characteristics rather than domain-specific conventions.

Stylometric features, organised into broad categories in \Cref{tab:stylometric_features}, span from lexical and character statistics to semantic and application-specific indicators, with the latter often being noisier due to reliance on processing tools~\citep{stamatatos_survey_2009}.

% Static features, such as function words, word-length distributions, vocabulary richness measures, are context-free.
% Dynamic features are context-dependent attributes and include n-grams and misspelled words~\citep{abbasi_writeprints_2008}.


\begin{table}[h]
    \centering
    \caption[Incomplete taxonomy of style markers.]{Incomplete taxonomy of style markers from~\citep{stamatatos_survey_2009}.}
    \label{tab:stylometric_features}
 
    \begin{tabular}{@{}ll@{}} % numbers should be right aligned, text left aligned
    \toprule
    \textbf{Category} & \textbf{Features} \\ 
    \midrule
    Lexical & Token-based \\ %(word/ sentence length, ...) \\
     & Vocabulary richness  \\
     & Word frequencies  \\
     & Word n-grams  \\
     & Errors \\
    %  \midrule
    % Character & Character types (letters, digits, ...)\\
    %  & Character n-grams (fixed length)  \\
    %  & Character n-grams (variable length) \\
    %  & Compression methods \\
    %  \midrule
    Syntactic & Part-of-Speech (POS)  \\
    %  & Chunks \\
    %  & Sentence and phrase structure  \\
    %  & Rewrite rule frequencies \\
     & Errors  \\
    %  \midrule
    Semantic & Synonyms \\
     & Semantic dependencies \\
    %  & Functional  \\
    %  \midrule
    Application-specific & Structural  \\
     & Content-specific\\
    %  & Language-specific \\
     \bottomrule
    \end{tabular}%

\end{table}

\paragraph{Lexical features.} 
Lexical features treat text as a sequence of tokens, such as characters, words, or sentences~\citep{stamatatos_survey_2009}. 
Distributions over character n-grams (i.e. sequences of $n$ consecutive characters) counts, word-length~\citep{stein_intrinsic_2011}, and sentence-length~\citep{stein_intrinsic_2011,abbasi_writeprints_2008} are typical for character, word, and sentence unit-based lexical features that disregard token order in the \ac{bow} fashion~\citep{neal_surveying_2018}, respectively. 
Character-level features are computationally inexpensive, language-independent, and relatively robust to noise, whereas word-level features require tokenization, are language-dependent, and more sensitive to noise~\citep{stamatatos_survey_2009}.
Idiosyncratic spelling or grammatical mistakes are also considered discriminative lexical features~\citep{abbasi_writeprints_2008,neal_surveying_2018}. 

\paragraph{Syntactic features.}
Syntactic features exploit structural information such as \ac{pos} n-grams, function words, or recurrent syntactic errors~\citep{stamatatos_survey_2009,abbasi_writeprints_2008}.
They require robust \ac{nlp} pipelines, making them computationally more expensive than lexical features and inherently language-dependent~\citep{neal_surveying_2018,stamatatos_survey_2009}.

\paragraph{Semantic features.}
Semantic features include synonym or hyponym relations, or semantic dependencies. 
They usually complement lexical and syntactic features. 
Semantic features depend on external resources (e.g. WordNet, a lexical database of semantic relations~\citep{wordnet_1995}).


\paragraph{Application-specific features.} % (structural + content)

Application-specific features can include structural or content-related aspects. 
Structural features capture general characteristics, such as layout, file extensions, font choices, sizes, and colours~\citep{abbasi_writeprints_2008,neal_surveying_2018}. 
Certain features, in contrast, refer to elements that are only meaningful within a particular domain, for instance, the use of salutations and signatures in emails or programming constructs in source code. 
Content-specific frequent words may be useful within a single genre, but they should not be used across topics or genres since they conflate authorial idiosyncrasies with semantic information~\citep{abbasi_writeprints_2008}.
% Domain-specific features include ratios of quoted words and external links, number of paragraphs, and paragraphs average length for the news article domain~\citep{potthast_stylometric_2018}


In summary, stylometry's objective is to extract informative style markers which minimise influences from genre, register, or topic. 
While lexical features remain the most widely used due to their simplicity, more advanced syntactic and semantic representations, as well as domain-specific markers, can be crucial for tackling diverse real-world scenarios.

\subsection{Digital Forensics}

Besides \ai{} for literary texts authored long before today, \ai{} is particularly relevant to digital forensics.
\ai{} shapes the combat against cybercrime where online anonymity encourages malicious behaviour of individuals~\citep{abbasi_writeprints_2008}.
Unsuspecting users fall victim to email scams, identity deceptions in social media applications, fake ratings of vendors, or advance-fee scams, where victims are argued into providing bank account information of or transfer money by elaborate deception~\citep{abbasi_writeprints_2008,neal_surveying_2018}.
Plagiarism ranges from arts to academia~\citep{neal_surveying_2018}.
\subsection{Technical Background}



% AV -> open-set
\ac{av} is an open-set problem, meaning that the author of an anonymous document 
may or may be not be part of the set of candidate authors.

% AA -> closed-set
\ac{aa} is a closed-set problem, meaning that the author of an anonymous document
is part of the set of candidate authors.
For each candidate author, writing samples are available.
The task is to determine the author of the anonymous document from the set of candidate authors.

% reduction: closed-set AA -> open-set AV
\citet{koppel_determining_2014} state that all closed-set \ac{aa} problems are reducible to the \ac{av} problem.
The reverse is not true.
To reduce the \ac{aa} problem to the \ac{av} problem, we solve a \ac{av} problem, i.e. if text was written by a candidate author, 
for each of the respective candidates.
Ideally, we receive one positive answer for the correct candidate author and negative answers for all other candidates.

% complexity
\citet{koppel_determining_2014} explain that the \ac{av} problem is more complex than the \ac{aa} problem.
They claim that the ability to solve a closed-set \ac{aa} problem does not imply the ability to solve an open-set \ac{av} problem.

% open-set identification/ AA = many candidates problem
\citet{koppel_determining_2014} define the many-candidates problem, or the so-called open-set identification problem:
Given a large set of candidate authors, determine which, if any, of them wrote a given anonymous document.
According to \citet{koppel_determining_2014}, the many-candidates problem can be solved reasonably well: \autoref{lst:many_candidate_algo}.


\begin{definition}
    [One-class classification]
    A classification problem where the classifier is trained on samples of a single class.
    If counterexamples, i.e. so-called outliers, are available, they are usually not considered to be representative of \textit{non-target class}.
    Hence, the classifier has to learn the concept of the target class in the absence of discriminating features 
    \citep{stein_intrinsic_2011,koppel_authorship_2004}.
    Examples of one-class classification are intrinsic plagiarism analysis and \ac{av}.
    Approaches to one-class classification fall into the following categories \citep{stein_intrinsic_2011}:
    \begin{itemize}
        \item One-class density estimation, e.g., Naive Bayes
        \item One-class boundary estimation
        \item One-class reconstruction
    \end{itemize}
\end{definition}

\begin{definition}
    [Open-set classification]
    The true author is not necessarily included in the set of candidate authors \citep{stamatatos_survey_2009,barlas_cross_domain_2020,neal_surveying_2018}.
    It is a generalization of the closed-set classification problem allowing for an unknown author using a threshold for similarity \citep{neal_surveying_2018}.
\end{definition}

\begin{definition}
    [Closed-set classification]
    The true author is one necessarily one of the candidate authors \citep{stamatatos_survey_2009,koppel_authorship_2011,barlas_cross_domain_2020,boenninghoff_o2d2_2021,neal_surveying_2018}.
    In other words: The set of all possible author classes is known a priori.
    Hence, closed-set problems can use supervised or unsupervised classification techniques \citep{abbasi_writeprints_2008}.
\end{definition}

\begin{definition}
    [Closed world]
    In the realm of plagiarism detection, closed world refers to the assumption 
    that a reference collection $D$ of documents, 
    that are supposed to be compared to the possibly plagiarized text, is given \citep{stein_intrinsic_2011}.
    In the realm of \ac{av} and \ac{aa} texts in the test set are assumed to be written by one of the authors in the training set \citep{boenninghoff_o2d2_2021,neal_surveying_2018}.
\end{definition}


\begin{definition}
    [Supervised techniques]
    Supervised techniques for stylometric analysis require (author-)class labels for categorization.
    Examples include \acp{svm}, \acp{nn}, decision trees, and linear discriminant analysis.
    \acp{svm} are very common in authorship analysis due to their robustness \citep{abbasi_writeprints_2008}.
\end{definition}

\begin{definition}
    [Unsupervised techniques]
    Unsupervised techniques make categorizations with no prior knowledge of author classes.
    Examples include \ac{pca} and cluster analysis.
    \ac{pca} has been used in previous authorship studies due to its ability to 
    capture essential variance across large number of features in a reduced dimensionality \citep{abbasi_writeprints_2008}.
\end{definition}

\begin{definition}
    [Covariate shift]
    The distribution of neural stylometric features changes between training and test set due to, for instance, topic variability \citep{boenninghoff_o2d2_2021}.
\end{definition}

\begin{definition}
    [train-test-validation split]
    \citet{bischoff_importance_2020,altakrori_topic_2021,boenninghoff_o2d2_2021} train their model on a selection of the dataset (i.e. training set), 
    optimize the model's hyperparameters on a second disjoint selection of the dataset (i.e. validation set),
    and evaluate the model on a third disjoint selection of the dataset (i.e. test set).
    \citet{bischoff_importance_2020} ensure that there is no data leakage between the training, validation and test sets 
    (i.e. prevent parts of one fanfiction being in more than one of the data splits).
    \citet{altakrori_topic_2021} ensure the classifier to be trained has no access to any information about the setup 
    (topic confusion: group configuration or topic labels).
\end{definition}

\begin{definition}
    [Cross-domain]
    Texts of known authorship (training set) differ from texts of disputed authorship (test set) 
    in topic (i.e. cross-topic) or genre (i.e. cross-genre) 
    \citep{barlas_cross_domain_2020}.
\end{definition}

\begin{definition}
    [Cross-topic]
    New, unseen topics are used in the testing phase \citep{altakrori_topic_2021}.
\end{definition}
\subsection{Tasks}

In the following we will outline the most prominent tasks of \ai{} or authorship analyse.
We will start with author profiling, resuming with author obfuscation, \ac{aa}, plagiarism detection and finally conclude with \ac{av}.

% author profiling
Building upon the foundation of stylometry which states that subconscious authorial idiosyncrasies are encoded in author's writing, there is plenty of information to extract from a given text.
Formally, inferring an extensive set of personal information from a collection of texts is denoted author profiling.
Extracted information may include sociolinguistic attributes like age, gender, occupation, education, socio-economic status, cultural background, language familiarity and mental health issues~\citep{emmery_adversarial_2021,stamatatos_survey_2009,elmanarelbouanani_authorship_2014}.
% The task is also referred to as author characterization \citep{stamatatos_survey_2009,elmanarelbouanani_authorship_2014}.
% Author obfuscation
Since the extracted information can contain sensitive personal details, there are efforts of altering texts to render an author's style imperceptible~\citep{bischoff_importance_2020}.
This process is called author obfuscation.
It is an adversarial task of preventing successful verification of the author by altering the text's style so that it no longer resembles the original author's style \citep{bevendorff_divergence_based_2020,gohsen_task_oriented_2024}.

% authorship attribution
What the \ai{} pioneers of the \nth{19} century did when they tried to answer who among Francis Bacon, Christopher Marlowe, and William Shakespeare was the true author of the Shakespearean plays is called \ac{aa}.

\begin{definition}
    [\acl{aa}]   
    \ac{aa} is the task of determining the author of a text given a set of candidate authors with undisputed writing samples.
    % \citep{stein_intrinsic_2011,koppel_authorship_2004,stamatatos_survey_2009,tyo_state_2022,bischoff_importance_2020,barlas_cross_domain_2020,altakrori_topic_2021,bevendorff_divergence_based_2020,elmanarelbouanani_authorship_2014,abbasi_writeprints_2008,llm_detection_av_2025,neal_surveying_2018}
    \textcolor{red}{TODO: too many references}
\end{definition}

\ac{aa} is a multiclass, single-label text classification task~\citep{stamatatos_survey_2009,koppel_authorship_2004,elmanarelbouanani_authorship_2014}.
Formally, \ac{aa} is defined as a tuple $(A,K,U)$, where $A$ is the set of authors, $K=\underset{a\in A}{\cup}K_a$ is the set of known texts and $U$ is the set of unknown texts.
For closed-set \ac{aa}, each text $d \in U$ is attributed to exactly one author $a \in A$.
In cross-topic or cross-genre scenarios, the topic or genre of documents in $d \in U$ is different from the topics or genres found in $K$~\citep{barlas_cross_domain_2020}. 

% The decision is made based on stylistic traits rather than the content of the document \citep{neal_surveying_2018}.

% The task is also referred to as author(ship) identification \citep{stamatatos_survey_2009,elmanarelbouanani_authorship_2014}.

% \citet{elmanarelbouanani_authorship_2014} describe the workflow of \ac{aa} as follows:
% \begin{enumerate}
%     \item Data cleaning
%     \item Feature extraction
%     \item Normalization
%     \item Converting each text into a feature vector, where author is the class label
%     \item Split the dataset into training and test set
% \end{enumerate}
% Common classifiers include \ac{svm}, decision trees, and \acp{nn} \citep{elmanarelbouanani_authorship_2014}.

% Plagiarism detection
More prominent than ever, with probabilistic \acp{lm} generating texts based on existing texts from their training data, we have to deal with plagiarism.
Plagiarism is the appropriation of another author's information, language, ideas, results or writing without properly acknowledging the original source~\citep{stein_intrinsic_2011,gohsen_task_oriented_2024}.
% \citet{elmanarelbouanani_authorship_2014} define plagiarism as the complete or partial replication 
% of a piece of work with or without permission of the original author.
Plagiarism detection is the task of identifying plagiarized text~\citep{stein_intrinsic_2011}.
% It includes finding similarities between two texts \citep{stamatatos_survey_2009}.
Different to \ac{aa}, it is not important identifying the author's identity but rather whether multiple pieces of work were produced by a single author~\citep{elmanarelbouanani_authorship_2014}.
Hence, plagiarism detection is an application of \ac{av}~\citep{rivera_soto_learning_2021}.

% authorship verification
\begin{definition}
    [\acl{av}]   
    Given a set of writing samples $K_a$ of author $A$ and a text $t$, the task is to determine whether $t$ was written by $A$.
    \textcolor{red}{TODO: too many references}
    % ~\citep{stein_intrinsic_2011,stamatatos_survey_2009,koppel_authorship_2011,tyo_state_2022,kocher_unine_2015,koppel_authorship_2004}.
\end{definition}

% Gespräch Martin Potthast 19.05.2025: problem formulation 2 is less common and in the context of very sparse (metadata) information:
% This task can also be formulated as whether two texts $t_1$ and $t_2$ are written by the same author 
% \citep{bevendorff_generalizing_2019,bevendorff_divergence_based_2020,embarcadero_ruiz_graph_based_2022,rivera_soto_learning_2021,ordonez_will_2020,futrzynski_pairwise_2021,weerasinghe_feature_vector_difference_2021,llm_detection_av_2025}.

To solve the \ac{av} problem, the disputed document $t$ is compared to documents from the candidate $K_a$.
Different to other one-class classification problems, this approach disregards negative samples, i.e. all texts not written by the known author.
In fact, it is neither possible to assemble an exhaustive, nor representative set of samples of the non-target class.
Hence, \ac{av} is a more general and difficult classification problem than \ac{aa}~\citep{llm_detection_av_2025,neal_surveying_2018,koppel_authorship_2004}.
It is evident that \ac{av} is the fundamental problem of \ac{aa}, where there is only one candidate author~\citep{barlas_cross_domain_2020,tyo_state_2022}.

The \ac{av} problem can be framed to fit multiple problem categorizations.
When considering its output \textit{same-author} or \textit{different-author}, we can view \ac{av} a binary classification problem, where \textit{different-author} renders \ac{av} as an open-set problem.
% ~\citep{neal_surveying_2018}.
% However, \citet{neal_surveying_2018} consider framing \ac{av} as a one-class classification problem as a common approach (cf. \citep{llm_detection_av_2025}).
% \citet{elmanarelbouanani_authorship_2014} consider \ac{av} a similarity detection task.
The \impAppr{}, among other approaches, generates more candidates during inference in order to obtain their final prediction.
Hence, such approaches artificially create an \ac{aa} problem to solve the original problem~\citep{neal_surveying_2018}.


\subsection{Approaches}

\subsubsection{Compression-based???}
\textcolor{red}{TODO}


\subsubsection{Unmasking Method}
\textcolor{red}{TODO}


\begin{figure}[htbp]
    \centering
    \includesvg[width=\textwidth]{images/unmasking/unmasking.svg}
    \caption{Unmasking.}
    \label{fig:unmasking}
\end{figure}
Our method is an extension of the original \impAppr{} by \citet{koppel_determining_2014}.
By varying the seed and temperature, we can generate as many texts as we want.
  
\subsubsection{\imp{} Method}
\label{sec:impostor_method_theory}

The \impAppr{} leverages random projections to lower dimensional spaces (i.e. random set of features set to zero is a projection).
\begin{figure}[htbp]
    \centering
    \includesvg[width=\textwidth]{images/imposter/imposter.svg}
    \caption{\imp{}.}
    \label{fig:impostor}
\end{figure}

\section{Impact of \acs{llm} on Authorship Verification}

With the advances in \ac{gai} come risks and opportunities.
We can frame \acp{llm} as authors weakening our trust in the authencity of digital texts, but also see them as valuable tools to enhance our regulatory methods.

\subsection{\acsp{llm} as Authors}
There is no general definition of when a text is \ac{llm} generated rather than co-created by humans with \ac{llm} assistance.
% Obviously, fully generated texts should be marked as \ac{llm} generated.
Minor human edits of \ac{llm} generated texts do not change the fact that the core content was \ac{llm} generated.
If \acp{llm} are used for grammar checking, polishing, and editing suggestion the primary substantial contribution was human.
One could denote these texts "\ac{ai}-revised Human-Written Text"~\citep{wang_stumbling_2024}.

With advances of generative models with regard to mimicking human writing, we have to face the fact that \acp{llm} will play a crucial role in any authorship analysis related tasks from now on.
\citet{llm_detection_av_2025}\ claim that \ac{llm} detection is not an \ac{aa} task, i.e. a closed-set binary classification where both classes are sufficiently discriminative, but an \ac{av} task, i.e. an open-set one-class classification problem. 

% differences: also more in ~\citep{wang_stumbling_2024}
Despite the advances, there are still some statistical differences on \ac{llm} generated and human authored texts.
\ac{llm} generated texts lack lexical diversity, overuses certain adjectives (e.g.\ "innovative") and produces longer, more complex sentences.
Moreover, \acp{llm} possess stylistic fingerprints and memorise patterns from the training data.
% lengths
Furthermore, word length averages and distributions across genre differ for \acp{llm} and humans.
% future of LLMs
As \acp{llm} continue to produce increasingly human-like text, \ac{llm} detection is expected to more closely resemble human authorship classification tasks~\citep{llm_detection_av_2025}.


\subsection{\acsp{llm} as Discriminators}
\label{sec:llm_discriminator}

While \acp{llm} can generate coherent text, their effectiveness as direct discriminators in \ac{aa} tasks varies. 
Words with similar meaning, such as "color" and "colour", are mapped to similar vector representations~\citep{altakrori_topic_2021}.
This reduces sensitivity to subtle differences in language, which are often critical for identifying an author. 
Consequently, relying on \acp{llm} as the primary discriminator in \ac{aa} or \ac{av} tasks may be suboptimal.

Rather than serving as standalone discriminators, \acp{llm} are more appropriately employed as supporting tools. 
For instance, they can generate cross-domain training data to improve model robustness. 
In regard to \ac{llm} detection, prior work assessed the extent to which text changes under \ac{llm}-based paraphrasing, since machine generated text tends to undergo minimal alteration, whereas human-authored text exhibits greater variation~\citep{mao_raidar_2024}.
Conversely, in contexts where privacy is a concern, \acp{llm} can facilitate author obfuscation through controlled paraphrasing. 

% \section{\ac{av} as One-Class categorization}
\label{sec:av_one_class}

% \citet{koppel_authorship_2004} claim research had shown that linear separators work well for text categorization.
% Linear models include Naive Bayes (linear separator for two classes), 
% Window and Exponential Gradient and linear \acp{svm} \citep{koppel_authorship_2004}. 

% \subsection{\ac{aa} framework}
% \citet{koppel_authorship_2004} state that the following framework solve a number of real world \ac{aa} problems:
% \begin{enumerate}
%     \item Construction of appropriate feature vectors
%     \item Construction of a distinguishing model via a learning algorithm
%     \item Assessment of effectiveness of methods using k-fold cross-validation or bootstrapping
% \end{enumerate}

\subsection{Chunks}
\citet{koppel_authorship_2004}\ propose chunking texts such that each chunk is of approximately equal length, 
and at least 500 words without breaking paragraphs. 

\subsection{Definitions}
For author $A$ and book $X$, \citet{koppel_authorship_2004}\ define the following:
If $A$ is not the author of $X$, $A_X$ is the set of all works by author $A$.
If $A$ is the author of $X$, $A_X$ is the set of all works by author $A$ except $X$.
A pair of $A_X$ and $X$ is called \emph{same-author} if X was authored by $A$.
A pair of $A_X$ and $X$ is called \emph{different-author} if $X$ was not authored by $A$.

\subsection{Initial feature set}
The initial feature set consists of the 250 words with the highest average (over $X$ and $X_A$) frequency \citep{koppel_authorship_2004}.

\subsection{Features for meta-classifier}
\citet{koppel_authorship_2004}\ propose the following features for the meta-classifier 
(where $i$ is the number of elimination steps):
\begin{itemize}
    \item accuracy after $i$ elimination steps
    \item accuracy difference between round $i$ and round $i+1$
    \item accuracy difference between round $i$ and round $i+2$
    \item $i^{th}$ highest accuracy drop in one iteration
    \item $i^{th}$ highest accuracy drop in two iterations
\end{itemize}

The vectors are grouped by \emph{same-author} and \emph{different-author} pairs and thus, 
used to train a meta-learning scheme.

\subsection{Negative examples for Elimination method}
\citet{koppel_authorship_2004}\ state that negative examples are neither exhaustive nor representative.
They propose using words of several authors $A_1, ..., A_n$ roughly filling the same profile as candidate $A$ 
in terms of geography, chronology, culture and genre.
$A_1, ..., A_n$ are said to collectively represent class not-$A$.

\subsection{Elimination method}

The elimination method is only used to overrule positive predictions.
Hence, it can eliminate \acp{fp}.
One can frame it as a filter which is applied after or before \unmasking{}.

% training
\citet{koppel_authorship_2004}\ learn a model for $A$ and against not-$A$, 
and multiple models for $A_i$ and against not-$A_i$.
% inference
Then, $X$ is tested against all of these models.
$A(X)$ is the percentage of examples of $X$ classed as $A$ rather than not-$A$ 
(i.e., $A_i(X)$ analogously).
If $A(X)$ is not larger than all $A_i(X)$, $A$ is not the author of $X$.
If $A(X)$ is larger than all $A_i(X)$, conclude nothing.




