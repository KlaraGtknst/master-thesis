\section{Contemporary Applications} % Contemporary Relevance/ Perspective
\label{sec:digital_forensics}

While early work in authorship analysis focused on resolving historical literary disputes, its present-day relevance lies primarily in digital forensics. % forensic linguistics: sapkota_cross_topic_2014
\ac{av} provides critical tools for addressing cybercrime, where anonymity enables identity deception, harassment, and financial fraud~\citep{abbasi_writeprints_2008,uchendu_authorship_2020,bhattacharjee_fighting_2024}. 
Concrete applications could include the detection of phishing, and fake news~\citep{mao_raidar_2024,li_learning_2025,baradia_mirror_2025}.%, and deepfakes~\citep{uchendu_authorship_2020}. 

Beyond forensic contexts, intrinsic plagiarism detection constitutes another application area.
Potential instances of plagiarism are revealed by analysing texts for undeclared shifts in writing style~\citep{stein_intrinsic_2011}. 
In this context, \ac{av} helps uncover unacknowledged reuse of material across domains ranging from literature to academia~\citep{neal_surveying_2018}. 
Collectively, these examples highlight the contemporary relevance of \ac{av} safeguarding digital integrity and supporting legal investigation.
