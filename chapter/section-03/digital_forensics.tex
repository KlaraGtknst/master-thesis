\section{Contemporary Applications} % Contemporary Relevance/ Perspective

While early work in authorship analysis focused on resolving historical literary disputes, its present-day relevance lies primarily in digital forensics. % forensic linguistics: sapkota_cross_topic_2014
\ac{av} provides critical tools for addressing cybercrime, where anonymity enables identity deception, harassment, and financial fraud~\citep{abbasi_writeprints_2008,uchendu_authorship_2020,bhattacharjee_fighting_2024}. 
Concrete applications include the detection of fraudulent e-mails, impersonation on social media, and the identification of fake product reviews that undermine trust in e-commerce platforms. 

Beyond forensic contexts, plagiarism detection constitutes another significant application area. 
Here, authorship analysis assists in identifying unacknowledged reuse of material across domains ranging from literature to academia~\citep{neal_surveying_2018}. 
Together, these examples highlight the contemporary relevance of authorship analysis safeguarding digital integrity and supporting legal investigation.
\textcolor{orange}{Studentischer Kontext?}