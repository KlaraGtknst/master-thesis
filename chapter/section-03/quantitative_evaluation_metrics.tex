\section{Quantitative evaluation measures}

We will present state-of-the-art quantitative evaluation measures for \ac{aa}, \ac{av} and paraphrase generation.
Ideally, quantitative measures are comparable and reproducible since they do not arise from human biased judgment.

\subsection{\ac{av} quality measures}
\label{subsec:av_quality_measures}

Common metrics for evaluating the \ac{av} performance are:
\begin{itemize}
    \item $Accuracy = \frac{TP + TN}{TP + TN + FP + FN}$ \citep{elmanarelbouanani_authorship_2014,neal_surveying_2018} 
    measures the percentage of classified correctly over all test cases \citep{neal_surveying_2018}.

    \item $Precision = \frac{TP}{TP + FP}$ \citep{elmanarelbouanani_authorship_2014,neal_surveying_2018,chen_web_2008} 
    measures how often a system gets positive classification correctly \citep{neal_surveying_2018}.

    \item $Recall = \frac{TP}{TP + FN}$ \citep{elmanarelbouanani_authorship_2014,neal_surveying_2018,chen_web_2008} 
    measures how often a system correctly classifies positive samples when it encounters them \citep{neal_surveying_2018}.
    Recall is also called sensitivity or \acl{tp} rate \citep{palivela_optimization_2021}

    \item $F-measure = \frac{2 \cdot precision \cdot recall}{precision + recall}$~\citep{chen_web_2008,abbasi_writeprints_2008}

    % PAN
    \item \ac{roc-auc} \citep{bevendorff_overview_2024,weerasinghe_feature_vector_difference_2021,kocher_unine_2015}
    where \ac{roc} plots \ac{fpr} $= \frac{FP}{FP+TN}$ on the x-axis against the \ac{tpr} $=\frac{TP}{TP+FN}$ on the y-axis 
    for varying thresholds \citep{kocher_unine_2015,neal_surveying_2018}.
    The maximum \ac{roc} value of 1.0 indicated a perfect performance \citep{kocher_unine_2015}.
    Greater \ac{auc} indicates a better performance \citep{neal_surveying_2018}.
    % The \ac{auc} of the \ac{roc} is biased since the \ac{roc} gives more emphasis 
    % on the first position and therefore increases the total \ac{auc}.
    % A misclassification with a lower probability is less penalized with \ac{roc-auc} \citep{kocher_unine_2015}.
    \citet{kocher_unine_2015} claim that both \ac{roc} and \ac{auc} are difficult to interpret.
    % LLM detection as AV task
    \citet{llm_detection_av_2025} argue that the reduction from the \ac{fpr}-\ac{tpr} curve of \ac{roc} to a single \ac{roc-auc} number 
    comes with information loss due to the absence of a fixed threshold and trade-off.
    Moreover, \ac{roc}'s \ac{fpr} and \ac{tpr} are independent of class prevalence, which is desirable.
    However, in highly imbalanced class scenarios \ac{roc} can be misleading (overly optimistic or pessimistic).
    
    \item BRIER: Complement of the Brier score \citep{bevendorff_overview_2024,weerasinghe_feature_vector_difference_2021}, 
    in \citet{bevendorff_overview_2024}'s case equivalent to the mean squared loss.
    The Brier score is used to evaluate the ability of \ac{av} methods to abstain from hard samples \citep{tyo_state_2022}.
    
    \item $C@1 = \frac{nc}{np}(1+\frac{nu}{np})$ where $np$ is the number of problems, $nc$ the number of correct answers, 
    and $nu$ the number of unanswered problems \citep{kocher_unine_2015}. 
    Modified version of the accuracy \citep{bevendorff_overview_2024}/ F1-score \citep{weerasinghe_feature_vector_difference_2021} score, 
    where the non-answers (abstained) \citep{llm_detection_av_2025} are assigned the average accuracy of the remaining cases \citep{bevendorff_overview_2024}. 
    It rewards systems that leave difficult problems unanswered \citep{weerasinghe_feature_vector_difference_2021}.
    
    \item $F_1 = 2 \cdot \frac{precision \cdot recall}{precision + recall}$~\citep{neal_surveying_2018}: Harmonic mean of precision and recall \citep{bevendorff_overview_2024,weerasinghe_feature_vector_difference_2021}.
    A higher value indicates a better performance \citep{neal_surveying_2018}.
    
    \item $F_{0.5u}$: Modified version of the $F_{0.5}$ score, where the non-answers are considered \acp{fn} \citep{bevendorff_overview_2024}. 
    A measure that puts more emphasis on deciding same-author cases correctly \citep{weerasinghe_feature_vector_difference_2021}.
    Used to evaluate the ability of \ac{av} methods to abstain from hard samples \citep{tyo_state_2022}.
\end{itemize}
$TP$ is the number of \aclp{tp}, $FP$ is the number of \aclp{fp} 
and $FN$ is the number of \aclp{fn}~\citep{chen_web_2008}.



% PAN AA & AV metrics: Abstaining from hard samples
% The \todo{F0.5u, C@1, and Brier Score metrics} are used to evaluate the ability of \ac{av} methods 
% to abstain from hard samples \citep{tyo_state_2022}.
% For each sample, a score $\in [0, 1]$ is assigned to the sample.
% A score of exactly 0.5 means the model abstains from the sample \citep{tyo_state_2022,bevendorff_overview_2024,kocher_unine_2015}.

% The AUC metric is used to evaluate the ability of methods to rank predictions.
% No threshold is required.
% \ac{pan} ignores any abstained samples when calculating the AUC metric \citep{tyo_state_2022}.

% != PAN
% \citet{tyo_state_2022} chose to adopt the macro-averaged accuracy metric, so-called macro-accuracy, for \ac{aa}, 
% and AUC \ac{av} tasks.

\subsection{Paraphrase quality measures}
\label{subsec:paraphrase_quality_measures}

% Syntactic (\ac{bleu}, \ac{rouge}-1, \ac{rouge}-L), semantic (BERTScore, cosine similarity of SBERT vectors, WMS), human evaluation (TODO)

% There is manual (by humans) evaluation and automatic evaluation for paraphrase generation \citep{fu_learning_2024,zhou_paraphrase_2021}.
% According to \citet{zhou_paraphrase_2021}, automatic evaluation metrics mainly focus on the n-gram overlaps instead of meaning, 
% and hence, human evaluation is more accurate and has a higher quality.
% In the following, we focus on automatic evaluation.

There is syntactic and semantic evaluation of paraphrases \citep{gohsen_captions_2023}.
% Metrics for syntactic evaluation include \ac{bleu}, \ac{rouge}-1, \ac{rouge}-L, 
% while metrics for semantic similarity include BERTScore, 
% cosine similarity of dense vector representations derived from a BERT-based sentence transformer, 
% and \ac{wmd} \citep{gohsen_captions_2023}.
% The \ac{wmd} computes the minimum amount of distance that embedded words of a text need to travel 
% to reach the embedded words of another text \citep{gohsen_captions_2023}.
% \citet{gohsen_captions_2023} normalized all metrics and averaged the semantic and syntactic scores separately.
Syntactic evaluation metrics mainly focus on the n-gram overlaps. % no reference
Common syntactic evaluation metrics include \ac{bleu}, \ac{meteor} \ac{rouge}-1, \ac{rouge}-L.
\textcolor{red}{\citet{kurt_pehlivanoglu_comparative_2024} claims \ac{meteor} is a semantic metric.}

% Generally, \citet{kurt_pehlivanoglu_comparative_2024} order the metrics by their contribution area:
% \begin{itemize}
%     \item semantic: BERTScore, STSB, \ac{meteor}
%     \item Fluency: CoLA
%     \item Diversity: \ac{rouge}1/2/L, \ac{bleu}, GLEU
% \end{itemize}

% \bluert{} is machine evaluation metric for paraphrase generation.
% \citet{fu_learning_2024} use \bluert{} to filter out incorrect paraphrases (i.e. using a threshold $\theta$).

\ac{bleu} (2002) was developed for machine translation \citep{palivela_optimization_2021,zhou_paraphrase_2021,papineni_bleu_2001}.
Its values range from 0 to 1 \citep{papineni_bleu_2001}.
\ac{bleu} is a precision measure \citep{kurt_pehlivanoglu_comparative_2024,papineni_bleu_2001}.
It counts the matching n-grams (unigrams) in the generated/candidate text that appear in any of the gold/ reference texts \citep{palivela_optimization_2021,papineni_bleu_2001}, 
and then divides them by the total number of n-grams (unigrams) in the candidate text \citep{papineni_bleu_2001}.
Since candidates consisting only of high-probability n-grams (e.g. "the") would receive a high score without deserving it, 
\citet{papineni_bleu_2001} introduced a clipping mechanism to limit the count of n-grams in the candidate text 
to the maximum count of that n-gram in any of the reference texts.
The clipped n-grams occurrences are added up and divided by the total number 
of unclipped n-grams in the candidate text \citep{papineni_bleu_2001}.
\citet{papineni_bleu_2001} state that unigrams are used to test adequacy, while longer n-grams are used to test fluency.
\ac{bleu}'s basic unit of evaluation is a sentence. 
In order to compute the \ac{bleu} score from \autoref{eq:bleu} for more than one sentence, 
one (1) computes the n-grams matches sentence by sentence, 
then (2) adds the clipped n-grams matches across all sentences, 
and finally (3) divides the total clipped n-grams matches by 
the total number of unclipped n-grams in all candidate sentences \citep{papineni_bleu_2001}.
\begin{equation}
    p_n = \frac{\sum_{\mathcal{C} \in \left\{ Candidates \right\}}\sum_{n-gram \in\mathcal{C}}Count_{clip}(n-gram)}{\sum_{\mathcal{C'} \in \left\{ Candidates \right\}}\sum_{n-gram' \in\mathcal{C'}}Count(n-gram')}
\label{eq:bleu}
\end{equation}
\ac{bleu} combines the scores for different n-grams (separately computed) using the average logarithm with uniform weights, 
which is equivalent to using the geometric mean of the scores \citep{papineni_bleu_2001,banerjee_METEOR_2005}.
\citet{gohsen_captions_2023} use up to 4-grams.
\ac{bleu} automatically penalizes n-grams appearing in the candidate text but not in the reference text, 
as well as n-grams appearing more often in the candidate than in the reference text \citep{papineni_bleu_2001}.
According to \citet{papineni_bleu_2001}, they need to add a brevity penalty to the \ac{bleu} score 
to enforce proper length of the candidate text. 
Hence, the brevity penalty $BP$ is defined as follows in \autoref{eq:bleu_brevity_penalty}:
\begin{equation}
    BP = \begin{cases}
        1 & \text{if } c > r \\
        e^{1 - \frac{r}{c}} & \text{if } c \leq r
    \end{cases}
\label{eq:bleu_brevity_penalty}
\end{equation}
Combining all these, the final \ac{bleu} score is computed as follows in \autoref{eq:bleu_final}:
\begin{equation}
    \text{BLEU} = BP \cdot \exp\left(\sum_{n=1}^{N} w_n \cdot \log p_n\right)
\label{eq:bleu_final}
\end{equation}
For multiple sentences, they (1) add the best match (among the reference texts) length for each candidate sentence, 
and (2) divide this sum $r$ by the total length of all candidate sentences $c$. 
They cannot use recall for length-related problems here, 
because \ac{bleu} uses multiple reference texts, which may have different lengths \citep{papineni_bleu_2001,banerjee_METEOR_2005}.
If the generated candidate is significantly shorter than the reference text, the brevity penalty $BP$ is applied.
A \ac{bleu} score approaching 1 signifies the candidate matches one reference almost exactly \citep{papineni_bleu_2001}, 
and thus, limited syntactic diversity (i.e. inadequate paraphrase) \citep{kurt_pehlivanoglu_comparative_2024}.
Note that more reference texts lead to higher \ac{bleu} scores \citep{papineni_bleu_2001}.
Unigrams are token-wise and bi-grams are word-pairs \citet{palivela_optimization_2021}.
According to \citet{zhou_paraphrase_2021}'s survey, \ac{bleu} is the most frequently used metric for paraphrase generation.
\ac{bleu} is unable to measure semantic equivalents \citep{kurt_pehlivanoglu_comparative_2024,zhou_paraphrase_2021} 
when applied to low-resource languages \citep{zhou_paraphrase_2021}.
Moreover, \ac{bleu} fails to capture good paraphrases that are not similar to the reference text \citep{zhou_paraphrase_2021}.
\citet{kurt_pehlivanoglu_comparative_2024} found that \ac{bleu} tends to overestimate the quality of paraphrases.
\citet{zhou_paraphrase_2021} suggest combining \ac{bleu} with human evaluation to overcome its limitations.

% GLEU (Google-\ac{bleu}) (ranges from 0 to 1 \citep{kurt_pehlivanoglu_comparative_2024}) is a variant of \ac{bleu} that was developed to be closer to human judgement, and to 
% overcome \ac{bleu}'s drawback of per sentence reward objective \citep{palivela_optimization_2021}.
% GLEU computes n-gram precisions (overlaps \citep{kurt_pehlivanoglu_comparative_2024}) overgold/reference paraphrases 
% and weighs n-grams by their change from the source text \citep{palivela_optimization_2021}.
% GLEU assesses the fluency, order of n-grams, structural and semantic accuracy 
% and penalizes shorter average m-gram lengths in the generated text compared to the reference \citep{kurt_pehlivanoglu_comparative_2024}.
% Lower GLEU scores indicate greater diversity \citep{kurt_pehlivanoglu_comparative_2024}.

\ac{meteor} (2014) aims to address \ac{bleu}'s shortcomings.
Its values range from 0 to 1 \citep{kurt_pehlivanoglu_comparative_2024}.
First a mapping, so-called alignment, between the unigrams in the candidate text and the reference texts is created \citep{banerjee_METEOR_2005}.
Each unigram has zero or one match.
This alignment is created incrementally in repeating two steps:
(1) List all possible unigram mappings derived from different modules (i.e. exact matches, Porter stemmed matches, synonym matches), 
and (2) select the largest subset of unigram mappings that constitute a valid alignment 
(matches obtained from different modules are treated the same).
(3) Choose the subset with the largest cardinality and if there are multiple, 
choose the one with the fewest unigram mapping crosses \citep{banerjee_METEOR_2005}.
\ac{meteor} computes a weighted F-score 
(unigram-precision, unigram-recall \citep{kurt_pehlivanoglu_comparative_2024,banerjee_METEOR_2005} 
and a measure of fragmentation \citep{banerjee_METEOR_2005,kurt_pehlivanoglu_comparative_2024})
with a penality function whenever an incorrect word is encountered \citep{palivela_optimization_2021} as displayed in \autoref{eq:meteor}.
\begin{equation}
    METEOR = F_{mean} = \frac{10 \cdot P \cdot R}{R + 9P} \cdot (1 - Penalty)
\label{eq:meteor}
\end{equation}
The penalty is designed to reduce the $F_{mean}$ score to $50\%$ if there are no bigram or longer matches \citep{banerjee_METEOR_2005}.
It has better correlation with human judgment at the sentence/segment level than \ac{bleu} \citep{zhou_paraphrase_2021}, 
because it not only consists of simple n-gram matching but also including synonymy and stemming \citep{kurt_pehlivanoglu_comparative_2024}.

\ac{rouge} (2004) is a recall-based metric developed for text summarization 
\citep{zhou_paraphrase_2021,palivela_optimization_2021,kurt_pehlivanoglu_comparative_2024,lin_rouge_2004}.
Its values range from 0 to 1 \citep{kurt_pehlivanoglu_comparative_2024}.
\ac{rouge} can focus on the word variations and diversity.
It has multiple versions, the most popular ones include 
\ac{rouge}-N (computing the n-gram recall) \citep{zhou_paraphrase_2021,palivela_optimization_2021,kurt_pehlivanoglu_comparative_2024}, 
\ac{rouge}-L (computing the longest common subsequence) \citep{zhou_paraphrase_2021,palivela_optimization_2021,kurt_pehlivanoglu_comparative_2024}, 
\ac{rouge}-W (Weighted longest common subsequence) \citep{palivela_optimization_2021}, 
\ac{rouge}-S (skip-bigram co-occurrence statistics) \citep{palivela_optimization_2021}.
\ac{rouge}-1 computes the recall by analysing the matching unigrams between the generated paraphrase and the reference paraphrase \citep{palivela_optimization_2021,kurt_pehlivanoglu_comparative_2024}.
% ROUGE-N
\ac{rouge}-N is an n-gram recall between the candidate text and the reference texts \citep{lin_rouge_2004} as displayed in \autoref{eq:rouge_n}.
\begin{equation}
    ROUGE-N = \frac{\sum_{\mathcal{S} \in \left\{ References \right\}}\sum_{n-gram \in\mathcal{S}}Count_{match}(n-gram)}{\sum_{\mathcal{S'} \in \left\{ References \right\}}\sum_{n-gram' \in\mathcal{S'}}Count(n-gram')}
\label{eq:rouge_n}
\end{equation}
$Count_{match}(n-gram)$ is the maximum number of n-grams co-occuring in the candidate text and the set of reference texts \citep{lin_rouge_2004}.
The nominator sums over all references and thus, gives more weight to matching n-grams that occur in multiple references (i.e. a consensus between references) \citep{lin_rouge_2004}.
Refer to \citet{lin_rouge_2004} for more details on the work with multiple references (I do not understand that, because I thought we already use multiple).
% ROUGE-L
For \ac{rouge}-L, the intuition is that the longer the longest common subsequence (LCS) between the candidate and reference texts, the more similar they are \citep{lin_rouge_2004}.
For a candidate $Y$ of length $n$ and a reference $X$ of length $m$, the \ac{rouge}-L score is defined as follows in \autoref{eq:rouge_l}:
\begin{equation}
    ROUGE-L = F_{lcs} = \frac{(1 + \beta^2)R_{lcs}P_{lcs}}{R_{lcs} + \beta^2 P_{lcs}}
\label{eq:rouge_l}
\end{equation}
where $R_{lcs} = \frac{LCS(X,Y)}{m}$ and $P_{lcs} = \frac{LCS(X,Y)}{n}$ \citep{lin_rouge_2004}.
\ac{rouge}-L requires in-sequence matches that reflect the sentence level word order as n-grams \citep{lin_rouge_2004}.
Moreover, no predefined $n$ is necessary, because \ac{rouge}-L includes the longest in-sequence common n-grams \citep{lin_rouge_2004}.
However, \ac{rouge}-L does not include shorter sequences or alternative LCSes in the final score \citep{lin_rouge_2004}.
% ROUGE-S
A skip-bigram is any pair of words in their sentence order, allowing for arbitrary gaps \citep{lin_rouge_2004}.
\ac{rouge}-S measures the overlap of skip-bigrams between the candidate text and the reference texts \citep{lin_rouge_2004}.
Hence, if the candidate text is the reverse of the reference text, the \ac{rouge}-S score is 0 even though it is not as bad as completely unrelated candidates \citep{lin_rouge_2004}.
\ac{rouge}-SU extends \ac{rouge}-S with unigrams to solve this issue \citep{lin_rouge_2004}.
% ROUGE generally
\citet{kurt_pehlivanoglu_comparative_2024} claim that \ac{rouge} may not be adequate to assess semantic similarity and fluency.
Lower \ac{rouge} scores indicate greater diversity \citep{kurt_pehlivanoglu_comparative_2024}.

% TER (2006) was developed for machine translation \citep{zhou_paraphrase_2021}.
% It computes the number of edits required to change the translation until it matches the reference translation.
% It ranges from 0 (i.e. no edits needed) to 1 (i.e. all words need to be changed).

% \citet{fu_learning_2024} describe the Gini Coefficient as a measure of inequality in a distribution, ranging from 0 (i.e. even distribution across categories) to 1.

% \citet{master_thesis_paraphrasing_2024} include a table (tab. 3.1, pg. 18) with numerous stylometric metric including readbility, vocabulary richness, and word/character counts.

% \citet{palivela_optimization_2021} state that accuracy, precision, recall and F1-score are suitable for \ac{pi}.
% For \ac{pg}, they suggest \ac{rouge}, \ac{bleu}, GLEU, WER (Word Error Rate), and \ac{meteor} as suitable metrics.
% WER is the number of substitutions (replacements of words), insertions (adding words) and deletions (removing words) 
% divided by the total number of words in the reference text \citep{palivela_optimization_2021}.

Common metrics for semantic similarity include BERTScore, 
cosine similarity of dense vector representations derived from a BERT-based sentence transformer, 
and \ac{wmd} \citep{gohsen_captions_2023}.

% \citet{kurt_pehlivanoglu_comparative_2024} additionally use T5-STSB.
% The metric is based on \ac{t5} model adapted to the Semantic Textual Similarity Benchmark (STSB).
% It evaluates semantic equivalence by assigning a similarity score from 0 (no similarity) to 5 (complete equivalence) \citep{kurt_pehlivanoglu_comparative_2024}.

BERTScore calculates the cosine similarity between the contextual embeddings of the reference and generated texts. 
Hence, is assesses semantic equivalence and correlates well with human judgment \citep{kurt_pehlivanoglu_comparative_2024}.
First, token vector representations are computed for both the reference and generated texts using a pre-trained BERT model \citep{hanna_fine_grained_2021}.
Let reference $z$ and candidate $\hat{z}$ be the vector representations of the reference and candidate texts, respectively.
Then, the BERTScore precision, recall and $F_1$ score is computed as follows in \autoref{eq:bert_p}, \autoref{eq:bert_r}, and \autoref{eq:bert_f1}, respectively:
\begin{equation}
    P_{BERT} = \frac{1}{|\hat{z}|} \sum_{\hat{z}_j \in \hat{z}} \max_{z_j \in z} z_i\top \hat{z}_j
\label{eq:bert_p}
\end{equation}
\begin{equation}
    R_{BERT} = \frac{1}{|z|} \sum_{z_j \in z} \max_{\hat{z}_j \in \hat{z}} z_i\top \hat{z}_j
\label{eq:bert_r}
\end{equation}
\begin{equation}
    F_1 = \frac{2 P_{BERT} R_{BERT}}{P_{BERT} + R_{BERT}} 
\label{eq:bert_f1}
\end{equation}
Since $F_1 \in \left[-1,1\right]$ it can be rescaled to $[0,1]$ by modifying the precision and recall calculation 
to $\hat{P}_{BERT} = \frac{P_{BERT} - a}{1 - a}$ ($R_{BERT}$ analogous), where $a$ is the empirical lower bound on the BERTScore \citep{hanna_fine_grained_2021}.
% The BERTScore has difficulties on datasets with lexically similar (i.e. lexical overlap of content words) incorrect candidates 
% opposed to lexically different more correct candidates \citep{hanna_fine_grained_2021}.

The \ac{wmd} computes the minimum amount of distance that embedded words of a text need to travel 
to reach the embedded words of another text \citep{gohsen_captions_2023}.

% \ac{t5}-CoLA metric (ranges from 0 to 5 \citep{kurt_pehlivanoglu_comparative_2024}) utilizes the Corpus of Linguistic Acceptability 
% (CoLA) to evaluate the grammatical correctness of sentences and thus, 
% contributes linguistic evaluation \citep{kurt_pehlivanoglu_comparative_2024}.

% \citet{krishna_paraphrasing_2023} compute the lexical diversity using unigram token overlap and call it F1 score.
% As a semantic similarity score, they use the ACL Antology 2022 published \href{https://aclanthology.org/2022.emnlp-demos.38.pdf}{P-SP}.
