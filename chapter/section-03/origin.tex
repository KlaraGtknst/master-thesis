\section{Historical Background and Early Approaches}

Research on authorship originated in the 19th century as an attempt to resolve disputes over the authorship of literary works. 
Augustus de Morgan (1851) was among the first to propose a quantitative approach, using word-length frequencies. 
Building on this idea, Thomas C. Mendenhall (1887) carried out the first systematic analysis of word-length distributions in the works of Bacon, Marlowe, and Shakespeare, aiming to shed light on the authorship of Shakespearean plays~\citep{neal_surveying_2018,stamatatos_survey_2009}.

% Popular examples of \ai{} include a collection of Hebrew-Aramaic letters supposedly by a rabbinic scholar in Baghdad in the late 19th century~\citep{koppel_authorship_2004}.
% The latter denied having authored the text collection, which was contested by \citet{koppel_authorship_2004}.
