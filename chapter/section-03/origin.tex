\section{Historical Background and Early Approaches}
\label{sec:origin}

Research on authorship originated in the 19th century as an attempt to resolve disputes over the authorship of literary works. 
Augustus de Morgan was among the first to propose a quantitative approach in 1851, using word-length frequencies. 
Building on this idea, in 1887 Thomas C. Mendenhall carried out the first systematic analysis of word-length distributions in the works of Bacon, Marlowe, and Shakespeare, aiming to shed light on the authorship of Shakespearean plays~\citep{neal_surveying_2018,stamatatos_survey_2009}.
Word-length distributions quantify the number of words on the vertical axis by their length in characters on the horizontal axis. 
\citet{wordlengths_mendenhall_1887} concluded that differences between the curves of two disputed texts may indicate different authorship, whereas similarity in the curves is less reliable as evidence of shared authorship.

% Popular examples of \ai{} include a collection of Hebrew-Aramaic letters supposedly by a rabbinic scholar in Baghdad in the late 19th century~\citep{koppel_authorship_2004}.
% The latter denied having authored the text collection, which was contested by \citet{koppel_authorship_2004}.
