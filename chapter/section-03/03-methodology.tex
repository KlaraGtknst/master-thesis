\chapter{Methodology}
\label{chap:methodology}

\subsection{Dataset}
\label{subsec:dataset}

\subsubsection{Original Data}
Due to this approach extending the original imposter approach by \citet{koppel_determining_2014}, 
we first use original data to establish the feasibility and reproducibility of the original approach. 
\citet{koppel_determining_2014} used \dataBlog{} and the \dataStudent{} dataset.
The  \dataStudent{} dataset is not publicly available, due to sensible information about the students.
J. W. Pennebaker was kind enough to provide us with the original data M. Koppel used in his paper.
The dataset contains student essays for five assignments from a class in 2006.
The assignments include stream of consciousness, talk about your childhood, describe your personality, 
Thematic Apperception Test, and Give four examples of four different theories.
For reproducing our work, due to privacy concerns, 
we refer to J. W. Pennebaker as the data holder  and authority of the \dataStudent{} dataset.
For establishing a baseline for the imposter approach, we use the \dataBlog{} and the \dataStudent{} dataset.

\subsubsection{Additional Data}
To extend our test scenario for the imposter approach, we opted to find an additional dataset that fulfilled the following criteria:
\begin{itemize}
    \item controlled confounders (e.g. genre, topic)
    \item undisputed authorship 
    
\end{itemize}

We found that both the \dataPan{} and \dataGutenberg{} datasets are suitable candidates.
The statistical properties of the datasets are shown in \autoref{tab:data_stats}.

\begin{table}[h]
\centering\small
\caption{Statistics of preprocessed datasets \dataPan{}, \dataBlog{} and \dataGutenberg{}.}
\label{tab:data_stats}
\resizebox{\textwidth}{!}{%
\begin{tabular}{@{}lrrrrrrrrr@{}}   % numbers should be right aligned, text left aligned
\toprule
dataset & num\_pairs & num\_authors & num\_same\_pairs & num\_different\_pairs & avg\_text\_len & min\_text\_len & max\_text\_len & std\_text\_len & median\_text\_len \\ 
\midrule
pan20     & 66912 & 52778 & 35620 & 31292 & 21435.53  & 20670 & 296887  & 2685.49   & 21284.00  \\ 
blog      & 64771 & 18961 & 33639 & 31132 & 1846.20   & 501   & 615409  & 3461.74   & 1244.00   \\ 
gutenberg & 36    & 4     & 15    & 21    & 293840.11 & 14092 & 1176438 & 339991.16 & 139370.00 \\ 
\bottomrule
\end{tabular}%
}
\end{table}
\textcolor{red}{above len in characters, but I have also stats for words‚}


\subsubsection{Dataset Preprocessing}
\label{subsubsec:dataset_preprocessing}

To further control confounders influencing authorial style, we preprocess the dataset 
(once before creating the arrow dataset file and once before using the detector).
The requirements for the preprocessing are:
\begin{itemize}
    \item The texts are stripped of all format/ layout information to obtain plain text \textcolor{red}{TODO: before saving them as arrow files}.
    \item The texts should be of similar length (detector crops texts to the length of the shorter text).
\end{itemize}
In order to produce a controlled testing environment four our imposter approach, 
we opted to refine few datasets rather than scaling up to larger datasets.
Removing layout information includes removing HTML artefacts, play artefacts, newlines, 
converting utf-8 to ASCII, lowercasing and stripping leading and trailing whitespace.

\subsubsection{Selection of Text Pairs}
\label{subsubsec:dataset_text_pair_selection}

We had to select pairs of texts for the \dataBlog{} and the \dataGutenberg{} dataset.
We decided to keep the existing pairs in the \dataPan{} arrow dataset for better comparability.
All datasets consist of same- and different-author pairs. 
As mentioned before, we aimed to control confounders when selecting pairs.

For the \dataBlog{} dataset, 
two texts of a pair are selected such that they share the same topic, year, gender and age, where the last to reference the text's author.
Test (20\%) and train split (80\%) have different topics.

For the \dataGutenberg{} dataset,
we selected pairs of texts that share the same genre and century.
Authors can either be in the train (80\%) or test (20\%) set.

Irrespective of the information used to select pairs, the final dataset contains only the columns \texttt{authors}, \texttt{pair}, and \texttt{same}.
The \texttt{pair} column contains the texts of the pair as a list of strings,
the \texttt{authors} column contains the authors of the texts as a list of strings,
and the \texttt{same} column indicates whether the texts are from the same author (\texttt{True}) or from different authors (\texttt{False}).


\subsection{Imposter Generation}
\label{subsec:impostor_generation}

% good impostors: hard negatives
Equivalent to the original impostor approach by \citet{koppel_determining_2014}, we denote ideal impostor texts as hard negatives.
In other words, texts that are not authored by the candidate author, but are difficult to distinguish from the candidate author's texts.
Note that the quality of the impostors directly contributes to the performance of the model, 
since easy impostors lead to \acp{fp} and too difficult ones to \acp{fn}.

\subsubsection{Traditional Imposter Generation}
\label{subsubsec:traditional_impostor_generation}

To assess the quality of our Imposter approach implementation, we first reimplement the original impostor approach by \citet{koppel_determining_2014}.
The parameters are set to the original default values.


\subsubsection{Novel Imposter Generation}
\label{subsubsec:novel_impostor_generation}
% obstacles for impostor generation in the past
The traditional generation techniques outlined by \citet{koppel_determining_2014} faced difficulties in controlling essential factors such as text topic or genre.
Since authorial style heavily depends on these factors, traditional impostors can differ a lot from the candidate text.
% model process of original text generation
The ideal generation technique would model the process of the original text generation, 
i.e. among others, the author's task, and references.
Hence, all external influential factors are specified to match the original conditions.
Unfortunately, due to the nature of this task 
(i.e. requirement of \ac{av} is often linked to a lack of information of the author either due to death in the case of literary texts or 
due to unwillingness of cooperation in case of plagiarism), this information is in most cases not available.
% heuristics: paraphrase
With \acp{llm} it is possible attempt to control external factors and 
as a heuristic to modelling the generation process, paraphrase the original text.
% lack of definition of paraphrase
Due to the lack of a universal definition of paraphrases~\citep{gohsen_task_oriented_2024}, the following criteria are used to determine the quality of the generated paraphrases:
% our criteria for good paraphrases
\begin{itemize}
    \item The generated text should belong to the same topic, genre and exhibit the same tone (\textcolor{red}{=tone???}) as the original text.
    \item The semantic information may differ (i.e. hallucination is allowed).
    \item The generated text should be different to the original text in terms of style, i.e. wording and sentence structure (i.e. syntactic similarity). \textcolor{red}{threshold to exclude near-duplicate paraphrases~\citep{gohsen_captions_2023}?}
\end{itemize}


% Notes
According to \citet{gohsen_task_oriented_2024}, there are two perspectives to paraphrasing: 
Lexical (i.e. changes at word level) and syntactic (i.e. changes at syntactic level).
Paraphrase types can be classified into surface and semantic level. Finer levels are outlined in \citep{gohsen_task_oriented_2024}.

Popular paraphrase categories include \citep{fu_learning_2024}:
\begin{itemize}
    \item Top-level classification perspective: 
        \begin{itemize}
            \item Lexicon-based changes
            \item Morphology-based changes
            \item others
        \end{itemize}
    \item Second-level classification perspective:
        \begin{itemize}
            \item Change of format
            \item Semantic-based
            \item Change of order
        \end{itemize}
\end{itemize}

\citet{zhou_paraphrase_2025} (pg. 3, tab.1, and examples afterwards) define a topology of paraphrase types:
\begin{itemize}
    \item Morphology based: inflection changes (e.g. singular to plural), derivation changes (e.g. adjective to verb), functional word substitution (e.g. this to that).
    \item Lexicon based: Same polarity substitution (e.g. synonym), opposite polarity substitution (e.g. antonym), converse substitution (e.g. opposite view point), spelling changes, synthetic/ analytic substitution, relational substitution
    \item Syntax based: Negation switching (i.e. other negation), diathesis alternation (e.g. change position of verb), etc.
    \item Discourse based: Indirect/direct substitutions, sentence modality changes, punctuation changes, etc.
    \item Other changes: Change of order, change of format, etc.
\end{itemize}

\citet{zhou_paraphrase_2021} claim it is difficult to control the style of generated paraphrases.

Another approach to paraphrasing is back-translation, which may limit diversity \citep{zhou_paraphrase_2025}.


\subsection{Perplexity}
\label{subsec:perplexity}

% Perplexity is measure to assess how surprised a language model is by a text.
Perplexity $PPL$ can be employed to compute the likelihood of a language model generating a text.
A low perplexity indicates that the sequence aligns with model's predictions, 
while a high perplexity indicates that the sequence is unexpected or unlikely according to the model.
Perplexity is computed as follows:
\begin{equation}
    PPL = \exp\left(-\frac{1}{t}\sum_{i=1}^{t}\log P(w_i|w_{<i})\right)
\end{equation}
where $t$ is the number of words or tokens in the sequence, 
$w_i$ is the $i$-th word/ token, and $P(w_i|w_{<i})$ is the probability of the $i$-th word/ token given all previous words/ tokens in the sequence.
The exponent is the cross-entropy loss between the model's predictions and the actual sequence.
The cross-entropy can be refactored to the sum of the entropy of the model's predictions and the KL divergence of the prediction and the data.
While Python libraries such as \texttt{PyTorch} and \texttt{TensorFlow} use the natural logarithm $\log$ for perplexity calculations,
traditional information theory uses the logarithm to base 2. 
Note, that different bases differ only by a constant factor.
For sequences longer than the context window of the model, 
perplexity is computed on the windows of $n$ tokens, where $n$ is the context window size.
% strides: not good
Depending on the tokenizer, perplexity can be computed on the word or sub-word level, 
where sub-word level perplexity is often smaller due to higher likelihoods of smaller character sequences.
Since larger vocabulary lead to lower likelihoods per token, perplexity is generally higher for larger vocabularies.
Hence, perplexity is not directly comparable across different tokenizers or models.
Moreover, perplexity requires access to the model's probabilities $P(w_i|w_{<i})$, which are often not available.



\begin{figure}[htbp]
    \centering
    \includesvg[width=\textwidth]{images/unmasking/unmasking.svg}
    \caption{Unmasking.}
    \label{fig:unmasking}
\end{figure}
Our method is an extension of the original impostor approach by \citet{koppel_determining_2014}.
By varying the seed and temperature, we can generate as many texts as we want.
  
    
The Imposter approach leverages random projections to lower dimensional spaces (i.e. random set of features set to zero is a projection).
\begin{figure}[htbp]
    \centering
    \includesvg[width=\textwidth]{images/imposter/imposter.svg}
    \caption{Imposter.}
    \label{fig:impostor}
\end{figure}


\section{Quantitative evaluation metrics}

\subsection{\ac{av} quality measures}
\label{subsec:av_quality_measures}

% General 
\citet{chen_web_2008} use precision, recall (also: sensitivity or \acl{tp} rate \cite{palivela_optimization_2021}) 
and F-measure to describe the performance:
$$precision = \frac{TP}{TP + FP}$$
$$recall = \frac{TP}{TP + FN}$$
$$F-measure = \frac{2 \cdot precision \cdot recall}{precision + recall}$$~\cite{chen_web_2008,abbasi_writeprints_2008}
where $TP$ is the number of \aclp{tp}, $FP$ is the number of \aclp{fp} 
and $FN$ is the number of \aclp{fn}.

% Authorship Analysis Survey
\citet{elmanarelbouanani_authorship_2014} claim that there is no feature set optimized and applicable ti all people and to all domains.

% metrics
\subsection{Metrics}
\citet{elmanarelbouanani_authorship_2014} state that the common metrics for evaluating the performance of a particular are:
\begin{itemize}
    \item $Accuracy = \frac{TP + TN}{TP + TN + FP + FN}$ \cite{elmanarelbouanani_authorship_2014,neal_surveying_2018} 
    measures the percentage of classified correctly over all test cases \cite{neal_surveying_2018}.

    \item $Precision = \frac{TP}{TP + FP}$ \cite{elmanarelbouanani_authorship_2014,neal_surveying_2018} 
    measures how often a system gets positive classification correctly \cite{neal_surveying_2018}.

    \item $Recall = \frac{TP}{TP + FN}$ \cite{elmanarelbouanani_authorship_2014,neal_surveying_2018} 
    measures how often a system correctly classifies positive samples when it encounters them \cite{neal_surveying_2018}.
\end{itemize}

% LLM detection as AV task
Metrics also include c@1.
\citet{llm_detection_av_2025} argue that the reduction from the \ac{fpr}-\ac{tpr} curve of \ac{roc} to a single \ac{roc-auc} number 
comes with information loss due to the absence of a fixed threshold and trade-off.
Moreover, \ac{roc}'s \ac{fpr} and \ac{tpr} are independent of class prevalence, which is desirable.
However, in highly imbalanced class scenarios \ac{roc} can be misleading (overly optimistic or pessimistic).

% PAN
\citet{ayele_overview_2024,bevendorff_overview_2024} evaluate the participants submissions averaging the datasets' evaluation measures.
\citet{ayele_overview_2024} claim the following measures are established in \ac{av}:
\begin{itemize}
    \item \ac{roc-auc} \cite{bevendorff_overview_2024,weerasinghe_feature_vector_difference_2021,kocher_unine_2015}
    
    \item BRIER: Complement of the Brier score \cite{bevendorff_overview_2024,weerasinghe_feature_vector_difference_2021}, in \citet{bevendorff_overview_2024}'s case equivalent to the mean squared loss.
    
    \item C@1: Modified version of the accuracy \cite{bevendorff_overview_2024}/ F1-score \cite{weerasinghe_feature_vector_difference_2021} score, 
    where the non-answers (abstained) \cite{llm_detection_av_2025} are assigned the average accuracy of the remaining cases \cite{bevendorff_overview_2024}. 
    It rewards systems that leave difficult problems unanswered \cite{weerasinghe_feature_vector_difference_2021}.
    $$c@1 = \frac{nc}{np}(1+\frac{nu}{np})$$ where $np$ is the number of problems, $nc$ the number of correct answers, 
    and $nu$ the number of unanswered problems \cite{kocher_unine_2015}.
    
    \item $F_1$: Harmonic mean of precision and recall \cite{bevendorff_overview_2024,weerasinghe_feature_vector_difference_2021}:
    $ F_1 = 2 \cdot \frac{precision \cdot recall}{precision + recall} $ \cite{neal_surveying_2018}.
    A higher value indicates a better performance \cite{neal_surveying_2018}.
    
    \item $F_{0.5u}$: Modified version of the $F_{0.5}$ score, where the non-answers are considered \acp{fn} \cite{bevendorff_overview_2024}. A measure that puts more emphasis on deciding same-author cases correctly \cite{weerasinghe_feature_vector_difference_2021}.
\end{itemize}

The \ac{roc} curve is generated according to the percentage of \acp{fp}, i.e. \ac{fpr} $= \frac{FP}{FP+TN}$, in the x-axis and 
the percentage of \acp{tp}, i.e. \ac{tpr} $=\frac{TP}{TP+FN}$, in the y-axis,
for varying thresholds \cite{kocher_unine_2015,neal_surveying_2018}.
The maximum value of 1.0 indicated a perfect performance \cite{kocher_unine_2015}.
The \ac{auc} is the area under the curve, where a greater \ac{auc} indicates a better performance \cite{neal_surveying_2018}.
\citet{kocher_unine_2015} claim that both \ac{roc} and \ac{auc} are difficult to interpret.
According to \citet{kocher_unine_2015}, usually, the \ac{auc} values should be consistent and comparable with the C@1 values.
The \ac{auc} of the \ac{roc} is biased since the \ac{roc} gives more emphasis 
on the first position and therefore increases the total \ac{auc}.
A misclassification with a lower probability is less penalized with \ac{roc-auc} \cite{kocher_unine_2015}.

% PAN AA & AV metrics
The \todo{F0.5u, C@1, and Brier Score metrics} are used to evaluate the ability of \ac{av} methods 
to abstain from hard samples \cite{tyo_state_2022}.
For each sample, a score $\in [0, 1]$ is assigned to the sample.
A score of exactly 0.5 means the model abstains from the sample \cite{tyo_state_2022,bevendorff_overview_2024,kocher_unine_2015}.

The AUC metric is used to evaluate the ability of methods to rank predictions.
No threshold is required.
\ac{pan} ignores any abstained samples when calculating the AUC metric \cite{tyo_state_2022}.

% != PAN
\citet{tyo_state_2022} chose to adopt the macro-averaged accuracy metric, so-called macro-accuracy, for \ac{aa}, 
and AUC \ac{av} tasks.

\subsection{Paraphrase quality measures}
\label{subsec:paraphrase_quality_measures}

Syntactic (BLEU, ROUGE-1, ROUGE-L), semantic (BERTScore, cosine similarity of SBERT vectors, WMS), human evaluation (TODO)

There is manual (by humans) evaluation and automatic evaluation for paraphrase generation \cite{fu_learning_2024,zhou_paraphrase_2021}.
According to \citet{zhou_paraphrase_2021}, automatic evaluation metrics mainly focus on the n-gram overlaps instead of meaning, 
and hence, human evaluation is more accurate and has a higher quality.
In the following, we focus on automatic evaluation.

There is syntactic and semantic evaluation of paraphrases \cite{gohsen_captions_2023}.
Metrics for syntactic evaluation include BLEU, ROUGE-1, ROUGE-L, 
while metrics for semantic similarity include BERTScore, 
cosine similarity of dense vector representations derived from a BERT-based sentence transformer, 
and Word Mover Distance \cite{gohsen_captions_2023}.
The Word Mover Distance computes the minimum amount of distance that embedded words of a text need to travel 
to reach the embedded words of another text \cite{gohsen_captions_2023}.
\citet{gohsen_captions_2023} normalized all metrics and averaged the semantic and syntactic scores separately.

\bluert{} is machine evaluation metric for paraphrase generation.
\citet{fu_learning_2024} use \bluert{} to filter out incorrect paraphrases (i.e. using a threshold $\theta$).

BLEU (Bilingual Evaluation Understudy \cite{palivela_optimization_2021,zhou_paraphrase_2025,papineni_bleu_2001}) (2002) 
was developed for machine translation \cite{zhou_paraphrase_2021,papineni_bleu_2001}.
It can take values from 0 to 1 \cite{papineni_bleu_2001}.
BLEU is a precision measure \cite{kurt_pehlivanoglu_comparative_2024,papineni_bleu_2001}.
It counts the matching n-grams (unigrams) in the generated/candidate text that appear in any of the gold/ reference texts \cite{palivela_optimization_2021,papineni_bleu_2001}, 
and then divides them by the total number of n-grams (unigrams) in the candidate text \cite{papineni_bleu_2001}.
Since candidates consisting only of high-probability n-grams (e.g. "the") would receive a high score without deserving it, 
\citet{papineni_bleu_2001} introduced a clipping mechanism to limit the count of n-grams in the candidate text to the maximum count of that n-gram in any of the reference texts.
The clipped n-grams occurences are added up and divided by the total number of unclipped n-grams in the candidate text \cite{papineni_bleu_2001}.
\citet{papineni_bleu_2001} state that unigrams are used to test adequacy, while longer n-grams are used to test fluency.
BLEU's basic unit of evaluation is a sentence. 
In order to compute the BLEU score from \autoref{eq:bleu} for more than one sentence, one (1) computes the n-grams matches sentence by sentence, 
then (2) add the clipped n-grams matches across all sentences, 
and finally (3) divides the total clipped n-grams matches by the total number of unclipped n-grams in all candidate sentences \cite{papineni_bleu_2001}.
\begin{equation}
    p_n = \frac{\sum_{\mathcal{C} \in \left\{ Candidates \right\}}\sum_{n-gram \in\mathcal{C}}Count_{clip}(n-gram)}{\sum_{\mathcal{C'} \in \left\{ Candidates \right\}}\sum_{n-gram' \in\mathcal{C'}}Count(n-gram')}
\label{eq:bleu}
\end{equation}
BLEU combines the scores for different n-grams (separately computed) using the average logarithm with uniform weights, 
which is equivalent to using the geometric mean of the scores \cite{papineni_bleu_2001,banerjee_METEOR_2005}.
\citet{gohsen_captions_2023} use up to 4-grams.
BLEU automatically penalizes n-grams appearing in the candidate text but not in the reference text, as well as n-grams appearing more often in the candidate than in the reference text \cite{papineni_bleu_2001}.
According to \citet{papineni_bleu_2001}, they need to add a brevity penalty to the BLEU score to enforce proper length of the candidate text. 
For multiple sentences, they (1) add the best match (among the reference texts) length for each candidate sentence, and (2) divide this sum $r$ by the total length of all candidate sentences $c$. 
Hence, the brevity penalty $BP$ is defined as follows in \autoref{eq:bleu_brevity_penalty}:
\begin{equation}
    BP = \begin{cases}
        1 & \text{if } c > r \\
        e^{1 - \frac{r}{c}} & \text{if } c \leq r
    \end{cases}
\label{eq:bleu_brevity_penalty}
\end{equation}
Combining all these, the final BLEU score is computed as follows in \autoref{eq:bleu_final}:
\begin{equation}
    \text{BLEU} = BP \cdot \exp\left(\sum_{n=1}^{N} w_n \cdot \log p_n\right)
\label{eq:bleu_final}
\end{equation}
They cannot use recall for length-related problems here, because BLEU uses multiple reference texts, which may have different lengths \cite{papineni_bleu_2001,banerjee_METEOR_2005}.
If the generated candidate is significantly shorter than the reference text, the brevity penalty $BP$ is applied.
A BLEU score approaching 1 signifies the candidate matches one reference almost exactly \cite{papineni_bleu_2001}, 
and thus, limited syntactic diversity (i.e. inadequate paraphrase) \cite{kurt_pehlivanoglu_comparative_2024}.
Note that more reference texts lead to higher BLEU scores \cite{papineni_bleu_2001}.
Unigrams are token-wise and bi-grams are word-pairs \citet{palivela_optimization_2021}.
According to \citet{zhou_paraphrase_2021}'s survey, BLEU is the most frequently used metric for paraphrase generation.
BLEU is unable to measure semantic equivalents \cite{kurt_pehlivanoglu_comparative_2024,zhou_paraphrase_2021} 
when applied to low-resource languages \cite{zhou_paraphrase_2021}.
Moreover, BLEU fails to capture good paraphrases that are not similar to the reference text \cite{zhou_paraphrase_2021}.
\citet{kurt_pehlivanoglu_comparative_2024} found that BLEU tends to overestimate the quality of paraphrases.
\citet{zhou_paraphrase_2021} suggest combining BLEU with human evaluation to overcome its limitations.

GLEU (Google-BLEU) (ranges from 0 to 1 \cite{kurt_pehlivanoglu_comparative_2024}) is a variant of BLEU that was developed to be closer to human judgement, and to 
overcome BLEU's drawback of per sentence reward objective \cite{palivela_optimization_2021}.
GLEU computes n-gram precisions (overlaps \cite{kurt_pehlivanoglu_comparative_2024}) overgold/reference paraphrases 
and weighs n-grams by their change from the source text \cite{palivela_optimization_2021}.
GLEU assesses the fluency, order of n-grams, structural and semantic accuracy 
and penalizes shorter average m-gram lengths in the generated text compared to the reference \cite{kurt_pehlivanoglu_comparative_2024}.
Lower GLEU scores indicate greater diversity \cite{kurt_pehlivanoglu_comparative_2024}.

METEOR (Metric for Evaluation of Translation with Explicit Ordering \cite{palivela_optimization_2021,banerjee_METEOR_2005}) 
(ranges from 0 to 1 \cite{kurt_pehlivanoglu_comparative_2024}) (2014) aims to address BLEU's shortcomings.
First a mapping, so-called aligment, between the unigrams in the candidate text and the reference texts is created \cite{banerjee_METEOR_2005}.
Each unigram has zero or one match.
This aligment is created incrementally in repeating two steps:
(1) List all possible unigram mappings derived from different modules (i.e. exact matches, Porter stemmed matches, synonym matches), 
and (2) select the largest subset of unigram mappings that constitute a valid alignment (matches obtained from different modules are treated the same).
(3) Choose the subset with the largest cardinality and if there are multiple, choose the one with the fewest unigram mapping crosses \cite{banerjee_METEOR_2005}.
METEOR computes a weighted F-score 
(unigram-precision, unigram-recall \cite{kurt_pehlivanoglu_comparative_2024,banerjee_METEOR_2005} 
and a measure of fragmentation \cite{banerjee_METEOR_2005,kurt_pehlivanoglu_comparative_2024})
with a penality function whenever an incorrect word is encountered \cite{palivela_optimization_2021} as displayed in \autoref{eq:meteor}.
\begin{equation}
    METEOR = F_{mean} = \frac{10 \cdot P \cdot R}{R + 9P} \cdot (1 - Penalty)
\label{eq:meteor}
\end{equation}
The penality is designed to reduce the $F_{mean}$ score to $50\%$ if there are no bigram or longer matches \cite{banerjee_METEOR_2005}.
It has better correlation with human judgement at the sentence/segment level than BLEU \cite{zhou_paraphrase_2021}, 
because it not only consists of simple n-gram matching but also including synonymy and stemming \cite{kurt_pehlivanoglu_comparative_2024}.

ROUGE (Recall-Oriented Understudy for Gisting Evaluation \cite{palivela_optimization_2021,lin_rouge_2004}) 
(ranges from 0 to 1 \cite{kurt_pehlivanoglu_comparative_2024}) (2004) 
is a recall-based metric developed for text summarization \cite{zhou_paraphrase_2021,palivela_optimization_2021,kurt_pehlivanoglu_comparative_2024,lin_rouge_2004}.
ROUGE can focus on the word variations and diversity.
It has multiple versions, the most popular ones include 
ROUGE-N (computing the n-gram recall) \cite{zhou_paraphrase_2021,palivela_optimization_2021,kurt_pehlivanoglu_comparative_2024}, 
ROUGE-L (computing the longest common subsequence) \cite{zhou_paraphrase_2021,palivela_optimization_2021,kurt_pehlivanoglu_comparative_2024}, 
ROUGE-W (Weighted longest common subsequence) \cite{palivela_optimization_2021}, 
ROUGE-S (skip-bigram co-occurrence statistics) \cite{palivela_optimization_2021}.
ROUGE-1 computes the recall by analysing the matching unigrams between the generated paraphrase and the reference paraphrase \cite{palivela_optimization_2021,kurt_pehlivanoglu_comparative_2024}.
% ROUGE-N
ROUGE-N is an n-gram recall between the candidate text and the reference texts \cite{lin_rouge_2004} as displayed in \autoref{eq:rouge_n}.
\begin{equation}
    ROUGE-N = \frac{\sum_{\mathcal{S} \in \left\{ References \right\}}\sum_{n-gram \in\mathcal{S}}Count_{match}(n-gram)}{\sum_{\mathcal{S'} \in \left\{ References \right\}}\sum_{n-gram' \in\mathcal{S'}}Count(n-gram')}
\label{eq:rouge_n}
\end{equation}
$Count_{match}(n-gram)$ is the maximum number of n-grams co-occuring in the candidate text and the set of reference texts \cite{lin_rouge_2004}.
The nominator sums over all references and thus, gives more weight to matching n-grams that occur in multiple references (i.e. a consensus between references) \cite{lin_rouge_2004}.
Refer to \citet{lin_rouge_2004} for more details on the work with multiple references (I do not understand that, because I thought we already use multiple).
% ROUGE-L
For ROUGE-L, the intuition is that the longer the longest common subsequence (LCS) between the candidate and reference texts, the more similar they are \cite{lin_rouge_2004}.
For a candidate $Y$ of length $n$ and a reference $X$ of length $m$, the ROUGE-L score is defined as follows in \autoref{eq:rouge_l}:
\begin{equation}
    ROUGE-L = F_{lcs} = \frac{(1 + \beta^2)R_{lcs}P_{lcs}}{R_{lcs} + \beta^2 P_{lcs}}
\label{eq:rouge_l}
\end{equation}
where $R_{lcs} = \frac{LCS(X,Y)}{m}$ and $P_{lcs} = \frac{LCS(X,Y)}{n}$ \cite{lin_rouge_2004}.
ROUGE-L requires in-sequence macthes that reflect the sentence level word order as n-grams \cite{lin_rouge_2004}.
Moreover, no predefined $n$ is necessary, because ROUGE-L includes the longest in-sequence common n-grams \cite{lin_rouge_2004}.
However, ROUGE-L does not include shorter sequences or alternative LCSes in the final score \cite{lin_rouge_2004}.
% ROUGE-S
A skip-bigram is any pair of words in their sentence order, allowing for arbitrary gaps \cite{lin_rouge_2004}.
ROUGE-S measures the overlap of skip-bigrams between the candidate text and the reference texts \cite{lin_rouge_2004}.
Hence, if the candidate text is the reverse of the reference text, the ROUGE-S score is 0 even though it is not as bad as completely unrelated candidates \cite{lin_rouge_2004}.
ROUGE-SU extends ROUGE-S with unigrams to solve this issue \cite{lin_rouge_2004}.
% ROUGE generally
\citet{kurt_pehlivanoglu_comparative_2024} claim that ROUGE may not be adequate to assess semantic similarity and fluency.
Lower ROUGE scores indicate greater diversity \cite{kurt_pehlivanoglu_comparative_2024}.

TER (2006) was developed for machine translation \cite{zhou_paraphrase_2021}.
It computes the number of edits required to change the translation until it matches the reference translation.
It ranges from 0 (i.e. no edits needed) to 1 (i.e. all words need to be changed).

\citet{fu_learning_2024} describe the Gini Coefficient as a measure of inequality in a distribution, ranging from 0 (i.e. even distribution across categories) to 1.

\citet{master_thesis_paraphrasing_2024} include a table (tab. 3.1, pg. 18) with numerous stylometric metric including readbility, vocabulary richness, and word/character counts.

\citet{palivela_optimization_2021} state that accuracy, precision, recall and F1-score are suitable for \ac{pi}.
For \ac{pg}, they suggest ROUGE, BLEU, GLEU, WER (Word Error Rate), and METEOR as suitable metrics.
WER is the number of substitutions (replacements of words), insertions (adding words) and deletions (removing words) 
divided by the total number of words in the reference text \cite{palivela_optimization_2021}.

\citet{kurt_pehlivanoglu_comparative_2024} additionally use T5-STSB.
The metric is based on \ac{t5} model adapted to the Semantic Textual Similarity Benchmark (STSB).
It evaluates semantic equivalence by assigning a similarity score from 0 (no similarity) to 5 (complete equivalence) \cite{kurt_pehlivanoglu_comparative_2024}.

BERTScore calculates the cosine similarity between the contextual embeddings of the reference and generated texts. 
Hence, is assesses semantic equivalence and correlates well with human judgement \cite{kurt_pehlivanoglu_comparative_2024}.
First, token vector representations are computed for both the reference and generated texts using a pre-trained BERT model \cite{hanna_fine_grained_2021}.
Let reference $z$ and candidate $\hat{z}$ be the vector representations of the reference and candidate texts, respectively.
Then, the BERTScore precision, recall and $F_1$ score is computed as follows in \autoref{eq:bert_p}, \autoref{eq:bert_r}, and \autoref{eq:bert_f1}, respectively:
\begin{equation}
    P_{BERT} = \frac{1}{|\hat{z}|} \sum_{\hat{z}_j \in \hat{z}} \max_{z_j \in z} z_i\top \hat{z}_j
\label{eq:bert_p}
\end{equation}
\begin{equation}
    R_{BERT} = \frac{1}{|z|} \sum_{z_j \in z} \max_{\hat{z}_j \in \hat{z}} z_i\top \hat{z}_j
\label{eq:bert_r}
\end{equation}
\begin{equation}
    F_1 = \frac{2 P_{BERT} R_{BERT}}{P_{BERT} + R_{BERT}} 
\label{eq:bert_f1}
\end{equation}
Since $F_1 \in \left[-1,1\right]$ it can be rescaled to $[0,1]$ by modifying the precision and recall calculation 
to $\hat{P}_{BERT} = \frac{P_{BERT} - a}{1 - a}$ ($R_{BERT}$ analoguous), where $a$ is the empirical lower bound on the BERTScore \cite{hanna_fine_grained_2021}.
The BERTScore has difficulties on datasets with lexically similar (i.e. lexical overlap of content words) incorrect candidates 
opposed to lexically different more correct candidates \cite{hanna_fine_grained_2021}.


\ac{t5}-CoLA metric (ranges from 0 to 5 \cite{kurt_pehlivanoglu_comparative_2024}) utilizes the Corpus of Linguistic Acceptability (CoLA) to evaluate the grammatical correctness of sentences and thus, 
contributes linguistic evaluation \cite{kurt_pehlivanoglu_comparative_2024}.

Generally, \citet{kurt_pehlivanoglu_comparative_2024} order the metrics by their contribution area:
\begin{itemize}
    \item semantic: BERTScore, STSB, METEOR
    \item Fluency: CoLA
    \item Diversity: ROUGE1/2/L, BLEU, GLEU
\end{itemize}

\citet{krishna_paraphrasing_2023} compute the lexical diversity using unigram token overlap and call it F1 score.
As a semantic similarity score, they use the ACL Antology 2022 published \href{https://aclanthology.org/2022.emnlp-demos.38.pdf}{P-SP}.

According to \citet{gohsen_task_oriented_2024}, there are two perspectives to paraphrasing: 
Lexical (i.e. changes at word level) and syntactic (i.e. changes at syntactic level).
Paraphrase types can be classified into surface and semantic level. Finer levels are outlined in \cite{gohsen_task_oriented_2024}.

\textcolor{red}{Do we want semantically similar paraphrases even though our task is style transfer belonging to semantically equivalent paraphrasing?}

Popular paraphrase categories include \cite{fu_learning_2024}:
\begin{itemize}
    \item Top-level classification perspective: 
        \begin{itemize}
            \item Lexicon-based changes
            \item Morphology-based changes
            \item others
        \end{itemize}
    \item Second-level classification perspective:
        \begin{itemize}
            \item Change of format
            \item Semantic-based
            \item Change of order
        \end{itemize}
\end{itemize}
\citet{zhou_paraphrase_2025} (pg. 3, tab.1, and examples afterwards) define a topology of paraphrase types:
\begin{itemize}
    \item Morphology based: inflection changes (e.g. singular to plural), derivation changes (e.g. adjective to verb), functional word substitution (e.g. this to that).
    \item Lexicon based: Same polarity substitution (e.g. synonym), opposite polarity substitution (e.g. antonym), converse substitution (e.g. opposite view point), spelling changes, synthetic/ analytic substitution, relational substitution
    \item Syntax based: Negation switching (i.e. other negation), diathesis alternation (e.g. change position of verb), etc.
    \item Discourse based: Indirect/direct substitutions, sentence modality changes, punctuation changes, etc.
    \item Other changes: Change of order, change of format, etc.
\end{itemize}

\citet{fu_learning_2024} give three prompt tips:
\begin{enumerate}
    \item Make question/ prompt as clear as possible even if some restrictive requirements may seem unnecessary.
    \item Place significant details, including restrictive elements such as time, place, manner, reason, purpose and conditions, at the beginning of the prompt.
    \item Pay attention to spelling, i.e. proper nouns, title, honorifics, abbreviations, acronyms and observe capitalization.
\end{enumerate}

\citet{zhou_paraphrase_2021} claim it is difficult to control the style of generated paraphrases.

Another approach to paraphrasing is back-translation, which may limit diversity \cite{zhou_paraphrase_2025}.
