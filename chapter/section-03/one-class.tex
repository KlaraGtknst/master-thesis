\section{\ac{av} as One-Class categorisation}
\label{sec:av_one_class}

% \citet{koppel_authorship_2004} claim research had shown that linear separators work well for text categorisation.
% Linear models include Naive Bayes (linear separator for two classes), 
% Window and Exponential Gradient and linear \acp{svm} \citep{koppel_authorship_2004}. 

% \subsection{\ac{aa} framework}
% \citet{koppel_authorship_2004} state that the following framework solve a number of real world \ac{aa} problems:
% \begin{enumerate}
%     \item Construction of appropriate feature vectors
%     \item Construction of a distinguishing model via a learning algorithm
%     \item Assessment of effectiveness of methods using k-fold cross-validation or bootstrapping
% \end{enumerate}

\subsection{Chunks}
\citet{koppel_authorship_2004}\ propose chunking texts such that each chunk is of approximately equal length, 
and at least 500 words without breaking paragraphs. 

\subsection{Definitions}
For author $A$ and book $X$, \citet{koppel_authorship_2004}\ define the following:
If $A$ is not the author of $X$, $A_X$ is the set of all works by author $A$.
If $A$ is the author of $X$, $A_X$ is the set of all works by author $A$ except $X$.
A pair of $A_X$ and $X$ is called \emph{same-author} if X was authored by $A$.
A pair of $A_X$ and $X$ is called \emph{different-author} if $X$ was not authored by $A$.

\subsection{Initial feature set}
The initial feature set consists of the 250 words with the highest average (over $X$ and $X_A$) frequency \citep{koppel_authorship_2004}.

\subsection{Features for meta-classifier}
\citet{koppel_authorship_2004}\ propose the following features for the meta-classifier 
(where $i$ is the number of elimination steps):
\begin{itemize}
    \item accuracy after $i$ elimination steps
    \item accuracy difference between round $i$ and round $i+1$
    \item accuracy difference between round $i$ and round $i+2$
    \item $i^{th}$ highest accuracy drop in one iteration
    \item $i^{th}$ highest accuracy drop in two iterations
\end{itemize}

The vectors are grouped by \emph{same-author} and \emph{different-author} pairs and thus, 
used to train a meta-learning scheme.

\subsection{Negative examples for Elimination method}
\citet{koppel_authorship_2004}\ state that negative examples are neither exhaustive nor representative.
They propose using words of several authors $A_1, ..., A_n$ roughly filling the same profile as candidate $A$ 
in terms of geography, chronology, culture and genre.
$A_1, ..., A_n$ are said to collectively represent class not-$A$.

\subsection{Elimination method}

The elimination method is only used to overrule positive predictions.
Hence, it can eliminate \acp{fp}.
One can frame it as a filter which is applied after or before \unmasking{}.

% training
\citet{koppel_authorship_2004}\ learn a model for $A$ and against not-$A$, 
and multiple models for $A_i$ and against not-$A_i$.
% inference
Then, $X$ is tested against all of these models.
$A(X)$ is the percentage of examples of $X$ classed as $A$ rather than not-$A$ 
(i.e.\ $A_i(X)$ analogously).
If $A(X)$ is not larger than all $A_i(X)$, $A$ is not the author of $X$.
If $A(X)$ is larger than all $A_i(X)$, conclude nothing.
