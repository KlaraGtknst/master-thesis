\subsection{\impAppr{}}
\label{sec:impostor_method_theory}

\citet{koppel_determining_2014} propose reducing open-set \ac{av} to open-set \ac{aa}, or the so-called many-candidates, which determines which (if any) candidate author from a pool of candidates authored a given anonymous text. 
They outline the problem hierarchy as displayed in \autoref{fig:problem_hierarchy}.
The candidates of the open-set \ac{aa} problem are retrieved via hard negative mining. 
In the \ac{aa} context, hard negatives are defined as text pairs from different authors that are nonetheless highly similar. 
To ensure difficulty, impostor texts must be drawn from the same domain as the candidate, where “domain” encompasses factors such as topic, genre, register, or time period~\citep{bischoff_importance_2020}. 

\begin{figure}[htbp]
    \centering
    \includesvg{notes/Koppel_imposter_2014/problem_hierarchy_complexity}
    \caption{Problem hierarchy and complexity.}
    \label{fig:problem_hierarchy}
\end{figure}

The disputed text and all candidates are represented via \ac{tfidf} vectors of the \num{100000} most frequent space-free character 4-grams.
Space free n-grams are strings of $n$ characters that do not include spaces, or strings of less than $n$ characters that are surrounded by spaces~\citep{koppel_authorship_2011,neal_surveying_2018}.
While the optimal $n$ is language dependent~\citep{neal_surveying_2018}, tri-grams are commonly used in stylistic analysis, due to their ability to capture inflections, % Flexion/ Beugung in Deutsch
morphemes, %  smallest meaningful constituents within a linguistic expression and particularly within a word
and other syntactic structures for Germanic languages.
Explorative study on character n-gram proposed that character n-grams are successful in \ac{av} tasks, due to their ability to capture different stylometry feature categories.
Affix n-grams capture morphology, and n-grams containing punctuation marks capture style information valuable for cross-domain settings.
Word n-grams capture topic information which is not necessary to obtain valid performance~\citep{Sapkota_ngrams_2015}.

The \impAppr{} builds on unmasking's repeated feature subsampling~\citep{koppel_authorship_2004}. 
However, unlike unmasking, the \impAppr{} uses random projections to transform features into lower-dimensional spaces.
After each projection step, the method records how often a candidate author is predicted as the source of the text. 
This score is aggregated across all elimination steps~\citep{tyo_state_2022}.
The workflow described above is illustrated in \autoref{fig:impostor}.

The process is carried out twice for each pair of texts, swapping the roles of the disputed text and the candidate text. 
In this way, each text serves once as the disputed text and once as the reference for impostor generation. 
The final prediction is obtained by averaging the aggregated scores and applying a threshold.


\begin{figure}[htbp]
    \centering
    \includesvg[width=\textwidth]{images/imposter/problem_hierarchy_complexity.svg}
    \caption{\impAppr{} workflow: (1) Impostor generation, (2) creation of feature vectors using frequent n-grams, (3) random dimensionality reduction, (4) similarity computation and selection of the most similar candidate.}
    \label{fig:impostor}
\end{figure}



% \begin{definition}
%     [within-domain]
%     Experiments with P=Q.
%     Hence, it is necessary to ensure all texts are mutually from the same domain \citep{bischoff_importance_2020}.
%     \begin{table}[tbp]
%         \centering
%         \caption{Typical scheme $S_1$ for \ac{aa} problem instances, where A, B, are authors and P, Q domains and 
%         the vertical mapping denotes which author has written in which domain. 
%         For training, texts from A and B take turn; for testing, previously unseen texts from A and B are used \citep{bischoff_importance_2020}.}
%         \label{tab:within_domain_aa}
%         \begin{tabular}{|l|ll|ll|}
%         \hline
%         \textbf{Scheme $S_1$} & \multicolumn{2}{l|}{\textbf{training}} & \multicolumn{2}{l|}{\textbf{testing}} \\ \hline
%         \textbf{authors} & \multicolumn{1}{l|}{A} & B & \multicolumn{1}{l|}{A} & B \\ \hline
%         \textbf{domains} & \multicolumn{1}{l|}{P} & Q & \multicolumn{1}{l|}{P} & Q \\ \hline
%         \end{tabular}%
%     \end{table}
% \end{definition}



% \begin{figure}[htbp]
%     \centering
%     \includesvg{notes/Koppel_imposter_2014/reduction_closed_set_AV_to_open_set_AA}
%     \caption{Reducing the \ac{av} problem to the many-candidates problem.}
%     \label{fig:problem_reduction}
% \end{figure}


\subsubsection{Traditional \Imp{} Generation}
\label{subsubsec:traditional_impostor_generation}

\citet{koppel_determining_2014} introduce three approaches to generating \imps{}. 
In the fixed approach, \imps{} are drawn at random from a predetermined pool of documents that bear no particular relation to the input pair.
In their study, this pool consisted of texts collected from random English Google queries. 
The on-the-fly approach, by contrast, derives impostors based on the candidate text. 
Specifically, sets of three to five medium-frequency words are sampled from the text, each set is submitted as a Google query, and the top 25 results are aggregated. 
The retrieved documents serve as \imps{} that are, at least in theory, on-topic. 
Finally, in the blog approach, impostors are selected at random from the \dataBlog{} dataset. 
This method, according to the authors, ensures that impostor texts share the same genre as the input pair.


First the $m$ most similar impostor documents in terms of min-max similarity are selected.
Then, $n < m$ random impostor documents are selected.
\citet{koppel_determining_2014} found that using a selection of $n$ \imps{} rather than all $m$ impostor documents produces better results.
The approach is insensitive to $m$ and $n$.