
\subsubsection{\imp{} Method}
\label{sec:impostor_method_theory}


\begin{definition}
    [\imp{} method]
    This method extends the ngram-unmasking method, i.e. iteratively omitting most influencely features (repeated feature subsampling \citep{koppel_determining_2014})
    from a trained classifier and classifying the accuracy drop.
    It takes score of how often an author is predicted after each feature-elimination step.
    The final prediction is made based on this score \citep{tyo_state_2022}.
\end{definition}


\begin{definition}
    [Hard Negative Mining]
    This method updates the model during training only with the most difficult examples in each batch.
    In the \ac{aa} context, difficult is defined as the most similar two texts from different authors, 
    which makes the decision the most difficult.
    \citet{tyo_state_2022} claim that the \ac{av} setting is strictly easier since 
    it most compare to only a single text.
    Due to the fact, that the most difficult example is model-dependent, \ac{av} problems can be made harder 
    but they can not exist of exactly the hardest negatives.
\end{definition}


\begin{definition}
    [Domain]
    The domain include topic, genre, register, idiolect, time period etc. \citep{bischoff_importance_2020}.
\end{definition}
  
\begin{definition}
    [Domain variables]
    These include topic, genre and language \citep{bischoff_importance_2020}.
\end{definition}

\begin{definition}
    [within-domain]
    Experiments with P=Q.
    Hence, it is necessary to ensure all texts are mutually from the same domain \citep{bischoff_importance_2020}.
    \begin{table}[tbp]
        \centering
        \caption{Typical scheme $S_1$ for \ac{aa} problem instances, where A, B, are authors and P, Q domains and 
        the vertical mapping denotes which author has written in which domain. 
        For training, texts from A and B take turn; for testing, previously unseen texts from A and B are used \citep{bischoff_importance_2020}.}
        \label{tab:within_domain_aa}
        \begin{tabular}{|l|ll|ll|}
        \hline
        \textbf{Scheme $S_1$} & \multicolumn{2}{l|}{\textbf{training}} & \multicolumn{2}{l|}{\textbf{testing}} \\ \hline
        \textbf{authors} & \multicolumn{1}{l|}{A} & B & \multicolumn{1}{l|}{A} & B \\ \hline
        \textbf{domains} & \multicolumn{1}{l|}{P} & Q & \multicolumn{1}{l|}{P} & Q \\ \hline
        \end{tabular}%
    \end{table}
\end{definition}


The \impAppr{} leverages random projections to lower dimensional spaces (i.e. random set of features set to zero is a projection).
\begin{figure}[htbp]
    \centering
    \includesvg[width=\textwidth]{images/imposter/imposter.svg}
    \caption{\imp{}.}
    \label{fig:impostor}
\end{figure}

% AV -> open-set
\ac{av} is an open-set problem, meaning that the author of an anonymous document 
may or may be not be part of the set of candidate authors.

% AA -> closed-set
\ac{aa} is a closed-set problem, meaning that the author of an anonymous document
is part of the set of candidate authors.
For each candidate author, writing samples are available.
The task is to determine the author of the anonymous document from the set of candidate authors.

% reduction: closed-set AA -> open-set AV
\citet{koppel_determining_2014} state that all closed-set \ac{aa} problems are reducible to the \ac{av} problem.
The reverse is not true.
To reduce the \ac{aa} problem to the \ac{av} problem, we solve a \ac{av} problem, i.e. if text was written by a candidate author, 
for each of the respective candidates.
Ideally, we receive one positive answer for the correct candidate author and negative answers for all other candidates.

% complexity
\citet{koppel_determining_2014} explain that the \ac{av} problem is more complex than the \ac{aa} problem.
They claim that the ability to solve a closed-set \ac{aa} problem does not imply the ability to solve an open-set \ac{av} problem.

% open-set identification/ AA = many candidates problem
\citet{koppel_determining_2014} define the many-candidates problem, or the so-called open-set identification problem:
Given a large set of candidate authors, determine which, if any, of them wrote a given anonymous document.
According to \citet{koppel_determining_2014}, the many-candidates problem can be solved reasonably well: \autoref{lst:many_candidate_algo}.