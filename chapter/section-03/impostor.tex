% \subsubsection{\imp{} Method}
% \label{sec:impostor_method_theory}


% The \impAppr{} extends the n-gram Unmasking method, i.e. classifying the accuracy drop from a trained classifier for repeated feature subsampling~\citep{koppel_determining_2014}.
% Similarly, it leverages random projections to lower dimensional spaces (i.e. random set of features set to zero is a projection).
% It takes score of how often an author is predicted after each feature-elimination step.
% The final prediction is made based on this score \citep{tyo_state_2022}.

% % AV -> open-set
% \ac{av} is an open-set problem, meaning that the author of an anonymous document may or may be not be part of the set of candidate authors.
% % reduction: closed-set AA -> open-set AV
% \citet{koppel_determining_2014} state that all closed-set \ac{aa} problems are reducible to the \ac{av} problem.
% For a closed-set problem, the author of an anonymous document is part of the set of candidate authors.
% For each candidate author, writing samples are available.
% The task is to determine the author of the anonymous document from the set of candidate authors.
% The universal reverse reduction is not true.
% To reduce the \ac{aa} problem to the \ac{av} problem, we solve a \ac{av} problem, i.e. if text was written by a candidate author, 
% for each of the respective candidates.
% Ideally, we receive one positive answer for the correct candidate author and negative answers for all other candidates.
% % complexity
% \citet{koppel_determining_2014} explain that the \ac{av} problem is more complex than the \ac{aa} problem.
% They claim that the ability to solve a closed-set \ac{aa} problem does not imply the ability to solve an open-set \ac{av} problem.
% % open-set identification/ AA = many candidates problem
% \citet{koppel_determining_2014} define the many-candidates problem, or the so-called open-set identification problem:
% Given a large set of candidate authors, determine which, if any, of them wrote a given anonymous document.
% According to \citet{koppel_determining_2014}, the many-candidates problem can be solved reasonably well: \autoref{lst:many_candidate_algo}.


% \citet{koppel_determining_2014}' features Hard Negative Mining: 
% It generates impostors as candidates for an artificial many-candidates problem.
% Hence, the model selects the most difficult examples in each batch during training.
% In the \ac{aa} context, hard negatives are defined as the most similar two texts from different authors.
% Therefore, impostors should belong to the same domain as the candidate text.
% The domain include topic, genre, register, idiolect, time period etc.~\citep{bischoff_importance_2020}.

% The workflow is displayed in \autoref{fig:impostor}.

% \begin{figure}[htbp]
%     \centering
%     \includesvg[width=\textwidth]{images/imposter/imposter.svg}
%     \caption{\imp{}.}
%     \label{fig:impostor}
% \end{figure}


\subsubsection{\impAppr{}}
\label{sec:impostor_method_theory}

The \impAppr{} builds on the n-gram Unmasking method, which classifies authorship by measuring how quickly classifier accuracy declines under repeated feature subsampling~\citep{koppel_authorship_2004}. 
Whereas the n-gram Unmasking method relies on complex, interdependent projections, the \impAppr{} uses random projections to transform features into lower-dimensional spaces.
After each projection step, the method records how often a candidate author is predicted as the source of the text. 
The aggregated score across all elimination steps forms the basis for the final prediction~\citep{tyo_state_2022}.

\ac{av} is an open-set problem: the true author of an anonymous text may or may not be the candidate author. 
As \citet{koppel_determining_2014} argue, every closed-set \ac{aa} problem can be reduced to an \ac{av} problem. 
This reduction is achieved by testing each candidate author individually: if the verifier confirms authorship for exactly one candidate, attribution is successful. 
However, the reverse does not hold: solving closed-set \ac{aa} does not guarantee the ability to solve the more complex open-set \ac{av} problem.

Extending \ac{av}, \citet{koppel_determining_2014} introduce the many-candidates (or: open-set \ac{aa}) problem: given a large pool of candidate authors, determine which (if any) authored a given anonymous text. 

A key component of the \impAppr{} is the use of hard negative mining. 
The \impAppr{} generates impostor candidates to make verification more robust. 
In the \ac{aa} context, hard negatives are defined as text pairs from different authors that are nonetheless highly similar. 
To ensure difficulty, impostor texts must be drawn from the same domain as the candidate, where “domain” encompasses factors such as topic, genre, register, idiolect, or time period~\citep{bischoff_importance_2020}. 

The overall workflow of the method is illustrated in \autoref{fig:impostor}.

\begin{figure}[htbp]
    \centering
    \includesvg[width=\textwidth]{images/imposter/impostor.svg}
    \caption{Workflow of the \imp{} method.}
    \label{fig:impostor}
\end{figure}




% \begin{definition}
%     [within-domain]
%     Experiments with P=Q.
%     Hence, it is necessary to ensure all texts are mutually from the same domain \citep{bischoff_importance_2020}.
%     \begin{table}[tbp]
%         \centering
%         \caption{Typical scheme $S_1$ for \ac{aa} problem instances, where A, B, are authors and P, Q domains and 
%         the vertical mapping denotes which author has written in which domain. 
%         For training, texts from A and B take turn; for testing, previously unseen texts from A and B are used \citep{bischoff_importance_2020}.}
%         \label{tab:within_domain_aa}
%         \begin{tabular}{|l|ll|ll|}
%         \hline
%         \textbf{Scheme $S_1$} & \multicolumn{2}{l|}{\textbf{training}} & \multicolumn{2}{l|}{\textbf{testing}} \\ \hline
%         \textbf{authors} & \multicolumn{1}{l|}{A} & B & \multicolumn{1}{l|}{A} & B \\ \hline
%         \textbf{domains} & \multicolumn{1}{l|}{P} & Q & \multicolumn{1}{l|}{P} & Q \\ \hline
%         \end{tabular}%
%     \end{table}
% \end{definition}





