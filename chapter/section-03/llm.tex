\section{Impact of \acs{llm} on Authorship Verification}

With the advances in \ac{gai} come risks and opportunities.
Similarly, we can frame \acp{llm} as authors weakening our trust in the authencity of digital texts, but also see them as valuable tools to enhance our regulatory methods.

\subsection{\acsp{llm} as Authors}
There is no general definition of when a text is \ac{llm} generated rather than co-created by humans with \ac{llm} assistance.
% Obviously, fully generated texts should be marked as \ac{llm} generated.
Minor human edits of \ac{llm} generated texts do not change the fact that the core content was \ac{llm} generated.
If \acp{llm} are used for grammar checking, polishing, and editing suggestion the primary substantial contribution was human.
One could denote these texts "\ac{ai}-revised Human-Written Text"~\citep{wang_stumbling_2024}.

With advances of generative models with regard to mimicking human writing, we have to face the fact that \acp{llm} will play a crucial role in any authorship analysis related tasks from now on.
\citet{llm_detection_av_2025} claim that \ac{llm} detection is not an \ac{aa} task, i.e., a closed-set binary classification where both classes are sufficiently discriminative, but an \ac{av} task, i.e., an open-set one-class classification problem. 

% differences: also more in ~\citep{wang_stumbling_2024}
Despite the advances, there are still some statistical differences on \ac{llm} generated and human authored texts.
\ac{llm} generated texts lack lexical diversity, overuses certain adjectives (e.g.\ "innovative") and produces longer, more complex sentences.
Moreover, \acp{llm} possess stylistic fingerprints and memorize patterns from the training data.
% lengths
Furthermore, word length averages and distributions across genre differ for \acp{llm} and humans~\citep{llm_detection_av_2025}.

% future of LLMs
However, as \acp{llm} progress, basic heuristics applied by human detectors no longer suffice.
\acp{llm} will become more human-like and thus, \ac{llm} detection will increasingly resemble a human authorship classification task~\citep{llm_detection_av_2025}.


\subsection{\acsp{llm} as Discriminators}
\label{sec:llm_discriminator}

While \acp{llm} can generate coherent text, their effectiveness as direct discriminators in \ac{aa} tasks varies. 
Words with similar meaning, such as "color" and "colour", are mapped to similar vector representations~\citep{altakrori_topic_2021}.
This reduces sensitivity to subtle language system differences, which are often critical for identifying an author. 
Consequently, relying on \acp{llm} as the primary discriminator in \ac{aa} or \ac{av} tasks may be suboptimal.

Rather than serving as standalone discriminators, \acp{llm} are more appropriately employed as supporting tools. 
For instance, they can generate cross-domain training data to improve model robustness. 
Moreover, prior methods assessed the extent to which text changes under \ac{llm}-based paraphrasing, since machine generated text tends to undergo minimal alteration, whereas human-authored text exhibits greater variation~\citep{mao_raidar_2024}.
Conversely, in contexts where privacy is a concern, \acp{llm} can facilitate author obfuscation through controlled paraphrasing. 