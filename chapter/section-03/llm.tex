\section{Impact of \acs{llm} on \ai{}}

With advances of generative \acp{llm} with regard to mimicking human writing, we have to face the fact that \acp{llm} will play a crucial role in any \ai{} related tasks from now on.
Since there is not \textit{the one} \ac{llm}, but a great variety of \acp{llm}, one may find denoting \acp{llm} an author deceptive.
This implicates individual authorships for different \acp{llm}.
Recent research~\citep{llm_detection_av_2025}, claimed that \ac{llm} detection is not an \ac{aa} task, i.e., a closed-set binary classification where both classes are sufficiently discriminative, but an \ac{av} task, i.e., an open-set one-class classification problem. 

There is also the problem that there is no general definition of when a text is \ac{llm} generated rather than co-created by humans with \ac{llm} assistance.
Obviously, fully generated texts should be marked as \ac{llm} generated.
Minor edits of \ac{llm} generated texts do not change the fact that the core content was \ac{llm} generated.
If \acp{llm} are used for grammar checking, polishing, and editing suggestion the primary substantial contribution was human.
One could denote these texts "\ac{ai}-revised Human-Written Text"~\citep{wang_stumbling_2024}.

% differences: also more in ~\citep{wang_stumbling_2024}
Despite the advances, there are still some statistical differences on \ac{llm} generated and human authored texts.
\ac{llm} generated text lacks lexical diversity, overuses certain adjectives ("commendable", "innovative", "meticulous", etc.) and produces longer, more complex sentences.
Moreover, \acp{llm} possess stylistic fingerprints and memorize patterns from the training data.
% lengths
Word length averages and distributions across genre differ for \acp{llm} and humans~\citep{llm_detection_av_2025}.

% future of LLMs
However, as \acp{llm} advance, basic heuristics applied by human detectors no longer suffice.
\acp{llm} will become more human-like and thus, \ac{llm} detection will increasingly resemble a human authorship classification task~\citep{llm_detection_av_2025}.


\subsection{Shortcomings of \acp{lm}}
\label{sec:topic_confusion_shortcomings_lm}
\citet{altakrori_topic_2021} suspect that due to the nature of \acp{lm}, where words of similar meaning, i.e. "color" and "colour", 
are mapped to a similar vector, they do not work well in \ac{aa} tasks.
In \ac{aa} tasks, language system differences are highly relevant since they reveal the author's identity.  