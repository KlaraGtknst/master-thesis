\section{Technical Background}

This section outlines the \acl{ml} principles and paradigms that underpin modern authorship analysis. 
We first introduce the main classification concepts relevant to \ac{av}. 
We then discuss training and evaluation practices, including domain shift scenarios. 
Finally, we present the principal categories of authorship analysis.

\subsection{\acl{ml} Principles}

\ac{aa} tasks are conventionally formulated as classification problems, where the objective is to attribute an anonymous text to one of a set of candidate authors. 
Unlike regression, which estimates a continuous value, classification yields a discrete label corresponding to the author’s identity.

\paragraph{Closed- vs. open-set classification.} 
In a closed-set scenario, the true author is guaranteed to be among the candidate set~\citep{koppel_authorship_2011}. 
In contrast, open-set classification acknowledges that the author of a disputed document may not belong to the candidate set~\citep{stamatatos_survey_2009}. 

\paragraph{One-class classification.} 
In some cases, training data is available for only a single class, and the task is to decide whether a new sample belongs to this class.
If counterexamples, i.e. so-called outliers, are available, they are usually not considered to be representative of non-target class. 
This is formalized as one-class classification, where the model learns the characteristics of the target class without reliable counterexamples~\citep{stein_intrinsic_2011,koppel_authorship_2004}.

\paragraph{Multi-class classification.} 
In the context of \ac{aa}, multi-class classification refers to the task of discriminating among a large number of candidate authors. 
This setting is particularly challenging because the class distribution is often highly imbalanced, with some authors being represented by many texts while others are represented by only a few~\citep{stamatatos_survey_2009,koppel_authorship_2004,elmanarelbouanani_authorship_2014}. 


\subsection{Training and Testing}

Models are typically trained on one portion of the data (training set), tuned on another (validation set), and evaluated on a disjoint partition (test set). 
Any overlap between these partitions constitutes data leakage and invalidates the results~\citep{bischoff_importance_2020,altakrori_topic_2021,boenninghoff_o2d2_2021}. 

A major challenge in stylometry is covariate shift, i.e., a mismatch between the distribution of training and test data. 
This often arises from topic variability~\citep{boenninghoff_o2d2_2021}. 
A common evaluation setting is cross-domain attribution, where training and test texts differ in topic or genre.
Variants include cross-topic and cross-genre settings~\citep{barlas_cross_domain_2020}.  


\paragraph{Supervised vs. unsupervised learning.}  
Supervised methods require labelled training data. 
Examples include classifiers such as \acp{svm}, decision trees, \acp{nn}, and linear discriminant analysis. 
\acp{svm} are particularly common in authorship analysis due to their robustness. 
Unsupervised methods do not rely on labels.
Clustering techniques or \ac{pca} have been used to uncover latent stylistic patterns or to reduce feature dimensionality~\citep{abbasi_writeprints_2008}.


\subsection{Authorship Analysis Methods}
\label{subsec:attribution_methods}

Approaches to authorship analysis can be grouped into three families~\citep{stamatatos_survey_2009}:

\paragraph{Profile-based methods.} 
All training texts of an author are concatenated into a single profile, from which a cumulative feature representation is extracted. 
This approach is effective when only short texts are available.
Profile-based methods ignore intra-author variation~\citep{stamatatos_survey_2009,elmanarelbouanani_authorship_2014,neal_surveying_2018}.  

\paragraph{Instance-based methods.} 
Here, each training text is treated as a separate instance of the author's style. 
This allows models to capture intra-author variability~\citep{stamatatos_survey_2009,altakrori_topic_2021,elmanarelbouanani_authorship_2014,neal_surveying_2018}.  

\paragraph{Hybrid methods.} 
Hybrid approaches combine both paradigms by representing texts individually while aggregating author profiles through feature-wise averages computed over an author’s texts~\citep{stamatatos_survey_2009}. 
