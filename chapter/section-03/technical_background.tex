\section{Technical Background}
\label{sec:technical_background}

This section outlines the \ac{ml} principles and paradigms that underpin modern \ac{av}. 
We first introduce the main classification concepts relevant to \ac{av}. 
We then discuss training and evaluation practices, including domain shift scenarios. 
Finally, we present the principal categories of authorship analysis.

\subsection{\acl{ml} Principles}

\ac{aa} tasks are conventionally formulated as classification problems, where the objective is to attribute an anonymous text to one of a set of candidate authors. 
Unlike regression, which estimates a continuous value, classification yields a discrete label corresponding to the author’s identity.

\paragraph{Closed- vs.\ open-set classification.} 
In a closed-set scenario, the true author is guaranteed to be among the candidate set~\citep{koppel_authorship_2011}. 
In contrast, open-set classification acknowledges that the author of a disputed document may not belong to the candidate set~\citep{stamatatos_survey_2009}. 

\paragraph{One-class classification.} 
When training data is available for only a single class, the task is to determine whether a new sample belongs to it. 
Counterexamples, or so-called outliers, if present, are typically not representative of a non-target class. 
This is formalised as one-class classification, where the model learns the target class characteristics without reliable counterexamples~\citep{stein_intrinsic_2011,koppel_authorship_2004}, as in \ac{av}.

\paragraph{Multi-class classification.} 
In the context of \ac{aa}, multi-class classification refers to the task of discriminating among a large number of candidate authors. 
This setting is particularly challenging because the class distribution is often highly imbalanced, with some authors being represented by many texts while others are represented by only a few~\citep{stamatatos_survey_2009,koppel_authorship_2004}. 


\subsection{Training and Testing}

Models are typically trained on one portion of the data (training set), tuned on a separate portion (validation set), and evaluated on an independent partition (test set). 
Any overlap between these partitions constitutes data leakage and invalidates the results~\citep{bischoff_importance_2020,altakrori_topic_2021,boenninghoff_o2d2_2021}. 

\paragraph{Covariate shift.} % epistemic/ model uncertainty -> reduce by more training data cf. O2O2 paper
A major challenge in stylometry is covariate shift, i.e.\ mismatched data distributions often caused by topic variability across datasets \citep{boenninghoff_o2d2_2021}. 
For instance, a model trained on the word-length distribution of a sports newspaper corpus may perform poorly when applied to a scientific newspaper corpus, as scientific vocabulary tends to comprise longer words, rendering thresholds learned on the original corpus ineffective. 
To assess robustness under such conditions, cross-domain attribution is performed using training and test texts that differ in topic or genre~\citep{barlas_cross_domain_2020}.

\paragraph{Supervised vs. unsupervised learning.}  
Supervised methods require labelled training data. 
Examples include classifiers such as \acp{svc}, decision trees, \acp{nn}, and linear discriminant analysis. 
\acp{svc} are particularly common in authorship analysis due to their robustness. 
Unsupervised methods do not rely on labels.
Clustering techniques or \acl{pca} have been used to uncover latent stylistic patterns or to reduce feature dimensionality~\citep{abbasi_writeprints_2008}.


\subsection{Authorship Analysis Methods}
\label{subsec:attribution_methods}

Approaches to authorship analysis can be grouped into three families~\citep{stamatatos_survey_2009}.

\paragraph{Profile-based methods.} 
All training texts of an author are concatenated into a single profile, from which a cumulative feature representation is extracted. 
This approach is effective when only short texts are available.
Profile-based methods ignore intra-author variation~\citep{stamatatos_survey_2009,elmanarelbouanani_authorship_2014,neal_surveying_2018}.
Hence, only general properties of an author are captured while differences between texts of that author are discarded~\citep{elmanarelbouanani_authorship_2014}, leading to the loss of style changes over time or across topics.

\paragraph{Instance-based methods.} 
Here, each training text is treated as a separate instance of the author's style. 
This allows models to capture intra-author variation~\citep{stamatatos_survey_2009,altakrori_topic_2021,elmanarelbouanani_authorship_2014,neal_surveying_2018}.  

\paragraph{Hybrid methods.} 
Hybrid approaches combine both paradigms by representing texts individually while aggregating author profiles through feature-wise averages computed over an author’s texts~\citep{stamatatos_survey_2009}. 
