\subsection{Dataset}
\label{subsec:dataset}

\subsubsection{Original Data}
Due to this approach extending the original imposter approach by \citet{koppel_determining_2014}, 
we first use original data to establish the feasibility and reproducibility of the original approach. 
\citet{koppel_determining_2014} used \dataBlog{} and the \dataStudent{} dataset.
Unfortunately, after contacting M. Koppel and J. W. Pennebaker, the \dataStudent{} dataset could not be obtained.
For establishing a baseline for the imposter approach, we only use the \dataBlog{} dataset.

\subsubsection{Additional Data}
To extend our test scenario for the imposter approach, we opted to find an additional dataset that fulfilled the following criteria:
\begin{itemize}
    \item controlled confounders (e.g. genre, topic)
    \item undisputed authorship 
    
\end{itemize}

We found that both the \dataPan{} and \dataGutenberg{} datasets are suitable candidates.
The statistical properties of the datasets are shown in \autoref{tab:data_stats}.

\begin{table}[h]
\centering\small
\caption{Statistics of preprocessed datasets \dataPan{}, \dataBlog{} and \dataGutenberg{}.}
\label{tab:data_stats}
\resizebox{\textwidth}{!}{%
\begin{tabular}{@{}lrrrrrrrrr@{}}   % numbers should be right aligned, text left aligned
\toprule
dataset & num\_pairs & num\_authors & num\_same\_pairs & num\_different\_pairs & avg\_text\_len & min\_text\_len & max\_text\_len & std\_text\_len & median\_text\_len \\ 
\midrule
pan20     & 66912 & 52778 & 35620 & 31292 & 21435.53  & 20670 & 296887  & 2685.49   & 21284.00  \\ 
blog      & 64771 & 18961 & 33639 & 31132 & 1846.20   & 501   & 615409  & 3461.74   & 1244.00   \\ 
gutenberg & 36    & 4     & 15    & 21    & 293840.11 & 14092 & 1176438 & 339991.16 & 139370.00 \\ 
\bottomrule
\end{tabular}%
}
\end{table}
\textcolor{red}{above len in characters, but I have also stats for words‚}


\subsubsection{Dataset Preprocessing}
\label{subsubsec:dataset_preprocessing}

To further control confounders influencing authorial style, we preprocess the dataset 
(once before creating the arrow dataset file and once before using the detector).
The requirements for the preprocessing are:
\begin{itemize}
    \item The texts are stripped of all format/ layout information to obtain plain text \textcolor{red}{TODO: before saving them as arrow files}.
    \item The texts should be of similar length (detector crops texts to the length of the shorter text).
\end{itemize}
In order to produce a controlled testing environment four our imposter approach, 
we opted to refine few datasets rather than scaling up to larger datasets.
Removing layout information includes removing HTML artefacts, play artefacts, newlines, 
converting utf-8 to ASCII, lowercasing and stripping leading and trailing whitespace.

\subsubsection{Selection of Text Pairs}
\label{subsubsec:dataset_text_pair_selection}

We had to select pairs of texts for the \dataBlog{} and the \dataGutenberg{} dataset.
We decided to keep the existing pairs in the \dataPan{} arrow dataset for better comparability.
All datasets consist of same- and different-author pairs. 
As mentioned before, we aimed to control confounders when selecting pairs.

For the \dataBlog{} dataset, 
two texts of a pair are selected such that they share the same topic, year, gender and age, where the last to reference the text's author.
Test (20\%) and train split (80\%) have different topics.

For the \dataGutenberg{} dataset,
we selected pairs of texts that share the same genre and century.
Authors can either be in the train (80\%) or test (20\%) set.

Irrespective of the information used to select pairs, the final dataset contains only the columns \texttt{authors}, \texttt{pair}, and \texttt{same}.
The \texttt{pair} column contains the texts of the pair as a list of strings,
the \texttt{authors} column contains the authors of the texts as a list of strings,
and the \texttt{same} column indicates whether the texts are from the same author (\texttt{True}) or from different authors (\texttt{False}).
