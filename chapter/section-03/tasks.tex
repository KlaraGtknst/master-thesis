\section{Tasks}

In this section we outline the main tasks of authorship analysis. 
We begin with the foundational problems of \ac{aa} and \ac{av}, before turning to derivative tasks such as \ac{llm} detection.

\subsection{Authorship Attribution}
\ac{aa} is the classical multi-class, single-label text classification task in which, given a set of candidate authors $A$ with known texts $K=\bigcup_{a\in A} K_a$, the goal is to determine which author wrote a disputed text $d \in U$, where $U$ is the set of texts of unknown authorship~\citep{koppel_authorship_2004,barlas_cross_domain_2020}. 
In the closed-set setting, each $d \in U$ must be attributed to exactly one $a \in A$. 
Variants such as cross-topic or cross-genre \ac{aa} introduce distributional shifts between training and test data, complicating the task~\citep{barlas_cross_domain_2020}.

\subsection{Authorship Verification}
\ac{av} addresses the problem of establishing whether a given text $d$ was written by a candidate author $a$, using a set of the author’s known writings $K_a$ as reference~\citep{koppel_authorship_2004}.
We denote disputed-candidate pair the input of the text of unknown authorship and the candidate author's text in the following.
In contrast to \ac{aa}, \ac{av} does not have reliable negative examples, since assembling a representative sample of all texts not authored by 
$a$ is impossible. 
This limitation makes \ac{av} a more challenging classification problem than \ac{aa}~\citep{llm_detection_av_2025,neal_surveying_2018,koppel_authorship_2004}.
\ac{av} is framed as a one-class, a binary, or a similarity detection task depending on the methodological perspective~\citep{neal_surveying_2018,koppel_authorship_2004}.  
\ac{aa} can be reduced to a series of \ac{av} problems, where the other direction is typically not true~\citep{barlas_cross_domain_2020,tyo_state_2022}.
% Gespräch Martin Potthast 19.05.2025: problem formulation 2 is less common and in the context of very sparse (metadata) information:
% This task can also be formulated as whether two texts $t_1$ and $t_2$ are written by the same author 
% \citep{bevendorff_generalizing_2019,bevendorff_divergence_based_2020,embarcadero_ruiz_graph_based_2022,rivera_soto_learning_2021,ordonez_will_2020,futrzynski_pairwise_2021,weerasinghe_feature_vector_difference_2021,llm_detection_av_2025}.


% \subsection{Author Profiling}
% Author profiling infers sociolinguistic attributes of an author, such as age, gender, education, or mental health, from a set of texts. 
% It is grounded in the assumption that subconscious idiosyncrasies of writing style encode personal traits~\citep{emmery_adversarial_2021,stamatatos_survey_2009,elmanarelbouanani_authorship_2014}. 
% Profiling raises substantial ethical and privacy concerns due to the sensitivity of the inferred attributes.

% \subsection{Author Obfuscation}
% Author obfuscation is an adversarial task in which a text is deliberately modified to conceal the author's identity while preserving its semantic content. 
% It directly opposes \ac{aa} and \ac{av} by aiming to neutralize stylometric features~\citep{bischoff_importance_2020,bevendorff_divergence_based_2020,gohsen_task_oriented_2024}. 

% \subsection{Intrinsic Plagiarism Detection}   % style change detection
% Plagiarism detection is the task of identifying reused or unattributed content in texts~\citep{stein_intrinsic_2011,gohsen_task_oriented_2024}. 
% Unlike \ac{aa}, the goal is not to establish authorship but to uncover overlap between documents, regardless of author identity~\citep{elmanarelbouanani_authorship_2014}.
% Hence, plagiarism detection is an application of \ac{av}~\citep{rivera_soto_learning_2021}.


\subsection{\acs{llm} Detection}
Given the ability of \acp{llm} to generate text closely resembling human writing, we can conceptualize \acp{llm} as one or multiple distinct authors.
Since there is not \textit{the one} \ac{llm}, but a great variety of \acp{llm}, one may find denoting \acp{llm} an author deceptive.
This implicates individual authorships for different \acp{llm}.
This perspective allows us to frame \ac{llm} detection as an \ac{av} task, where given two texts (i.e. one of unknown authorship and one known to be generated by an \ac{llm} author) the objective is to determine whether they share the same author~\citep{llm_detection_av_2025}.


