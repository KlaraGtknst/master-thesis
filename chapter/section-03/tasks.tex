\subsection{Tasks}

In the following we will outline the most prominent tasks of \ai{} or authorship analyse.
We will start with author profiling, resuming with author obfuscation, \ac{aa}, plagiarism detection and finally conclude with \ac{av}.

% author profiling
Building upon the foundation of stylometry which states that subconscious authorial idiosyncrasies are encoded in author's writing, there is plenty of information to extract from a given text.
Formally, inferring an extensive set of personal information from a collection of texts is denoted author profiling.
Extracted information may include sociolinguistic attributes like age, gender, occupation, education, socio-economic status, cultural background, language familiarity and mental health issues~\citep{emmery_adversarial_2021,stamatatos_survey_2009,elmanarelbouanani_authorship_2014}.
% The task is also referred to as author characterization \citep{stamatatos_survey_2009,elmanarelbouanani_authorship_2014}.
% Author obfuscation
Since the extracted information can contain sensitive personal details, there are efforts of altering texts to render an author's style imperceptible~\citep{bischoff_importance_2020}.
This process is called author obfuscation.
It is an adversarial task of preventing successful verification of the author by altering the text's style so that it no longer resembles the original author's style \citep{bevendorff_divergence_based_2020,gohsen_task_oriented_2024}.

% authorship attribution
What the \ai{} pioneers of the \nth{19} century did when they tried to answer who among Francis Bacon, Christopher Marlowe, and William Shakespeare was the true author of the Shakespearean plays is called \ac{aa}.

\begin{definition}
    [\acl{aa}]   
    \ac{aa} is the task of determining the author of a text given a set of candidate authors with undisputed writing samples.
    % \citep{stein_intrinsic_2011,koppel_authorship_2004,stamatatos_survey_2009,tyo_state_2022,bischoff_importance_2020,barlas_cross_domain_2020,altakrori_topic_2021,bevendorff_divergence_based_2020,elmanarelbouanani_authorship_2014,abbasi_writeprints_2008,llm_detection_av_2025,neal_surveying_2018}
    \textcolor{red}{TODO: too many references}
\end{definition}

\ac{aa} is a multiclass, single-label text classification task~\citep{stamatatos_survey_2009,koppel_authorship_2004,elmanarelbouanani_authorship_2014}.
Formally, \ac{aa} is defined as a tuple $(A,K,U)$, where $A$ is the set of authors, $K=\underset{a\in A}{\cup}K_a$ is the set of known texts and $U$ is the set of unknown texts.
For closed-set \ac{aa}, each text $d \in U$ is attributed to exactly one author $a \in A$.
In cross-topic or cross-genre scenarios, the topic or genre of documents in $d \in U$ is different from the topics or genres found in $K$~\citep{barlas_cross_domain_2020}. 

% The decision is made based on stylistic traits rather than the content of the document \citep{neal_surveying_2018}.

% The task is also referred to as author(ship) identification \citep{stamatatos_survey_2009,elmanarelbouanani_authorship_2014}.

% \citet{elmanarelbouanani_authorship_2014} describe the workflow of \ac{aa} as follows:
% \begin{enumerate}
%     \item Data cleaning
%     \item Feature extraction
%     \item Normalization
%     \item Converting each text into a feature vector, where author is the class label
%     \item Split the dataset into training and test set
% \end{enumerate}
% Common classifiers include \ac{svm}, decision trees, and \acp{nn} \citep{elmanarelbouanani_authorship_2014}.

% Plagiarism detection
More prominent than ever, with probabilistic \acp{lm} generating texts based on existing texts from their training data, we have to deal with plagiarism.
Plagiarism is the appropriation of another author's information, language, ideas, results or writing without properly acknowledging the original source~\citep{stein_intrinsic_2011,gohsen_task_oriented_2024}.
% \citet{elmanarelbouanani_authorship_2014} define plagiarism as the complete or partial replication 
% of a piece of work with or without permission of the original author.
Plagiarism detection is the task of identifying plagiarized text~\citep{stein_intrinsic_2011}.
% It includes finding similarities between two texts \citep{stamatatos_survey_2009}.
Different to \ac{aa}, it is not important identifying the author's identity but rather whether multiple pieces of work were produced by a single author~\citep{elmanarelbouanani_authorship_2014}.
Hence, plagiarism detection is an application of \ac{av}~\citep{rivera_soto_learning_2021}.

% authorship verification
\begin{definition}
    [\acl{av}]   
    Given a set of writing samples $K_a$ of author $A$ and a text $t$, the task is to determine whether $t$ was written by $A$.
    \textcolor{red}{TODO: too many references}
    % ~\citep{stein_intrinsic_2011,stamatatos_survey_2009,koppel_authorship_2011,tyo_state_2022,kocher_unine_2015,koppel_authorship_2004}.
\end{definition}

% Gespräch Martin Potthast 19.05.2025: problem formulation 2 is less common and in the context of very sparse (metadata) information:
% This task can also be formulated as whether two texts $t_1$ and $t_2$ are written by the same author 
% \citep{bevendorff_generalizing_2019,bevendorff_divergence_based_2020,embarcadero_ruiz_graph_based_2022,rivera_soto_learning_2021,ordonez_will_2020,futrzynski_pairwise_2021,weerasinghe_feature_vector_difference_2021,llm_detection_av_2025}.

To solve the \ac{av} problem, the disputed document $t$ is compared to documents from the candidate $K_a$.
Different to other one-class classification problems, this approach disregards negative samples, i.e. all texts not written by the known author.
In fact, it is neither possible to assemble an exhaustive, nor representative set of samples of the non-target class.
Hence, \ac{av} is a more general and difficult classification problem than \ac{aa}~\citep{llm_detection_av_2025,neal_surveying_2018,koppel_authorship_2004}.
It is evident that \ac{av} is the fundamental problem of \ac{aa}, where there is only one candidate author~\citep{barlas_cross_domain_2020,tyo_state_2022}.

The \ac{av} problem can be framed to fit multiple problem categorizations.
When considering its output \textit{same-author} or \textit{different-author}, we can view \ac{av} a binary classification problem, where \textit{different-author} renders \ac{av} as an open-set problem.
% ~\citep{neal_surveying_2018}.
% However, \citet{neal_surveying_2018} consider framing \ac{av} as a one-class classification problem as a common approach (cf. \citep{llm_detection_av_2025}).
% \citet{elmanarelbouanani_authorship_2014} consider \ac{av} a similarity detection task.
The \impAppr{}, among other approaches, generates more candidates during inference in order to obtain their final prediction.
Hence, such approaches artificially create an \ac{aa} problem to solve the original problem~\citep{neal_surveying_2018}.

