% \section{First Approaches}

% Approaches of the \nth{19} century were limited to the analysis of word lengths.
The \nth{20} century brought new measures, such as Zipf's law (1932) and Yule's Characteristic (1944).
Zipf's law describes the relationship between rank and frequency of words.
Yule's Characteristic was proposed by George Yule as a first attempt to measure vocabulary richness via word frequencies~\citep{neal_surveying_2018,stamatatos_survey_2009}.

A massive milestone was proposed in 1964 by Mosteller and Wallace with their computer-assisted stylometry on "The Federalist Papers" from the \nth{18} century.
Their discriminations were built on a Bayesian statistical analysis of the frequencies of a small set of common words such as "and" or "to".
This approach instigated stylometry, the research of finding features that quantify writing style.
In the peak years of stylometry between 1964 and the late 1990s, around \num{1000} different measurements were proposed.
Such measures include sentence or words lengths, word or character frequencies, and vocabulary richness.
The main shortcomings of stylometry in the \nth{20} century included the lack of method comparison due to the absence of suitable benchmark data and the mostly subjective evaluation of individual methods via visual inspection of scatterplots~\citep{stamatatos_survey_2009}.

The late 1990s initiated a new era of \ai{}.
The growing amount of electronic texts and benchmark data marked significant progress towards objective comparative evaluation of methods.
Moreover, \ac{ml} algorithms allowed for more expressive text representations than before. 
The focus shifted from answering disputed literary questions to solving real-world problems and creating practical applications~\citep{stamatatos_survey_2009}.
According to \citet{abbasi_writeprints_2008}, until 2008 stylometry not only lacked of scalability in terms of number of authors and across genres, but also was applied exclusively to closed-world scenarios.


% % potential introduction to style chapter
% \citet{bischoff_importance_2020} assume that each author has a unique style, unconsciously encoded in their writing.
% This style depends on the author's personal traits, customs an author adopts due to genre, register, type, and topic.
% These concepts are too vague and ill-separable to be efficiently operationalized.
% The goal is to discover a set of style markers more likely to be determined by the author's personality than by domain customs.

% They claim that features frequent function words and word length have a high correlation with topic. \todo{????}
% Hence, traditional author style models are highly susceptible to learning domain-specific features and 
% prone to pick up domain artefacts unless domains are controlled, 
% which imposes severe practical limitations.

% % char 3-gram
% \citet{bischoff_importance_2020} analyse the robustness of character trigrams as a feature for \ac{aa}.
% They find that the character trigrams feature set is not robust in a cross-topic setting, but across two genres.


