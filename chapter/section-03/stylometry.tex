% \subsection{Stylometry}

\begin{definition}
    [Stylometry]
    Stylometry refers to a linguistic research area, where literary style is quantified by statistical features.
\end{definition}
% In other words, stylometry is the statistical analysis of literary style between one writer or genre and another \citep{tyo_state_2022}.
Researchers working on stylometric features believe that subconscious authorial syntactic idiosyncrasies are quantifiably measurable and sufficient to define an author's unique style~\citep{neal_surveying_2018}. 
These stylometric features are also referred to as style markers, or stylistic features if they are the most effective discriminators of authorship. 
Static features, such as function words, word-length distributions, vocabulary richness measures, are context-free.
Dynamic features are context-dependent attributes and include $n$-grams and misspelled words~\citep{abbasi_writeprints_2008}.
% Stylometric features include lexical, syntactic and structural features \citep{stein_intrinsic_2011}.
% Research includes five subtasks \citep{neal_surveying_2018}:
% \begin{itemize}
%     \item \ac{aa}
%     \item \ac{av}
%     \item Author profiling
%     \item Stylochronometry
%     \item adversarial stylometry
% \end{itemize}
Generally speaking, the more detailed the style marker extraction process, the more noisy are the produced measures~\citep{stamatatos_survey_2009}.
We outline prominent style marker taxonomy categories from \autoref{tab:stylometric_features} in the following. 

\begin{table}[]
    \centering
    \caption{Incomplete taxonomy of style markers from \citep{stamatatos_survey_2009}.}
    \label{tab:stylometric_features}
 
    \begin{tabular}{@{}ll@{}} % numbers should be right aligned, text left aligned
    \toprule
    \textbf{Category} & \textbf{Features} \\ 
    \midrule
    Lexical & Token-based \\ %(word/ sentence length, ...) \\
     & Vocabulary richness  \\
     & Word frequencies  \\
     & Word n-grams  \\
     & Errors \\
    %  \midrule
    % Character & Character types (letters, digits, ...)\\
    %  & Character n-grams (fixed length)  \\
    %  & Character n-grams (variable length) \\
    %  & Compression methods \\
    %  \midrule
    Syntactic & Part-of-Speech (POS)  \\
    %  & Chunks \\
    %  & Sentence and phrase structure  \\
    %  & Rewrite rule frequencies \\
     & Errors  \\
    %  \midrule
    Semantic & Synonyms \\
     & Semantic dependencies \\
    %  & Functional  \\
    %  \midrule
    Application-specific & Structural  \\
     & Content-specific\\
    %  & Language-specific \\
     \bottomrule
    \end{tabular}%

\end{table}

% lexical
Lexical features treat text as a mere sequence tokens~\citep{stamatatos_survey_2009}.
Token units include character, word, and sentence.
In \autoref{tab:comp_lexical}, we present characteristics of the character and word feature units.
Character (n-gram) counts, word lengths~\citep{stein_intrinsic_2011}, sentence lengths~\citep{stein_intrinsic_2011,abbasi_writeprints_2008}, are examples for character, word, and sentence unit-based lexical features.
% line length~\citep{abbasi_writeprints_2008}, word length distribution~\citep{abbasi_writeprints_2008}, 
% vocabulary richness~\citep{abbasi_writeprints_2008,neal_surveying_2018} ...) 
Errors in \autoref{tab:stylometric_features} refers to idiosyncratic features, which include misspellings, grammatical mistakes, and other usage anomalies~\citep{abbasi_writeprints_2008,neal_surveying_2018}.


\begin{table}[]
\centering
\caption{Comparison of a subset of lexical features~\citep{stamatatos_survey_2009}. Requirements refer to computational requirements such as a tokenizer.}
\label{tab:comp_lexical}
\resizebox{\textwidth}{!}{%
\begin{tabular}{lllll}
    \toprule
    \textbf{Unit} & \textbf{Complexity} & \textbf{Language-independent} & \textbf{Noise-sensitivity} & \textbf{Requirements} \\
    \midrule
    character & low & \checkmark & low & \xmark  \\
    word & high & \xmark  & high  & \checkmark    \\
    \bottomrule
    \end{tabular}%
}
\end{table}

% syntactic
Syntax or syntactic structure is the structural organization of sentences \citep{kurt_pehlivanoglu_comparative_2024}.
Given robust \ac{nlp} tools, syntactic features are considered more reliable than lexical features.
Well-established syntactic features built on function words, or on syntactic errors such as mismatched tense or sentence fragments, or use morpho-syntactic \ac{pos} tags for each word token for \ac{pos} tag (n-gram) frequencies~\citep{abbasi_writeprints_2008,stamatatos_survey_2009}.
It is evident, that every syntactic feature requires parsing or processing of natural language and is thus, language-dependent~\citep{neal_surveying_2018,stamatatos_survey_2009}.

% semantic
Semantic features capture meaning behind words, phrases, and sentences, such as through analysis of synonyms and semantic dependencies \citep{neal_surveying_2018}.
Semantic similarities between words, synonym or hyponym relationships are derived using WordNet, any thesaurus or latent semantic analysis.
Semantic dependencies such as the specification of a person with a location, can boost classification performance combined with lexical and syntactic information~\cite{stamatatos_survey_2009}.


% application specific (structural + content)
Application-specific features vary by domain.
E-mails may include structural elements such as greetings absent in, for instance, e-commerce platform's item descriptions or source code.
More general structural features include layout, file extensions, font, sizes, and colours~\citep{abbasi_writeprints_2008,neal_surveying_2018}.
Given texts belonging to a distinct genre and topic, frequent words are a content-specific feature.
Different to syntactic features built on function words, frequent words carry semantic information.
It is important to note that frequent words features should not be used in cross-topic or cross-genre scenarios and that there is no instruction on how to choose of such words~\citep{abbasi_writeprints_2008}.
% Domain-specific features include ratios of quoted words and external links, number of paragraphs, and paragraphs average length for the news article domain~\citep{potthast_stylometric_2018}
