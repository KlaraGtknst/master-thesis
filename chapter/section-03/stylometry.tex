% \subsection{Stylometry}

\begin{definition}
    [Stylometry]
    Stylometry is the quantitative analysis of linguistic style through measurable textual features, with the aim of distinguishing between authors.
\end{definition}
The central assumption of stylometry is that authors unconsciously leave behind consistent idiosyncratic syntactic traces, so-called style markers, that can be exploited for \ai tasks~\citep{neal_surveying_2018,bischoff_importance_2020}. 
These traces are presumed to reflect relatively stable personal traits and stylistic habits rather than superficial properties imposed by genre, register, or topic. 
The practical challenge, therefore, lies in identifying markers that are more strongly determined by authorial traits than by domain conventions.

Stylometric features are commonly grouped into several broad categories~\citep{stamatatos_survey_2009}, outlined in \autoref{tab:stylometric_features}. 
They range from low-level lexical and character-based statistics to high-level semantic and application-specific indicators.
High-level features tend to be prone to noise since they rely on processing tools.

% Static features, such as function words, word-length distributions, vocabulary richness measures, are context-free.
% Dynamic features are context-dependent attributes and include $n$-grams and misspelled words~\citep{abbasi_writeprints_2008}.


\begin{table}[]
    \centering
    \caption{Incomplete taxonomy of style markers from \citep{stamatatos_survey_2009}.}
    \label{tab:stylometric_features}
 
    \begin{tabular}{@{}ll@{}} % numbers should be right aligned, text left aligned
    \toprule
    \textbf{Category} & \textbf{Features} \\ 
    \midrule
    Lexical & Token-based \\ %(word/ sentence length, ...) \\
     & Vocabulary richness  \\
     & Word frequencies  \\
     & Word n-grams  \\
     & Errors \\
    %  \midrule
    % Character & Character types (letters, digits, ...)\\
    %  & Character n-grams (fixed length)  \\
    %  & Character n-grams (variable length) \\
    %  & Compression methods \\
    %  \midrule
    Syntactic & Part-of-Speech (POS)  \\
    %  & Chunks \\
    %  & Sentence and phrase structure  \\
    %  & Rewrite rule frequencies \\
     & Errors  \\
    %  \midrule
    Semantic & Synonyms \\
     & Semantic dependencies \\
    %  & Functional  \\
    %  \midrule
    Application-specific & Structural  \\
     & Content-specific\\
    %  & Language-specific \\
     \bottomrule
    \end{tabular}%

\end{table}

\paragraph{Lexical features.} 
Lexical features treats text as a sequence of tokens, such as characters, words, or sentences~\citep{stamatatos_survey_2009}. 
Distributions over character $n$-grams counts, word-length~\citep{stein_intrinsic_2011}, and sentence-length~\citep{stein_intrinsic_2011,abbasi_writeprints_2008} are classic for character, word, and sentence unit-based lexical features, respectively. 
Character-level features are computationally inexpensive, language-independent, and relatively robust to noise, whereas word-level features require tokenization, are language-dependent, and more sensitive to noise~\citep{stamatatos_survey_2009}.
Idiosyncratic spelling or grammatical mistakes are also considered discriminative lexical features~\citep{abbasi_writeprints_2008,neal_surveying_2018}. 

\paragraph{Syntactic features.}
Syntactic features exploit structural information such as \ac{pos} $n$-grams, function words, or recurrent syntactic errors~\citep{stamatatos_survey_2009,abbasi_writeprints_2008}.
% ,kurt_pehlivanoglu_comparative_2024}. defines syntax
They are generally more reliable than lexical features but require robust \ac{nlp} pipelines, making them computationally more expensive and inherently language-dependent~\citep{neal_surveying_2018,stamatatos_survey_2009}.

\paragraph{Semantic features.}
Semantic features, such as synonym or hyponym relations, or semantic dependencies, can complement lexical and syntactic features. 
For example, a semantic dependency between a person and a location may provide additional discriminatory power~\citep{stamatatos_survey_2009,neal_surveying_2018}. 
However, semantic features depend on external resources (e.g., WordNet, a lexical database of semantic relations~\citep{zhou_paraphrase_2025}).


\paragraph{Application-specific features.} % (structural + content)
General structural features include layout, file extensions, font, sizes, and colours~\citep{abbasi_writeprints_2008,neal_surveying_2018}.
Certain features are only meaningful in specific domains. 
For instance, e-mails often contain characteristic salutations and signatures, while source code contains structural conventions specific to the programming language. 
Content-specific frequent words may be useful within a single genre, but they should not be used across topics or genres since they conflate authorial idiosyncrasies with semantic information~\citep{abbasi_writeprints_2008}.
% Domain-specific features include ratios of quoted words and external links, number of paragraphs, and paragraphs average length for the news article domain~\citep{potthast_stylometric_2018}


In summary, stylometry rests on the delicate balance between extracting informative style markers and minimizing confounding influences from genre, register, or topic. 
While lexical features remain the most widely used due to their simplicity, more advanced syntactic and semantic representations, as well as domain-specific markers, can be crucial for tackling diverse real-world scenarios.
