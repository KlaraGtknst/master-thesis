\subsection{Traditional Quantitative Paraphrase Evaluation Measures}
\label{subsec:traditional_quantitative_evaluation_measures}

The evaluation of paraphrases can be broadly divided into syntactic and semantic approaches~\citep{gohsen_captions_2023}. 
Syntactic metrics primarily assess measure n-gram overlaps between a candidate text and reference texts~\citep{zhou_paraphrase_2021}. 
By contrast, semantic metrics aim to capture equivalence of meaning. 
Combining both perspectives allows identification of good paraphrases which balance syntactic diversity and semantic similarity.

% \citet{gohsen_captions_2023} normalized all metrics and averaged the semantic and syntactic scores separately.

\subsubsection{Syntactic Measures}
Syntactic evaluation metrics mainly focus on the n-gram overlaps~\citet{zhou_paraphrase_2021}. 
Common syntactic evaluation metrics include \ac{bleu}, \ac{meteor} \ac{rouge}-1, \ac{rouge}-L.
\textcolor{red}{\citet{kurt_pehlivanoglu_comparative_2024} claims \ac{meteor} is a semantic metric.}

BLEU~\citep{papineni_bleu_2001} was originally developed for machine translation and has since become a standard metric for evaluating paraphrase generation \citep{zhou_paraphrase_2021}. 
BLEU is based on precision, computing the fraction of generated n-grams that appear in any reference text~\citep{kurt_pehlivanoglu_comparative_2024,palivela_optimization_2021,papineni_bleu_2001}. 
To prevent inflated scores due to repetition of frequent tokens (e.g., “the”), BLEU introduces a clipping mechanism that caps the count of n-grams at their maximum reference frequency~\citep{papineni_bleu_2001}. 
Precision for each $n \in \mathbb{N}_{>0}$ is given by \autoref{eq:bleu}.
\ac{bleu}'s basic unit of evaluation is a sentence. 
In order to compute the \ac{bleu} score from \autoref{eq:bleu} for more than one sentence, 
one (1) computes the n-grams matches sentence by sentence, 
then (2) adds the clipped n-grams matches across all sentences, 
and finally (3) divides the total clipped n-grams matches by 
the total number of unclipped n-grams in all candidate sentences \citep{papineni_bleu_2001}.

\begin{equation}
    p_n = \frac{\sum_{\mathcal{C} \in \left\{ Candidates \right\}}\sum_{n-gram \in\mathcal{C}}Count_{clip}(n-gram)}{\sum_{\mathcal{C'} \in \left\{ Candidates \right\}}\sum_{n-gram' \in\mathcal{C'}}Count(n-gram')}
\label{eq:bleu}
\end{equation}

Unigrams are used to test adequacy, while longer n-grams are used to test fluency~\citep{papineni_bleu_2001}.
Combined scores across different n-gram orders are computed via the geometric mean, weighted uniformly across different $n$~\citep{papineni_bleu_2001,banerjee_METEOR_2005}. 
The brevity penalty $BP$ from \autoref{eq:bleu_brevity_penalty} is additionally applied to discourage excessively short candidates~\citep{papineni_bleu_2001}.

\begin{equation}
    BP = \begin{cases}
        1 & \text{if } c > r \\
        e^{1 - \frac{r}{c}} & \text{if } c \leq r
    \end{cases}
\label{eq:bleu_brevity_penalty}
\end{equation}

\autoref{eq:bleu} and \autoref{eq:bleu_brevity_penalty} combined leads to the final score in\autoref{eq:bleu_final}.

\begin{equation}
    \text{BLEU} = BP \cdot \exp\left(\sum_{n=1}^{N} w_n \cdot \log p_n\right)
\label{eq:bleu_final}
\end{equation}

Although BLEU is the most frequently applied metric in paraphrase evaluation, it has well-known shortcomings. 
It disregards semantic similarity completely and therefore judges paraphrases only based on n-gram overlap. 
As such, it is generally recommended being supplemented with human evaluation~\citep{zhou_paraphrase_2021}.

% Its values range from 0 to 1 \citep{papineni_bleu_2001}.

% \ac{bleu} automatically penalizes n-grams appearing in the candidate text but not in the reference text, 
% as well as n-grams appearing more often in the candidate than in the reference text \citep{papineni_bleu_2001}.

% For multiple sentences, they (1) add the best match (among the reference texts) length for each candidate sentence, 
% and (2) divide this sum $r$ by the total length of all candidate sentences $c$. 
% They cannot use recall for length-related problems here, 
% because \ac{bleu} uses multiple reference texts, which may have different lengths \citep{papineni_bleu_2001,banerjee_METEOR_2005}.
% If the generated candidate is significantly shorter than the reference text, the brevity penalty $BP$ is applied.
% A \ac{bleu} score approaching 1 signifies the candidate matches one reference almost exactly \citep{papineni_bleu_2001}, 
% and thus, limited syntactic diversity (i.e. inadequate paraphrase) \citep{kurt_pehlivanoglu_comparative_2024}.
% Note that more reference texts lead to higher \ac{bleu} scores \citep{papineni_bleu_2001}.



METEOR was proposed to address BLEU’s deficiencies. 
Unlike BLEU, which is precision-oriented, METEOR explicitly incorporates recall and includes mechanisms for handling morphological and semantic variation, such as stemming and synonym matching. 
Candidate and reference unigrams are first aligned based on modules (exact match, Porter stemmed match, synonymy), after which the best largest subset of unigram mappings is selected according to cardinality and minimal crossing. 
From this alignment, METEOR computes the weighted $F$-score from \autoref{eq:meteor}~\citep{banerjee_METEOR_2005}.

\begin{equation}
    METEOR = F_{mean} = \frac{10 \cdot P \cdot R}{R + 9P} \cdot (1 - Penalty)
\label{eq:meteor}
\end{equation}

The penalty function discourages fragmented matches and reduces the score to $50\%$ bigram or longer matches are absent~\citep{banerjee_METEOR_2005}. 
METEOR has been shown to correlate more strongly with human judgments than BLEU, particularly at the sentence or segment level, due to its sensitivity to lexical and semantic variation \citep{zhou_paraphrase_2021,kurt_pehlivanoglu_comparative_2024}.
% Its values range from 0 to 1 \citep{kurt_pehlivanoglu_comparative_2024}.
There is penality function whenever an incorrect word is encountered~\citep{palivela_optimization_2021} as displayed in \autoref{eq:meteor}.



\ac{rouge} (2004) is a recall-based metric developed for text summarization 
\citep{zhou_paraphrase_2021,palivela_optimization_2021,kurt_pehlivanoglu_comparative_2024,lin_rouge_2004}.
Its values range from 0 to 1 \citep{kurt_pehlivanoglu_comparative_2024}.
\ac{rouge} can focus on the word variations and diversity.
It has multiple versions, the most popular ones include 
\ac{rouge}-N (computing the n-gram recall) \citep{zhou_paraphrase_2021,palivela_optimization_2021,kurt_pehlivanoglu_comparative_2024}, 
\ac{rouge}-L (computing the longest common subsequence) \citep{zhou_paraphrase_2021,palivela_optimization_2021,kurt_pehlivanoglu_comparative_2024}, 
\ac{rouge}-W (Weighted longest common subsequence) \citep{palivela_optimization_2021}, 
\ac{rouge}-S (skip-bigram co-occurrence statistics) \citep{palivela_optimization_2021}.
\ac{rouge}-1 computes the recall by analysing the matching unigrams between the generated paraphrase and the reference paraphrase \citep{palivela_optimization_2021,kurt_pehlivanoglu_comparative_2024}.
% ROUGE-N
\ac{rouge}-N is an n-gram recall between the candidate text and the reference texts \citep{lin_rouge_2004} as displayed in \autoref{eq:rouge_n}.
\begin{equation}
    ROUGE-N = \frac{\sum_{\mathcal{S} \in \left\{ References \right\}}\sum_{n-gram \in\mathcal{S}}Count_{match}(n-gram)}{\sum_{\mathcal{S'} \in \left\{ References \right\}}\sum_{n-gram' \in\mathcal{S'}}Count(n-gram')}
\label{eq:rouge_n}
\end{equation}
$Count_{match}(n-gram)$ is the maximum number of n-grams co-occuring in the candidate text and the set of reference texts \citep{lin_rouge_2004}.
The nominator sums over all references and thus, gives more weight to matching n-grams that occur in multiple references (i.e. a consensus between references) \citep{lin_rouge_2004}.
Refer to \citet{lin_rouge_2004} for more details on the work with multiple references (I do not understand that, because I thought we already use multiple).
% ROUGE-L
For \ac{rouge}-L, the intuition is that the longer the longest common subsequence (LCS) between the candidate and reference texts, the more similar they are \citep{lin_rouge_2004}.
For a candidate $Y$ of length $n$ and a reference $X$ of length $m$, the \ac{rouge}-L score is defined as follows in \autoref{eq:rouge_l}:
\begin{equation}
    ROUGE-L = F_{lcs} = \frac{(1 + \beta^2)R_{lcs}P_{lcs}}{R_{lcs} + \beta^2 P_{lcs}}
\label{eq:rouge_l}
\end{equation}
where $R_{lcs} = \frac{LCS(X,Y)}{m}$ and $P_{lcs} = \frac{LCS(X,Y)}{n}$ \citep{lin_rouge_2004}.
\ac{rouge}-L requires in-sequence matches that reflect the sentence level word order as n-grams \citep{lin_rouge_2004}.
Moreover, no predefined $n$ is necessary, because \ac{rouge}-L includes the longest in-sequence common n-grams \citep{lin_rouge_2004}.
However, \ac{rouge}-L does not include shorter sequences or alternative LCSes in the final score \citep{lin_rouge_2004}.
% ROUGE-S
A skip-bigram is any pair of words in their sentence order, allowing for arbitrary gaps \citep{lin_rouge_2004}.
\ac{rouge}-S measures the overlap of skip-bigrams between the candidate text and the reference texts \citep{lin_rouge_2004}.
Hence, if the candidate text is the reverse of the reference text, the \ac{rouge}-S score is 0 even though it is not as bad as completely unrelated candidates \citep{lin_rouge_2004}.
\ac{rouge}-SU extends \ac{rouge}-S with unigrams to solve this issue \citep{lin_rouge_2004}.
% ROUGE generally
\citet{kurt_pehlivanoglu_comparative_2024} claim that \ac{rouge} may not be adequate to assess semantic similarity and fluency.
Lower \ac{rouge} scores indicate greater diversity \citep{kurt_pehlivanoglu_comparative_2024}.

Common metrics for semantic similarity include BERTScore, 
cosine similarity of dense vector representations derived from a BERT-based sentence transformer, 
and \ac{wmd} \citep{gohsen_captions_2023}.

BERTScore calculates the cosine similarity between the contextual embeddings of the reference and generated texts. 
Hence, is assesses semantic equivalence and correlates well with human judgment \citep{kurt_pehlivanoglu_comparative_2024}.
First, token vector representations are computed for both the reference and generated texts using a pre-trained BERT model \citep{hanna_fine_grained_2021}.
Let reference $z$ and candidate $\hat{z}$ be the vector representations of the reference and candidate texts, respectively.
Then, the BERTScore precision, recall and $F_1$ score is computed as follows in \autoref{eq:bert_p}, \autoref{eq:bert_r}, and \autoref{eq:bert_f1}, respectively:
\begin{equation}
    P_{BERT} = \frac{1}{|\hat{z}|} \sum_{\hat{z}_j \in \hat{z}} \max_{z_j \in z} z_i\top \hat{z}_j
\label{eq:bert_p}
\end{equation}
\begin{equation}
    R_{BERT} = \frac{1}{|z|} \sum_{z_j \in z} \max_{\hat{z}_j \in \hat{z}} z_i\top \hat{z}_j
\label{eq:bert_r}
\end{equation}
\begin{equation}
    F_1 = \frac{2 P_{BERT} R_{BERT}}{P_{BERT} + R_{BERT}} 
\label{eq:bert_f1}
\end{equation}
Since $F_1 \in \left[-1,1\right]$ it can be rescaled to $[0,1]$ by modifying the precision and recall calculation 
to $\hat{P}_{BERT} = \frac{P_{BERT} - a}{1 - a}$ ($R_{BERT}$ analogous), where $a$ is the empirical lower bound on the BERTScore \citep{hanna_fine_grained_2021}.
% The BERTScore has difficulties on datasets with lexically similar (i.e. lexical overlap of content words) incorrect candidates 
% opposed to lexically different more correct candidates \citep{hanna_fine_grained_2021}.

The \ac{wmd} computes the minimum amount of distance that embedded words of a text need to travel 
to reach the embedded words of another text \citep{gohsen_captions_2023}.

