
\subsection{Paraphrase evaluation}
\label{subsec:paraphrase_evaluation}

Evaluating paraphrases can be reduced to central \ac{nlp} problems such as summarization or translation evaluation.
Yet, there is no universal definition of what constitutes a paraphrase. 
Definitions vary in degree of semantic equivalence required. 
This conceptual ambiguity makes the task of evaluation especially challenging, since different applications may prioritize different aspects such as fidelity to meaning, stylistic variation, or grammatical well-formedness.
Because of this, paraphrase evaluation cannot be reduced to the syntactic dimension alone. 
A meaningful assessment must account both for syntactic diversity and for the extent to which semantic content is preserved. 

Existing approaches can broadly be grouped into automatic and human-based methods. 
Automatic measures attempt to quantify the similarity or equivalence between a candidate and a reference paraphrase using algorithmic techniques. 
These methods can be further distinguished by the linguistic level at which they operate. 
Some focus on syntactic structure and word overlap, while others evaluate whether the meaning preservation. 
Human evaluation, in contrast, remains the gold standard, as it naturally incorporates all of these dimensions.

% Syntactic (\ac{bleu}, \ac{rouge}-1, \ac{rouge}-L), semantic (BERTScore, cosine similarity of SBERT vectors, WMS), human evaluation (TODO)

% There is manual (by humans) evaluation and automatic evaluation for paraphrase generation \citep{fu_learning_2024,zhou_paraphrase_2021}.
% According to \citet{zhou_paraphrase_2021}, automatic evaluation metrics mainly focus on the n-gram overlaps instead of meaning, 
% and hence, human evaluation is more accurate and has a higher quality.
% In the following, we focus on automatic evaluation.

In the following, we examine both automatic and human evaluation strategies. 
The automatic measures are divided into syntactic and semantic approaches, reflecting the different dimensions along which paraphrases can be compared. 
This is followed by a discussion of human evaluation, which complements automatic measures by providing a more comprehensive assessment of paraphrase quality.


\subsection{Traditional Quantitative Paraphrase Evaluation Measures}
\label{subsec:traditional_quantitative_evaluation_measures}

Evaluating paraphrases can be reduced to summarization or translation evaluation.
The evaluation of paraphrases can be divided into syntactic and semantic approaches. 
% \citet{gohsen_captions_2023} normalized all metrics and averaged the semantic and syntactic scores separately.

\subsubsection{Syntactic Measures}
Syntactic evaluation metrics mainly focus on the n-gram overlaps~\citep{zhou_paraphrase_2021}. 
Common syntactic evaluation metrics include \acs{bleu}, \acs{rouge}-1, and \acs{rouge}-L.

\paragraph{\ac{bleu}}
\ac{bleu}~\citep{papineni_bleu_2001} was originally developed for machine translation~\citep{zhou_paraphrase_2021,anantha_pearson_metrics_2021}. 
\ac{bleu}'s basic unit of evaluation is a sentence. 
\ac{bleu} is based on precision, i.e.\ computing the fraction of generated word n-grams $n\text{-}gram \in \mathcal{C}$ that appear in any reference text $r$~\citep{kurt_pehlivanoglu_comparative_2024,palivela_optimization_2021,papineni_bleu_2001,anantha_pearson_metrics_2021}. 
In our case, there is only one reference.
We consider the reference the ground truth text.
To prevent inflated precision scores $p_n$ due to repetition of frequent tokens (e.g.\ "the"), \ac{bleu} introduces a clipping mechanism $\operatorname{Count_{match}}(.)$ that caps n-gram counts at their maximum reference frequency~\citep{papineni_bleu_2001}. 
Precision $p_n$ for $n \in \mathbb{N}_{>0}$ is given by \Cref{eq:bleu}.

\begin{equation}
    p_n = \sum_{\mathcal{C} \in \left\{ Candidates \right\}}\sum_{n\text{-}gram \in\mathcal{C}} \frac{\operatorname{Count_{match}}(n\text{-}gram)}{\operatorname{Count}(n\text{-}gram)}
\label{eq:bleu}
\end{equation}

The choice of $n$ determines what syntactic characteristic is evaluated.
Uni-grams are used to test adequacy, while longer n-grams are used to test fluency~\citep{papineni_bleu_2001}. 
The brevity penalty $\operatorname{BP}$ from \Cref{eq:bleu_brevity_penalty} is applied to discourage excessively short candidates $c$~\citep{papineni_bleu_2001}.

\begin{equation}
    \operatorname{BP} = \begin{cases}
        1 & \text{if } \operatorname{len}(c) > \operatorname{len}(r) \\
        e^{1 - \frac{r}{c}} & \text{else}
    \end{cases}
\label{eq:bleu_brevity_penalty}
\end{equation}

In order to compute the \ac{bleu} score from \Cref{eq:bleu} for more than one sentence, 
one (1) computes the clipped n-gram matches sentence by sentence, 
then (2) adds them across all sentences, 
and finally (3) divides the total clipped n-gram matches by 
the total number of unclipped n-grams in all candidate sentences~\citep{papineni_bleu_2001,cordeiro_bleu_2007}.

Combined scores across different n-gram orders are computed via the geometric mean, weighted uniformly (i.e.\ $w_n$) across different $n$~\citep{papineni_bleu_2001,banerjee_METEOR_2005}.
Combining precision $p_n$ (\Cref{eq:bleu}) and brevity penalty $\operatorname{BP}$ (\Cref{eq:bleu_brevity_penalty}) leads to the final score in \Cref{eq:bleu_final}.

\begin{equation}
    \operatorname{BLEU} = \operatorname{BP}  \exp\left(\sum_{n=1}^{N} w_n  \log p_n\right)
\label{eq:bleu_final}
\end{equation}

\ac{bleu} disregards semantic similarity completely and therefore judges paraphrases only based on n-gram overlap. 
As such, it is generally recommended being supplemented with human evaluation~\citep{zhou_paraphrase_2021}.



% Its values range from 0 to 1 \citep{papineni_bleu_2001}.

% \ac{bleu} automatically penalises n-grams appearing in the candidate text but not in the reference text, 
% as well as n-grams appearing more often in the candidate than in the reference text \citep{papineni_bleu_2001}.

% For multiple sentences, they (1) add the best match (among the reference texts) length for each candidate sentence, 
% and (2) divide this sum $r$ by the total length of all candidate sentences $c$. 
% They cannot use recall for length-related problems here, 
% because \ac{bleu} uses multiple reference texts, which may have different lengths \citep{papineni_bleu_2001,banerjee_METEOR_2005}.
% If the generated candidate is significantly shorter than the reference text, the brevity penalty $\operatorname{BP}$ is applied.
% A \ac{bleu} score approaching 1 signifies the candidate matches one reference almost exactly \citep{papineni_bleu_2001}, 
% and thus, limited syntactic diversity (i.e.\ inadequate paraphrase) \citep{kurt_pehlivanoglu_comparative_2024}.
% Note that more reference texts lead to higher \ac{bleu} scores \citep{papineni_bleu_2001}.

\paragraph{\ac{rouge}.}
\ac{rouge} \citep{lin_rouge_2004}, initially developed for summarisation, is recall-oriented and emphasises coverage of reference content in the candidate text. 
Lower \ac{rouge} scores indicate greater diversity \citep{kurt_pehlivanoglu_comparative_2024}.
Several variants exist, including \ac{rouge}-N, which computes word n-gram recall, and \ac{rouge}-L, which measures the \ac{lcs}~\citep{zhou_paraphrase_2021,palivela_optimization_2021,kurt_pehlivanoglu_comparative_2024}. 

% ROUGE-N
\ac{rouge}-N is an n-gram recall between the candidate text $c$ and the reference text $r$~\citep{lin_rouge_2004}.
Since we only consider single reference scenarios we use the simpler version for one reference in \Cref{eq:rouge_n}.
\begin{equation}
    \operatorname{ROUGE-N} = \sum_{s \in r}\sum_{n \text{-} gram \in s} \frac{\operatorname{Count_{match}}(n \text{-} gram)}{\operatorname{Count}(n \text{-} gram)}
\label{eq:rouge_n}
\end{equation}
Both the nominator and the denominator iterate over all n-grams in all sentences of the reference, i.e.\ ground truth, text.
The nominator sums up $\operatorname{Count_{match}}(n \text{-} gram)$, i.e.\ the maximum number of occurrences of this reference sentence n-gram in any of the candidate texts.
Since this sum measures the number of co-occurrences of that n-gram in both candidate and reference text, it is naturally capped to the number of occurrences in the reference.
This ensures that after normalising with the denominator, the \ac{rouge}-N values range from 0 to 1~\citep{kurt_pehlivanoglu_comparative_2024}.
% The nominator sums over all references and thus, gives more weight to matching n-grams that occur in multiple references (i.e.\ a consensus between references) \citep{lin_rouge_2004}.    % do not use, bc we have only one reference
The denominator does not consider matches but only sums up the number of times an n-gram appears in the reference sentences~\citep{lin_rouge_2004}.
If every n-gram from the reference sentences would appear equally often in the candidate, the \ac{rouge}-N would be one since it measures the n-gram overlap between reference and candidate from a reference or recall perspective.

\ac{rouge}-L for a candidate $c$ of length $n$ and a reference $r$ of length $m$ is defined in \Cref{eq:rouge_l}.
The length is measured in number of words.
The intuition is that the length of the \ac{lcs} between the candidate and reference texts correlates with their similarity.
\ac{rouge}-L does not include shorter sequences or alternative \ac{lcs} in the final score~\citep{lin_rouge_2004}.
$\beta$ is defined as $\frac{\mathrm{P_{lcs}}}{\mathrm{R_{lcs}}}$.

\begin{equation}
\mathrm{P_{lcs}} = \frac{\operatorname{LCS}(r,c)}{n}
\label{eq:rouge_l_precision}
\end{equation}

\begin{equation}
\mathrm{R_{lcs}} = \frac{\operatorname{LCS}(r,c)}{m}
\label{eq:rouge_l_recall}
\end{equation}

\begin{equation}
\operatorname{ROUGE-L} = \frac{(1 + \beta^2)  \mathrm{R_{lcs}}  \mathrm{P_{lcs}}}{\mathrm{R_{lcs}} + \beta^2  \mathrm{P_{lcs}}}
\label{eq:rouge_l}
\end{equation}

\ac{rouge}-Lsum is a summary-level \ac{lcs} \ac{rouge} variant summing over the union of \ac{lcs} matches $\operatorname{LCS}_\cup(r_i,C)$ between the $u$ reference summary sentences $r_i, i \in [1,u]$ and each of the $v$ candidate sentence $c_j, j \in [1,v]$.
The total number of words in the references and candidates is $m$ and $n$, respectively.
While the formula for the final score $F_{lcs}$ from \Cref{eq:rouge_l} remains the same, the computation of precision and recall replace $\operatorname{LCS}(r,c)$ by $\sum_{i=1}^{u}\operatorname{LCS}_\cup(r_i,C)$.


\paragraph{METEOR}
% Its values range from 0 to 1 \citep{kurt_pehlivanoglu_comparative_2024}.
METEOR was proposed to address the limitations of \ac{bleu}. 
Unlike \ac{bleu}, METEOR explicitly incorporates recall. 
We consider METEOR primarily a syntactic metric due to its conceptual similarity to \ac{bleu}, but it also captures semantic aspects through stemming and synonym matching modules~\citep{kurt_pehlivanoglu_comparative_2024}. 

The order of modules reflects their priority in the alignment process. 
When the first module is exact matching, all possible mappings of candidate word unigrams to exact matches in the reference text are considered. 
Although valid alignments may restrict each unigram to a single mapping, multiple mappings are allowed in this initial stage. 
In the second stage, the best subset of unigram mappings is selected according to cardinality and minimal crossing. 
Unigrams that have not yet been mapped are then eligible for alignment using the next module in order, such as Porter-stemmed matching or synonym matching, producing multiple sets of mappings between candidate and reference. 
From the resulting alignments, METEOR computes a weighted $F$-score, as defined in \autoref{eq:meteor}. 
In this formulation, unigram precision $P$ is the fraction of candidate unigrams that are mapped to reference unigrams relative to the total number of candidate unigrams. 
Conversely, unigram recall $R$ is the fraction of candidate unigrams that are mapped to reference unigrams relative to the total number of reference unigrams~\citep{banerjee_METEOR_2005}.

\begin{equation}
    METEOR = F_{mean} = \frac{10 \cdot P \cdot R}{R + 9P} \cdot (1 - Penalty)
\label{eq:meteor}
\end{equation}

The penalty function discourages fragmented alignments and reduces the score by up to $50\%$ if bigram or longer matches are absent~\citep{banerjee_METEOR_2005}. 
METEOR has been shown to correlate more strongly with human judgments than \ac{bleu}, particularly at the sentence or segment level, due to its sensitivity to both lexical and semantic variation~\citep{zhou_paraphrase_2021,kurt_pehlivanoglu_comparative_2024}.


% \tikzstyle{startstop} = [rectangle, rounded corners, minimum width=3cm, minimum height=1cm,text centered, draw=black, fill=red!30]
% \tikzstyle{process} = [rectangle, minimum width=3cm, minimum height=1cm, text centered, draw=black, fill=blue!20]
% \tikzstyle{decision} = [diamond, minimum width=3cm, minimum height=1cm, text centered, draw=black, fill=green!30]
% \tikzstyle{arrow} = [thick,->,>=stealth]


% \begin{figure}[h!]
% \centering
% % \resizebox{\textwidth}{!}{%
% \begin{tikzpicture}[node distance=2.5cm, every node/.style={minimum width=3cm, minimum height=1cm, text centered, draw, fill=blue!20}]

% % Nodes in a circular layout
% \node (start) [rectangle, rounded corners, fill=red!30] at (90:4cm) {Candidate \& Reference Sentences};
% \node (matching) at (30:4cm) {Matching};
% \node (bestsubset) at (150:4cm) {Select Subset of Mappings};
% \node (fscore) at (180:4cm) {Compute F-Score};
% \node (end) [rectangle, rounded corners, fill=red!30] at (200:4cm) {METEOR Score};

% % Arrows
% \draw[->, thick] (start) -- (matching);
% \draw[<->, thick] (matching) -- (bestsubset);
% \draw[->, thick] (bestsubset) -- (fscore);
% \draw[->, thick] (fscore) -- (end);

% \end{tikzpicture}%
% % }
% \caption{Circular visualization of METEOR score computation steps, from candidate and reference sentences to the final weighted F-score with penalty.}
% \label{fig:meteor_circular}
% \end{figure}





\subsubsection{Semantic Measures}
Syntactic measures are inadequate when the goal is to evaluate paraphrases that prioritize semantic preservation over lexical similarity. 
To address this limitation, semantic metrics leverage distributed representations of words or sentences.
We compute semantic similarity between transformer based models~\citep{gohsen_captions_2023}.

BERTScore~\citep{hanna_fine_grained_2021} computes similarity between contextual BERT embeddings of candidate and reference texts. 
For reference vectors $r$ and candidate vectors $c$, precision and recall are defined as \autoref{eq:bert_p} and \autoref{eq:bert_r}, respectively.

\begin{equation}
    P_{BERT} = \frac{1}{|c|} \sum_{c_i \in c} \max_{z_j \in r} r_j\top c_i
\label{eq:bert_p}
\end{equation}
\begin{equation}
    R_{BERT} = \frac{1}{|r|} \sum_{r_i \in r} \max_{c_j \in c} r_i\top c_j
\label{eq:bert_r}
\end{equation}

% \begin{equation}
%     F_1 = \frac{2 P_{BERT} R_{BERT}}{P_{BERT} + R_{BERT}} 
% \label{eq:bert_f1}
% \end{equation}
% Since $F_1 \in \left[-1,1\right]$ it can be rescaled to $[0,1]$ by modifying the precision and recall calculation 
% to $\hat{P}_{BERT} = \frac{P_{BERT} - a}{1 - a}$ ($R_{BERT}$ analogous), where $a$ is the empirical lower bound on the BERTScore \citep{hanna_fine_grained_2021}.


BERTScore correlates with human judgment at the semantic level \citep{kurt_pehlivanoglu_comparative_2024}, although it may struggle when lexically overlapping but semantically incorrect candidates are present \citep{hanna_fine_grained_2021}.

\ac{wmd} measures the minimal transport cost of aligning word embeddings from one text to another \citep{gohsen_captions_2023}. 
\textcolor{red}{TODO: Rechnung}

\textcolor{red}{TODO: SBERT cosine similarity}
The cosine similarity between dense vector representations of a SBERT model~\citep{gohsen_captions_2023}.

\subsubsection{Gohsen Delta $\Delta_{sem,syn}$}
First, all syntactic and semantic measures are normalized to a scale from zero to one.
Then, the average syntactic similarity $\diameter_{syn}$ and the average semantic similarity $\diameter_{sem}$ is calculated.
Syntactic metrics include \ac{rouge}-1, \ac{rouge}-L, and \ac{bleu}.
Semantic measures include \ac{wms}, BERT, and cosine similarity of the SBERT embeddings.
Finally, $\Delta_{sem,syn}$ is defined as in \autoref{eq:gohsen_delta}, i.e. the difference of semantic and syntactic average distance~\citep{gohsen_captions_2023}.
\begin{equation}
    \Delta_{sem,syn}=\diameter_{sem}-\diameter_{syn}
    \label{eq:gohsen_delta}
\end{equation}
Hence, high $\Delta_{sem,syn}$ values indicate structurally and lexically diverse and semantically similar text pairs.



\subsubsection{Qualitative Evaluation}
\label{subsec:qualitative_evaluation}

Human qualitative evaluation can combine syntactic and semantic dimensions more reliable than any automatic metric proposed.
Naturally, when being asked to evaluate the quality of a paraphrase, individuals will score syntactic difference from the reference text, the readability from the paraphrase and semantic similarity to the reference text.
Evaluation is usually formalized via a Likert scale~\citep{gohsen_captions_2023}.
