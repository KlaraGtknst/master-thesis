\section{Dataset}
\label{sec:dataset}

Since our method extends the original \impAppr{} proposed by \citet{koppel_determining_2014}, we first obtained the datasets used in their study to validate our implementation and reproduce their results. 
The original experiments were carried out on the \dataBlog{} and \dataStudent{} datasets, described in detail in \Cref{subsec:original_data}.
In addition to these, we incorporated two supplementary datasets, \dataPan{} and \dataGutenberg{}, presented in \Cref{subsec:additional_data}. 
Following the general description of all datasets, we outline our preprocessing pipeline in \Cref{subsec:dataset_preprocessing} and conclude with the text-pair selection procedure in \Cref{subsec:dataset_text_pair_selection}.


\subsection{Original Data}
\label{subsec:original_data}

% Blog
The \dataBlog{} corpus~\citep{blog_dataset_2006} consists of blog posts collected from \textit{blogger.com} on or before August 2004, with each blog authored by a single user.
According to the Kaggle repository~\footnote{\href{https://www.kaggle.com/datasets/rtatman/blog-authorship-corpus?resource=download}{Kaggle dataset \texttt{rtatman/blog-authorship-corpus}} (26.07.2025)}, the dataset contains \num{681288}~posts from \num{19320}~bloggers, averaging approximately 35~posts and \num{7250}~words per author.
Each record includes the following metadata: \texttt{id}, \texttt{gender}, \texttt{age}, \texttt{topic}, 
\texttt{sign} (referring to the author's zodiac sign), \texttt{date}, and \texttt{text}.

% student essays
The \dataStudent{} dataset is not publicly available due to the presence of sensitive student information. 
% We gratefully acknowledge J. W. Pennebaker for granting access to the original data used by \citet{koppel_determining_2014} in their study. 
Researchers seeking access to the \dataStudent{} dataset should contact J. W. Pennebaker, the official data custodian. 
The dataset comprises \num{7052}~student essays written for five assignments by a cohort of 950 university students in 2006~\citep{koppel_determining_2014}.

The assignments include (1) a stream-of-consciousness task, (2) a reflections on childhood, (3) a self-assessment of personality, (4) a thematic apperception test, and (5) four examples of four different theories.
% Most files are named solely by the author ID, whereas those from the first assignment follow the format \texttt{2006\_authorID}.
Following \citet{koppel_determining_2014}, our dataset is built from the first four assignments. 
The dataset provides metadata both collected from structured columns in the \texttt{.dat} files and derived from file names, which we combined into a unified resource. 
Metadata includes year, author ID, author name, political orientation, task, sex, ethnicity, and teacher.


\subsection{Additional Data}
\label{subsec:additional_data}
To broaden the evaluation scope of the \impAppr{}, we incorporated additional datasets that both control for confounding factors such as genre and topic and consist of texts with verified, undisputed authorship. 
Both the \dataPan{} and \dataGutenberg{} datasets meet these criteria.

% PAN20: Fanfiction
The \dataPan{} corpus~\citep{bischoff_importance_2020} comprises fanfiction texts sourced from \textit{fanfiction.net}.
Each text belongs exclusively to one fandom (i.e.\ thematic category), with no crossovers between fandoms.
According to \href{https://pan.webis.de/clef20/pan20-web/author-identification.html}{the official \acs{pan} website}, 
train and test set originate from two different fanfictions.
Dataset features include \texttt{id}, \texttt{fandoms}, and \texttt{pair}, where the latter contains the paired texts.
An additional \texttt{jsonl} file provides the ground truth for each pair, specifying \texttt{id}, \texttt{same} (ground truth label for \ac{av}), and \texttt{authors}.

% Gutenberg
The \dataGutenberg{} dataset~\footnote{\url{https://www.gutenberg.org/} (26.07.2025)} contains a curated selection of literary works from Project Gutenberg, a digital library dedicated primarily to older works whose U.S. copyrights have expired.
As of this writing, the collection contains over \num{75000}~digitized and proofread e-books contributed by volunteers according to their website.
% For our experiments, we selected 19 works authored by 7 writers from the 16th to 19th centuries, the distribution of genres is given in \Cref{tab:genre_counts_gutenberg}.
For our experiments, we selected 19~works authored by 7~writers from the 16th to 19th centuries, covering nine dramas, nine fiction texts and one poetry work.
Metadata for these works was manually extracted from the \href{https://www.gutenberg.org/}{Project Gutenberg website} and Wikipedia.

% \begin{table}[]
% \centering
% \caption{Unique value count of genres of \dataGutenberg{} dataset.}
% \label{tab:genre_counts_gutenberg}
% % \resizebox{\textwidth}{!}{%
% \begin{tabular}{@{}ll@{}}
%     \toprule
% \textbf{Genre} & \textbf{Count} \\
% \midrule
% Drama          & 9              \\
% Fiction        & 9              \\
% Poetry         & 1    \\
% \bottomrule         
% \end{tabular}%
% % }
% \end{table}

% \textcolor{orange}{We augment the \dataStudent{} dataset with artificial generated texts and denote this dataset \dataArtificialStudent{}.
% It contains pairs of student essays written in response to different academic assignments. 
% The dataset includes both human-written essays and artificially generated paraphrases created by \acp{llm} instructed to simulate student writing. 
% Artificially generated essays are produced using \acp{llm} prompted to emulate an 18-year-old college freshman's voice, taking into account demographic features like sex, ethnicity, and political orientation.
% Each pair is labelled as same-author or different-author indicating whether the text was generated artificially. 
% }

\subsection{Dataset Preprocessing}
\label{subsec:dataset_preprocessing}

To control confounding factors that influence authorial style, we preprocess each dataset twice:
(1) Prior to generating arrow dataset file and (2) before using the \impAppr{}.
This two-stage approach addresses both experimental scenarios, where all material is prepared in advance, and inference scenarios in which the \impAppr{} is applied directly to texts.
The preprocessing process was designed to meet the following requirements:
\begin{enumerate}
    \item Removal of all formatting and layout information to produce plain text
    \item Cropping texts to match the length of the shorter text in each pair
    \item Removal of texts with less than 700 words
\end{enumerate}
For a controlled evaluation environment in our \impAppr{}, we opted to work with relatively small, curated datasets rather than scaling to larger collections.  
Text-length filtering is essential to the creation of optimal testbeds for \ac{av}.
To determine the necessary preprocessing steps, we examined our datasets and analysed their respective artefacts.

Since both the \dataBlog{} and the \dataPan{} dataset originate from the Internet, some of their documents contain HTML fragments such as \texttt{< >} enclosed tags.
Upon inspection of the \dataGutenberg{} dataset, we found that certain patterns reappear due to the presences of many plays.
Based on these findings, we collected eight preprocessing steps we deemed potentially useful. 
Notably, we do not consider actor instructions (e.g. character cues) or structural elements (e.g. chapter headings) as part of authorial style.
Hence, we remove these patterns using regular expressions.
You may find the regular expression attached in \Cref{app:regex_preproc}. % of \Cref{ch:appendix}.
Note that the perceived utility of these steps should be reevaluated for other datasets.
Next, we analysed the effect of individual preprocessing steps on vocabulary size.  
We define the vocabulary of a dataset as unique tokens across all texts (regardless of their text length or split) of that dataset.
In line with \citep{koppel_determining_2014}, tokens are space-free character 4-grams.
We leave the capitalization of tokens unchanged. 
The results are displayed in \Cref{fig:preprocesing_impact_vocab_size}.

We found that HTML-specific preprocessing steps had minimal impact on the size of vocabularies.
Similarly, the removal of artefacts specific to theatre plays had little effect on the overall vocabulary size.
The results aligned with expectations, as the vocabularies are defined over space-free 4-gram units that, due to their limited length, exhibit a high likelihood of redundancy.
Consequently, repetitive patterns are mapped on the same vocabulary item.
Once the repetitive patterns are filtered out, the vocabulary lacks their respective items, which are not many.

Based on these considerations, our preprocessing steps include removing HTML artefacts, play artefacts, newlines, converting UTF-8 to ASCII, and stripping leading and trailing whitespace.
We opted to forgo lowercasing the texts, as our preliminary analysis indicated that lowercasing had no meaningful effect on any dataset while potentially discarding deliberate authorial capitalization choices.
Due to preprocessing, our \dataPan{} version differs from those applied elsewhere.

\begin{figure}[htbp]
    \centering
    \includesvg[width=\textwidth]{images/dataset/impact_preprocessing_steps.svg}
    \caption[Effect of preprocessing steps on vocabulary size.]{Effect of preprocessing steps on vocabulary size. The vocabulary contains unique space-free character 4-grams.
    For this experiment, there was no minimum text length.
    For \dataPan{}, train and test split were combined.}
    \label{fig:preprocesing_impact_vocab_size}
\end{figure}


\subsection{Selection of Text Pairs}
\label{subsec:dataset_text_pair_selection}

For the \dataBlog{}, \dataStudent{}, and \dataGutenberg{} datasets, we selected pairs of texts according to specific criteria to control potential confounding factors.
Only texts with a minimum length of \num{700}~words were considered eligible. 
For the \dataPan{} dataset, we retained the existing pairs in the arrow dataset, but filtered out texts with less than \num{700}~words. 
All datasets include both same-author and different-author pairs. 

For the \dataBlog{} dataset, we matched text pairs on topic, year, gender, and author age. The training set contains 80\% of the data and the test set 20\%, with distinct topics in each split.

For the \dataStudent{} dataset, following \citet{koppel_determining_2014}, we drew all text pairs from different tasks, with authors matched by sex, ethnicity, and political orientation. Pairs were split into training (70\%) and test (30\%) sets.
The test portion is larger than in the \dataBlog{} dataset because each author typically contributes only one essay per task, so including only one of four tasks in the test set would prevent pair formation.

For the \dataGutenberg{} dataset, we selected text pairs that share the same genre and century and split the authors into training (80\%) and test (20\%) sets.

% minimum length necessesary for AV/ AA
The choice of minimum text length was informed by literature research.
\citet{bevendorff_generalizing_2019}\ used text chunks of at least 700 words for an unmasking approach, while \citet{koppel_authorship_2004}\ set the minimum to \num{500}~words.
Recent work~\citep{llm_detection_av_2025} identifies \num{2500}-\num{4000}~characters to be sufficient for effective \ac{llm} detection framed as \ac{av}.

Regardless of the selection criteria, the final datasets contain only three columns: \texttt{authors}, \texttt{pair}, and \texttt{same}.
The \texttt{pair} column contains the texts of the pair as a list of strings,
the \texttt{authors} column contains the authors of the texts as a list of strings,
and the \texttt{same} column indicates whether the texts originate from the same author (\texttt{True}) or from different authors (\texttt{False}).
Descriptive statistics for all preprocessed datasets are provided in Table~\ref{tab:data_stats}.

% The \dataArtificialStudent{} dataset is split into training and test sets using a stratified approach, ensuring that all combinations of author type (human vs. \ac{llm}), pair type (same- vs. different-author), and artificial generation (True vs. False) are proportionally represented in both splits. 
% Since the \dataArtificialStudent{} dataset is artificially created, its feature differ from the other datasets. 
% Each record contains the assignment names and descriptions, the paired texts, the authors, and metadata flags indicating author sameness and artificial generation.

% The presence of short artefacts in the \dataArtificialStudent{} dataset cannot be attributed to human-authored texts, as these have been filtered to meet a minimum length.
% Hence, the artificially generated texts exhibit shorter average lengths compared to other datasets.

\begin{table}[H]
% \begin{sidewaystable}
\centering\small
\caption[Statistics of preprocessed datasets.]{Statistics of preprocessed datasets \dataBlog{}, \dataGutenberg{}, \dataPan{}, and \dataStudent{}. %, and \dataArtificialStudent{}.
$p_s$, $p_{\neg s}$ denote same-author and different-authors pairs, while $l_w$, $l_c$ denote text length in words and characters, respectively.
}
\label{tab:data_stats}
\resizebox{\textwidth}{!}{%
\begin{tabular}{@{}lrrrrrrrrr@{}}   % numbers should be right aligned, text left aligned
\toprule
dataset & \# pairs & \# authors & \# $p_s$ & \# $p_{\neg s}$ & \diameter $l_w$ ($l_c$) & max $l_w$ & $\sigma_{l_w}$ & median $l_w$ \\
\midrule
\dataBlog{}            & 11565 & 5997  & 6204 & 5361  & 6249.94 (1154.25)     & 115365 & 1493.97 & 913 \\
\dataGutenberg{}       & 12    & 7     & 6     & 6     & 437870.75 (78698.79) & 297704 & 68329.91 & 60282 \\
\dataPan{}           & 66905 & 52771 & 35616 & 31289 & 21418.76 (3914.76)   & 55413 & 512.19 & 3889 \\
\dataStudent{} & 224  & 163   & 112   & 112  & 4403.73 (851.45)     & 1520 & 138.15 & 807  \\
% \dataArtificialStudent{} & 110 & 32 & 50 & 60 & 661.39 (3581.46) & 1769 & 267.08 & 703 \\
\bottomrule
\end{tabular}%
}
\end{table}
% \end{sidewaystable}
