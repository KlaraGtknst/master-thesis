\subsection{Authorship Verification Quality Measures}
\label{subsec:av_quality_measures}

Since \ac{av} forms the core of \ac{aa}, and because every \ac{aa} task can be reduced to \ac{av}, this section focuses primarily on \ac{av}. 
Evaluation relies on standard classification metrics, each with distinct advantages and limitations.

% \paragraph{Accuracy}
% The most straightforward metric is accuracy (\Cref{eq:accuracy}), which measures the proportion of correctly classified cases across all samples. 
% While intuitive, accuracy can be misleading in scenarios with class imbalance. 

% \begin{equation}\label{eq:accuracy}
%     \text{Accuracy} = \frac{\text{\acs{tp}} + \acs{tn}}{\acs{tp} + \acs{tn} + \acs{fp} + \acs{fn}}
% \end{equation}

% \paragraph{Precision, Recall \& $\operatorname{F_{1}}$}
Precision $\operatorname{P}$ and recall $\operatorname{R}$ address the limitations of accuracy in class imbalance scenarios by focusing on the effectiveness for the positive class, that is, the correct identification of same-author pairs. 
Precision measures the proportion of positive predictions that are correct, while recall quantifies the proportion of \ac{tp} cases that are successfully detected. 
Because these metrics often trade off against each other, their harmonic mean, the $\operatorname{F_{1}}$ score, is commonly used to provide a balanced assessment of effectiveness. 
\Cref{eq:f1} shows the computation of the $\operatorname{F_{1}}$ score~\citep{neal_surveying_2018}.

\begin{equation}\label{eq:f1}
     \operatorname{F_{1}} = \frac{2\mathrm{P}  \mathrm{R}}{\mathrm{P} + \mathrm{R}}
\end{equation}

% \paragraph{$\operatorname{F_{0.5u}}$ \& $\operatorname{c@1}$}
% Beyond these basic measures, \ac{av} research has introduced metrics that explicitly account for difficult borderline cases. 
% The modified $\operatorname{F_{0.5u}}$ score penalises non-answers by treating them as \acp{fn}~\citep{bevendorff_overview_2024}. 
% This places additional weight on correctly deciding same-author cases~\citep{weerasinghe_feature_vector_difference_2021}, thereby evaluating the ability of \ac{av} methods to abstain from hard samples~\citep{tyo_state_2022}. 
% The $\operatorname{c@1}$ metric, by contrast, was designed to reward abstention from particularly ambiguous cases. 
% It does so by granting unanswered problems partial credit, equal to the average accuracy on the remaining cases~\citep{bevendorff_overview_2024}, thus encouraging systems to remain silent when uncertain.

% However, in the present work, $\operatorname{c@1}$ and $\operatorname{F_{0.5u}}$ are not appropriate measures. 
Although we also used $\operatorname{F_{1}}$ and accuracy in our work, we opted to exclude them in this thesis to keep results simple and comparable to the original work by \citet{koppel_determining_2014}.
More advanced metrics, such as $\operatorname{F_{0.5u}}$ and $\operatorname{c@1}$, assume that models can explicitly abstain by assigning a score of exactly 0.5, a convention used in \acs{pan}'s shared tasks~\citep{tyo_state_2022,bevendorff_overview_2024,kocher_unine_2015}. 
In our setting, the model produces only the binary outcomes "same author" or "don't know". 
Since there is no natural mechanism for producing a calibrated 0.5 score, abstention cannot be meaningfully represented. 
One could introduce a second threshold to create an artificial abstention region, but this would not align with the open-set nature of \ac{av}. 
The different-author class is inherently ill-defined, as the set of possible authors is unbounded and cannot be exhaustively represented. 

% Moreover, we do not report average effectiveness scores across different recall values, since we do not need good results across different thresholds, but only high effectiveness for one threshold.
% For this reason, the focus in this thesis remains on precision, recall, and $\operatorname{F_{1}}$. 
