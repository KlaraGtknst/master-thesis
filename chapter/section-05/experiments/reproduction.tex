\subsection{Exp. 1: Reproduction of Original Work}

To assess the validity of our extension to the traditional \impAppr{}, we first verified the correctness of our implementation. 
For this purpose, we designed two experiments, which we ran on a subset of 100 pairs from the training and test sets of the \dataBlog{} and \dataStudent{} dataset respectively. 
Half of the selected samples belong to the same author class.

\paragraph{Exp. 1(a): Varying number of \imps{}.}
The first experiment evaluates the effect of varying the number of \imps{} while setting the \imp{} generation method to \texttt{fixed}.
All other hyperparameter values were set to the default values reported by \citet{koppel_determining_2014} (cf. Table~\ref{tab:repr_exp1}). 
Adhering \citet{koppel_determining_2014}, we computed precision and recall scores across different thresholds.
For comparison, reference precision-recall points reported by \citet{koppel_determining_2014} were included in our visualization. 
Based on their description, we deduced that their reported scores were obtained using the \dataBlog{} dataset.


\begin{table}[h]
\centering\small
\caption{Exp. 1(a): \impAppr{} configuration.}
\label{tab:repr_exp1}
\begin{tabular}{@{}llrrl@{}}   % numbers should be right aligned, text left aligned
\toprule
\# Impostors & Generation & Rounds & Top $n$ & Upsample \\
\midrule
\textit{Variable} & Fixed & 100 & \num{100000} & False \\
\bottomrule
\end{tabular}%
\end{table}

% Exp 1c: find best threshold via different metrics
% A detector instance was trained on each training set, and the optimal decision threshold was determined using Youden's J statistic. 
% This threshold was then applied to the 15 test set pairs to generate final predictions, which were summarized in a confusion matrix.

% We also considered using thresholds that maximized alternative metrics, such as the F1 score, but rejected this approach because it produced imbalanced detector classifications. 

\paragraph{Exp. 1(b): Varying impostor generation.}
The second experiment evaluates different \imp{} generation methods while keeping the number of \imps{} fixed.
Again, all other hyperparameter values were set to the default values reported by \citet{koppel_determining_2014} (cf. Table~\ref{tab:repr_exp2}). 
Following \citet{koppel_determining_2014}, we compared the \texttt{fixed} and \texttt{on-the-fly} \imp{} generation methods with the baseline approaches unsupervised min-max similarity, unsupervised cosine similarity, and supervised linear \ac{svm}.

\begin{table}[h]
\centering\small
\caption{Exp. 1(b): \impAppr{} configuration.}
\label{tab:repr_exp2}
\begin{tabular}{@{}rlrrl@{}}   % numbers should be right aligned, text left aligned
\toprule
\# Impostors & Generation & Rounds & Top $n$ & Upsample \\
\midrule
50 & \textit{Variable} & 100 & \num{100000} & False \\
\bottomrule
\end{tabular}%
\end{table}

As in the first experiment, precision and recall were used as the primary evaluation metrics. 
Consistent with \citet{koppel_determining_2014}, we calculated precision and recall with respect to both the same author and different author class, alternately treating each as the reference class.
We note that the different author class if ill-defined as \ac{av} is a one-class classification problem.
