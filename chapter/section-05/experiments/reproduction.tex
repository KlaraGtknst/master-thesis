\subsection{Exp.\ 1: Reproduction of Original Work}

To assess the validity of our extension to the traditional \impAppr{}, we first verified the correctness of our implementation. 
For this purpose, we designed two experiments, which we ran on a subset of 100~pairs from the training and test sets of the \dataBlog{} and \dataStudent{} dataset respectively. 
Half of the selected samples belong to the same-author class.

\begin{table}[h]
\centering\small
\caption{Exp.\ 1: \impAppr{} configuration.}
\label{tab:config_exp1}
\begin{tabular}{@{}rrlrrl@{}}   % numbers should be right aligned, text left aligned
\toprule
Experiment & \# Impostors & Generation & Rounds & Top $n$ & Upsample \\
\midrule
1(a) & \textit{Variable} & Fixed & 100 & \num{100000} & False \\
1(b) & 50 & \textit{Variable} & 100 & \num{100000} & False \\
\bottomrule
\end{tabular}%
\end{table}

\paragraph{Exp.\ 1(a): Varying number of \imps{}.}
The first experiment evaluates the effect of varying the number of \imps{} while setting the \imp{} generation method to \texttt{fixed}.
All other hyperparameter values are set to the default values reported by \citet{koppel_determining_2014} (cf. Table~\ref{tab:config_exp1}). 
Adhering \citet{koppel_determining_2014}, we compute precision and recall scores across different thresholds.
\textcolor{orange}{For comparison, reference precision-recall points reported by \citet{koppel_determining_2014} are included in our visualization.} 
Based on their description, we deduced that their reported scores were obtained using the \dataBlog{} dataset.

\paragraph{Exp.\ 1(b): Varying \imp{} generation.}
The second experiment evaluates different \imp{} generation methods while keeping the number of \imps{} fixed.
Again, all other hyperparameter values are set to the default values reported by \citet{koppel_determining_2014} (cf. Table~\ref{tab:config_exp1}). 
Following \citet{koppel_determining_2014}, we compare the \texttt{fixed} and \texttt{on-the-fly} \imp{} generation methods with the baseline approaches unsupervised min-max similarity, unsupervised cosine similarity, and supervised linear \ac{svm}.

As in the first experiment, precision and recall are used as the primary evaluation metrics. 
Consistent with \citet{koppel_determining_2014}, we calculate precision and recall with respect to both the same-author and different-author class, alternately treating each as the reference class.
We note that the different-author class is ill-defined, as \ac{av} constitutes a one-class classification scenario in which covering representative instances of this class is infeasible.