\subsection{Exp.\ 2: Comparison of Paraphrasers}
\label{subsec:comp_paraphrasers_setup}

Next, we wanted to assess our paraphrasing approaches.
We hence designed two experiments.
The first experiment computes state-of-the-art paraphrasing measures for all paraphrasers on different datasets.
The second experiment aims to evaluate the ability of our two-step models to adhere to instructions.
We tested their proficiency extracting metadata and generating paraphrases of similar length as the reference text.
While the \dataBlog{} ground truth metadata comes with its CSV dataset, the \dataStudent{} metadata is derived from existing information about and in the dataset, and the \dataGutenberg{} metadata is manually curated.
We omit \dataPan{} from this experiment due to infeasible manual metadata curation.

\paragraph{Exp.\ 2(a): Quantitative evaluation.}

We select one text from the \dataBlog{}, the \dataGutenberg{} and the \dataStudent{} dataset, respectively.
For the \dataGutenberg{} dataset, we load ground truth metadata. % Student Essay: not used, even though existent
The paraphraser configurations contain two different temperatures for two-step paraphrasers, and two different prompts for one-step paraphrasers.
We create one paraphrase for each text configuration pair.
Evaluation measures include \ac{bleu}, \ac{rouge}-1, \ac{rouge}-2, \ac{rouge}-L, \ac{rouge}\-Lsum, METEOR, \ac{bert}\-Score Precision, \ac{bert}\-Score Recall, \ac{bert}\-Score F1, \ac{sbert} \ac{wms}, \ac{sbert} cosine similarity.
Based on these we also compute average syntactic and average semantic similarity, as well as Gohsen Delta $\Delta_{sem,syn}$~\citep{gohsen_captions_2023}.
We save the most extrem (min, max) paraphrases per metric.
The scores are subsequently visualized via syntactic-semantic scatters, score distributions, and radar plots per paraphraser and per prompt. 

\paragraph{Exp.\ 2(b): Evaluation of prompt adherence.}
For the second experiment, we selected five samples each from the \dataBlog{}, \dataGutenberg{}, and \dataStudent{} dataset. 
Our \pextractor{} identifies the genre, topic, and century of each input text. 
Both extracted and ground truth values are lowercased and stripped of leading and trailing whitespace. 
Genre and topic values may consist of multiple items separated by commas. 
For genre extraction, we split the \pextractor{}'s output by commas to obtain individual genre values. 
Similarly, ground truth topics, which often include multiple items separated by commas, are split into a list. 
Cosine similarities between the \ac{sbert} embeddings of the extracted and ground truth values are computed for each individual genre and topic, and the final score is taken as the maximum similarity among these single values. 

For century matching, we preprocess the \pextractor{} output by mapping terms such as \textit{present}, \textit{current}, and \textit{now} to 21, and, for numbers with three or more digits, dropping the last two digits and adding one. 
Examples are shown in \Cref{tab:examples_extract_century}. 
Ground truth century values are cast to dates.
We compute the ratio $\frac{a}{b}$ where $a$ is the extracted century and $b$ is the ground truth.

Moreover, we obtain the relative length difference of the paraphrase and original text for every selected sample. 

\begin{table}[h]
\centering
\caption{Examples for century processing.}
\label{tab:examples_extract_century}
% \resizebox{\textwidth}{!}{%
\begin{tabular}{@{}ll@{}}
    \toprule
\textbf{Original} & \textbf{Processeed} \\
\midrule
1964              & 20                  \\
now               & 21                  \\
190               & 2     \\
\bottomrule             
\end{tabular}%
% }
\end{table}

