\subsection{Comparing \citet{koppel_determining_2014}'s to \ac{llm}-based \imps{}}
\label{subsec:imp_gen}

We evaluated the performance of traditional impostor generation~\citep{koppel_determining_2014}, against \ac{llm}-based impostor generation on the \dataBlog{} and \dataStudent{} datasets. 
For each dataset, one sample was selected from both the training and test splits. 
The detector configuration is shown in Table~\ref{tab:imp_gen_comp}.

\begin{table}[h]
\centering\small
\caption{Configurations for detector.}
\label{tab:imp_gen_comp}
\begin{tabular}{@{}rlrrl@{}}   % numbers should be right aligned, text left aligned
\toprule
\# impostors & impostor generation & rounds & top\_n & upsample \\
\midrule
50 & \textit{variable} & 100 & \num{100000} & False \\
\bottomrule
\end{tabular}%
\end{table}

The detector was trained on the training set, and the optimal decision threshold was determined using Youden’s J statistic. 
Predictions on the test set were obtained by applying this threshold to the detector’s scores.

For each impostor generation method, we computed accuracy, precision, recall, and F1 score. 
The results are presented as bar plots, with separate plots for each metric to facilitate direct comparison. 
We also examined the effect of varying decision thresholds on detector performance, plotting all impostor generation methods together for each metric to enable side-by-side evaluation.
