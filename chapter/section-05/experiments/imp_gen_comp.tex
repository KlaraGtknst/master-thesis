\subsection{Exp. 5: Comparing \acs{av} Methods in Traditional Human-Human Scenario}
\label{subsec:imp_gen}

We want to answer the question of how our \ac{llm}-based \imp{} generation performs compared to (a) traditional \imp{} generation methods in the \impAppr{}~\citep{koppel_determining_2014}, and compared to (b) \acl{sota} \ac{av} methods in the traditional \ac{av} scenario.
We thus, create 10 same- and 10 different-author pairs from the \dataStudent{}. % (llm_detection_scenarios.py)
% We thus, create 100 same- and 100 different-author pairs from the \dataStudent{} (comp_av.py)
% 1 each on and the \dataBlog{} datasets % 1 for Blog, 100 for Student Essays
It is noteworthy, that an approach predicting only one output will obtain an accuracy of $0.5$.
The \impAppr{} and \unmasking{} detector configuration are shown in \autoref{tab:exp5_imp_config} and \autoref{tab:exp5_unmasking_config}, respectively.

\begin{table}[h]
\centering\small
\caption{Exp. 5: \impAppr{} configurations.}
\label{tab:exp5_imp_config}
\begin{tabular}{@{}rlrrl@{}}   % numbers should be right aligned, text left aligned
\toprule
\# Impostors & Generation & Rounds & Top $n$ & Upsample \\
\midrule
50 & \textit{Variable} & 100 & \num{100000} & False \\
\bottomrule
\end{tabular}%
\end{table}

\begin{table}[h]
\centering\small
\caption[Exp. 5: Unmasking configurations.]{Exp. 5: Unmasking configurations. CV denotes cross-validation.}
\label{tab:exp5_unmasking_config}
\begin{tabular}{@{}rrrrl@{}}   % numbers should be right aligned, text left aligned
\toprule
\# CV Folds & \# Chunks & Rounds & Top $n$ & Upsample \\
\midrule
3 & 60 & 30 & \num{250} & False \\
\bottomrule
\end{tabular}%
\end{table}

For each impostor generation method, we computed accuracy, precision, recall, and F1 score for different thresholds. 
