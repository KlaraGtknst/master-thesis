\subsection{Exp.\ 4: Impact of Syntactic Similarity on \impApprTitle{} Performance}
\label{sec:syn_sim_impact_}

We designed this experiment in order to assess whether the syntactic similarity of generated paraphrases, i.e.\ the difficulty of hard negatives, influences the effectiveness of the \impAppr{}.
We conducted this experiment on both the \dataBlog{} and \dataStudent{} datasets, selecting 15 samples each from the training and test splits. 
The detector was configured according to Table~\ref{tab:imp_syn_sim_config}.

\begin{table}[h]
\centering\small
\caption{Exp.\ 4: \impAppr{} configuration.}
\label{tab:imp_syn_sim_config}
\begin{tabular}{@{}rlrrl@{}}   % numbers should be right aligned, text left aligned
\toprule
\# Impostors & Generation & Rounds & Top $n$ & Upsample \\
\midrule
50 & LLM & 100 & \num{100000} & False \\
\bottomrule
\end{tabular}%
\end{table}

For generation, we loop through all \ac{llm}-based paraphrasers until we successfully created 50 \imps{}.
One-step paraphrasers are used with both prompts.
Predictions on the test set were obtained by thresholding the detector’s scores with a decision threshold was determined using Youden’s J statistic on the training set.
We computed the average syntactic similarity on the test set. 
Following \citet{gohsen_captions_2023}, we define average syntactic similarity $\diameter_{syn}$ as the mean of the \ac{bleu}, \ac{rouge}-1, and \ac{rouge}-L scores. 
For each input pair in the test set, we calculated
(1) the average syntactic similarity between the two texts in the pair, (2) the mean average syntactic similarity between the candidate reference text and its paraphrases, and (3) the mean average syntactic similarity between the disputed text and the paraphrases.

We further grouped samples based on (1), (2), (3) and the difference (2)–(1). 
For each group, we computed accuracy, precision, recall, and F1 score of the detector’s predictions. 
The average values for each metric in a bin are presented in a bar chart.

