\subsection{Exp. 6: Comparing \acs{av} Methods in \acs{llm} Author Scenarios}

This experiment evaluates the performance of the \impAppr{} relative to \acl{sota} \ac{av} methods. 
As baselines, we employ generalized \unmasking{}~\citep{bevendorff_generalizing_2019}, the compression-based \ac{ppmd} approach~\citep{stamatatos_survey_2009}, and the baselines from the original \impAppr{} study~\citep{koppel_determining_2014}. 
In contrast to Experiment 5, this experiment uses the \dataArtificialStudent{} dataset, where each text pair contains at least one \ac{llm}-generated text. 
We consider three scenarios. 

The first scenario simulates \ac{llm} detection (cf. \autoref{fig:idea_llm_detection}), where the task is to output \textit{true} if the disputed text was generated by an \ac{llm}. 
Here, candidate texts are \ac{llm}-generated, while disputed texts may be either human-authored or \ac{llm}-generated. 
A pair is labelled \textit{true} if the disputed text originates from an \ac{llm}, regardless of the specific model. 

The second scenario represents a \ac{llm}-based \ac{av} task (cf. \autoref{fig:idea_llm_AV}), using the same dataset as the first scenario, but only same-author pairs are labelled \textit{true}. 

The third scenario consists entirely of \ac{llm}-generated texts, with \textit{true} labels assigned according to the same scheme as in the second scenario (cf. \autoref{fig:idea_llm_AV}). 

\begin{figure}[htbp]
  \centering
  \begin{subfigure}{0.45\textwidth}
    \centering
    \includesvg[width=\linewidth]{images/AV_comparison/detection_scenarios/idea/LLM_detection.svg}
    \caption{\ac{llm} detection}
    \label{fig:idea_llm_detection}
  \end{subfigure}
  \hfill
  \begin{subfigure}{0.45\textwidth}
    \centering
    \includesvg[width=\linewidth]{images/AV_comparison/detection_scenarios/idea/LLM_AV.svg}
    \caption{\ac{llm} \ac{av}}
    \label{fig:idea_llm_AV}
  \end{subfigure}
  \caption[Scenarios tested in Exp. 6.]{Visual description of scenarios tested in Exp. 6.
  Green denotes \textit{true} and red denotes \textit{false}.
  }
  \label{fig:ideas_exp6}
\end{figure}

Following the experimental setup described in Experiment 5 (cf. \autoref{subsec:imp_gen}) for both \impAppr{} and \unmasking{}, we evaluate performance using accuracy, precision, recall, and F1 score.

% \begin{table}[h]
% \centering\small
% \caption{Exp. 6: \impAppr{} configurations.}
% \label{tab:av_comp}
% \begin{tabular}{@{}rlrrl@{}}   % numbers should be right aligned, text left aligned
% \toprule
% \# Impostors & Generation & Rounds & Top $n$ & Upsample \\
% \midrule
% 50 & \textit{Variable} & 100 & \num{100000} & False \\
% \bottomrule
% \end{tabular}%
% \end{table}

% \begin{table}[h]
% \centering\small
% \caption{Exp. 6: Unmasking configurations.}
% \label{tab:exp6_unmasking_config}
% \begin{tabular}{@{}rrrrl@{}}   % numbers should be right aligned, text left aligned
% \toprule
% \# CV Folds & \# Chunks & Rounds & Top $n$ & Upsample \\
% \midrule
% 3 & 60 & 30 & \num{250} & False \\
% \bottomrule
% \end{tabular}%
% \end{table}
