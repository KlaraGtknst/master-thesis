\subsection{Exp. 6: Comparing \ac{av} Methods in \acs{llm} author scenarios}

This experiment evaluates the performance of the \impAppr{} in comparison to established \ac{av} methods.
As baselines, we employ generalized unmasking~\citep{bevendorff_generalizing_2019} and the compression-based approach PPMD approach and baselines from the original work~\citep{koppel_determining_2014}.
Different to experiment 5, each input pair contains at least one \ac{llm} generated text.
We compare three scenarios.
The first one simulates \ac{llm} detection, i.e. output true if the disputed text is generated by an \ac{llm}.
We design our dataset, such that candidate texts are \ac{llm} generated and the disputed texts can be either \ac{llm} generated or human authored.
A pair receives the truth label if the disputed text is \ac{llm} generated irrespective of the model used.
The second group is a \ac{llm} \ac{av} scenario where the dataset is the same as the one before, but we only label same author pairs with true.
The third scenario contains only \ac{llm} generated texts.
We denote this a more difficult \ac{llm} \ac{av} scenario.


Following the experimental setup described in \autoref{tab:av_comp} and \autoref{tab:exp6_unmasking_config} for the \impAppr{} and unmasking respectively, we assess performance using  accuracy, precision, recall, and the F1 score. 

\begin{table}[h]
\centering\small
\caption{Exp. 6: \impAppr{} configurations.}
\label{tab:av_comp}
\begin{tabular}{@{}rlrrl@{}}   % numbers should be right aligned, text left aligned
\toprule
\# Impostors & Generation & Rounds & Top $n$ & Upsample \\
\midrule
50 & \textit{Variable} & 100 & \num{100000} & False \\
\bottomrule
\end{tabular}%
\end{table}

\begin{table}[h]
\centering\small
\caption{Exp. 6: Unmasking configurations.}
\label{tab:exp6_unmasking_config}
\begin{tabular}{@{}rrrrl@{}}   % numbers should be right aligned, text left aligned
\toprule
\# CV Folds & \# Chunks & Rounds & Top $n$ & Upsample \\
\midrule
3 & 60 & 30 & \num{250} & False \\
\bottomrule
\end{tabular}%
\end{table}