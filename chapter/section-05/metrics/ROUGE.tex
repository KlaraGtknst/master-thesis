
\paragraph{\ac{rouge}}
\ac{rouge} \citep{lin_rouge_2004}, initially developed for summarization, is recall-oriented and emphasizes coverage of reference content in the candidate text. 
Lower \ac{rouge} scores indicate greater diversity \citep{kurt_pehlivanoglu_comparative_2024}.
Several variants exist, including \ac{rouge}-N, which computes word n-gram recall, and \ac{rouge}-L, which measures the \ac{lcs}~\citep{zhou_paraphrase_2021,palivela_optimization_2021,kurt_pehlivanoglu_comparative_2024}. 

% ROUGE-N
\ac{rouge}-N is an n-gram recall between the candidate text $c$ and the reference text $r$~\citep{lin_rouge_2004}.
Since we only consider single reference scenarios we use the simpler version for one reference in \autoref{eq:rouge_n}.
\begin{equation}
    \operatorname{ROUGE-N} = \sum_{s \in r}\sum_{n \text{-} gram \in s} \frac{\operatorname{Count_{match}}(n \text{-} gram)}{\operatorname{Count}(n \text{-} gram)}
\label{eq:rouge_n}
\end{equation}
Both the nominator and the denominator iterate over all n-grams in all sentences of the reference, i.e.\ ground truth, text.
The nominator sums up $\operatorname{Count_{match}}(n \text{-} gram)$, i.e.\ the maximum number of occurrences of this reference sentence n-gram in any of the candidate texts.
Since this sum measures the number of co-occurrences of that n-gram in both candidate and reference text, it is naturally capped to the number of occurrences in the reference.
This ensures that after normalizing with the denominator, the \ac{rouge}-N values range from 0 to 1~\citep{kurt_pehlivanoglu_comparative_2024}.
% The nominator sums over all references and thus, gives more weight to matching n-grams that occur in multiple references (i.e.\ a consensus between references) \citep{lin_rouge_2004}.    % do not use, bc we have only one reference
The denominator does not consider matches but only sums up the number of times an n-gram appears in the reference sentences~\citep{lin_rouge_2004}.
If every n-gram from the reference sentences would appear equally often in the candidate, the \ac{rouge}-N would be one since it measures the n-gram overlap between reference and candidate from a reference or recall perspective.

\ac{rouge}-L for a candidate $c$ of length $n$ and a reference $r$ of length $m$ is defined in \autoref{eq:rouge_l}.
The length is measured in number of words.
The intuition is that the length of the \ac{lcs} between the candidate and reference texts correlates with their similarity.
\ac{rouge}-L does not include shorter sequences or alternative \ac{lcs} in the final score~\citep{lin_rouge_2004}.
$\beta$ is defined as $\frac{\mathrm{P_{lcs}}}{\mathrm{R_{lcs}}}$.

\begin{equation}
\mathrm{P_{lcs}} = \frac{\operatorname{LCS}(r,c)}{n},
\label{eq:rouge_l_precision}
\end{equation}

\begin{equation}
\mathrm{R_{lcs}} = \frac{\operatorname{LCS}(r,c)}{m},
\label{eq:rouge_l_recall}
\end{equation}

\begin{equation}
\operatorname{ROUGE-L} = \frac{(1 + \beta^2)  \mathrm{R_{lcs}}  \mathrm{P_{lcs}}}{\mathrm{R_{lcs}} + \beta^2  \mathrm{P_{lcs}}},
\label{eq:rouge_l}
\end{equation}

\ac{rouge}-Lsum is a summary-level \ac{lcs} \ac{rouge} variant summing over the union \ac{lcs} matches $\operatorname{LCS}_\cup(r_i,C)$ between the $u$ reference summary sentences $r_i, i \in [1,u]$ and each of the $v$ candidate sentence $c_j, j \in [1,v]$.
The total number of words in the references and candidates is $m$ and $n$, respectively.
While the formula for the final score $F_{lcs}$ from \autoref{eq:rouge_l} remains the same, the computation of precision and recall replace $\operatorname{LCS}(r,c)$ by $\sum_{i=1}^{u}\operatorname{LCS}_\cup(r_i,C)$.
