\chapter{Introduction}
\label{chap:introduction}

% motivation
\ac{av} denotes the task of determining whether two texts were written by the same author. 
It forms the foundation of broader authorship-related problems such as authorship attribution. 
Historically, \ac{av} has been applied to literary disputes, but contemporary concerns have shifted. 
In an era where large amounts of text can be copied, paraphrased, or fabricated with ease, \ac{av} has gained renewed importance in practical contexts such as plagiarism detection in digital communication.

The emergence of \acp{llm} adds an additional layer of complexity. 
While these models are widely embraced for beneficial applications such as summarization, code generation, and customer service automation, their ability to convincingly imitate human writing creates new risks. 
\acp{llm} can be exploited to generate misinformation, fabricate academic submissions, or impersonate individuals, thereby undermining trust in digital communication. 
Since \acp{llm} can be conceptualized as authors their detection naturally falls within the scope of \ac{av} rather than requiring a separate methodological framework. 
Thus, instead of treating \ac{llm} detection as an isolated task, it is more consistent to frame it as a specialized case of \ac{av}.

% specificity rather than generality
Despite significant advances in \ac{av}, many existing approaches pursue generalized solutions, training a model once and applying it across domains. 
Evidence shows that such models struggle in \ac{ood} settings, where topic or genre diverges from the training data. 
This shortcoming motivates a shift toward scenario-specific solutions, where models are trained anew for narrowly defined cases. 
Such single-case approaches enable more precise control over contextual factors and place greater emphasis on stylistic idiosyncrasies rather than domain-level variation.

% AV
Among the techniques developed for \ac{av}, the \impAppr{} by \citet{koppel_determining_2014} represents a particularly influential approach. 
It introduces the idea of generating \imp{} texts, i.e. hard negatives used to sharpen the discrimination between genuine and false authorship matches. 
However, the method's effectiveness is limited by the quality and contextual adequacy of these \imp{} texts. 
Previous work did not fully address how to construct challenging \imps{} via controlled contextual variables.

This thesis extends the \impAppr{} by leveraging \acl{sota} \acp{llm} to generate hard negative examples in a controlled scenario. 
By integrating \ac{llm}-based \imp{} generation, we bridge the gap between traditional \ac{av} and modern challenges involving \acp{llm}. 
The contribution lies not in proposing a new detection paradigm, but in enhancing an established verification method by improving its treatment of contextual factors and strengthening its ability to discriminate between genuine and artificial authorship.


\section{Research Questions}
\label{sec:research_questions}
We formulate the following research questions to investigate whether \ac{llm}-generated paraphrases can improve the effectiveness of the \imp{} method by providing better control over confounding factors than traditional hard negative sampling strategies.
\begin{questions}
    \item \textbf{How suitable are \acp{llm} for \imp{} generation?} \label{enum:rq1} \hfill \\
    For \imps{} to be effective, hard negatives need to adhere to constraints similar to the candidate text, including confounding variables such as genre, topic, and target audience.
    We employ \ac{llm}-based paraphrasing model the original confounding variables. 
    By conditioning \acp{llm} on confounding variables, the \impAppr{} emphasises differences in authorial style rather than in content. 
    However, the precise syntactic similarity threshold at which paraphrases are attributable to the candidate author remains uncertain, potentially limiting detection accuracy. 
    % To investigate the suitability of artificial \imp{} generation, we design an experiment that contrasts one-step \imp{} generation with traditional baselines using standard classification metrics.

    \item \textbf{Which metrics are appropriate to evaluate the quality of generated \imps{}?} \label{enum:rq2} \hfill \\
    Assessing \imps{} requires evaluating paraphrases, a task complicated by the absence of a universally accepted definition of paraphrasing. 
    Prior work often adopts evaluation metrics from adjacent \ac{nlp} tasks such as machine translation and summarisation~\citep{gohsen_captions_2023}. 
    The two central dimensions are semantic similarity and syntactic similarity. 
    Contrary to naive assumptions, high syntactic similarity is not always desirable, as it may indicate minimal modification of the original text.
    Instead, the objective is to achieve strong semantic similarity with sufficient syntactic variation, typically measured by automatic scores, while a subset of samples additionally validated through human judgments of adequacy.
    % We therefore construct experiments to evaluate paraphrases using state-of-the-art measures for paraphrase quality. 

    \item \textbf{How does the \ac{llm}-based \impAppr{} perform compared to established \ac{av} methods?} \label{enum:rq3} \hfill \\
    To benchmark our approach, we compare it against baselines introduced by \citet{koppel_determining_2014}, \unmasking{}, and \acs{ppmd}.
    % , \textcolor{red}{as well as \unmasking{} and compression-based \ac{ppmd}}. 
    The evaluation is conducted on human-authored text pairs from the \dataStudent{} dataset used in the original study, measuring performance in terms of precision and recall.
    
\end{questions}

\newpage
\section{Contributions}
\label{sec:contributions}
The main contributions of this thesis are as follows:
\begin{itemize}
    \item We collected and preprocessed the original datasets, including a controlled candidate-pair selection procedure aligned with the methodology of \citet{koppel_determining_2014} and informed by direct consultation with the authors (cf.~\Cref{sec:dataset}).
    \item This work reimplements the traditional \impAppr{} \citep{koppel_determining_2014}\ (cf.~\Cref{sec:impostor_method_theory}).
    \item It extends the \impAppr{} with \ac{llm}-based hard negative generation proposing scenario-specific solution to \ac{av} problems (cf.~\Cref{sec:impostor_generation}). 
    \item We further analyse prompt design, its influence on paraphrase length as a confounding variable (cf.~\Cref{subsec:prompt_impact_res}/~\ref{enum:rq1} and \ref{enum:rq2}) and on quantitative metrics adapted from summarisation and translation tasks (hereafter paraphrasing scores) (cf.~\Cref{sec:comp_paraphrases}/~\ref{enum:rq2}).
    We also investigate the effect of text segmentation on paraphrasing scores (cf.~\Cref{sec:results_chunks}/~\ref{enum:rq2}).
    \item We compare the proposed \ac{llm}-based \impAppr{} with established \ac{av} techniques, specifically the baselines introduced by \citet{koppel_determining_2014}, \unmasking{} and \acs{ppmd}, 
    using pairs of human-authored texts in the traditional \ac{av} setting (cf.~\Cref{subsec:imp_gen_res}/~\ref{enum:rq3}).
   
\end{itemize}



\section{Thesis Structure}
\label{sec:thesis_structure}

The remainder of this thesis is organised as follows: 
To begin, \Cref{chap:authorship_identification} provides the background on \ac{av}, covering early approaches, state-of-the-art methods, stylometry, and practical applications.
\Cref{chap:related_work} then examines related research in greater detail.
Building on this, \Cref{chap:llm_impostor_method} introduces our extension to the \impAppr{}.
The experimental work is detailed in \Cref{chap:experimental_setup}, with \Cref{chap:experimental_results} presenting the corresponding results.
Finally, \Cref{chap:discussion} interprets the findings, and \Cref{chap:conclusion} summarises the thesis while outlining directions for future research.
