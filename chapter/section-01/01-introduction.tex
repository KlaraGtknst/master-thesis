\chapter{Introduction}
\label{chap:introduction}



% motivation
Historically, authorship analysis focused on literary disputes~\citep{neal_surveying_2018,stamatatos_survey_2009}, but contemporary concerns have shifted towards practical applications.
In an era where large amounts of text can be copied, paraphrased, or fabricated with ease, determining the true author of a text is crucial for maintaining trust in communication. 
Scenarios include detecting plagiarised passages of texts~\citep{stein_intrinsic_2011}, and verifying the authenticity of online content or student submissions. 
Formally, we refer to these problems as \acf{av} or \acf{aa}, where every \ac{aa} task can be formulated as a sequence of \ac{av} problems~\citep{tyo_state_2022,barlas_cross_domain_2020}.

The emergence of \acp{llm} adds an additional layer of complexity. 
While these models are widely embraced for beneficial applications such as summarisation, information seeking, and assistive writing~\citep{wang_stumbling_2024}, their ability to convincingly imitate human writing creates new risks. 
\acp{llm} can be used to generate misinformation, impair academic honesty, or impersonate individuals, thereby inflicting harm on individuals who fall victim to these schemes~\citep{mitchell_detectgpt_2023,li_learning_2025,wang_stumbling_2024,bhattacharjee_fighting_2024}. 
Since \acp{llm} can be conceptualised as authors, their detection naturally falls within the scope of \ac{av}. 
Thus, instead of treating \ac{llm} detection as an isolated task, it is more consistent to frame it as a specialised case of \ac{av}~\citep{llm_detection_av_2025}.

% specificity rather than generality
Existing approaches to generalisation typically train a single model and apply it across domains.
Despite significant advances in \ac{av}, prior work finds that such models struggle in \ac{ood} settings, where the topic or genre diverges from the training data~\citep{Sundararajan_style_18,bischoff_importance_2020,li_learning_2025}. 
This shortcoming motivates a shift towards scenario-specific solutions, i.e.\ models are trained anew for narrowly defined cases. 
Such single-case approaches enable more precise control over contextual factors and place greater emphasis on stylistic idiosyncrasies rather than domain-level variation.

% AV
The \impAppr{} by \citet{koppel_determining_2014}\ introduces the idea of generating \imp{} texts, i.e.\ hard negatives, used to sharpen the discrimination between genuine and false authorship matches. 
However, the method's effectiveness is limited by the quality and contextual adequacy of these \imp{} texts. 
Previous work did not fully address how to construct challenging \imps{} via controlled contextual variables.

The thesis extends the \impAppr{} by leveraging \acl{sota} \acp{llm} to generate paraphrases as \imps{}, enabling control over multiple confounding factors such as genre, topic, and target audience. 
In doing so, the approach shifts the focus towards authorial style rather than domain differences, yielding improved precision–recall on the \dataStudent{} dataset compared with the original sampling strategies, \unmasking{} and \acs{ppmd}.


\section{Research Questions}
\label{sec:research_questions}
To guide this objective, we formulate the following research questions:
\begin{questions}
    \item \textbf{What is the suitability of \acp{llm} for \imp{} generation?} \label{enum:rq1} \hfill \\
    Optimal \imps{} function as hard negatives in \ac{av}.
    By generating paraphrases with \acp{llm} that control genre, topic, and other contextual factors, similarity measures in the \imp{} method primarily emphasize differences in authorial style rather than content.
    However, the threshold of syntactic similarity at which paraphrases are attributed to the candidate author remains unclear, potentially reducing detection accuracy.
    We design an experiment comparing simple one-step and advanced two-step \ac{llm} paraphrasing techniques against traditional baselines using standard classification metrics.
    Additionally, we analyse how syntactic similarity between the candidate reference and \ac{llm}-generated paraphrases affects detection performance.
    % We obtain this controlled environment by utilizing \acp{llm} to paraphrase the original text.
    % There are different approaches to paraphrasing text using \acp{llm}.
    % They include (a) directly asking the \ac{llm} to paraphrase the text, 
    % (b) first extracting specific information from the original text and subsequently generating a paraphrase based on the information.
    % This thesis compares both approaches on \dataStudent{}, \dataBlog{}, \dataGutenberg{} and \dataPan{}.

    \item \textbf{Which metrics are appropriate to evaluate the quality of generated \imps{}?} \label{enum:rq2} \hfill \\
    Optimal \imps{} are generated under the same constraints as the candidate text.
    We use \ac{llm}-generated paraphrases as a heuristic of the original generation process.
    Thus, evaluating \imps{} equates to evaluating paraphrases, a task complicated by the lack of a universal definition of paraphrasing.
    Previous studies often adapt metrics from related \ac{nlp} tasks such as machine translation and summarization.
    Two primary evaluation dimensions are semantic similarity and syntactic similarity.
    Contrary to initial assumptions, high syntactic similarity is not always desirable, as it may indicate minimal modification of the original text.
    Our emphasis is on achieving high semantic similarity while preserving syntactic diversity to ensure genuine rephrasing.
    Moreover, relatively low scores can be acceptable if qualitative human assessments confirm paraphrase adequacy.
    We construct an experiment to compare both our prompts and paraphrasers in terms of state-of-the-art paraphrasing measures. 

    % \item \textbf{Which features are used for the \ac{av} problem?} \label{enum:rq3} \hfill \\
    % Traditional features include character tri-gram features, while newer research has proposed using \ac{llm} such as BERT.

    \item \textbf{How does the \ac{llm}-based \impAppr{} perform compared to established \ac{av} methods?} \label{enum:rq3} \hfill \\
    We use Unmasking and PPMD as baselines to evaluate our \ac{llm}-based \impAppr{}.
    Although computationally intensive, our method is designed as a solution tailored to specific detection tasks  rather than a general-purpose \ac{llm} detection tool.
    Performance is assessed across three scenarios: (a) human-\ac{llm} input pairs,
    (b) \ac{llm}-\ac{llm} pairs using the same model, and (c) \ac{llm}-\ac{llm} pairs with different models.
    Comparisons are conducted on the \dataStudent{}, \dataBlog{}, \dataGutenberg{}, and \dataPan{} datasets using standard classification metrics.
    
\end{questions}

% \section*{Idea}
% \label{sec:idea}

% Given a text of unknown authorship (i.e., human or \ac{llm}), 
% construct a set of \imp{} texts using state-of-the-art \acp{llm} based on the original text.
% Obtain the author by \ac{aa}/ \ac{av} methods, such as unmasking, to \textit{confidently}, i.e. high precision, identify \ac{llm} generated texts
% (and possibly which \ac{llm}).



\section{Contributions}
\label{sec:contributions}
The main contributions of this thesis are as follows:
\begin{itemize}
    \item We reimplemented the traditional \imp{} method~\citep{koppel_determining_2014} (cf. \autoref{sec:impostor_method_theory}).
    \item Our work includes developing an implementation of \ac{llm}-based \imp{} generation (cf. \autoref{chap:llm_impostor_method}). 
    \item To evaluate the use of \acp{llm} for generating hard negative \imps{}, we conducted an empirical assessment focusing on (a) the influence of syntactic similarity between the reference candidate and the paraphrase, and (b) performance relative to traditional \imp{} generation techniques proposed by \citet{koppel_determining_2014} (cf. \autoref{chap:experimental_results}/ \ref{enum:rq1})
    \item We compared paraphrasing approaches quantitatively using standard summarization and translation-based techniques (cf. \autoref{chap:experimental_results}/ \ref{enum:rq2})
    \item The proposed \ac{llm}-based \imp{} method was evaluated against established \ac{av} techniques, namely Unmasking and PPMD (cf. \autoref{chap:experimental_results}/ \ref{enum:rq3})
    \item We analysed how chunked paraphrases affect detector scores by varying the number of chunks and assessing detection performance (cf. \autoref{chap:experimental_results}).
    \item We collected and preprocessed the original datasets, including a deliberate candidate-pair selection procedure aligned with the methodology of the original study by \citet{koppel_determining_2014}, informed through direct consultation with the original authors (cf. \autoref{sec:dataset}).
\end{itemize}

\textcolor{red}{correct chapter refernces}


\section{Thesis Structure}
\label{sec:thesis_structure}
The thesis is structured as follows:
\textcolor{red}{TODO}