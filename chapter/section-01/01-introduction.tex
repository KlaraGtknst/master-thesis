\chapter{Introduction}
\label{chap:introduction}

\section{Problem}
\label{sec:problem}

Given a set of text-author pairs and a text of unknown authorship, 
the task at hand is to determine whether text was written by a human or a \ac{lm}.
% AV
This problem can be framed as a \ac{av} task, where the core problem stripped of all metadata 
is to determine whether two text were written by the same author or not.
% Genre
It can also be framed as a genre classification task, where the goal is to determine whether a text belongs 
to the genre human or to the genre \ac{lm} \todo{Martins neues Paper}.
% One-class classification
It can be framed as a one-class classification task, where all \ac{lm} authors are considered as one class.
The goal is to identify \ac{lm} generated texts with high confidence/ \todo{precision?}.

\section{Research Questions}
\label{sec:research_questions}

\begin{questions}
    \item \textbf{Given the text of candidate author, how can we instruct a \ac{llm} via prompting to paraphrase the text such that it has other stylistic properties (i.e., those of the \ac{llm}) but the same content?} \label{enum:rq1} \hfill \\
    The goal is to create hard negatives for the \ac{av} task.
    Imposter texts are generated by letting \acp{llm} paraphrase the original text.
    Paraphrases should have the same genre and topic as the original text, 
    since stylistic elements are influenced by those elements and thus, 
    the classification problem would be too easy in a cross-genre approach 
    due to similarity measures representing every aspect without the option to differentiate topic or genre influences.
    There are different approaches to paraphrasing text using \acp{llm}.
    They include (a) directly asking the \ac{llm} to paraphrase the text, 
    (b) first obtaining bullet-points of the texts' content, as well as its genre, and using them as input to the \ac{llm} with the task to generate a text based on these bulletpoints, 
    or (c) first obtaining thematic keywords or extraction of the text's content, and then using them as input to the \ac{llm} with the task to generate a text based on these keywords.
    %using a chain-of-thought prompting approach.
    This thesis will compare these approaches and possibly more.
    It will also include a study on how similar the generated text should be to the original text.
    The approaches will tested on the dataset of \citet{koppel_determining_2014} and \ac{pan}.

    \item \textbf{Given an original text and its paraphrase, how do we evaluate the quality of the parapharse?} \label{enum:rq2} \hfill \\
    The evaluation of the quality of the paraphrase includes keeping original content, while changing the style.

    \item \textbf{Which features are used for the \ac{av} problem?} \label{enum:rq3} \hfill \\
    Traditional features include character tri-gram features, while newer research has proposed using \ac{llm} such as BERT.

    \item \textbf{How does the imposter approach perform compared to state-of-the-art models submitted at \ac{pan}?} \label{enum:rq4} \hfill \\
    The evaluation will use \ac{pan} 2024/2025 data/ models.
    Our approach will most likely be more computationally expensive.
\end{questions}

\section{Idea}
\label{sec:idea}

Given a text of unknown authorship (i.e., human or \ac{llm}), 
construct a set of imposter texts using state-of-the-art \acp{llm} based on the original text.
Obtain the author by \ac{aa}/ \ac{av} methods, such as unmasking, to \textit{confidently}, i.e. high precision, identify \ac{llm} generated texts
(and possibly which \ac{llm}).
% Experiments
Construct experiments where
\begin{enumerate}
    \item the disputed text is human generated,
    \item the disputed text is \ac{llm} generated,
    \item the disputed text is \ac{llm} generated but the \ac{llm} is not part of the imposters.
    \item More?
\end{enumerate}


\section{Contributions}
\label{sec:contributions}
The contributions of this thesis are: