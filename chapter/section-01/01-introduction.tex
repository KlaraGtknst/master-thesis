\chapter{Introduction}
\label{chap:introduction}

% motivation
Historically, authorship analysis focused on literary disputes~\citep{neal_surveying_2018,stamatatos_survey_2009}, but contemporary concerns have shifted towards practical applications.
In an era where large amounts of text can be copied, paraphrased, or fabricated with ease, determining the true author of a text is crucial for maintaining trust in communication. 
Scenarios include detecting plagiarised passages of texts~\citep{stein_intrinsic_2011}, and verifying the authenticity of online content or student submissions. 
Formally, we refer to these problems as \acf{av} or \acf{aa}, where every \ac{aa} task can be formulated as a sequence of \ac{av} problems~\citep{tyo_state_2022,barlas_cross_domain_2020}.

The emergence of \acp{llm} adds an additional layer of complexity. 
While these models are widely embraced for beneficial applications such as summarisation, information seeking, and assistive writing~\citep{wang_stumbling_2024}, their ability to convincingly imitate human writing creates new risks. 
\acp{llm} can be used to generate misinformation, impair academic honesty, or impersonate individuals, thereby inflicting harm on individuals who fall victim to these schemes~\citep{mitchell_detectgpt_2023,li_learning_2025,wang_stumbling_2024,bhattacharjee_fighting_2024}. 
Since \acp{llm} can be conceptualised as authors, their detection naturally falls within the scope of \ac{av}. 
Thus, instead of treating \ac{llm} detection as an isolated task, it is more consistent to frame it as a specialised case of \ac{av}~\citep{llm_detection_av_2025}.

% specificity rather than generality
Existing approaches to generalisation typically train a single model and apply it across domains.
Despite significant advances in \ac{av}, prior work finds that such models struggle in \ac{ood} settings, where the topic or genre diverges from the training data~\citep{Sundararajan_style_18,bischoff_importance_2020,li_learning_2025}. 
This shortcoming motivates a shift towards scenario-specific solutions, i.e.\ models are trained anew for narrowly defined cases. 
Such single-case approaches enable more precise control over contextual factors and place greater emphasis on stylistic idiosyncrasies rather than domain-level variation.

% AV
The \impAppr{} by \citet{koppel_determining_2014}\ introduces the idea of generating \imp{} texts, i.e.\ hard negatives, used to sharpen the discrimination between genuine and false authorship matches. 
However, the method's effectiveness is limited by the quality and contextual adequacy of these \imp{} texts. 
Previous work did not fully address how to construct challenging \imps{} via controlled contextual variables.

The thesis extends the \impAppr{} by leveraging \acl{sota} \acp{llm} to generate paraphrases as \imps{}, enabling control over multiple confounding factors such as genre, topic, and target audience. 
In doing so, the approach shifts the focus towards authorial style rather than domain differences, yielding improved precision–recall on the \dataStudent{} dataset compared with the original sampling strategies, \unmasking{} and \acs{ppmd}.


\section{Research Questions}
\label{sec:research_questions}
To address the overarching objective of this thesis, we formulate the following research questions:
\begin{questions}
    \item \textbf{How suitable are \acp{llm} for \imp{} generation?} \label{enum:rq1} \hfill \\
    Effective \imps{}, i.e. hard negatives, must be generated under constraints comparable to those of the candidate text. 
    We employ \ac{llm}-based paraphrasing as a proxy for the original generative process. 
    By conditioning \acp{llm} on genre, topic, and other contextual parameters, the \imp{} method emphasises differences in authorial style rather than in content. 
    However, the precise syntactic similarity threshold at which paraphrases are attributable to the candidate author remains uncertain, potentially limiting detection accuracy. 
    To investigate the suitability of artificial \imp{} generation, we design an experiment that contrasts one-step \imp{} generation with traditional baselines using standard classification metrics.

    \item \textbf{Which metrics are appropriate to evaluate the quality of generated \imps{}?} \label{enum:rq2} \hfill \\
    Assessing \imps{} requires evaluating paraphrases, a task complicated by the absence of a universally accepted definition of paraphrasing. 
    Prior work often adopts evaluation metrics from adjacent \ac{nlp} tasks such as machine translation and summarisation. 
    The two central dimensions are semantic similarity and syntactic similarity. 
    Contrary to naive assumptions, high syntactic similarity is not always desirable, as it may indicate minimal modification of the original text.
    Instead, the objective is to achieve strong semantic alignment while ensuring sufficient syntactic variation to capture genuine rephrasing. 
    Furthermore, relatively low automatic scores may still be acceptable if supported by human judgments of adequacy. 
    We therefore construct experiments to evaluate paraphrases using \acl{sota} measures for paraphrase quality. 

    \item \textbf{How does the \ac{llm}-based \impAppr{} perform compared to \acl{sota} \ac{av} methods?} \label{enum:rq3} \hfill \\
    To benchmark our approach, we compare it against established baselines introduced by \citet{koppel_determining_2014}.
    % , \textcolor{red}{as well as \unmasking{} and compression-based \ac{ppmd}}. 
    The evaluation is conducted on human-authored text pairs from the \dataStudent{} dataset, applying standard classification metrics. 
    
\end{questions}



\section{Contributions}
\label{sec:contributions}
The main contributions of this thesis are as follows:
\begin{itemize}
    \item Collection and preprocessing of original datasets, including a controlled candidate-pair selection procedure aligned with the methodology of \citet{koppel_determining_2014} and informed by direct consultation with the authors (cf.~\Cref{sec:dataset}).
    \item Reimplementation of the traditional \impAppr{}~\citep{koppel_determining_2014}\ (cf.~\Cref{sec:impostor_method_theory}).
    \item Development of an \ac{llm}-based \impAppr{} extending the work of \citet{koppel_determining_2014} (cf.~\Cref{sec:impostor_generation}). 
    % \item Empirical evaluation of \ac{llm}-generated hard negatives, with a focus on prompts influencing the control of confounder variables during the artificial \imp{} generation process (cf.~\Cref{subsec:prompt_impact_res}/~\ref{enum:rq1}).
    \item Analysis of prompt design and its influence on paraphrase length as a confounding variable (cf.~\Cref{subsec:prompt_impact_res}/~\ref{enum:rq1} and \ref{enum:rq2}).
    \item Quantitative comparison of paraphrasing approaches using metrics adapted from summarisation and translation tasks (hereafter paraphrasing scores) (cf.~\Cref{sec:comp_paraphrases}/~\ref{enum:rq2}).
    \item Analysis of the effect of text segmentation (i.e. number of chunks) on paraphrasing scores (cf.~\Cref{sec:results_chunks}/~\ref{enum:rq2}).
    \item Evaluation of the proposed \ac{llm}-based \impAppr{} against established \ac{av} techniques, specifically the baselines introduced by \citet{koppel_determining_2014}
    % , \textcolor{red}{\unmasking{} and \ac{ppmd}}, 
    within the traditional \ac{av} setting using pairs of human-authored texts (cf.~\Cref{subsec:imp_gen_res}/~\ref{enum:rq3}).
   
\end{itemize}



\section{Thesis Structure}
\label{sec:thesis_structure}

The remainder of this thesis is organised as follows. 
\Cref{chap:authorship_identification} introduces the background of \ac{av}, including early approaches, \acl{sota} methods, stylometry, and applications. 
\Cref{chap:related_work} surveys related research in greater detail. 
\Cref{chap:llm_impostor_method} presents our extension to the \impAppr{}. 
\Cref{chap:experimental_setup} outlines the experimental setup, and \Cref{chap:experimental_results} reports the results. 
Finally, \Cref{chap:discussion} interprets the findings, and \Cref{chap:conclusion} concludes the thesis while outlining future research directions. 
