\chapter{Introduction}
\label{chap:introduction}



% motivation
Historically, authorship analysis focused on literary disputes~\citep{neal_surveying_2018,stamatatos_survey_2009}, but contemporary concerns have shifted towards practical applications.
In an era where large amounts of text can be copied, paraphrased, or fabricated with ease, determining the true author of a text is crucial for maintaining trust in communication. 
Scenarios include detecting plagiarised passages of texts~\citep{stein_intrinsic_2011}, and verifying the authenticity of online content or student submissions. 
Formally, we refer to these problems as \acf{av} or \acf{aa}, where every \ac{aa} task can be formulated as a sequence of \ac{av} problems~\citep{tyo_state_2022,barlas_cross_domain_2020}.

The emergence of \acp{llm} adds an additional layer of complexity. 
While these models are widely embraced for beneficial applications such as summarisation, information seeking, and assistive writing~\citep{wang_stumbling_2024}, their ability to convincingly imitate human writing creates new risks. 
\acp{llm} can be used to generate misinformation, impair academic honesty, or impersonate individuals, thereby inflicting harm on individuals who fall victim to these schemes~\citep{mitchell_detectgpt_2023,li_learning_2025,wang_stumbling_2024,bhattacharjee_fighting_2024}. 
Since \acp{llm} can be conceptualised as authors, their detection naturally falls within the scope of \ac{av}. 
Thus, instead of treating \ac{llm} detection as an isolated task, it is more consistent to frame it as a specialised case of \ac{av}~\citep{llm_detection_av_2025}.

% specificity rather than generality
Existing approaches to generalisation typically train a single model and apply it across domains.
Despite significant advances in \ac{av}, prior work finds that such models struggle in \ac{ood} settings, where the topic or genre diverges from the training data~\citep{Sundararajan_style_18,bischoff_importance_2020,li_learning_2025}. 
This shortcoming motivates a shift towards scenario-specific solutions, i.e.\ models are trained anew for narrowly defined cases. 
Such single-case approaches enable more precise control over contextual factors and place greater emphasis on stylistic idiosyncrasies rather than domain-level variation.

% AV
The \impAppr{} by \citet{koppel_determining_2014}\ introduces the idea of generating \imp{} texts, i.e.\ hard negatives, used to sharpen the discrimination between genuine and false authorship matches. 
However, the method's effectiveness is limited by the quality and contextual adequacy of these \imp{} texts. 
Previous work did not fully address how to construct challenging \imps{} via controlled contextual variables.

The thesis extends the \impAppr{} by leveraging \acl{sota} \acp{llm} to generate paraphrases as \imps{}, enabling control over multiple confounding factors such as genre, topic, and target audience. 
In doing so, the approach shifts the focus towards authorial style rather than domain differences, yielding improved precision–recall on the \dataStudent{} dataset compared with the original sampling strategies, \unmasking{} and \acs{ppmd}.


\section*{Research Questions}
\label{sec:research_questions}

\begin{questions}
    \item \textbf{Given the text of candidate author, how can we instruct a \ac{llm} via prompting to paraphrase the text such that it has other stylistic properties (i.e., those of the \ac{llm}) but the same content?} \label{enum:rq1} \hfill \\
    The goal is to create hard negatives for the \ac{av} task.
    Imposter texts are generated by letting \acp{llm} paraphrase the original text.
    Paraphrases should have the same genre and topic as the original text, 
    since stylistic elements are influenced by those elements and thus, 
    the classification problem would be too easy in a cross-genre approach 
    due to similarity measures representing every aspect without the option to differentiate topic or genre influences.
    There are different approaches to paraphrasing text using \acp{llm}.
    They include (a) directly asking the \ac{llm} to paraphrase the text, 
    (b) first obtaining bullet-points of the texts' content, as well as its genre, and using them as input to the \ac{llm} with the task to generate a text based on these bulletpoints, 
    or (c) first obtaining thematic keywords or extraction of the text's content, and then using them as input to the \ac{llm} with the task to generate a text based on these keywords.
    %using a chain-of-thought prompting approach.
    This thesis will compare these approaches and possibly more.
    It will also include a study on how similar the generated text should be to the original text.
    The approaches will tested on the dataset of \citet{koppel_determining_2014} and \ac{pan}.

    \item \textbf{Given an original text and its paraphrase, how do we evaluate the quality of the parapharse?} \label{enum:rq2} \hfill \\
    The evaluation of the quality of the paraphrase includes keeping original content, while changing the style.

    \item \textbf{Which features are used for the \ac{av} problem?} \label{enum:rq3} \hfill \\
    Traditional features include character tri-gram features, while newer research has proposed using \ac{llm} such as BERT.

    \item \textbf{How does the imposter approach perform compared to state-of-the-art models submitted at \ac{pan}?} \label{enum:rq4} \hfill \\
    The evaluation will use \ac{pan} 2024/2025 data/ models.
    Our approach will most likely be more computationally expensive.
\end{questions}

\section*{Idea}
\label{sec:idea}

Given a text of unknown authorship (i.e., human or \ac{llm}), 
construct a set of imposter texts using state-of-the-art \acp{llm} based on the original text.
Obtain the author by \ac{aa}/ \ac{av} methods, such as unmasking, to \textit{confidently}, i.e. high precision, identify \ac{llm} generated texts
(and possibly which \ac{llm}).
% Experiments
Construct experiments where
\begin{enumerate}
    \item the disputed text is human generated,
    \item the disputed text is \ac{llm} generated,
    \item the disputed text is \ac{llm} generated but the \ac{llm} is not part of the imposters.
    \item More?
\end{enumerate}


\section*{Contributions}
\label{sec:contributions}
The contributions of this thesis are:
\begin{enumerate}
    \item Reimplementation of the traditional Imposter approach.
    \item Extension of the imposter approach with \ac{llm} generated imposters for line-up of difficult opponents. 
    \item Frame \ac{llm} detection as a \ac{av} problem: Use \ac{llm} generated text as candidate for text of "unknown" authorship.
\end{enumerate}
