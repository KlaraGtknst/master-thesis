\chapter{Introduction}
\label{chap:introduction}



% motivation
\ac{av} denotes the task of determining whether two texts were written by the same author. 
It forms the foundation of broader authorship-related problems such as authorship attribution. 
Historically, \ac{av} has been applied to literary disputes, but contemporary concerns have shifted. 
In an era where large amounts of text can be copied, paraphrased, or fabricated with ease, \ac{av} has gained renewed importance in practical contexts such as plagiarism detection in digital communication.

The emergence of \acp{llm} adds an additional layer of complexity. 
While these models are widely embraced for beneficial applications such as summarization, code generation, and customer service automation, their ability to convincingly imitate human writing creates new risks. 
\acp{llm} can be exploited to generate misinformation, fabricate academic submissions, or impersonate individuals, thereby undermining trust in digital communication. 
Since \acp{llm} can be conceptualized as authors their detection naturally falls within the scope of \ac{av} rather than requiring a separate methodological framework. 
Thus, instead of treating \ac{llm} detection as an isolated task, it is more consistent to frame it as a specialized case of \ac{av}.

% specificity rather than generality
Despite significant advances in \ac{av}, many existing approaches pursue generalized solutions, training a model once and applying it across domains. 
Evidence shows that such models struggle in \ac{ood} settings, where topic or genre diverges from the training data. 
This shortcoming motivates a shift toward scenario-specific solutions, where models are trained anew for narrowly defined cases. 
Such single-case approaches enable more precise control over contextual factors and place greater emphasis on stylistic idiosyncrasies rather than domain-level variation.

% AV
Among the techniques developed for \ac{av}, the \impAppr{} by \citet{koppel_determining_2014} represents a particularly influential approach. 
It introduces the idea of generating \imp{} texts, i.e. hard negatives used to sharpen the discrimination between genuine and false authorship matches. 
However, the method's effectiveness is limited by the quality and contextual adequacy of these \imp{} texts. 
Previous work did not fully address how to construct challenging \imps{} via controlled contextual variables.

This thesis extends the \impAppr{} by leveraging \acl{sota} \acp{llm} to generate hard negative examples in a controlled scenario. 
By integrating \ac{llm}-based \imp{} generation, we bridge the gap between traditional \ac{av} and modern challenges involving \acp{llm}. 
The contribution lies not in proposing a new detection paradigm, but in enhancing an established verification method by improving its treatment of contextual factors and strengthening its ability to discriminate between genuine and artificial authorship.


% \section*{Research Questions}
% \label{sec:research_questions}
To guide this objective, we formulate the following research questions:
\begin{questions}
    \item \textbf{Given the text of a candidate author, how can we instruct a \ac{llm} via prompting to paraphrase the text such that it has the \ac{llm}'s stylistic properties and the same content?} \label{enum:rq1} \hfill \\
    The goal is to create hard negatives for the \ac{av} task.
    Imposter texts are generated by letting \acp{llm} paraphrase the original text.
    Paraphrases should have the same genre and topic as the original text, 
    since stylistic elements are influenced by those elements and thus, 
    the classification problem would be too easy in a cross-genre approach 
    due to similarity measures representing every aspect without the option to differentiate topic or genre influences.
    There are different approaches to paraphrasing text using \acp{llm}.
    They include (a) directly asking the \ac{llm} to paraphrase the text, 
    (b) first obtaining bullet-points of the texts' content, as well as its genre, and using them as input to the \ac{llm} with the task to generate a text based on these bulletpoints, 
    or (c) first obtaining thematic keywords or extraction of the text's content, and then using them as input to the \ac{llm} with the task to generate a text based on these keywords.
    %using a chain-of-thought prompting approach.
    This thesis will compare these approaches and possibly more.
    It will also include a study on how similar the generated text should be to the original text.
    The approaches will tested on the dataset of \citet{koppel_determining_2014} and \ac{pan}.

    \item \textbf{Given an original text and its paraphrase, how do we evaluate the quality of the parapharse?} \label{enum:rq2} \hfill \\
    The evaluation of the quality of the paraphrase includes keeping original content, while changing the style.

    \item \textbf{Which features are used for the \ac{av} problem?} \label{enum:rq3} \hfill \\
    Traditional features include character tri-gram features, while newer research has proposed using \ac{llm} such as BERT.

    \item \textbf{How does the impostor approach perform compared to state-of-the-art models submitted at \ac{pan}?} \label{enum:rq4} \hfill \\
    The evaluation will use \ac{pan} 2024/2025 data/ models.
    Our approach will most likely be more computationally expensive.
\end{questions}

\section*{Idea}
\label{sec:idea}

Given a text of unknown authorship (i.e., human or \ac{llm}), 
construct a set of impostor texts using state-of-the-art \acp{llm} based on the original text.
Obtain the author by \ac{aa}/ \ac{av} methods, such as unmasking, to \textit{confidently}, i.e. high precision, identify \ac{llm} generated texts
(and possibly which \ac{llm}).
% Experiments
Construct experiments where
\begin{enumerate}
    \item the disputed text is human generated,
    \item the disputed text is \ac{llm} generated,
    \item the disputed text is \ac{llm} generated but the \ac{llm} is not part of the impostors.
    \item More?
\end{enumerate}


\section*{Contributions}
\label{sec:contributions}
The contributions of this thesis are:
\begin{enumerate}
    \item Reimplementation of the traditional Imposter approach.
    \item Extension of the impostor approach with \ac{llm} generated impostors for line-up of difficult opponents. 
    \item Frame \ac{llm} detection as a \ac{av} problem: Use \ac{llm} generated text as candidate for text of "unknown" authorship.
\end{enumerate}
