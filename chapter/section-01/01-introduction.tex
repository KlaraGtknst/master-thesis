\chapter{Introduction}
\label{chap:introduction}



% motivation
Historically, authorship analysis focused on literary disputes~\citep{neal_surveying_2018,stamatatos_survey_2009}, but contemporary concerns have shifted towards practical applications.
In an era where large amounts of text can be copied, paraphrased, or fabricated with ease, determining the true author of a text is crucial for maintaining trust in communication. 
Scenarios include detecting plagiarised passages of texts~\citep{stein_intrinsic_2011}, and verifying the authenticity of online content or student submissions. 
Formally, we refer to these problems as \acf{av} or \acf{aa}, where every \ac{aa} task can be formulated as a sequence of \ac{av} problems~\citep{tyo_state_2022,barlas_cross_domain_2020}.

The emergence of \acp{llm} adds an additional layer of complexity. 
While these models are widely embraced for beneficial applications such as summarisation, information seeking, and assistive writing~\citep{wang_stumbling_2024}, their ability to convincingly imitate human writing creates new risks. 
\acp{llm} can be used to generate misinformation, impair academic honesty, or impersonate individuals, thereby inflicting harm on individuals who fall victim to these schemes~\citep{mitchell_detectgpt_2023,li_learning_2025,wang_stumbling_2024,bhattacharjee_fighting_2024}. 
Since \acp{llm} can be conceptualised as authors, their detection naturally falls within the scope of \ac{av}. 
Thus, instead of treating \ac{llm} detection as an isolated task, it is more consistent to frame it as a specialised case of \ac{av}~\citep{llm_detection_av_2025}.

% specificity rather than generality
Existing approaches to generalisation typically train a single model and apply it across domains.
Despite significant advances in \ac{av}, prior work finds that such models struggle in \ac{ood} settings, where the topic or genre diverges from the training data~\citep{Sundararajan_style_18,bischoff_importance_2020,li_learning_2025}. 
This shortcoming motivates a shift towards scenario-specific solutions, i.e.\ models are trained anew for narrowly defined cases. 
Such single-case approaches enable more precise control over contextual factors and place greater emphasis on stylistic idiosyncrasies rather than domain-level variation.

% AV
The \impAppr{} by \citet{koppel_determining_2014}\ introduces the idea of generating \imp{} texts, i.e.\ hard negatives, used to sharpen the discrimination between genuine and false authorship matches. 
However, the method's effectiveness is limited by the quality and contextual adequacy of these \imp{} texts. 
Previous work did not fully address how to construct challenging \imps{} via controlled contextual variables.

The thesis extends the \impAppr{} by leveraging \acl{sota} \acp{llm} to generate paraphrases as \imps{}, enabling control over multiple confounding factors such as genre, topic, and target audience. 
In doing so, the approach shifts the focus towards authorial style rather than domain differences, yielding improved precision–recall on the \dataStudent{} dataset compared with the original sampling strategies, \unmasking{} and \acs{ppmd}.


% \section*{Research Questions}
% \label{sec:research_questions}
To guide this objective, we formulate the following research questions:
\begin{questions}
    \item \textbf{How can we instruct a \ac{llm} to paraphrase the text of a candidate author such that it captures the \ac{llm}'s stylistic properties?} \label{enum:rq1} \hfill \\
    The goal is to create hard negatives for the Impostor method by controlling contextual factors.
    By controlling genre, topic and other factors, similarity measures primarily focus on differences in authorial style rather than the impact of content on style.
    We obtain this controlled environment by utilizing \acp{llm} to paraphrase the original text.
    There are different approaches to paraphrasing text using \acp{llm}.
    They include (a) directly asking the \ac{llm} to paraphrase the text, 
    (b) first extracting specific information from the original text and subsequently generating a paraphrase based on the information.
    This thesis compares both approaches on \dataStudent{}, \dataBlog{}, \dataGutenberg{} and \dataPan{}.

    \item \textbf{How do we evaluate the quality of paraphrases?} \label{enum:rq2} \hfill \\
    Paraphrase evaluation is inherently challenging, as there is no universally agreed-upon definition of what constitutes a paraphrase. 
    Prior research often adapts evaluation metrics from related \ac{nlp} tasks such as machine translation or summarization. 
    Two key dimensions are typically considered: semantic similarity and syntactic similarity.
    Contrary to initial intuition, high syntactic similarity is not necessarily desirable, as it may indicate that the \ac{llm} has merely copied the original text with minimal changes. 
    Instead, our focus lies on achieving high semantic similarity while maintaining syntactic diversity to ensure genuine rephrasing.
    Furthermore, we acknowledge that relatively low automatic scores can still be acceptable if qualitative human evaluation confirms the paraphrase’s adequacy.

    % \item \textbf{Which features are used for the \ac{av} problem?} \label{enum:rq3} \hfill \\
    % Traditional features include character tri-gram features, while newer research has proposed using \ac{llm} such as BERT.

    \item \textbf{How does the \ac{llm}-based impostor approach perform compared to state-of-the-art models?} \label{enum:rq4} \hfill \\
    Though our approach is computationally expensive, we argue that it is not a general purpose \ac{llm} detection method, but rather a single case solution tailored to specific detection tasks.
    We evaluate its performance in scenarios where (a) the disputed text is human generated,
    (b) the disputed text is \ac{llm} generated and the candidate is the same \ac{llm}, and
    (c) the disputed text is \ac{llm} generated, but the candidate is a different \ac{llm}.
    In terms of performance, we compare our method to other \ac{av} approaches on the \dataStudent{}, \dataBlog{}, \dataGutenberg{} and \dataPan{} datasets.
    
\end{questions}

% \section*{Idea}
% \label{sec:idea}

% Given a text of unknown authorship (i.e., human or \ac{llm}), 
% construct a set of impostor texts using state-of-the-art \acp{llm} based on the original text.
% Obtain the author by \ac{aa}/ \ac{av} methods, such as unmasking, to \textit{confidently}, i.e. high precision, identify \ac{llm} generated texts
% (and possibly which \ac{llm}).



% \section*{Contributions}
% \label{sec:contributions}
The contributions of this thesis are:
\begin{enumerate}
    \item Reimplementation of the traditional Impostor approach (cf. \autoref{chap:implementation}).
    \item Extension of the impostor approach with \ac{llm} generated impostors for line-up of difficult opponents (cf. \autoref{chap:methodology}). 
    \item Frame \ac{llm} detection as a \ac{av} problem and use \ac{llm} generated text as candidate for text of "unknown" authorship.
\end{enumerate}
