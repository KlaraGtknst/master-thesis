% motivation
Historically, authorship analysis focused on literary disputes, such as the authorship of Shakespearean plays~\citep{neal_surveying_2018,stamatatos_survey_2009}, but contemporary concerns have shifted towards applications with everyday relevance.
In an era where large amounts of text can be copied, paraphrased, or fabricated with ease, identifying the true author of a text is crucial for maintaining trust in communication. 
%
When the goal is to determine who, if any, from a candidate set of authors produced a disputed text, the task is known as \ac{aa}. 
Conversely, when the aim is to establish whether two texts were authored by the same individual, the problem is referred to as \ac{av}~\citep{koppel_authorship_2004}. 
By solving \ac{av} tasks for each candidate in an \ac{aa} setting, the \ac{aa} problem can be approached as a sequence of \ac{av} decisions~\citep{tyo_state_2022,barlas_cross_domain_2020}.
%
Both \ac{aa} and \ac{av} remain highly relevant today, as contemporary challenges, such as detecting plagiarised passages~\citep{stein_intrinsic_2011} and verifying the authenticity of online content or student submissions, can be addressed using these approaches. 
For example, detecting plagiarised passages can be formulated as an \ac{av} task by determining whether any two passages were written by different authors.


% The emergence of \acp{llm} adds an additional layer of complexity. 
% While these models are widely embraced for beneficial applications such as summarisation, information seeking, and assistive writing~\citep{wang_stumbling_2024}, their ability to convincingly imitate human writing creates new risks. 
% \acp{llm} can be used to generate misinformation, impair academic honesty, or imitate individual authorial style, thereby inflicting harm on individuals who fall victim to these schemes~\citep{mitchell_detectgpt_2023,li_learning_2025,wang_stumbling_2024,bhattacharjee_fighting_2024}. 
% Since \acp{llm} can be conceptualised as authors, their detection can be modelled as an \ac{av} problem with human-\ac{llm} input pairs. 
% Thus, instead of treating \ac{llm} detection as an isolated task, it is more consistent to frame it as \ac{av}~\citep{llm_detection_av_2025}.

% specificity rather than generality
Existing approaches to generalisation typically train a single model and apply it across domains.
Despite significant advances in \ac{av}, prior work finds that such models struggle in out-of-distribution settings, where the topic or genre diverges from the training data~\citep{Sundararajan_style_18,bischoff_importance_2020,li_learning_2025}. 
This shortcoming motivates a shift towards scenario-specific solutions, where models are trained anew for narrowly defined cases. 
Such single-case approaches enable more precise control over domain variables which facilitates emphasising on stylistic idiosyncrasies rather than domain-level variation.

% AV
The \impAppr{} by \citet{koppel_determining_2014}\ introduces the idea of sampling so-called \imp{} texts, i.e.\ hard negatives, intended to sharpen the discrimination between genuine and false authorship matches. 
Since impostor texts are drawn anew for every input text pair, the method should generalise effectively to cross-domain scenarios.
However, the method's effectiveness is limited by the quality and contextual adequacy of these \imp{} texts. 
Traditional strategies cannot simultaneously control multiple confounding factors such as genre, topic, and target audience, limiting the discriminatory power of the hard negatives.

The thesis extends the \impAppr{} by leveraging state-of-the-art \acp{llm} to generate paraphrases as \imps{} rather than sampling existing texts.
This allows for simultaneous control over multiple confounding factors.
Collection and preprocessing of the original datasets, including a controlled candidate-pair selection procedure aligned with the methodology of \citet{koppel_determining_2014} and informed by direct consultation with the authors, ensures a robust evaluation setup. 
The extended \impAppr{} achieves improved precision–recall on the \dataStudent{} dataset compared with the original sampling strategies, \unmasking{} and \acs{ppmd}.
