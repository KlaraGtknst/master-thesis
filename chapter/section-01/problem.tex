% motivation
\Acf{av} denotes the task of determining whether two texts were written by the same author. 
It forms the foundation of broader authorship-related problems such as authorship attribution. 
Historically, \ac{av} has been applied to literary disputes, but contemporary concerns have shifted. 
In an era where large amounts of text can be copied, paraphrased, or fabricated with ease, \ac{av} has gained renewed importance in practical contexts such as intrinsic plagiarism detection in digital communication.

The emergence of \acp{llm} adds an additional layer of complexity. 
While these models are widely embraced for beneficial applications such as summarisation, code generation, and customer service automation, their ability to convincingly imitate human writing creates new risks. 
\acp{llm} can be exploited to generate misinformation, engender academic submissions, or impersonate individuals, thereby inflict harm on individuals who fall victim to these schemes. 
Since \acp{llm} can be conceptualised as authors, their detection naturally falls within the scope of \ac{av}. 
Thus, instead of treating \ac{llm} detection as an isolated task, it is more consistent to frame it as a specialised case of~\ac{av}.

% specificity rather than generality
Despite significant advances in \ac{av}, many existing approaches pursue generalised solutions, training a model once and applying it across domains. 
Evidence shows that such models struggle in \ac{ood} settings, where topic or genre diverges from the training data. 
This shortcoming motivates a shift toward scenario-specific solutions, i.e.\ models are trained anew for narrowly defined cases. 
Such single-case approaches enable more precise control over contextual factors and place greater emphasis on stylistic idiosyncrasies rather than domain-level variation.

% AV
Among the techniques developed for \ac{av}, the \impAppr{} by \citet{koppel_determining_2014}\ represents a particularly influential approach. 
It introduces the idea of generating \imp{} texts, i.e.\ hard negatives used to sharpen the discrimination between genuine and false authorship matches. 
However, the method's effectiveness is limited by the quality and contextual adequacy of these \imp{} texts. 
Previous work did not fully address how to construct challenging \imps{} via controlled contextual variables.

This thesis extends the \impAppr{} by leveraging \acl{sota} \acp{llm} to generate hard negative examples in a controlled scenario as \imps{}. 

