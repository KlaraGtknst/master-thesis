\section{Problem}
\label{sec:problem}

% LLM detection
\ac{llm} detection denotes the task of determining whether a text was written by a human or an \ac{llm}.
% specificity rather than generality
Due to the diversity of text domains \textcolor{red}{(/topics)}, we believe that generalized solutions for \ac{llm} detection are not feasible.
Generalized solutions refer to models that are trained once on a dataset and then used for all texts.
% bhattacharjee_fighting_2024: ChatGPT not equally good across different LLMs
% li_learning_2025: Scores drop for Out of Distribution (OOD) texts
Literature suggests that many models fail to detect all \acp{llm} especially in \ac{ood} scenarios~\citep{bhattacharjee_fighting_2024,li_learning_2025}.

% frame problem
We focus on single case solutions, i.e. models trained tailored to a specific scenario.
Formally, single case solutions include creating \ac{id} \ac{llm} detection scenarios, 
where we can control all variables (train new model for each dataset).
% What do we want from the solution?
Further requirements for the solution are high precision, i.e. if the model predicts that a text was written by an \ac{llm}, it should be correct.

% AV
\textcolor{gray}{
This problem can be framed as a \ac{av} task, where given two texts 
(i.e. the one of unknown authorship and the one written by an \ac{llm} author) 
the goal is to determine whether they were written by the same author or not.
% AA
Given set of texts generated by \acp{llm} and a text of unknown authorship,
the task can be framed as open-set \ac{aa} problem, 
where the goal is to determine whether the text was written by one of the \acp{llm} or not.
% Genre
It can also be framed as a genre classification task, where the goal is to determine whether a text belongs 
to the genre human or to the genre \ac{lm} \todo{Martins neues Paper}.
% One-class classification
It can be framed as a one-class classification task, where all \ac{lm} authors are considered as one class.
The goal is to identify \ac{lm} generated texts with high confidence/ \todo{precision?}.
}
