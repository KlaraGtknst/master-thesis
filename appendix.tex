\appendix
\chapter{Appendix}
\label{ch:appendix}


\section{Extractor Prompts used for Two-Step Paraphrasing}
\label{app:extractor_prompts}
% bullet point
\paragraph{Bullet point approach}
\begin{quote}
\textit{
Summarize the text above in five to six short bullet points. 
Respond ONLY with a JSON object in the following format: \\
$\{$ \\
\hspace{1em}"bullet\_points": "<list of bullet points>", \\
\hspace{1em}"tone": "<tone>", \\
\hspace{1em}"time\_period": "<time\_period>", \\
\hspace{1em}"language\_register": "<register>", \\
\hspace{1em}"target\_audience": "<target\_audience>", \\
\hspace{1em}"genre": "<genre>" \\
$\}$. 
Do not use direct quotes.
}
\end{quote}


% task
\paragraph{Task approach}
\begin{quote}
\textit{
Act as the author of the text above. From that perspective, infer your role or identity, the topic being addressed, and the purpose or instruction behind writing the text. 
Combine these elements into a concise task prompt that you would give to an LLM to reproduce the text. 
Respond ONLY with a JSON object in the following format: \\
$\{$ \\
\hspace{1em}"task": "<task>", \\
\hspace{1em}"tone": "<tone>", \\
\hspace{1em}"time\_period": "<time\_period>", \\
\hspace{1em}"language\_register": "<register>", \\
\hspace{1em}"target\_audience": "<target\_audience>", \\
\hspace{1em}"genre": "<genre>" \\
$\}$. 
Do not use direct quotes.
}
\end{quote}


% topic
\paragraph{Topic approach}
\begin{quote}
\textit{
Extract the topic, tone, time period, register, target audience, and genre from the text above. 
Respond ONLY with a JSON object in the following format: \\
$\{$ \\
\hspace{1em}"topic": "<topic>", \\
\hspace{1em}"tone": "<tone>", \\
\hspace{1em}"time\_period": "<time\_period>", \\
\hspace{1em}"language\_register": "<register>", \\
\hspace{1em}"target\_audience": "<target\_audience>", \\
\hspace{1em}"genre": "<genre>" \\
$\}$. 
Do not use direct quotes.
}
\end{quote}


% title
\paragraph{Title approach}
\begin{quote}
\textit{
Find a concise title for the text and extract the tone, time period, register, target audience, and genre from the text above. 
Respond ONLY with a JSON object in the following format: \\
$\{$ \\
\hspace{1em}"title": "<title>", \\
\hspace{1em}"tone": "<tone>", \\
\hspace{1em}"time\_period": "<time\_period>", \\
\hspace{1em}"language\_register": "<register>", \\
\hspace{1em}"target\_audience": "<target\_audience>", \\
\hspace{1em}"genre": "<genre>" \\
$\}$. 
Do not use direct quotes.
}
\end{quote}


\section{Generator Prompts used for Two-Step Paraphrasing}
\label{app:generator_prompts}

% bullet point
\paragraph{Bullet point approach}
\begin{minted}{python}
prompt = "Write a text which covers the following items:\n" 
    + "\n".join(f"- {bp}" for bp in bullet_points)
\end{minted}

% task
\paragraph{Task approach}
\begin{minted}[breaklines]{python}
generator_prompt = "Write a text of about {l} words with a {tone} tone, a {genre} genre, in the {register} register for the target audience of {target_audience} and in the {time_period} time period, covering the following task:\n{task}. Do not use asterisks. Only output the text without any additional commentary.".format(
            l=len(text.split()),
            tone=tone,
            genre=genre,
            task=task,
            time_period=time_period,
            register=register,
            target_audience=target_audience,
        )
\end{minted}

% topic
\paragraph{Topic approach}
\begin{minted}[breaklines]{python}
generator_prompt = "Write a text of about {l} words with a {topic} topic, {tone} tone, a {genre} genre, in the {register} register for the target audience of {target_audience} and in the {time_period} time period. Do not use asterisks. Only output the text without any additional commentary.".format(
            l=len(text.split()),
            tone=tone,
            genre=genre,
            topic=topic,
            time_period=time_period,
            register=register,
            target_audience=target_audience,
        )
\end{minted}

% title
\paragraph{Title approach}
\begin{minted}[breaklines]{python}
generator_prompt = "Write a text of about {l} words with a {title} title, {tone} tone, a {genre} genre, in the {register} register for the target audience of {target_audience} and in the {time_period} time period. Do not use asterisks. Only output the text without any additional commentary.".format(
            l=len(text.split()),
            tone=tone,
            genre=genre,
            title=title,
            time_period=time_period,
            register=register,
            target_audience=target_audience,
        )
\end{minted}

\section{Preprocessing Regular Expressions}
\label{app:regex_preproc}

% \begin{table}[]
% \centering
% \caption{Regular Expressions used for preprocessing}
% \label{tab:preproc_regex}
% \resizebox{\textwidth}{!}{%
% \begin{tabular}{ll}
%     \toprule
% \textbf{Description} & \textbf{Regular Expression}      
% \midrule                                                                                                        \\
% html tags            & rr %r"\textless{}{\[}\textasciicircum{}\textgreater{}{\]}+\textgreater{}"                                                                      \\
% % play artifacts       & r"\textasciicircum{}{\[}A-Z\textbackslash{}s{\]}+{\[}.:{\]}?\$"                                                                              \\
% % chapter artifacts    & r"\textasciicircum{}\textbackslash{}s*Chapter\textbackslash{}s+\textbackslash{}w+.*\textbackslash{}n\textbackslash{}s*\textbackslash{}n" \\
% % play artifacts &
% %   \begin{tabular}[c]{@{}l@{}}r"\textbackslash{}s*ACT\textbackslash{}s+\textbackslash{}w+\textbackslash{}b\textbackslash{}.?"\\ r"\textbackslash{}s*SCENE\textbackslash{}s+\textbackslash{}w+\textbackslash{}b\textbackslash{}.?"\\ r"\[(.*?)\]" -\textgreater r"\textbackslash{}1"\\ r"\_+"r"\textbackslash{}s+\textbackslash{}d+\textbackslash{}s*\$"\end{tabular} \\
% % whitespace           & r"\textbackslash{}s+"  \\
% \bottomrule                                                                                                                 
% \end{tabular}%
% }
% \end{table}

\textcolor{red}{FIXME}
\begin{table}[H]% extra page (usually for large figures/tables)
  \caption{Tables have their captions above, figures below.}
  \label{table-with-numbers}%
  \centering\small
  \begin{tabular}{@{}ll@{}} % use @{} to remove spacing; numbers should be right aligned
    \toprule
    \textbf{Description} & \textbf{Regular Expression}  \\
    \midrule
            html tags            & <[\^>]+> \\
            play artifacts       & \^[A-Z$\backslash$s]+[.:]?\$ \\
            % chapter artifacts    & \^$\backslash$s\*Chapter$\backslash$s+$\backslash$w+.*$\backslash$n$\backslash$s*$\backslash$n \\
             play artifacts & $\backslash$s*ACT$\backslash$s+$\backslash$w+$\backslash$b$\backslash$.? \\
& $\backslash$s*SCENE$\backslash$s+$\backslash$w+$\backslash$b$\backslash$.? \\
% & $\backslash$[(.*?)$\backslash$]" $\rightarrow$ $\backslash$1 \\
% & _+"r"$\backslash$s+$\backslash$d+$\backslash$s*\$ \\
whitespace           & $\backslash$s+ \\


    \bottomrule
  \end{tabular}
\end{table}



\section{Exp.\ 1(b): Varying \Imp{} Generation}

\begin{figure}[H]
  \centering
  \begin{subfigure}{\textwidth}
    \centering
    \includesvg[width=0.6\linewidth]{images/imposter/reproduction_koppel_figures/fig4/student_essays/student_roc_prec_recall_curve_r100_top100000_Same_Author_dif_imp_gen.svg}
    \caption{Same-author reference class}
    \label{fig:student_essays_same_author}
  \end{subfigure}
  \begin{subfigure}{\textwidth}
    \centering
    \includesvg[width=0.6\linewidth]{images/imposter/reproduction_koppel_figures/fig4/student_essays/student_roc_prec_recall_curve_r100_top100000_Different_Author_dif_imp_gen.svg}
    \caption{Different-author reference class}
    \label{fig:student_essays_different_author}
  \end{subfigure}
  \caption[Recall-precision curves for the \dataStudent{}.]{Recall-precision curves for the \dataStudent{}. 
Due to API limit restrictions, the test set for on-the-fly was smaller which is visible in the respective curves.
  }
  \label{fig:diff_imp_gen_student_essays}
\end{figure}


\section{Exp. 2: Comparison of Different Paraphrasers}
\label{sec:app_paraphrases}

\begin{figure}[H]
    \centering
    \includesvg[width=0.9\textwidth]{images/paraphrasing/experiments/radar/Blog_paraphrasing_metrics_grouped_by_Paraphraser_radar_chart.svg}
    \caption{Radar chart of syntactic and semantic paraphrase evaluation measures different paraphrasers on the \dataBlog{} dataset.}
    \label{fig:radar_blog}
\end{figure}


\begin{figure}[H]
    \centering
    \includesvg[width=0.9\textwidth]{images/paraphrasing/experiments/sem_syn_scatter/Gutenberg_sem_syn_scatter_grouped_by_Paraphraser.svg}
    \caption{Average semantic and syntactic similarity for different paraphraser on the \dataGutenberg{}.}
    \label{fig:sem_syn_gutenberg}
\end{figure}

\begin{figure}[H]
    \centering
    \includesvg[width=0.9\textwidth]{images/paraphrasing/experiments/radar/Gutenberg_paraphrasing_metrics_grouped_by_Paraphraser_radar_chart.svg}
    \caption{Radar chart of syntactic and semantic paraphrase evaluation measures different paraphrasers on the \dataGutenberg{} dataset.}
    \label{fig:radar_gutenberg}
\end{figure}


\section{Exp. 3: Paraphrasing Chunks}
\label{sec:app_chunks}

\begin{figure}[htbp]
    \centering
    \includesvg[width=\textwidth]{images/paraphrasing/experiments/chunks/setup/results/Gutenberg/Translation_metrics_plot_category_Gutenberg.svg}
    \caption{Different paraphrasing scores for the Translation model. 
    Changing the number of chunks has little effect on the model scores, but a small reduction of the standard deviation.}
    \label{fig:abl_chunks_gutenberg_translation}
\end{figure}

\begin{figure}[htbp]
    \centering
    \includesvg[width=\textwidth]{images/paraphrasing/experiments/chunks/setup/results/Student_Essays/Task_metrics_plot_category_Student Essays.svg}
    \caption{Different paraphrasing scores for the BulletPoint model. 
    This model is not affected by the number of chunks.}
    \label{fig:abl_chunks_student_essays_task}
\end{figure}

% appendix
\section{Exp. 6: Comparing \ac{av} Methods in different scenarios}
\label{sec:app_detection_scenarios}

% \begin{figure}[htbp]
% \centering
%     \includesvg[width=\linewidth]{}
%   \caption{Accuracy curves for the class same-author across different threshold on artificially augmented \dataStudent{} dataset. 
%   }
%   \label{fig:human-human_acc}
% \end{figure}


\begin{figure}[htbp]
  \centering
  \begin{subfigure}[b]{0.52\textwidth}
    \centering
    \includesvg[width=\linewidth]{images/AV_comparison/detection_scenarios/recall/student_essays_Human-Human_threshold_recalls_curves_all_incl_baselines.svg}
    \caption{Human-Human}
    \label{fig:detec_scen_human-human_recall}
  \end{subfigure}
  \hfill
  \begin{subfigure}[b]{0.52\textwidth}
    \centering
    \includesvg[width=\linewidth]{images/AV_comparison/detection_scenarios/recall/student_essays_Human-LLM_threshold_recalls_curves_all_incl_baselines.svg}
    \caption{Human-\ac{llm}}
    \label{fig:detec_scen_human-llm_recall}
  \end{subfigure}
  \hfill
  \begin{subfigure}[b]{0.52\textwidth}
    \centering
    \includesvg[width=\linewidth]{images/AV_comparison/detection_scenarios/recall/student_essays_LLM-LLM_same_threshold_recalls_curves_all_incl_baselines.svg}
    \caption{\ac{llm}-\ac{llm}}
    \label{fig:detec_scen_llm-llm_recall}
    \end{subfigure}
  \caption{Recall curves for the class same-author across different threshold on artificially augmented \dataStudent{} dataset.}
  \label{fig:detec_scen_recall}
\end{figure}


% \section{Libraries used}
% \label{app:libraries}

% \begin{table}[]
% \centering
% \caption{Libraries used}
% \label{tab:libs}
% \begin{tabular}{ll}
%     \toprule
% \textbf{Functionality} & \textbf{Library} \\
% \midrule
% Sentence tokenizer & \texttt{nltk}'s \texttt{sent\_tokenize}             \\
% \ac{rouge}              & Google Research\footnote{https://huggingface.co/spaces/evaluate-metric/rouge} \\
% TODO               & TODO             \\
% \bottomrule
% \end{tabular}%
% \end{table}
