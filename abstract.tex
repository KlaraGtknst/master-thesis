\chapter*{Abstract}
\markboth{Abstract}{Abstract}

\Acl{av} seeks to determine whether two texts share the same author. 
Existing approaches often perform poorly in cross-domain scenarios. 
The influential \impAppr{} of \citet{koppel_determining_2014} introduced hard negative sampling to sharpen predictions, yet it struggles to fix confounding variables during inference. 
This thesis investigates whether \aclp{llm} can help address these limitations. 
Specifically, we employ \acs{llm}-generated paraphrases as hard negatives. 
Since these paraphrases aim to preserve the confounding variables of the original text, they mitigate domain-related biases during inference. 
Moreover, by constructing a tailored case for each input text pair, the approach eliminates \acl{ood} issues, ensuring that comparisons remain within the distribution defined by the pair itself.
We evaluate the approach in the traditional human-authored \acl{av} setting and find that the \acs{llm}-based extension outperforms the original baselines. 
At the same time, our results reveal the practical and conceptual challenges of integrating \acsp{llm} into \acl{av}, including issues of reliability, hallucination, and control over paraphrase quality.



% The original \impAppr{} compares a disputed text to a candidate text and hard negatives, considering disputed and candidate text to share an author if the candidate is consistently the most similar across random feature projections. 