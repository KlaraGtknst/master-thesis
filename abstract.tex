\chapter*{Abstract}
\markboth{Abstract}{Abstract}

\Acl{av} seeks to determine whether two texts share the same author, a task critical for ensuring the integrity of academic submissions or online content.
Existing approaches typically exhibit poor generalisation across domains.
The \impAppr{} introduced hard negative sampling to create input-pair-specific settings to improve cross-domain generalisation. 
However, traditional sampling strategies are unable to simultaneously control for multiple confounding variables, such as genre and topic.
%, highlighting the need for improved sampling strategies.
This thesis investigates whether \aclp{llm} can help address these limitations. 
Specifically, we employ \acs{llm}-generated paraphrases as hard negatives. 
% Since these paraphrases aim to preserve the confounding variables of the original text, they mitigate domain-related biases during inference. 
% Moreover, by constructing a tailored case for each input text pair, the approach eliminates \acl{ood} issues, ensuring that comparisons remain within the distribution defined by the pair itself.
Evaluation on the \dataStudent{} dataset from the original study shows that the \acs{llm}-based extension surpasses the original baselines by \citet{koppel_determining_2014}, \acs{ppmd}, and \unmasking{} in terms of precision and recall.
At the same time, our results reveal the practical and conceptual challenges of integrating \acsp{llm} into \acl{av}, including issues of reliability, hallucination, and control over paraphrase quality.



% The original \impAppr{} compares a disputed text to a candidate text and hard negatives, considering disputed and candidate text to share an author if the candidate is consistently the most similar across random feature projections. 