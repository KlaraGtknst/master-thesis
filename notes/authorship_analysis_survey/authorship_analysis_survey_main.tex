\section{Authorship Analysis Survey}
\label{sec:authorship_analysis_survey}

% features
\citet{elmanarelbouanani_authorship_2014} claim that there is no feature set optimized and applicable to all people and to all domains.

% metrics
\subsection{Metrics}
\citet{elmanarelbouanani_authorship_2014} state that the common metrics for evaluating the performance of a particular are:
\begin{itemize}
    \item $Accuracy = \frac{TP + TN}{TP + TN + FP + FN}$ \citep{elmanarelbouanani_authorship_2014,neal_surveying_2018} 
    measures the percentage of classified correctly over all test cases \citep{neal_surveying_2018}.

    \item $Precision = \frac{TP}{TP + FP}$ \citep{elmanarelbouanani_authorship_2014,neal_surveying_2018} 
    measures how often a system gets positive classification correctly \citep{neal_surveying_2018}.

    \item $Recall = \frac{TP}{TP + FN}$ \citep{elmanarelbouanani_authorship_2014,neal_surveying_2018} 
    measures how often a system correctly classifies positive samples when it encounters them \citep{neal_surveying_2018}.
\end{itemize}


% distance measures
\subsection{Distance Measures}
\citet{elmanarelbouanani_authorship_2014} state that the most common distance measures are:
\begin{itemize}
    \item Delta measure
    \item Chi-Square distance
    \item \ac{kld}
\end{itemize}
They claim the Delta measure outperforms the other measures.