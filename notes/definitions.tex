\section{Definitions}
\label{sec:definitions}


\begin{definition}
    [Author profiling]
    Task of inferring an extensive set of (sensitive) personal information.
    This includes age, gender, education, socio-economic status and mental health issues \cite{emmery_adversarial_2021,stamatatos_survey_2009}.
    The task is also referred to as author characterization \cite{stamatatos_survey_2009}.
\end{definition}

\begin{definition}
    [Stylometry]
    The construction of models for the quantification of writing style, text complexity, and grading level assessment.
    Stylometric features include lexical, syntactic and structural features \cite{stein_intrinsic_2011}.
\end{definition}

\begin{definition}
    [Style markers]
    Taxonomies of features to quantify the writing style \cite{stamatatos_survey_2009}.
\end{definition}

\begin{definition}
    [Lexical features]
    Lexical features are common features used in stylometry.
    They are character- (i.e. character (n-gram) frequency, ...) 
    or word-based (i.e. average word/ sentence length, ...) features \cite{stein_intrinsic_2011}. 
    These features consider text as a mere sequence of word-tokens or characters, respectively \cite{stamatatos_survey_2009}.
    Word features are more complex than character features \cite{stamatatos_survey_2009}.
\end{definition}

\begin{definition}
    [Adversarial Stylometry]
    Attack models that automatically infer a variety of potentially sensitive author information.
    These attacks are not to be confused with adversarial learning \cite{emmery_adversarial_2021}.
\end{definition}

\begin{definition}
    [Closed world]
    In the realm of plagiarism detection, closed world refers to the assumption that a reference collection $D$ of documents, that are supposed to be compared to the possibly plagiarized text, is given \cite{stein_intrinsic_2011}.
\end{definition}

\begin{definition}
    [Plagiarism]
    In the context of texts, plagiarism is the usage of another author's information, language, or writing without properly acknowledging the original source \cite{stein_intrinsic_2011}.
\end{definition}

\begin{definition}
    [Plagiarism detection]
    The task of identifying plagiarized text \cite{stein_intrinsic_2011}, i.e. finding similarities between two texts \cite{stamatatos_survey_2009}.
\end{definition}

\begin{definition}
    [Intrinsic plagiarism detection]
    This task can be understood as a more general form of authorship verification.
    By analysing undeclared changes in writing style, potential plagiarism can be detected.
    Opposed to authorship verification, where the decision is made based on the whole text, 
    intrinsic plagiarism detects plagiarism on a section level \cite{stein_intrinsic_2011}.
    Intrinsic analysis does not use any information on authorship from external sources \cite{zangerle_overview_2024}.
\end{definition}

\begin{definition}
    [Authorship analysing]
    This problem devises into two \textcolor{orange}{(i.e. first two acc. to \cite{stein_intrinsic_2011}, all acc. to \cite{stamatatos_survey_2009})} subproblems \cite{stein_intrinsic_2011}:
    \begin{itemize}
        \item Authorship attribution \cite{stein_intrinsic_2011}
        \item Authorship verification \cite{stein_intrinsic_2011,stamatatos_survey_2009}
        \item Plagiarism detection \cite{stamatatos_survey_2009}
        \item Author profiling \cite{stamatatos_survey_2009}
        \item Detection of stylistic inconsistencies (i.e. in collaborative writing) \cite{stamatatos_survey_2009}
    \end{itemize}
\end{definition}

\begin{definition}
    [Authorship attribution]
    The task of determining the author of a text based on textual features 
    given a set of candidate authors with undisputed writing samples \cite{stein_intrinsic_2011,stamatatos_survey_2009}.
    In terms of machine learning, this is a multiclass, single-label text categorization task \cite{stamatatos_survey_2009}.
    The task is also referred to as author(ship) identification \cite{stamatatos_survey_2009}.
\end{definition}

\begin{definition}
    [Authorship verification]
    Given a set of writing samples of author $A$ and a text $t$, 
    the task is to determine whether $t$ was written by $A$ \cite{stein_intrinsic_2011,stamatatos_survey_2009}.
    Related research areas include \cite{stein_intrinsic_2011}:
    \begin{itemize}
        \item stylometry
        \item outlier analysis and meta learning
        \item symbolic knowledge representation, i.e. \todo{knownledge representation, deduction, heuristic inference}
    \end{itemize}
\end{definition}

\begin{table}[]
    \centering
    \caption{Building blocks for authorship verification from \cite{stein_intrinsic_2011}.}
    \label{tab:authorship_verification_blocks}
    \resizebox{\textwidth}{!}{%
    \begin{tabular}{|l|lll|l|}
    \hline
    \rowcolor[HTML]{EFEFEF} 
    Pre-analysis & \multicolumn{3}{l|}{\cellcolor[HTML]{EFEFEF}Modeling and classifier methods} & Post-processing \\ \hline
    \rowcolor[HTML]{EFEFEF} 
    Impurity assessment & \multicolumn{1}{l|}{\cellcolor[HTML]{EFEFEF}Decomposition strategy} & \multicolumn{1}{l|}{\cellcolor[HTML]{EFEFEF}Style model construction} & Outlier identification & Outlier post-processing \\ \hline
    Document length analysis & \multicolumn{1}{l|}{Uniform length} & \multicolumn{1}{l|}{Lexical character features} & One-class density estimation & Heuristric voting \\
    Genre Analysis & \multicolumn{1}{l|}{Structural boundaries} & \multicolumn{1}{l|}{Lexical word features} & One-class boundary estimation & Citation analysis \\
    Analysis of issuing institution & \multicolumn{1}{l|}{Text element boundaries} & \multicolumn{1}{l|}{Syntactical features} & One-class reconstruction & Human inspection \\
     & \multicolumn{1}{l|}{Topical boundaries} & \multicolumn{1}{l|}{Structural features} & Two-class discrimination & Unmasking \\
     & \multicolumn{1}{l|}{Stylistic boundaries} & \multicolumn{1}{l|}{Language modeling} &  & Qsum \\
     & \multicolumn{1}{l|}{} & \multicolumn{1}{l|}{} &  & Batch means
    \end{tabular}%
    }
\end{table}
Post-processing to avoid false positives, c.f. \citet{stein_intrinsic_2011} for approaches.

\begin{definition}
    [Meta learning]
    \todo{Based on learning successes and failures, the system learns to learn. }
    Approaches include:
    \begin{itemize}
        \item Unmasking: Measurement of reconstruction errors starting from a good reconstruction and iteratively impairing the reconstruction. % example on page 9 of Benno's paper
        \item Qsum heuristic: Compares the growth rates of two cumulative sums over a sequence of sentences. The sums are calculated via the deviations from the mean sentence length and the deviations of the function words.
        \item Batch means: \todo{For a series of values the variance development of the sample mean is measured while the sample size is successively increased.}
    \end{itemize}
\end{definition}

\begin{definition}
    [Unmasking]
    The idea of this meta learning approach is that with progressively omitting more and more frequent words, 
    topic specific words are excluded and thus, leaving only writing style specific words.
    After several iterations, remaining features which are not powerful enough to discriminate two documents indicate that 
    these documents originate from the same author \cite{stein_intrinsic_2011}.
\end{definition}

\begin{definition}
    [Stop words vs. function words]
    Function words are the most common words (articles, prepositions, pronouns, etc.) \cite{stamatatos_survey_2009}.
    Most function words are stop words, but not all stop words are function words \cite{stein_intrinsic_2011}.
\end{definition}

\begin{definition}
    [One-class classification]
    A classification problem where the classifier is trained on samples of a single class.
    If counterexamples, i.e. so-called outliers, are available, they are usually not considered to be representative of \textit{non-target class}.
    Hence, the classifier has to learn the concept of the target class in the absence of discriminating features.
    Examples of one-class classification are intrinsic plagiarism analysis and authorship verification.
    Approaches to one-class classification fall into the following categories \cite{stein_intrinsic_2011}:
    \begin{itemize}
        \item One-class density estimation, e.g., Naive Bayes
        \item One-class boundary estimation
        \item One-class reconstruction
    \end{itemize}
\end{definition}

\begin{definition}
    [Open-set classification]
    The true author is not necessarily included in the set of candidate authors \cite{stamatatos_survey_2009}.
\end{definition}

\begin{definition}
    [Closed-set classification]
    The true author should be one of the candidate authors \cite{stamatatos_survey_2009}.
\end{definition}

\begin{definition}
    [n-gram]
    $n$ contiguous words also known as word collocations.
\end{definition}