\section{Suppressing Domain Style}

% potential introduction to style chapter
\citet{bischoff_importance_2020} assume that each author has a unique style, unconsciously encoded in their writing.
This style depends on the author's personal traits, customs an author adopts due to genre, register, type, and topic.
These concepts are too vague to be efficiently operationalized.
The goal is to discover a set of style markers more likely to be determined by the author's personality than by domain customs.

They claim that features frequent function words and word length have a high correlation with topic. \todo{????}

% char 3-gram
\citet{bischoff_importance_2020} analyse the robustness of character trigrams as a feature for \ac{aa}.
They find that the character trigrams feature set is not robust in a cross-topic setting, but across two genres.

% this approach
\citet{bischoff_importance_2020} propose an approach where they train with respect to the author labels, 
and adversarially train on the texts with respect to their domain labels.
They claim that this results in discriminative features for the task of \ac{aa} and in 
indiscriminative for the text domain differences.

% Domain swapping
\citet{bischoff_importance_2020} quantify the influence of domains rather than author style on the performance of models by 
computing the difference $\Delta(S_1,S_2)$ of the model performances in both the traditional scheme $S_1$ 
and the domain-swapped scheme $S_2$ (cf. \autoref{sec:definitions}).

% min len
\citet{bischoff_importance_2020} claim that the minimum sufficient length to measure author style is 500 words.