\section{UniNE CLEF 2015 \ac{av} Imposter}
\label{sec:UniNE_CLEF2015_AV_imposter}

The idea of \cite{kocher_unine_2015} is to apply a distance measure to the candidate author and the disputed text and 
compare the distance to those between the disputed text and a set of imposters.
The authors build profiles of candidate authors by concatenating the texts of the candidate author.
The distance is then calculated between the candidate author profile and the disputed text.
%Hence, they need a sample text of the candidate author and texts from the imposters.???

% features
\subsection{Features}
% \label{sec:features}

\citet{kocher_unine_2015} use the $k$ most frequent terms (without stemming) of the disputed texts 
where isolated words and punctuation symbols are considered terms.
Words had to occur at least twice to be considered.
The state a reasonable range for $k$ is 200 to 300.


\subsection{Distance measure}

\citet{kocher_unine_2015} measure the distance between the candidate author and the disputed text using the PSATIUM-L1 distance:
$$\Delta(Q,A)=\Delta_0=\sum_{i=1}^{k}\left| P_Q\left[ t_i \right] -P_A\left[ t_i \right] \right|$$
where $P_Q$ and $P_A$ are the estimated occurrence probability of the term $t_i$ in the query text $Q$ and in the author profile $A$, respectively.
The probabilities are estimated by dividing the term occurrence frequency $tf_i$ by the length of tokens $n$ in the corresponding text and thus, 
$P\left[ t_i \right]=\frac{tf_i}{n}$.
Since the authors do not apply smoothing, they accept a 0.0 probability.
$k$ represents the number of terms (i.e., words or punctuation symbols).
In order to contextualize the magnitude of the distance $\Delta_0$, a set of imposter distances is calculated.
They create $m$ imposter profiles resulting in $m$ distances $\Delta_1, \ldots, \Delta_m$.
Only the minimum distance among the set of imposter distances is kept.
They repeat this stage $r=5$ times resulting in $r$ distance measures $\Delta_{m1}, \ldots, \Delta_{mr}$.
They compute the arithmetic mean $\Delta_m$ of the $r$ distances.

The decision rule is as follows:


\subsection{Set of Imposters}

\citet{kocher_unine_2015} claim \ac{aa} is not more difficult than \ac{av} 
due to the fact that in \ac{av} tasks one has to compare the disputed text with a set of all representative imposter texts.
The authors claim it is difficult to determine whether one has included all imposters, 
i.e., other writers having a similar style to the candidate author.

They choose $m=3$ random imposters from the test set of the candidate author 
where those with similar number of known documents to the candidate author are favoured.
Each set of imposter documents is used to create a profile.
$m=3$ is chosen arbitrarily.

\subsection{\tira{}}
\label{sec:tira}

The evaluation of shared \ac{pan} CLEF tasks is done via the \tira{} platform.
\tira{} is an automated tool for deployment and evaluation of the software submitted.
During runs of the submitted software, no data leakage to the participants is possible since \tira{} encapsulates the software.
\citet{kocher_unine_2015} claim that \tira{} offers a fair evaluation of the tome needed to produce an answer.
However, system incompatibilities may occur \cite{kocher_unine_2015}.

