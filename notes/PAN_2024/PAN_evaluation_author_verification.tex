\section{\acs{pan} evaluation}
\label{sec:pan_evaluation}

\citet{ayele_overview_2024,bevendorff_overview_2024} evaluate the participants submissions averaging the datasets' evaluation measures.
\citet{ayele_overview_2024} claim the following measures are established in \ac{av}:
\begin{itemize}
    \item \ac{roc-auc} \citep{bevendorff_overview_2024,weerasinghe_feature_vector_difference_2021,kocher_unine_2015}
    
    \item BRIER: Complement of the Brier score \citep{bevendorff_overview_2024,weerasinghe_feature_vector_difference_2021}, in \citet{bevendorff_overview_2024}'s case equivalent to the mean squared loss.
    
    \item C@1: Modified version of the accuracy \citep{bevendorff_overview_2024}/ F1-score \citep{weerasinghe_feature_vector_difference_2021} score, 
    where the non-answers (abstained) \citep{llm_detection_av_2025} are assigned the average accuracy of the remaining cases \citep{bevendorff_overview_2024}. 
    It rewards systems that leave difficult problems unanswered \citep{weerasinghe_feature_vector_difference_2021}.
    $$c@1 = \frac{nc}{np}(1+\frac{nu}{np})$$ where $np$ is the number of problems, $nc$ the number of correct answers, 
    and $nu$ the number of unanswered problems \citep{kocher_unine_2015}.
    
    \item $F_1$: Harmonic mean of precision and recall \citep{bevendorff_overview_2024,weerasinghe_feature_vector_difference_2021}:
    $ F_1 = 2 \cdot \frac{precision \cdot recall}{precision + recall} $ \citep{neal_surveying_2018}.
    A higher value indicates a better performance \citep{neal_surveying_2018}.
    
    \item $F_{0.5u}$: Modified version of the $F_{0.5}$ score, where the non-answers are considered \acp{fn} \citep{bevendorff_overview_2024}. A measure that puts more emphasis on deciding same-author cases correctly \citep{weerasinghe_feature_vector_difference_2021}.
\end{itemize}

The \ac{roc} curve is generated according to the percentage of \acp{fp}, i.e.\ \ac{fpr} $= \frac{FP}{FP+TN}$, in the x-axis and 
the percentage of \acp{tp}, i.e.\ \ac{tpr} $=\frac{TP}{TP+FN}$, in the y-axis,
for varying thresholds \citep{kocher_unine_2015,neal_surveying_2018}.
The maximum value of 1.0 indicated a perfect performance \citep{kocher_unine_2015}.
The \ac{auc} is the area under the curve, where a greater \ac{auc} indicates a better performance \citep{neal_surveying_2018}.
\citet{kocher_unine_2015} claim that both \ac{roc} and \ac{auc} are difficult to interpret.
According to \citet{kocher_unine_2015}, usually, the \ac{auc} values should be consistent and comparable with the C@1 values.
The \ac{auc} of the \ac{roc} is biased since the \ac{roc} gives more emphasis 
on the first position and therefore increases the total \ac{auc}.
A misclassification with a lower probability is less penalized with \ac{roc-auc} \citep{kocher_unine_2015}.