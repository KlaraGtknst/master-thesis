\section{Topic Confusion}
\label{sec:topic_confusion}

% error
\citet{altakrori_topic_2021} propose topic confusion.
It is a task specifically designed to evaluate \ac{aa} models' effectiveness.
The task splits the error into two parts:
\begin{itemize}
    \item \textbf{Models' confusion}: The model is confused about the topic of the text.
    \item \textbf{Features' inability}: The features are unable to capture the authors' writing style.
\end{itemize}

% dependency of author and topic
\citet{altakrori_topic_2021} present an example to explain the dependency of author and topic:
Consider a topic defined by a unique word distribution.
An author selects a subset of words from the topic due to the limited length of documents.
Since the authors choose wordings, i.e. synonyms, the dependency of the topic on the author varies from one author to another.
Based on this, they model the attribution process as follows:

\begin{equation}
    \label{eq:topic_confusion_model_world}
    P(A,T,D) = P(A) P(T \mid A)P(D\mid A,T)
\end{equation}

\begin{equation}
    \label{eq:topic_confusion_attribution_process}
    P(A=a \mid D) \propto \sum_{T}^{t}\left[ P(A=a) P(T=t \mid A = a)P(D\mid T=t,A=a) \right]
\end{equation}

% topic confusion
Opposed to traditional cross-topic \ac{aa} tasks, 
which do not shed any light on whether the error arose from the topic or the features themselves, 
\citet{altakrori_topic_2021}'s topic confusion task offers insights in the error sources and 
whether features are indicative of writing style or topic.
% data constraints
The task requires writing samples written by $N$ authors on $T$ topics where $N \geq 4$ and $T \geq 3$ and 
each author has written approximately the same number of samples on each topic.
% procedure
First, authors are divided into two groups.
During training, the model receive only topic $A$ for authors in group one and topic $B$ for authors in group two.
Topic $A$ and topic $B$ are chosen randomly.
During testing, the group-topic configuration is flipped.
Unused writing samples are used for the validation set.

% metrics
\citet{altakrori_topic_2021} define three metrics:
\begin{itemize}
    \item \textbf{Correct (\%)}: Percentage of correctly classified samples.
    \item \textbf{Same-group error (\%)}: Percentage of samples that were attributed to the wrong author but within the same group 
    as the correct author.
    \item \textbf{Cross-group error (\%)}: Percentage of samples that were attributed to the wrong author and to the author group 
    that does not contain the correct author.
\end{itemize}

% explanation
Topic invariant features capture the writing style of the author.
Hence, the model should perform well on the test set.
Features that capture topic rather than writing style lead to a model classifying according to topic, 
resulting in cross-group errors.
Other error sources result in same-group errors.

\subsection{Common classifiers and features}
\label{sec:topic_confusion_classifiers}
% common models
Common options of classifiers are \acp{svm}, \ac{nb} and decision trees.
\citet{altakrori_topic_2021} attach their hyperparameter configuration for a \ac{svm} classifier to their paper.
They also state n-grams are a common approach to represent documents in \ac{aa} tasks.
Commonly, tokenization for n-grams is done either on the word level or on the character level.
\citet{altakrori_topic_2021} also include \ac{pos}-level n-grams in their feature set 
which are considered an essential indication of style.
If large pre-trained language models are used with the comparably small training datasets, embedding layers are frozen and 
only the classification layer is trained.

\subsection{Features}
\label{sec:topic_confusion_features}
% features
% stylometric features
Based on empirical evidence (one paper, one dataset), stylometric features are rather topic-invariant \cite{altakrori_topic_2021}.
% POS
According to \citet{altakrori_topic_2021}, \ac{pos} tags capture stylistic variations in language grammar between authors.
% n-grams
Character-level n-grams are favourable over word-level n-grams in terms of cross-group error.
Hence, character-level n-grams are more topic-invariant than word-level n-grams.

\subsection{Masking}
\label{sec:topic_confusion_masking}
% masking
\citet{altakrori_topic_2021} also include a masking technique to replace every character in masked words with (*) and all digits by (\#).
Low frequency words in the British National Corpus (BNC) are masked.
After masking, the original document structure is recreated, based on which the n-gram features are extracted.

% Imbalanced dataset
\subsection{Imbalanced Dataset}
\label{sec:topic_confusion_imbalanced_dataset}
\citet{altakrori_topic_2021} state that proper metrics for imbalanced datasets include 
weighted accuracy, precision, recall and F-Score.


\subsection{\ac{ap} model}
\label{sec:topic_confusion_ap_model}
\citet{altakrori_topic_2021} train a separate neural language model for each author.
Each model is called an \ac{ap} model.
Embedding layers are intialized with pre-trained equivalences.

\subsection{How \ac{ap} model works}
\label{sec:topic_confusion_ap_model_works}
For a document of disputed authorship, the average perplexity of all \ac{ap} models is calculated.
The perplexity scores are normalized using a normalization vector $\begin{pmatrix}
    n_0
     \\\vdots 
     \\n_k
    
    \end{pmatrix}$
where $n_i$ is the average perplexity of author $A_i$ on the normalization corpus.
\citet{altakrori_topic_2021} propose two normalization corpora:
\begin{itemize}
    \item \textbf{Training set}
    \item \textbf{Testing set without labels} (unrealistic scenario, since often only one document is available during inference)
\end{itemize}
The lowest perplexity score is assigned the author of the document.

\subsection{Shortcomings of \acp{lm}}
\label{sec:topic_confusion_shortcomings_lm}
\citet{altakrori_topic_2021} suspect that due to the nature of \acp{lm}, where words of similar meaning, i.e. "color" and "colour", 
are mapped to a similar vector, they do not work well in \ac{aa} tasks.
In \ac{aa} tasks, language system differences are highly relevant since they reveal the author's identity.  