\section{Heuristic Author Obfuscation}
\label{sec:heuristic_author_obfuscation}

\citet{bevendorff_divergence_based_2020} cast the task of author obfuscation as a heuristic search problem.
They look for a cost-minimum sequence of tailored paraphrasing operations for a significant increase in the distance 
of the paraphrased text to other texts from the author.
Estimated text quality reduction is used as a cost function.

% trigram
\subsection{Trigram}
In the realm of authorship analysis, character 3-grams are considered effective \citep{bevendorff_divergence_based_2020}, 
even though it is not systematically proven.
\citet{bevendorff_divergence_based_2020} claim that current (up to 2020) \ac{av} algorithms analyse (implicitly/explicitly) 
character 3-grams.
According to \citet{bevendorff_divergence_based_2020}, without loss of generality, since character 3-gram are the lowest-level feature that 
still captures word boundaries and morphology, % Lehre über Worte
directly influence higher-level features, such as word n-grams and \ac{pos} features.
This holds true for most European languages \citep{bevendorff_divergence_based_2020}.
3-grams encode vocabulary, morphology, and punctuation.

% distance measure
\subsection{Distance Measure}
% KL divergence
The \ac{kld} is a character-based style distance measure.
It acts as a feature- and task agnostic % skeptisch
information-theoretic divergence measure.
Moreover, it can be used as a simple and computationally feasible stopping criterion for an obfuscation process.
Furthermore, based on \ac{kld}, a normalization criterion for obfuscating texts of different lengths can be derived.
Finally, \ac{kld}'s derivative can serve as a selection criterion for parts of text that 
will yield the highest obfuscation gain if changed \citep{bevendorff_divergence_based_2020}.
This could be used for a greedy algorithm.
However, \citet{bevendorff_divergence_based_2020} claim greedy strategies can be reverse engineered.
$$KLD(P\mid \mid Q) = \sum_{i}^{}P[i] log\frac{P[i]}{Q[i]}$$
where $P,Q$ are the discrete probability distribution corresponding to the relative frequencies of character 3-grams in the 
to-be-obfuscated text and the known texts respectively.
For true probability distributions, the \ac{kld} is always non-negative.
However, the \ac{kld} is not symmetric, i.e.\ $KLD(P\mid \mid Q) \neq KLD(Q\mid \mid P)$.
Moreover, it is only defined for distributions $P,Q$ where $Q[i] = 0$ implies $P[i] = 0$ (which yields zero summands).
% Jensen-Shannon divergence
To overcome \ac{kld}'s shortcomings, 
\citet{bevendorff_divergence_based_2020} use the $JS_\Delta$ metric, based on the \ac{jsd}:
$$JSD(P\mid \mid Q) = \frac{KLD(P\mid \mid M)+KLD(Q\mid \mid M)}{2}$$
where $M = \frac{P + Q}{2}$ is the average of the two distributions.
The $JS_\Delta$ metric is defined as:
$$JS_\Delta(P,Q) = \sqrt{2\cdot JSD(P\mid \mid Q)}$$

% writeprints
\begin{definition}
    [Writeprints]
    \label{def:writeprints}
    It is a technique for \ac{aa} and similarity detection \citep{elmanarelbouanani_authorship_2014,neal_surveying_2018} 
    on a set of over twenty lexical, syntactic, and structural text feature types \citep{bevendorff_divergence_based_2020}.
    It is an individual-author-level feature extraction model \citep{neal_surveying_2018}.
    Texts to be attributed are compared with all other entities, a score is calculated and if the score is above a certain threshold, 
    the text is clustered with the entity \citep{elmanarelbouanani_authorship_2014}.
\end{definition}

% task
\subsection{Task}
After \ac{pan}'s author obfuscation tasks from 2016 to 2018 received either poor rule-based approaches or such that 
produced unreadable texts, they chose to refine the task to paraphrase the text.

\begin{definition}
    [Paraphrase]
    The meaning should stay the same and the text should be still readable \citep{bevendorff_divergence_based_2020}.
\end{definition}

% proposed approach
\subsection{Proposed Approach}
\citet{bevendorff_divergence_based_2020} propose obfuscation by reduction.
First, they rank 3-grams based on their influence on $JS_\Delta$ via their partial \ac{kld} derivative.
They remove one occurrence of the most influential 3-gram from the to-be-obfuscated text via naively omitting it without replacement, 
targeted paraphrasing (semantically equivalent text passage without the 3-gram).
The problem is defined as potentially infinite space with possible (text) states, 
in which each state is reachable from one or multiple nodes in a graph spanned over the entire space by operators.
Edges are labelled with the cost of the operator.
The task is to find a minimum-cost path from a starting node to a node that satisfies a pre-determined goal condition 
(i.e.\ sufficient obfuscation).
Applying an operator has highly non-linear effects on the text quality and may restrict the set of applicable operators in the same text.
They define the informed heuristic search A*.

The verifier most resilient against this specific reductive obfuscation (obfuscating only n-grams that are already rare for maximum effect) 
was based on an \impAppr{} on most frequent words.

% dataset
\subsection{Dataset}
The dataset used for the task is the Webis \acl{av} corpus 2019.