\newcommand{\writeprints}{Writeprints}
\section{\writeprints{}}
\label{sec:writeprints}

\citet{abbasi_writeprints_2008} propose the \writeprints{} technique for \ac{aa} and similarity detection.
\writeprints{} is an unsupervised method. % Karhunen-Loeve is supervised?!
It is a Karhunen-Loeve transform-based technique that uses a sliding window and 
pattern disruption algorithm with individual author-level feature sets.

\writeprints{} is constructed for each author using the author's key features, 
i.e.\ individual-author-level (cf. \autoref{sec:definitions}) feature sets.
The algorithm consists of two parts as outlined in \autoref{alg:writeprints}:
% part 1: creation
Via applying Karhunen-Loeve transforms with a sliding window, patterns are created.
These \writeprints{} patterns project usage variance into a lower-dimension space for each author feature.
Each lower-dimensional point reflects a single window instance.
% part 2: pattern disruption
Features an author never uses are treated as pattern disruptors (i.e., red flags), 
which means that the occurrence of these features in the disputed text decreases the similarity score 
between the author and the disputed text.

\begin{algorithm}
\caption{Writeprints Steps}
\label{alg:writeprints}
\begin{algorithmic}[1]
    \Procedure{Writeprints}{$X$, $authors$, $feature\_sets$}

        \State \textbf{Step 1: Creation of \writeprints{} Patterns}
        \ForAll{individual-level features with occurrence frequency > 0}    \Comment{Individual per author}
            \State Extract feature vectors for each sliding window instance \Comment{sliding window size $L=1500$ characters (i.e., roughly 250 words) \& jump interval $J=L$ characters (i.e., no overlap)}
            \State Derive basis matrix (set of eigenvectors) from feature usage covariance matrix using Karhunen-Loeve transforms \Comment{number basis vectors via Kaiser-Guttmann stopping rule}
            \State Compute \writeprints{} patterns (= Coordinates/ Principal Components for each window instance) by multiplying window feature vectors with basis
        \EndFor
       

        \State \textbf{Step 2: Pattern Disruption}  
        \ForAll{author features with occurrence frequency = 0} \Comment{From set of features of all authors}
            \State Compute feature disruption value as product of information gain, synonymy usage, and disruption constant $K$
            \State Append features' disruption values to basis matrix
            \State Apply disruptor based on pattern orientations
          
        \EndFor

        \State Repeat step 1-2 for each identity
    \EndProcedure
\end{algorithmic}
\end{algorithm}



\subsection{Karhunen-Loeve}
\label{sec:karhunen-loeve}

Karhunen-Loeve transforms are a supervised form of \ac{pca} that allows for inclusion of class information in the transform process.
It consists of a dimensionality reduction technique where the transformation is done by deriving the basis matrix (set of eigenvectors) and 
then projecting the feature usage matrix into a lower-dimension space.
While \ac{pca} captures the variance across a set of authors (interclass variance) using a single feature set and basis matrix, 
Karhunen-Loeve can be applied to each author (intraclass variance) by only considering the author's feature set and basis matrix.
Thus, Karhunen-Loeve can be used as an individual-level (cf. \autoref{sec:definitions}) similarity detection technique 
(more about that in \citep{abbasi_writeprints_2008} Chap. 2 last paragraph).


\subsection{Feature set dimension}
\label{sec:feature-set-dimension}

\citet{abbasi_writeprints_2008} define feature set breadth as the number of feature categories and 
the feature set depth as the number of features.

\subsection{Features}

First, the \citet{abbasi_writeprints_2008} filter out all message signatures in order remove obvious identifies.
Then, they extract two sets of features: Baseline feature set consisting of 327 static author-group-level features and an 
extended feature set consisting of static and dynamic (author-group-level and individual-identity level) features.
They provide a table with quantities and description of features both of the baseline and extended feature set.
\citet{abbasi_writeprints_2008} choose their features using the information gain heuristic $IG$.
The information gain of a feature $j$ across a set of classes $c$ is derived as $IG(c,j)=H(c)-H(c|j)$, 
where $H(c)$ is the overall entropy across author classes and $H(c|j)$ is the conditional entropy for feature $j$.

For \ac{pos} tagging, they used the Arizona noun-phrase extractor, which uses the Penn Treebank tag set.