\section{Voigt-Kampff}
\label{sec:voigt_kampff}
\newcommand{\voigtkampff}{Voigt-Kampff}

% problem definition
\citet{bevendorff_overview_2024} present results from the \voigtkampff{} challenge at 
\ac{pan} and \ac{eloquent} at \ac{clef} 2024.
The task is to determine whether a text was \ac{ai} generated or written by a human.
The authors want to establish a baseline and thus, choose to formulate the task in its simplest form:
\begin{quote}
    \textit{Given two texts, one authored by a human, one by a machine: pick out the human.}
\end{quote}
As in its fictional inspiration, 
\textit{Blade Runner} where an officer uses the \voigtkampff{} machine to test whether a subject is a replicant, 
the challenge is presented in a builder-breaker format:
\begin{itemize}
    \item \textbf{Builder at \ac{pan}}: Participants are asked to build a system that can detect 
    whether a text was written by a human or an \ac{ai}.
    \item \textbf{Breaker at \ac{eloquent}}: Participants are asked to build a system that can 
    generate texts that are indistinguishable from human-written texts.
\end{itemize}
Participants' submissions compete in an adversarial setting.

% similarity to Authorship identification
\citet{bevendorff_overview_2024} note the similarity to \ac{aa} and \ac{av} tasks, 
where the \ac{ai} is considered an author that exhibits identifiable characteristics.
They want prior research of authorship tasks in this \ac{ai} detection task to be considered.
Baselines for the \ac{ai} detection task include two \ac{av} systems, 
where each text is split in half and comparing them under the assumption that \ac{llm} texts 
are more self-similar than human texts.
The baselines are a compression model (\todo{PPMd Compression-based Cosine}) and 
short-text authorship unmasking.
\citet{bevendorff_overview_2024} also assume that \ac{llm}-generated texts show 
weaker coherence and textual entailment between sentences.

% machine/ AI
In this context, \ac{ai} refers to a language model.
They recommended using the following prompt to instruct the \ac{llm} to generate a text 
based on the bulletpoint summary of a news article generated as described in \cite{pan_dataset_authorship_verification}:
\begin{quote}
    \textit{Write a text of about 500 words which covers the following items:}
\end{quote}
% 9 different language models
The language models used in the challenge are:
\begin{itemize}
    \item \textbf{Poro} is an open-source, freely available multilingual decoder-only model.
    \item \textbf{Mistral}is an open-source, freely available model and trained for conversational data.
    \item and more cf. pg.8 \cite{bevendorff_overview_2024}
\end{itemize}
\citet{bevendorff_overview_2024} manually reviewed the generated texts and removed obvious artefacts 
(i.e., typical \ac{llm} chat phrases, excessive bullet lists, etc.).
They found that \ac{llm} texts failed to incorporate the quotes in the text and count the number of words.
The majority of open-source models were found to be less prone to these artefacts than closed-source \acp{llm}.
Finally, \citet{bevendorff_overview_2024} truncated the human texts to the same length as the \ac{llm} texts by 
fitting a \todo{log-normal distribution} to the \ac{llm} text lengths and drawing from the distribution for each human text.
The drawn samples were used as starting point and texts were truncated after the nearest paragraph ending in a window of at most 200 characters.

% test case
\citet{bevendorff_overview_2024} chose to hold back a variant of Gemini Pro model on the training set to test 
how robust submitted solutions are against the unseen \ac{llm} models.

% model output
\citet{bevendorff_overview_2024} chose to adapt the \ac{pan} \ac{av} scoring scheme.
During inference, the models received pairs of human and \ac{llm} texts.
For each sample, a score $\in [0, 1]$ is assigned to the sample.
For a score $<0.5$, the model predicts the left text to be human,
and for a score $>0.5$, the model predicts the right text to be human.
A score of exactly 0.5 means the model abstains from the sample.

% model metrics
They defined the effectiveness of the systems, i.e. models, as the arithmetic mean of the metrics defined in \autoref{sec:pan_evaluation}, 
where each metric is $\in [0, 1]$.
They correct all metrics by discounting half a standard deviation, estimated on each dataset individually, 
from the system's scores \todo{with $n-1$ degrees of freedom}.
This penalizes unstable systems with widely varying scores on the individual dataset variants, 
and promotes systems more robust to the dataset variants' alterations, including text obfuscation and modification.
Finally, they computed the macro-averaged scores across all datasets since they consider all datasets evenly important 
(even those with fewer number of examples).
With regard to the leader board, i.e. ranking of the approaches, they computed a \todo{mean score correlation coefficient (Kendall's $\tau$)}.

% dataset metrics
They defined the difficulty of the datasets as the inverse mean effectiveness score of the detection systems.


% source: Ben
micro/ macro in the context of multi-class confusion matrix (bc precision/ accuracy/ recall for binary classification)
micro average score = same as binary, but sums over class indices in both nominator and denominator, e.g.
$micro\_precision = \frac{\sum_{i=1}^n TP_i}{\sum_{i=1}^n (TP_i + FP_i)}$
macro average score = average of the class-wise scores, e.g.
$macro\_precision = \frac{1}{n} \sum_{i=1}^n \frac{TP_i}{TP_i + FP_i}$
where $n$ is the number of classes.