\citet{zangerle_overview_nodate} provide \autoref{tab:hierarchy_authorship_verification_problems}, i.e., 
the decomposition of authorship verification into multiple subtasks. 
They order the subtasks in terms of their complexity.
The first task on the one hand, is considered the easiest, since we know that out of two text, one is guaranteed to be human-generated while the other one is \ac{llm}-generated.
The last task on the other hand, is denoted as the most difficult, since we do not know whether the text is human- or \ac{llm}-generated.


\begin{table}[tbp]
    \centering
    \caption{Hierarchy of authorship verification problems from easiest (1) to most difficult (7), 
    where A, B corresponds to human-authored text and M denotes \ac{llm}-generated text.}
    \label{tab:hierarchy_authorship_verification_problems}
    \resizebox{\textwidth}{!}{%
    \begin{tabular}{lll}
        \toprule
    \rowcolor[HTML]{EFEFEF} 
    \textbf{Difficulty} & \textbf{Input/ Task} & \textbf{Possible Assignment Patterns} \\  \midrule
    1 & \{?,?\} & \{A,M\} \\ 
    2 & \{?,?\} & \{A,M\}, \{A,A\} \\
    3 & \{?,?\} & \{A,M\}, \{M,M\} \\
    4 & \{?,?\} & \{A,M\}, \{A,A\}, \{M,M\} \\
    5 & \{?,?\} & \{A,M\}, \{A,A\}, \{A,B\} \\
    6 & \{?,?\} & \{A,M\}, \{A,A\}, \{A,B\}, \{M,M\} \\
    7 & ? & A, M \\ \bottomrule
    \end{tabular}%
    }
    \end{table}