\section{Cross-Domain Pre-Trained Language Models}
\label{sec:cross_domain_pre_trained_LM}

\subsection{Features}
% topic invariant features
\citep{barlas_cross_domain_2020} claim function words or \ac{pos} n-grams are topic invariant features.
% efficient in AA
\citep{barlas_cross_domain_2020} state that character n-grams associated with word affixes and punctuation marks 
are most efficient in cross-topic \ac{aa} tasks.
% pre-processing
It is possible to pre-process texts to mask topic-related information while keeping the text structure including 
function words and punctuation marks.

\subsection{Dataset splits}
\label{sec:dataset_splits}

\citet{barlas_cross_domain_2020} propose Leave-One-Topic-Out cross-validation splits for cross-topic \ac{aa} tasks:
All texts on a specific topic (within a certain genre) are included in the test set,
while the remaining texts (in that genre) are used for training.
Mean classification accuracy is reported over all topics.

Analogously, \citet{barlas_cross_domain_2020} propose Leave-One-Genre-Out cross-validation splits for cross-genre \ac{aa} tasks:
All texts of a specific genre (within a certain topic) are included in the test set,
while the remaining texts (in that topic) are used for training.
Mean classification accuracy is reported over all genres.

\subsection{Normalization corpus} 
\label{sec:normalization_corpus}

In \ac{aa} tasks, it is important for the normalization corpus to have 
exactly the same properties with documents of unknown authorship.
This is not feasible in practice \citep{barlas_cross_domain_2020}.