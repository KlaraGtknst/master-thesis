\section{Imposter method}
\label{sec:imposter_method}

The problem to solve is the \ac{av} problem:
For two documents $X, Y$ determine if they were written by the same author.
As displayed in \autoref{fig:problem_reduction}, \citet{koppel_determining_2014} propose reducing the \ac{av} problem to the many-candidates problem.

\begin{figure}[htbp]
    \centering
    \includesvg{notes/Koppel_imposter_2014/reduction_closed_set_AV_to_open_set_AA}
    \caption{Reducing the \ac{av} problem to the many-candidates problem.}
    \label{fig:problem_reduction}
\end{figure}

This method produces a set of \textit{imposter} documents.
Then, \citet{koppel_determining_2014} determine whether $X$ is sufficiently more similar to $Y$ than to the imposter documents.
There are multiple important settings:
\begin{itemize}
    \item Proper methods to select the imposter documents.
    \item Proper methods to measure the similarity between documents.
\end{itemize}
Similar to unmasking, \citet{koppel_determining_2014} repeatedly select random subsets of features that serve as the basis for comparing documents.
If a documents $Y$ is more similar to document $X$ than any other document for many feature subsets, 
it is likely that $X$ and $Y$ are by the same author.
% FIXME: autoref doesn't know lst 
The algorithmic approach is displayed in \autoref{lst:imposter_algo}.
\citet{koppel_determining_2014} state that k=100 iterations are sufficient.
The threshold $\sigma^*$ varies the recall-precision tradeoff.
\citet{koppel_determining_2014} claim that the imposter method obtains strong results even for documents with no more than 500 words.
Moreover, they highlight the similarity between the imposter method and an ensemble of classifiers learning different subset of features.
% TODO: two paragraphs below conflicting? cf. koppel_determining_2014 pg. 181, 182
Furthermore, the performance of the method improves as the number of candidate author diminishes.
In an open-set scenario, fewer candidate authors makes the problem more difficult for the imposter method 
since one author being consistently more similar to the text than the others candidates across multiple feature subsets is less likely in large sets of competing candidate authors.
Small sets of candidate authors are more likely to generate \acp{fp} 
while large sets may generate \acp{fn} and thus, creating a tradeoff.

\todo{add algo}
\begin{algorithm}
    \caption{Author Identification via Random Feature Subsets}
    \label{lst:imposter_algo}
    \begin{algorithmic}[1]
        \Procedure{GetAuthor}{$X$, $features$, $candidate\_authors$, $k$, $\sigma^*$}
           TODO

        \EndProcedure
    \end{algorithmic}
\end{algorithm}
