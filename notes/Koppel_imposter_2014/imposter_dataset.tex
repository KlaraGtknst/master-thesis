\section{Impostor dataset}
\label{sec:impostor_dataset}

\citet{koppel_determining_2014} construct a corpus of 500 documents pairs $<X,Y>$ of blog posts.
Half of them are by the same author, half by different authors.
No single author is represented more than once.
The task is to determine whether the two documents are by the same author.
Since the model is not supplied any labelled samples of any author, the method is considered unsupervised.

A second dataset uses students' essays.
This dataset is constructed such that neither same-author pairs nor different-author pairs 
are ever about a single topic or on the same subgenre.
Impostor documents for a document $Y$ are selected from the same subgenre as $Y$.
Same-authors pairs in this dataset differ both in terms of subgenre and topic, leaving fewer common features to exploit.
Hence, this dataset ensures that models do not classify based on topic or subgenre.
However, this scenario proves more difficult for the imposter method.