\section{Problem hierarchy}
\label{sec:problem_hierarchy}

% koppel_determining_2014 use "identification" for attribution/ AA
\todo{change identification to attribution for consistency in terminology}

% AV -> open-set
\ac{av} is an open-set problem, meaning that the author of an anonymous document 
may or may be not be part of the set of candidate authors.

% AA -> closed-set
\ac{aa} is a closed-set problem, meaning that the author of an anonymous document
is part of the set of candidate authors.
For each candidate author, writing samples are available.
The task is to determine the author of the anonymous document from the set of candidate authors.

% reduction: closed-set AA -> open-set AV
\citet{koppel_determining_2014} state that all closed-set \ac{aa} problems are reducible to the \ac{av} problem.
The reverse is not true.
To reduce the \ac{aa} problem to the \ac{av} problem, we solve a \ac{av} problem, i.e.\ if text was written by a candidate author, 
for each of the respective candidates.
Ideally, we receive one positive answer for the correct candidate author and negative answers for all other candidates.

% complexity
\citet{koppel_determining_2014} explain that the \ac{av} problem is more complex than the \ac{aa} problem.
They claim that the ability to solve a closed-set \ac{aa} problem does not imply the ability to solve an open-set \ac{av} problem.

% open-set identification/ AA = many candidates problem
\citet{koppel_determining_2014} define the many-candidates problem, or the so-called open-set identification problem:
Given a large set of candidate authors, determine which, if any, of them wrote a given anonymous document.
According to \citet{koppel_determining_2014}, the many-candidates problem can be solved reasonably well: \Cref{lst:many_candidate_algo}.

\begin{algorithm}
    \caption{Author Identification via Random Feature Subsets}
    \label{lst:many_candidate_algo}
    \begin{algorithmic}[1]
        \Procedure{GetAuthor}{$X$, $features$, $candidate\_authors$, $k$, $\sigma^*$}
            \State Initialize count vector $M[c] \gets 0$ for all $c \in candidate\_authors$

            \For{$i = 1$ to $k$} \Comment{Number of iterations $k$}
                \State $F_i \gets \text{RandomSubset}(features)$ \Comment{Set of features $features$}
                \For{$c \in candidate\_authors$}
                    \State Compute similarity: $s_c \gets \text{Similarity}(X, Y_c|_{F_i})$ \Comment{Input text $X$}
                \EndFor
                \State $c^* \gets \arg\max_{c} (s_c)$
                \State $M[c^*] \gets M[c^*] + 1$  \Comment{Increment match count for best match}
            \EndFor

            \For{$c \in candidate\_authors$}
                \State $\text{Score}(c) \gets \frac{M[c]}{k}$ \Comment{Proportion of times $c$ was top match}
            \EndFor

            \State $c^{\text{final}} \gets \arg\max_{c} \text{Score}(c)$
            \If{$\text{Score}(c^{\text{final}}) > \sigma^*$} \Comment{Threshold $\sigma^*$}
                \State \Return $c^{\text{final}}$
            \Else
                \State \Return $\text{Don't know}$
            \EndIf
        \EndProcedure
    \end{algorithmic}
\end{algorithm}

% AV -> many-candidates problem
To reduce the \ac{av} problem to the many-candidates problem, we need to generate a large set of imposter candidates.
% open-set identification/ AA -> open-set AV => equivalence
\citet{koppel_determining_2014} state that the many-candidates problem is equivalent to the open-set \ac{av} problem.
% closed-set AA/ identification -> many-candidates problem/ open-set AV/ identification
The closed-set identification is reducible to the many-candidates problem.

\begin{figure}[htbp]
    \centering
    \includesvg{notes/Koppel_imposter_2014/problem_hierarchy_complexity}
    \caption{Problem hierarchy and complexity.}
    \label{fig:problem_hierarchy}
\end{figure}