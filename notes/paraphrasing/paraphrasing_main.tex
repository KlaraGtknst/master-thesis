\section{Paraphrasing}
\label{sec:paraphrasing}
\newcommand{\bluert}{\href{https://github.com/google-research/bleurt}{BLUERT}}

\citet{fu_learning_2024} propose PEARL, a black-box paraphrasing model to meet \ac{llm} expression style.

\citet{fu_learning_2024} state that paraphrase divergence (cf. \autoref{sec:definitions}) can be explained as language models 
not only learn knowledge but also expression patterns associated with the knowledge from a corpus during pre-training.
Ideally, prompt learning should be independent of questions, but in reality, prompts are task- or domain-specific.
Hence, preferences for a certain format within a particular task or domain learnt during training persist in the model.
In other words, \acp{llm} may exhibit different preferences for various semantics.

\bluert{} is machine evaluation metric for paraphrase generation.
\citet{fu_learning_2024} use \bluert{} to filter out incorrect paraphrases (i.e. using a threshold $\theta$).

UPRISE is a universal prompt auto-retrieval method that tunes a lightweight prompt retriever based on contrastive learning \cite{fu_learning_2024}.
There is also a official recommended template for manual prompt construction \cite{fu_learning_2024}.

Popular paraphrase categories include \cite{fu_learning_2024}:
\begin{itemize}
    \item Top-level classification perspective: 
        \begin{itemize}
            \item Lexicon-based changes
            \item Morphology-based changes
            \item others
        \end{itemize}
    \item Second-level classification perspective:
        \begin{itemize}
            \item Change of format
            \item Semantic-based
            \item Change of order
        \end{itemize}
\end{itemize}

\citet{fu_learning_2024} give three prompt tips:
\begin{enumerate}
    \item Make question/ prompt as clear as possible even if some restrictive requirements may seem unnecessary.
    \item Place significant details, including restrictive elements such as time, place, manner, reason, purpose and conditions, at the beginning of the prompt.
    \item Pay attention to spelling, i.e. proper nouns, title, honorifics, abbreviations, acronyms and observe capitalization.
\end{enumerate}


% metric
\citet{fu_learning_2024} describe the Gini Coefficient as a measure of inequality in a distribution, ranging from 0 (i.e. even distribution across categories) to 1.
