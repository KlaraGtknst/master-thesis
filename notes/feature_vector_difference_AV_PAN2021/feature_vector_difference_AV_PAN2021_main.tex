\section{Feature Vector Difference}
\label{sec:feature_vector_difference}

\citet{weerasinghe_feature_vector_difference_2021}'s goal is to create a topic agnostic \ac{av} model that works well on the open-world setting 
where authors and topics in test set are not in the training set. They propose a new feature vector difference method that is based on the \ac{av} model of \citet{stamatatos_2020} and show that it outperforms the state-of-the-art methods in the open-world setting.
The input to the model is a feature vector $X$ encoding the two documents $D_i,D_j$ 
and the target variable $Y$ indicating whether the two documents are written by the same author or not. 
Hence, they model the problem as a binary classification task.

% nltk library
\subsection{\ac{nltk} Library usage}
They use \ac{nltk}'s \texttt{casual\_tokenize} method, which uses their \texttt{TweetTokenizer} to tokenize the documents.
For \ac{pos} tagging, the use \ac{nltk}'s Perceptron Tagger.
In order to generate a \todo{partial parse tree} (i.e. \ac{pos} chunks), they train a Maxent (Maximum Entropy) classifier.
The parse trees are used to extract so-called syntactic dependency-based n-grams of \ac{pos} tags, i.e, stylometric features.

\subsection{Sklearn Library usage}
They use \ac{sklearn}'s \texttt{TFIDFVectorizer} to create the \ac{tf-idf} vectors for their features (cf. below).
They set \texttt{min\_df=0.1} to ignore terms that have less than a 10 \% document frequency.

\citet{weerasinghe_feature_vector_difference_2021} used a Stochastic Gradient descent training algorithm with logarithmic loss function, 
which results in a logistic regression classifier (by sklearn's  \texttt{SGDClassifier}) since the complete feature matrix cannot be stored in-memory.

\subsection{Features}
\textcolor{brown}{cf. Table 1, \cite{weerasinghe_feature_vector_difference_2021} Chapter 2.3 for urls}
\citet{weerasinghe_feature_vector_difference_2021} used some features from Writeprints feature set.
Character n-grams are \ac{tfidf} values for character n-grams of size 1 to 3.
They claim $n>3$ is not useful for \ac{av} because it is affected by topic similarities.
\ac{pos}-Tag n-grams are \ac{tfidf} values for \ac{pos}-tag 3-grams.
Special characters are \ac{tfidf} values for 31 pre-defined special characters.
Frequency of function words (where \cite{weerasinghe_feature_vector_difference_2021} linked a stop word website!) 
are frequencies for 851 common English words.
They also computed the average number of characters per word. %, the average number of words per sentence,
Moreover, they included the distribution of word-lengths (fraction of tokens of length $l \in [1,10]$).
They employed Python's \texttt{textComplexity} library to compute a variety of vocabulary richness measures 
(cf. page 4, \cite{weerasinghe_feature_vector_difference_2021}).
The \ac{pos}- tags chunks are the \ac{tfidf} values for the second level \ac{pos}-tag chunks (e.g., 'NP', 'VP', 'IN').
The \ac{pos}-tag construction are \ac{tfidf} values of each noun phrase, verb phrase, and prepositional phrase expansion.
They also define the stop-word and \ac{pos}-tag hybrid tri-grams to capture the syntactic information about the word order, 
inspired by text distortion:
In order to prevent topic related biases, all non-function words are replaced with the \ac{pos}-tag of the word.
Based on this, \ac{tfidf} values are computed for tri-grams.
The \ac{pos}-tag ratio calculated the portion of all \ac{pos}-tags in the \todo{Penn Treebank \ac{pos} Tag} collection.
They also include unique spelling, where the fraction of words present in the document that belong to each of the following dictionaries: 
Commonly misspelled English words, common typos when communicating online, common errors with determiner, 
British spelling of English words and popular online abbreviations.

\subsection{Pipeline}
Feature extractors were fit on training data.
Data was standarized (i.e., mean=0, std=1).
An absolute vector difference was computed for each feature vector pair and standarized again.
This standarized vector difference was used as input to the classifier.



