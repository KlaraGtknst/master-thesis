\documentclass[
    %draft, % Mit % kommentieren, um Bilder sichtbar zu machen und Links zu aktivieren
    english,
    master,
    ks, % University of Kassel
    % pdftex,
    % a4paper,
    twoside,
    % parskip=half,
    % numbers=noenddot,
    % listof=totoc,
    bibliography=totoc,
    % hyperfootnotes=false,
    % english,
    % openright
    open=right,
]{webisthesis}

% FIXME: this doesn't work: https://blog.rtwilson.com/how-to-add-simple-new-commands-to-latex-to-help-with-writing-papers/ (11.04.2025)
\newcommand{\todo}[1] {\textbf{\textcolor{red}{TODO:}} #1}
\newcommand{\thesistypedesc}{Department of Electrical Engineering and Computer Science \\
    University of Kassel}
\newcommand{\thesisauthorname}{Klara Maximiliane Gutekunst}
\newcommand{\thesisauthorhometown}{34119 Kassel}
\newcommand{\thesisauthormatrikelnumber}{35677772}
\newcommand{\thesisauthoremail}{klara.gutekunst@uni-kassel.de}
\newcommand{\thesisdepartment}{Chair Deep Semantic Learning}
\newcommand{\thesisfirstreviewer}{Prof.\ Dr.\ Martin Potthast}
\newcommand{\thesissecondreviewer}{Prof.\ Dr.\ Gerd Stumme}
\newcommand{\thesisdate}{\today}

\ThesisSetTitle{Leveraging \acs{llm}-Generated\\Hard Negatives for the\\ \impApprTitle{}}
% \ThesisSetTitle{Leveraging \acsp{llm} for the \impApprTitle{}}
\ThesisSetKeywords{\ac{av}, stylometry, hard negatives, \acsp{llm}} % only for PDF meta attributes
\ThesisSetLocation{\thesisauthorhometown{}} 

\ThesisSetAuthor{\thesisauthorname{}}
\ThesisSetStudentNumber{\thesisauthormatrikelnumber{}}
\ThesisSetDateOfBirth{25}{09}{2001}
\ThesisSetPlaceOfBirth{Brilon, Germany}

% Supervisors should usually be Professors from the candidate's university. A second supervisor is not always needed. 
\ThesisSetSupervisors{\thesisfirstreviewer{},\thesissecondreviewer{}}

\ThesisSetSubmissionDate{26}{09}{2025}


% content-specific commands
\newcommand{\tira}{TIRA}
\newcommand{\pextractor}{Paraphrase Extractor}
\newcommand{\pgenerator}{Paraphrase Generator}
\newcommand{\dataGutenberg}{Gutenberg}
\newcommand{\dataBlog}{Blog Posts}
\newcommand{\dataPan}{PAN20}
\newcommand{\dataStudent}{Student Essay}
\newcommand{\dataArtificialStudent}{Artificial Student Essay}
\newcommand{\dataCustom}{Custom CNN News}
\newcommand{\bluert}{\href{https://github.com/google-research/bleurt}{BLUERT}}

%    For lots of awesome glyphs: https://mirror.physik.tu-berlin.de/pub/CTAN/fonts/fontawesome/doc/fontawesome.pdf
% \usepackage[printonlyused]{acronym}
\pagenumbering{roman} % lowercase roman numerals
% Select input encoding, usually utf8 is the best choice, on windows, \usepackage[latin1]{inputenc} maybe required
\usepackage[utf8]{inputenc}
\usepackage[T1]{fontenc}
\usepackage[english]{babel}
\usepackage{csquotes}
\usepackage{xcolor}

\MakeOuterQuote{"} % Damit ist es möglich, " " zu verwenden ohne Umlaut zu erzeugen
\defaulthyphenchar=127 % Dadurch werden auch Wörter mit Bindestrich getrennt, die schon Bindestriche enthalten.

% geometry
\usepackage[bindingoffset=1cm, left=2.5cm, right=2.5cm, top=2.5cm, bottom=2.5cm]{geometry}

% Headline
\usepackage{fancyhdr}
\pagestyle{fancy}
\renewcommand{\chaptermark}[1]{\markboth{\thechapter\ #1}{}}
\lhead{\leftmark} \rhead{\thepage}
\cfoot{}
\fancypagestyle{plain}{}

\RedeclareSectionCommand[beforeskip=1.5cm,afterskip=1cm]{chapter}

% Colors
\usepackage{color}
\usepackage{colortbl}

% Tables
\usepackage{tabularx}
\usepackage{multirow}
\setlength{\tabcolsep}{4pt}

% Drawing graphs etc.
\usepackage{pgf}
\usepackage{tikz}
\usetikzlibrary{arrows,automata}

% Footnotes
\usepackage{footmisc}
\usepackage{xspace}
\newcommand{\sic}{[\acs{sic}]\xspace}

% math
\usepackage{amsmath}
\usepackage{amssymb}

\usepackage{siunitx}

% lists
\usepackage{paralist}

% Figures
\usepackage{graphicx, wrapfig}

% Hyperlinks
\usepackage[hyphens]{url}
\usepackage{hyperref}
\hypersetup{colorlinks, citecolor=black, linkcolor=black, urlcolor=black}

% Minted
\usepackage[chapter]{minted}
%\usemintedstyle{xcode}
\setminted{frame=single,tabsize=2,linenos,autogobble}

\newmintinline[code]{text}{breaklines}

\newminted[mdcodeblock]{md}{autogobble,frame=none,linenos=false,breaklines}

% list of abbreviations
\usepackage[printonlyused]{acronym}

% Set line pitch
\usepackage{setspace}
\onehalfspacing              % anderthalbzeilig (oder auch \doublespace)

%fancyBox
%\usepackage{fancybox}

% Layout corrections (Schusterjungen)
\clubpenalty = 10000
% Layout corrections (Hurenkinder)
\widowpenalty = 10000
\displaywidowpenalty = 10000

% Figures
\usepackage{caption}
\usepackage[hypcap=true,labelformat=simple]{subcaption}
\renewcommand{\thesubfigure}{(\alph{subfigure})}
\usepackage[inkscapelatex=false]{svg}

% Tables
\usepackage{booktabs}
%\usepackage[table,xcdraw]{xcolor}

% enumerate
\usepackage{enumitem}
\newlist{questions}{enumerate}{2}
\setlist[questions,1]{label=RQ\arabic*.,ref=RQ\arabic*}
\setlist{nolistsep}

% Bibliography
\usepackage[square,numbers]{natbib}
\bibliographystyle{plainnat} % or plainnat abbrvnat unsrtnat

% Frequently used column types
\newcolumntype{C}[1]{>{\centering\arraybackslash}p{#1}} % centering column type with fixed width
\newcolumntype{R}[1]{>{\raggedleft\arraybackslash}p{#1}} % right aligned column type with fixed width
\newcolumntype{L}[1]{>{\raggedright\arraybackslash}p{#1}} % left aligned column type with fixed width

% Shortcuts for referencing floats:
\newcommand{\fig}[1]{\figurename~\ref{#1}} %shortcut for a figure reference
\newcommand{\tab}[1]{Table~\ref{#1}} %shortcut for a table reference
\newcommand{\eq}[1]{(\ref{#1})} %shortcut for an equation reference
\newcommand{\lst}[1]{Listing~\ref{#1}} %shortcut for a listing reference
\newcommand{\sect}[1]{Section~\ref{#1}} %shortcut for a Section reference
\addto\extrasenglish{%
  \renewcommand{\sectionautorefname}{Section}%
  \renewcommand{\subsectionautorefname}{Subsection}%
  \renewcommand{\chapterautorefname}{Chapter}%
}

% Shortcut for terms
\newcommand{\databaseName}{Elasticsearch}
\newcommand{\flask}{Flask}
\newcommand{\angular}{Angular}
\newcommand{\infersent}{InferSent}
\newcommand{\wordcloud}{word cloud}
\newcommand{\localMaschineStats}{Apple M2 Pro MNW83D/A with 16 \ac{gb} RAM and 12 cores}
\newcommand{\slurm}{Slurm}




\begin{document}
    \begin{frontmatter}
        \cleardoublepage 
        \chapter*{Abstract}
\markboth{Abstract}{Abstract}

\Acl{av} seeks to determine whether two texts share the same author, a task critical for ensuring the integrity of academic submissions or online content.
Existing approaches typically exhibit poor generalisation across domains.
The \impAppr{} introduced hard negative sampling to create input-pair-specific settings to improve cross-domain generalisation. 
However, traditional sampling strategies are unable to simultaneously control for multiple confounding variables, such as genre and topic.
%, highlighting the need for improved sampling strategies.
This thesis investigates whether \aclp{llm} can help address these limitations. 
Specifically, we employ \acs{llm}-generated paraphrases as hard negatives. 
% Since these paraphrases aim to preserve the confounding variables of the original text, they mitigate domain-related biases during inference. 
% Moreover, by constructing a tailored case for each input text pair, the approach eliminates out-of-distribution issues, ensuring that comparisons remain within the distribution defined by the pair itself.
Evaluation on the \dataStudent{} dataset from the original study shows that the \acs{llm}-based extension surpasses the original baselines by \citet{koppel_determining_2014}, \acs{ppmd}, and \unmasking{} in terms of precision and recall.
At the same time, our results reveal the practical and conceptual challenges of integrating \acsp{llm} into \acl{av}, including issues of reliability, hallucination, and control over paraphrase quality.



% The original \impAppr{} compares a disputed text to a candidate text and hard negatives, considering disputed and candidate text to share an author if the candidate is consistently the most similar across random feature projections. 
        \cleardoublepage 

        \tableofcontents
        \cleardoublepage 
        \chapter*{List of abbreviations}
\markboth{List of abbreviations}{List of abbreviations}

\begin{acronym}[XXXXXXXXX]
    \acro{ai}[AI]{Artificial Intelligence}
    \acro{ir}[IR]{Information Retrieval}
    \acro{nlp}[NLP]{Natural Language Processing}
    \acro{llm}[LLM]{Large Language Model}
    \acro{roc-auc}[ROC-AUC]{Area Under the Receiver Operating Characteristic Curve}
    \acro{pan}[PAN]{Plagiarism Analysis and Authorship Mining} % TODO: find out if correct
    \acro{bert}[BERT]{Bidirectional Encoder Representations from Transformers}
    \acro{bow}[BoW]{Bag-of-Words}

    % \acro{}[]{}
\end{acronym}

    \end{frontmatter}
    \cleardoublepage 
    \pagenumbering{arabic}

    \chapter{Introduction}
\label{chap:introduction}



% motivation
Historically, authorship analysis focused on literary disputes~\citep{neal_surveying_2018,stamatatos_survey_2009}, but contemporary concerns have shifted towards practical applications.
In an era where large amounts of text can be copied, paraphrased, or fabricated with ease, determining the true author of a text is crucial for maintaining trust in communication. 
Scenarios include detecting plagiarised passages of texts~\citep{stein_intrinsic_2011}, and verifying the authenticity of online content or student submissions. 
Formally, we refer to these problems as \acf{av} or \acf{aa}, where every \ac{aa} task can be formulated as a sequence of \ac{av} problems~\citep{tyo_state_2022,barlas_cross_domain_2020}.

The emergence of \acp{llm} adds an additional layer of complexity. 
While these models are widely embraced for beneficial applications such as summarisation, information seeking, and assistive writing~\citep{wang_stumbling_2024}, their ability to convincingly imitate human writing creates new risks. 
\acp{llm} can be used to generate misinformation, impair academic honesty, or impersonate individuals, thereby inflicting harm on individuals who fall victim to these schemes~\citep{mitchell_detectgpt_2023,li_learning_2025,wang_stumbling_2024,bhattacharjee_fighting_2024}. 
Since \acp{llm} can be conceptualised as authors, their detection naturally falls within the scope of \ac{av}. 
Thus, instead of treating \ac{llm} detection as an isolated task, it is more consistent to frame it as a specialised case of \ac{av}~\citep{llm_detection_av_2025}.

% specificity rather than generality
Existing approaches to generalisation typically train a single model and apply it across domains.
Despite significant advances in \ac{av}, prior work finds that such models struggle in \ac{ood} settings, where the topic or genre diverges from the training data~\citep{Sundararajan_style_18,bischoff_importance_2020,li_learning_2025}. 
This shortcoming motivates a shift towards scenario-specific solutions, i.e.\ models are trained anew for narrowly defined cases. 
Such single-case approaches enable more precise control over contextual factors and place greater emphasis on stylistic idiosyncrasies rather than domain-level variation.

% AV
The \impAppr{} by \citet{koppel_determining_2014}\ introduces the idea of generating \imp{} texts, i.e.\ hard negatives, used to sharpen the discrimination between genuine and false authorship matches. 
However, the method's effectiveness is limited by the quality and contextual adequacy of these \imp{} texts. 
Previous work did not fully address how to construct challenging \imps{} via controlled contextual variables.

The thesis extends the \impAppr{} by leveraging \acl{sota} \acp{llm} to generate paraphrases as \imps{}, enabling control over multiple confounding factors such as genre, topic, and target audience. 
In doing so, the approach shifts the focus towards authorial style rather than domain differences, yielding improved precision–recall on the \dataStudent{} dataset compared with the original sampling strategies, \unmasking{} and \acs{ppmd}.


\section{Research Questions}
\label{sec:research_questions}
To guide this objective, we formulate the following research questions:
\begin{questions}
    \item \textbf{How can we instruct a \ac{llm} to paraphrase the text of a candidate author such that it captures the \ac{llm}'s stylistic properties?} \label{enum:rq1} \hfill \\
    The goal is to create hard negatives for the Impostor method by controlling contextual factors.
    By controlling genre, topic and other factors, similarity measures primarily focus on differences in authorial style rather than the impact of content on style.
    We obtain this controlled environment by utilizing \acp{llm} to paraphrase the original text.
    There are different approaches to paraphrasing text using \acp{llm}.
    They include (a) directly asking the \ac{llm} to paraphrase the text, 
    (b) first extracting specific information from the original text and subsequently generating a paraphrase based on the information.
    This thesis compares both approaches on \dataStudent{}, \dataBlog{}, \dataGutenberg{} and \dataPan{}.

    \item \textbf{How do we evaluate the quality of paraphrases?} \label{enum:rq2} \hfill \\
    Paraphrase evaluation is inherently challenging, as there is no universally agreed-upon definition of what constitutes a paraphrase. 
    Prior research often adapts evaluation metrics from related \ac{nlp} tasks such as machine translation or summarization. 
    Two key dimensions are typically considered: semantic similarity and syntactic similarity.
    Contrary to initial intuition, high syntactic similarity is not necessarily desirable, as it may indicate that the \ac{llm} has merely copied the original text with minimal changes. 
    Instead, our focus lies on achieving high semantic similarity while maintaining syntactic diversity to ensure genuine rephrasing.
    Furthermore, we acknowledge that relatively low automatic scores can still be acceptable if qualitative human evaluation confirms the paraphrase’s adequacy.

    % \item \textbf{Which features are used for the \ac{av} problem?} \label{enum:rq3} \hfill \\
    % Traditional features include character tri-gram features, while newer research has proposed using \ac{llm} such as BERT.

    \item \textbf{How does the \ac{llm}-based impostor approach perform compared to state-of-the-art models?} \label{enum:rq4} \hfill \\
    Though our approach is computationally expensive, we argue that it is not a general purpose \ac{llm} detection method, but rather a single case solution tailored to specific detection tasks.
    We evaluate its performance in scenarios where (a) the disputed text is human generated,
    (b) the disputed text is \ac{llm} generated and the candidate is the same \ac{llm}, and
    (c) the disputed text is \ac{llm} generated, but the candidate is a different \ac{llm}.
    In terms of performance, we compare our method to other \ac{av} approaches on the \dataStudent{}, \dataBlog{}, \dataGutenberg{} and \dataPan{} datasets.
    
\end{questions}

% \section*{Idea}
% \label{sec:idea}

% Given a text of unknown authorship (i.e., human or \ac{llm}), 
% construct a set of impostor texts using state-of-the-art \acp{llm} based on the original text.
% Obtain the author by \ac{aa}/ \ac{av} methods, such as unmasking, to \textit{confidently}, i.e. high precision, identify \ac{llm} generated texts
% (and possibly which \ac{llm}).



\section{Contributions}
\label{sec:contributions}
The contributions of this thesis are:
\begin{enumerate}
    \item Reimplementation of the traditional Impostor approach (cf. \autoref{chap:implementation}).
    \item Extension of the impostor approach with \ac{llm} generated impostors for line-up of difficult opponents (cf. \autoref{chap:methodology}). 
    \item Frame \ac{llm} detection as a \ac{av} problem and use \ac{llm} generated text as candidate for text of "unknown" authorship.
\end{enumerate}


\section{Thesis Structure}
\label{sec:thesis_structure}
The thesis is structured as follows:
    \cleardoublepage 
    \chapter{Background on \ai{}}
\label{chap:authorship_identification}
%    AI
%     - Woher kommt es?
%         - initial statistische analyse
%         - Dylo hypethese???
%         - stylometry
%         - Use case 1/ original use case: Literatur Forschung
%         - Use case 2: Digital text forensics
%     - AA
%     - AV als Kernproblem von allem
%         - Fokus hier, und LLM detection istr ein Plus, eher im Anhang
%     - technischer Hintergrund: open/ closed set, one-class classification
%     - state of the art models
%     - \imp{} method

This section offers background information on \ai{}.
Starting exploring with the origin of \ai{} in \ac{aa} of historical texts, we outline the initial statistical approaches and touch upon stylometry.
Next, we highlight why \ai{} is still relevant today, and break down some essential tasks of \ai{}.
We proceed with the technical background and \ac{sota} approaches.
We conclude with an outlook on the impact of \acp{llm} on \ai{}.

\textcolor{red}{TODO}

\section{Historical Background and Early Approaches}
\label{sec:origin}

Research on authorship originated in the 19th century as an attempt to resolve disputes over the authorship of literary works. 
Augustus de Morgan was among the first to propose a quantitative approach in 1851, using word-length frequencies. 
Building on this idea, in 1887 Thomas C. Mendenhall carried out the first systematic analysis of word-length distributions in the works of Bacon, Marlowe, and Shakespeare, aiming to shed light on the authorship of Shakespearean plays~\citep{neal_surveying_2018,stamatatos_survey_2009}.
Word-length distributions quantify the number of words on the vertical axis by their length in characters on the horizontal axis. 
\citet{wordlengths_mendenhall_1887} concluded that differences between the curves of two disputed texts may indicate different authorship, whereas similarity in the curves is less reliable as evidence of shared authorship.

% Popular examples of \ai{} include a collection of Hebrew-Aramaic letters supposedly by a rabbinic scholar in Baghdad in the late 19th century~\citep{koppel_authorship_2004}.
% The latter denied having authored the text collection, which was contested by \citet{koppel_authorship_2004}.

% \section{First Approaches}

% Approaches of the \nth{19} century were limited to the analysis of word lengths.
In the early \nth{20}, statistical measures such as Zipf's law (1932) and Yule's characteristic (1944) were introduced. 
These measures captured word frequency distributions and vocabulary richness, laying the foundation for more formalized analyses of writing style~\citep{neal_surveying_2018,stamatatos_survey_2009}.

A decisive milestone came with the work of Mosteller and Wallace (1964), who applied Bayesian statistical analysis to the "The Federalist Papers".
Their computer-assisted approach, based on the frequencies of a subset of common function words (e.g., "and", "to"), demonstrated the potential of rigorous quantitative methods for \ac{aa}. 
This work is often regarded as the beginning of modern stylometry, the systematic study of quantifiable features of writing style.

Between the 1960s and the late 1990s, stylometric research flourished.
Around \num{1000} features were proposed, including sentence and word lengths, character and word frequencies, and vocabulary richness. 
However, the field faced two major limitations: the absence of benchmark datasets prevented systematic comparison of methods, and evaluation often relied on subjective inspection of visualizations (e.g., scatterplots) rather than standardized metrics~\citep{stamatatos_survey_2009}.

The late 1990s marked the transition into a new era. 
The growing availability of electronic text collections enabled the construction of benchmark datasets and more thorough evaluation. 
At the same time, \ac{ml} algorithms facilitated more expressive text representations, moving beyond simple frequency counts toward feature-rich models. 
As a result, the scope of \ai{} expanded from resolving historical literary questions to addressing practical, real-world applications such as \ac{av} in digital forensics~\citep{stamatatos_survey_2009}. 
Nevertheless, as \citet{abbasi_writeprints_2008} note, even by 2008 stylometric methods struggled with scalability across large author sets, diverse genres, and open-world scenarios.

% \subsection{Stylometry}

\begin{definition}
    [Stylometry]
    Stylometry refers to a linguistic research area, where literary style is quantified by statistical features.
\end{definition}
% In other words, stylometry is the statistical analysis of literary style between one writer or genre and another \citep{tyo_state_2022}.
Researchers working on stylometric features believe that subconscious authorial syntactic idiosyncrasies are quantifiably measurable and sufficient to define an author's unique style~\citep{neal_surveying_2018}. 
These stylometric features are also referred to as style markers, or stylistic features if they are the most effective discriminators of authorship. 
Static features are context-free such as function words, word-length distributions, vocabulary richness measures, and dynamic features context-dependent attributes and include $n$-grams and misspelled words~\citep{abbasi_writeprints_2008}.
% Stylometric features include lexical, syntactic and structural features \citep{stein_intrinsic_2011}.
% Research includes five subtasks \citep{neal_surveying_2018}:
% \begin{itemize}
%     \item \ac{aa}
%     \item \ac{av}
%     \item Author profiling
%     \item Stylochronometry
%     \item adversarial stylometry
% \end{itemize}
We outline prominent style marker taxonomy categories in \autoref{tab:stylometric_features} in the following. 
Generally speaking, the more detailed the style marker extraction process, the more noisy are the produced measures~\citep{stamatatos_survey_2009}.

% lexical
Lexical features treat text as a mere sequence tokens~\citep{stamatatos_survey_2009}.
Token units include character, word, and sentence.
We \autoref{tab:comp_lexical} we present characteristics of the character and word feature units.
Character (n-gram) counts, average word~\citep{stein_intrinsic_2011}, sentence length~\citep{stein_intrinsic_2011,abbasi_writeprints_2008}, are examples for character, sentence, and sentence unit-based lexical features.
% line length~\citep{abbasi_writeprints_2008}, word length distribution~\citep{abbasi_writeprints_2008}, 
% vocabulary richness~\citep{abbasi_writeprints_2008,neal_surveying_2018} ...) 
Errors in \autoref{tab:stylometric_features} refers to idiosyncratic features include misspellings, grammatical mistakes, and other usage anomalies~\citep{abbasi_writeprints_2008,neal_surveying_2018}.


\begin{table}[]
\centering
\caption{Comparison of a subset of lexical features~\citep{stamatatos_survey_2009}. Requirements refer to computational requirements such as a tokenizer.}
\label{tab:comp_lexical}
\resizebox{\textwidth}{!}{%
\begin{tabular}{lllll}
    \toprule
    \textbf{unit} & \textbf{complexity} & \textbf{language-independent} & \textbf{affected by noise} & \textbf{requirements} \\
    \midrule
    character & low & \checkmark & low & \xmark  \\
    word & high & \xmark  & high  & \checkmark    \\
    \bottomrule
    \end{tabular}%
}
\end{table}

% syntactic
Syntax or syntactic structure is the structural organization of sentences \citep{kurt_pehlivanoglu_comparative_2024}.
Given robust \ac{nlp} tools, syntactic features are considered more reliable than lexical features.
Well-established syntactic features built on function words, or on syntactic errors such as mismatched tense or sentence fragments, or use morpho-syntactic \ac{pos} tags for each word token for \ac{pos} tag (n-gram) frequencies~\citep{abbasi_writeprints_2008,stamatatos_survey_2009}.
It is evident, that every syntactic feature requires parsing or processing of natural language and is thus, language-dependent~\citep{neal_surveying_2018,stamatatos_survey_2009}.

% semantic
Semantic features capture meaning behind words, phrases, and sentences, such as through analysis of synonyms and semantic dependencies \citep{neal_surveying_2018}.
Semantic similarities between words, synonym or hyponym relationships are derived using WordNet, any thesaurus or latent semantic analysis.
Semantic dependencies such as the specification of a person with location, can boost classification performance when combined with lexical and syntactic information~\cite{stamatatos_survey_2009}.


% application specific (structural + content)
Structural features include text organization, layout, file extensions, font, sizes, colours, 
use of braces and comments (for analysing computer programs)~\citep{abbasi_writeprints_2008,neal_surveying_2018}.
Content-specific features include important keywords and phrases on certain topics such as word $n$-grams~\citep{abbasi_writeprints_2008}.
Domain-specific features include ratios of quoted words and external links, number of paragraphs, and paragraphs average length for the news article domain~\citep{potthast_stylometric_2018}


\begin{table}[]
    \centering
    \caption{Incomplete taxonomy of style markers from \citep{stamatatos_survey_2009}.}
    \label{tab:stylometric_features}
 
    \begin{tabular}{@{}ll@{}} % numbers should be right aligned, text left aligned
    \toprule
    \textbf{Category} & \textbf{Features} \\ 
    \midrule
    Lexical & Token-based \\ %(word/ sentence length, ...) \\
     & Vocabulary richness  \\
     & Word frequencies  \\
     & Word n-grams  \\
     & Errors \\
    %  \midrule
    % Character & Character types (letters, digits, ...)\\
    %  & Character n-grams (fixed length)  \\
    %  & Character n-grams (variable length) \\
    %  & Compression methods \\
    %  \midrule
    Syntactic & Part-of-Speech (POS)  \\
    %  & Chunks \\
    %  & Sentence and phrase structure  \\
    %  & Rewrite rule frequencies \\
     & Errors  \\
    %  \midrule
    Semantic & Synonyms \\
     & Semantic dependencies \\
    %  & Functional  \\
    %  \midrule
    Application-specific & Structural  \\
     & Content-specific\\
    %  & Language-specific \\
     \bottomrule
    \end{tabular}%

\end{table}
\section{Contemporary Applications} % Contemporary Relevance/ Perspective

While early work in \ai{} focused on resolving historical literary disputes, its present-day relevance lies primarily in digital forensics. % forensic linguistics: sapkota_cross_topic_2014
\ac{aa} provides critical tools for addressing cybercrime, where anonymity enables identity deception, harassment, and financial fraud~\citep{abbasi_writeprints_2008,chendu_authorship_2020,bhattacharjee_fighting_2024}. 
Concrete applications include the detection of fraudulent e-mails, impersonation on social media, and the identification of fake product reviews that undermine trust in e-commerce platforms. 

Beyond forensic contexts, plagiarism detection constitutes another significant application area. 
Here, authorship analysis assists in identifying unacknowledged reuse of material across domains ranging from literature to academia~\citep{neal_surveying_2018}. 
Together, these examples highlight the contemporary relevance of \ai{} in safeguarding digital integrity, supporting legal investigation, and upholding standards of authorship in both professional and academic contexts.

\section{Tasks}

In this section we outline the main tasks of authorship analysis. 
We begin with the foundational problems of \ac{aa} and \ac{av}, before turning to derivative tasks such as author profiling,  author obfuscation, and plagiarism detection.

\subsection{\Acl{aa}}
\ac{aa} is the classical multiclass, single-label text classification task in which, given a set of candidate authors with known writings, the goal is to determine which author wrote a disputed text~\citep{koppel_authorship_2004}. 
Formally, let $A$ be the set of authors, $K=\bigcup_{a\in A} K_a$ the set of known texts, and $U$ the set of unknown texts. 
In the closed-set setting, each $d \in U$ must be attributed to exactly one $a \in A$. 
Variants such as cross-topic or cross-genre \ac{aa} introduce distributional shifts between training and test data, complicating the task~\citep{barlas_cross_domain_2020}.

\subsection{\Acl{av}}
\ac{av} addresses the problem of establishing whether a given text $t$ was written by a candidate author $a$, using a set of the author’s known writings $K_a$ as reference~\citep{koppel_authorship_2004}.
Disputed-candidate pair denotes the input of the text of unknown authorship and the candidate author's text in the following.
In contrast to \ac{aa}, \ac{av} does not have reliable negative examples, since assembling a representative sample of all texts not authored by 
$a$ is impossible. 
This limitation makes \ac{av} a more challenging classification problem than \ac{aa}~\citep{llm_detection_av_2025,neal_surveying_2018,koppel_authorship_2004}.
\ac{av} is framed as a one-class, a binary, or a similarity detection task depending on the methodological perspective~\citep{neal_surveying_2018,koppel_authorship_2004}.  
\ac{aa} can be reduced to a series of \ac{av} problems, where the other direction is typically not true~\citep{barlas_cross_domain_2020,tyo_state_2022}.
% Gespräch Martin Potthast 19.05.2025: problem formulation 2 is less common and in the context of very sparse (metadata) information:
% This task can also be formulated as whether two texts $t_1$ and $t_2$ are written by the same author 
% \citep{bevendorff_generalizing_2019,bevendorff_divergence_based_2020,embarcadero_ruiz_graph_based_2022,rivera_soto_learning_2021,ordonez_will_2020,futrzynski_pairwise_2021,weerasinghe_feature_vector_difference_2021,llm_detection_av_2025}.


% \subsection{Author Profiling}
% Author profiling infers sociolinguistic attributes of an author, such as age, gender, education, or mental health, from a set of texts. 
% It is grounded in the assumption that subconscious idiosyncrasies of writing style encode personal traits~\citep{emmery_adversarial_2021,stamatatos_survey_2009,elmanarelbouanani_authorship_2014}. 
% Profiling raises substantial ethical and privacy concerns due to the sensitivity of the inferred attributes.

% \subsection{Author Obfuscation}
% Author obfuscation is an adversarial task in which a text is deliberately modified to conceal the author's identity while preserving its semantic content. 
% It directly opposes \ac{aa} and \ac{av} by aiming to neutralize stylometric features~\citep{bischoff_importance_2020,bevendorff_divergence_based_2020,gohsen_task_oriented_2024}. 

% \subsection{Intrinsic Plagiarism Detection}   % style change detection
% Plagiarism detection is the task of identifying reused or unattributed content in texts~\citep{stein_intrinsic_2011,gohsen_task_oriented_2024}. 
% Unlike \ac{aa}, the goal is not to establish authorship but to uncover overlap between documents, regardless of author identity~\citep{elmanarelbouanani_authorship_2014}.
% Hence, plagiarism detection is an application of \ac{av}~\citep{rivera_soto_learning_2021}.


\subsection{\acs{llm} detection}
Given the ability of \acp{llm} to generate text closely resembling human writing, we can conceptualize \acp{llm} as one or multiple distinct authors.
This perspective allows us to frame \ac{llm} detection as an \ac{av} task, where given two texts (i.e. one of unknown authorship and one known to be generated by an \ac{llm} author) the objective is to determine whether they share the same author~\citep{llm_detection_av_2025}.



\section{Technical Background}

This section outlines the \ac{ml} principles and paradigms that underpin modern authorship analysis. 
We first introduce the main classification concepts relevant to \ac{av}. 
We then discuss training and evaluation practices, including domain shift scenarios. 
Finally, we present the principal categories of authorship analysis methods.

\subsection{\acl{ml} Principles}

\ac{aa} tasks are conventionally formulated as classification problems, where the objective is to attribute an anonymous text to one of a set of candidate authors. 
Unlike regression, which estimates a continuous value, classification yields a discrete label corresponding to the author’s identity.

\paragraph{Closed- vs. open-set classification.} 
In a closed-set scenario, the true author is guaranteed to be among the candidate set~\citep{stamatatos_survey_2009,koppel_authorship_2011,barlas_cross_domain_2020,boenninghoff_o2d2_2021,neal_surveying_2018}. 
In contrast, open-set classification acknowledges that the author of a disputed document may not belong to the candidate set~\citep{stamatatos_survey_2009,barlas_cross_domain_2020,neal_surveying_2018}. 

\paragraph{One-class classification.} 
In some cases, training data is available for only a single class, and the task is to decide whether a new sample belongs to this class.
If counterexamples, i.e. so-called outliers, are available, they are usually not considered to be representative of non-target class. 
This is formalized as one-class classification, where the model learns the characteristics of the target class without reliable counterexamples~\citep{stein_intrinsic_2011,koppel_authorship_2004}.

\paragraph{Multi-class classification.} 
The standard formulation involves multiple authors, each represented by several training texts. 
Here, the challenge is to discriminate among a large and often imbalanced set of classes~\citep{stamatatos_survey_2009,koppel_authorship_2004,elmanarelbouanani_authorship_2014} .


\subsection{Training and Testing}

Models are typically trained on one portion of the data (training set), tuned on another (validation set), and evaluated on a disjoint partition (test set). 
Any overlap between these partitions constitutes data leakage and invalidates the results~\citep{bischoff_importance_2020,altakrori_topic_2021,boenninghoff_o2d2_2021}. 

A major challenge in stylometry is covariate shift, i.e., a mismatch between the distribution of training and test data. 
This often arises from topic variability~\citep{boenninghoff_o2d2_2021}. 
Two common evaluation settings are:
\begin{itemize}
    \item \textbf{Cross-topic attribution}, where models trained on one set of topics are tested on previously unseen topics~\citep{altakrori_topic_2021}.  
    \item \textbf{Cross-domain attribution}, where training and test texts differ in topic or genre~\citep{barlas_cross_domain_2020}.  
\end{itemize}

\paragraph{Supervised vs. unsupervised learning.}  
Supervised methods require labelled training data and include classifiers such as \acp{svm}, decision trees, \acp{nn}, and linear discriminant analysis. 
\acp{svm} are particularly common in authorship analysis due to their robustness. 
Unsupervised methods do not rely on labels.
Clustering techniques or \ac{pca} have been used to uncover latent stylistic patterns or to reduce feature dimensionality~\citep{abbasi_writeprints_2008}.


\subsection{Authorship Analysis Methods}
\label{subsec:attribution_methods}

Approaches to authorship analysis can be grouped into three families~\citep{stamatatos_survey_2009}:

\paragraph{Profile-based methods.} 
All training texts of an author are concatenated into a single profile, from which a cumulative feature representation is extracted. 
This approach is effective when only short texts are available.
Profile-based methods ignore intra-author variation~\citep{stamatatos_survey_2009,elmanarelbouanani_authorship_2014,neal_surveying_2018}.  

\paragraph{Instance-based methods.} 
Here, each training text is treated as a separate instance of the author's style. 
This allows models to capture intra-author variability~\citep{stamatatos_survey_2009,altakrori_topic_2021,elmanarelbouanani_authorship_2014,neal_surveying_2018}.  

\paragraph{Hybrid methods.} 
Hybrid approaches combine both paradigms by representing texts individually while aggregating author profiles through feature-wise averages computed over an author’s texts~\citep{stamatatos_survey_2009}. 

\section{Canonical Methods}

The following sections introduce \ac{sota} approaches to \ac{av}.
We begin with compression-based methods, continue with traditional and generalized Unmasking and end with the traditional \impAppr{}.

\subsubsection{Compression-based}
% compression models, e.g. RAR or GZIP 
This category of \ac{aa} approaches is based on general-purpose compression models such as RAR or PPMD. %(i.e. a variant of \ac{ppm}~\citep{tyo_state_2022}), LZW, GZIP, BZIP2 and 7ZIP.
Such models capture textual characteristics by exploiting repeated character sequences~\citep{stamatatos_survey_2009,neal_surveying_2018}. 
Natural language, in particular, allows for high compression ratios due to its strong predictability (English has an entropy of at most 1.75 bits per character). 
For example, PPMD employs finite-order Markov language models for compression, which are highly effective in predicting character sequences in natural text but are also sensitive to increased entropy caused by text obfuscation~\citep{bevendorff_divergence_based_2020}.
Accordingly, compression-based \ac{aa} methods are considered character-based approaches.

They are further classified as profile-based methods. In this framework, an author profile is first constructed for each candidate author by concatenating all texts attributed to them and then compressing the resulting sequence. 
The disputed text is subsequently concatenated with each author profile and compressed as well. 
The difference in compression size between (i) the concatenated profile with the disputed text and (ii) the profile alone is then calculated~\citep{stamatatos_survey_2009,elmanarelbouanani_authorship_2014,neal_surveying_2018}. 
The author whose profile yields the smallest difference is selected as the most likely author~\citep{stamatatos_survey_2009,elmanarelbouanani_authorship_2014}.

The rationale behind this approach is that texts written by the same author can typically be compressed more efficiently than texts produced by different authors~\citep{stamatatos_survey_2009,elmanarelbouanani_authorship_2014}.

% RAR is the most accurate one \citep{elmanarelbouanani_authorship_2014}.
% \citet{elmanarelbouanani_authorship_2014} include the Normalized Compressor Distance (NCD) as a distance measure for compression-based methods. % Chap. 4.2
% \citet{stamatatos_survey_2009} claim that probabilistic approaches are faster in comparison to compression models.
% \citet{neal_surveying_2018} state that LZ77 is a lossless data compression algorithm that is used to compress data by detecting duplicates.

% Compression-based models can also be considered similarity-based measures which are slow 
% since the compression algorithm is called for each training text \citep{stamatatos_survey_2009,neal_surveying_2018}.



\subsubsection{Unmasking Method}
\label{subsec:unmasking}

The meta-learning approach Unmasking algorithm was first proposed by \citet{koppel_authorship_2004}.
Meta learning is a technique where the system learns to learn based on learning successes and failures.
It is based on the idea that omitting discriminant features and the consequent drop in accuracy of the classifier can be used for inference of the author of the unseen text.
For Unmasking, (1) an unseen text is chunked, such that the non-overlapping chunks compose multiple samples belonging either to the author or to a different author.
Next, one \ac{svm} is trained for each candidate author to discriminate the disputed texts' chunks from the candidate author's texts.
The \acp{svm}' features are usually frequencies over the $n=250$ highest average frequency words.
(2) The 10-fold cross validation accuracy for the trained model are obtained.
(3) For the next iteration, omit the most discriminating features among those left.
(4) Repeat steps (3) and (4).
(5) Another linear \ac{svm} classifier is trained on the accuracy curve, its central-difference gradients (first- and second order), 
and its gradients sorted by magnitude.
This classifier is used to predict the whether the texts originate from the same author.

After a few iterations, the classifier is no longer able to discriminate between the unseen text and the texts of the true author~\citep{stein_intrinsic_2011,tyo_state_2022,bevendorff_divergence_based_2020,koppel_authorship_2004,stamatatos_survey_2009} 
Two texts are probably written by different authors if the differences between are robust to changes in the underlying feature set used to represent the documents.

To operationalize this idea, differences are measured using classification via cross-validation accuracy~\citep{koppel_authorship_2011,bevendorff_generalizing_2019,bevendorff_divergence_based_2020,potthast_stylometric_2018,koppel_authorship_2004}, 
creating a performance degradation curve~\citep{tyo_state_2022,koppel_authorship_2004}.
An \ac{svm} is trained to classify the degradation curve to determine whether two text originated from the same author~\citep{tyo_state_2022,bevendorff_generalizing_2019,koppel_authorship_2004}.
Steep decrease in the curve indicates that the two texts are similar, and thus, written by the same authors~\citep{potthast_stylometric_2018,koppel_authorship_2004}.
% Provided that the unseen text is very large, this method can handle small open candidate sets \citep{koppel_authorship_2011}.
% koppel_determining_2014, pg. 1 + bevendorff_generalizing_2019 chap. 3.1 incl. algo: based on text chunks of length >= 500 words each
% \citet{koppel_determining_2014,bevendorff_generalizing_2019} claim that effective unmasking requires input documents to be large 
% (i.e. > 10000 words~\citep{koppel_determining_2014}, book-length~\citep{bevendorff_generalizing_2019}, 
% $\geq$ 5000 words (500 words per chunk) \citep{bevendorff_divergence_based_2020}).
% Otherwise the training set becomes too sparse and no descriptive curves can be generated 
% \citep{bevendorff_generalizing_2019,bevendorff_divergence_based_2020}.

% generalized unmasking
\citet{bevendorff_generalizing_2019,bevendorff_divergence_based_2020} propose creating chunks by oversampling words in a bootstrap aggregating manner. 
Each text is a pool of words, from which words are sampled without replacement.
The pool is replenished if it is exhausted before the chunk has sufficiently many words.
Since the random sampling of unmasking features introduces variance, unmasking is performed multiple times and the curves are averaged.
The algorithm is displayed in \autoref{alg:generalized_unmasking}.
The content of the while loop is, except the number of removed features (\citep{koppel_authorship_2004}: 6 total), similar to the original unmasking algorithm \citep{koppel_authorship_2004}.

\begin{algorithm}
    \caption{Generalized Unmasking Algorithm \citep{bevendorff_generalizing_2019,bevendorff_divergence_based_2020}}
    \label{alg:generalized_unmasking}
    \begin{algorithmic}[1]
    \Procedure{Unmasking}{$A$, $B$}
        \Comment{$A$, $B$: input documents}
    
        \State $\mathcal{C}_A \gets \text{RandomChunks}(A, 30, 700)$ \Comment{30 chunks, 700 words each}
        \State $\mathcal{C}_B \gets \text{RandomChunks}(B, 30, 700)$
        \State $\mathcal{F} \gets \text{TopFreqWords}(A, B, 250)$
        \State $\mathcal{C} \gets \mathcal{C}_A \cup \mathcal{C}_B$

        
        \While{$|\mathcal{F}| \geq 0$}
        \State $a \gets \text{CVAcc}(\mathcal{C}_A, \mathcal{C}_B, \mathcal{F}, linSVM)$ \Comment{Append $10$-fold cross-validation accuracy}
        \State $\mathcal{F} \gets \mathcal{F} \setminus \mathcal{F}_{\text{top}}^{\pm}$ \Comment{Remove top $5$ most significant positive and negative features}
    
        \EndWhile
    
        \State \Return List of recorded accuracies $a$
    \EndProcedure
    \end{algorithmic}
\end{algorithm}

% hyperparameters
% The most important hyperparameters are the number of chunks, the number of words per chunk, the size of feature vectors, 
% the number of word removals per round, and the number of averaged unmasking runs.
% More chunks result in generally shallower curves while shorter features vectors or more removals produce steep curves.
% Ideally, curves are not too steep and granular enough to allow distinguishing between different same and different author pairs.
% \citet{bevendorff_bias_2019} recommend 25 to 50 chunks, vector sizes of 250 to 400 features, not fewer than 5, yet not more than 20 removals per round, 
% between 500 and 700 words per chunk and about 10 runs to average for a curve.
% They increase the minimal distance between the \ac{svm} hyperplane and the decision boundary, i.e. their confidence parameter $c$, to increase precision.
% In a medium- to high-assurance scenario (where \acp{fp} should be avoided, but are not entirely critical), they recommend $c \geq 0.6$.
% If \acp{fp} should be avoided at all costs, they recommend $c \geq 0.7$.
% \citet{bevendorff_bias_2019} claim that, for this approach, hyperparameter tuning is simpler than for black box approaches.

% % metric results
% \citet{bevendorff_bias_2019} report that the generalized unmasking approach heavily prioritizes precision 
% opposed to compression-based approaches that balance precision and recall.
    






\begin{figure}[htbp]
    \centering
    \includesvg[width=\textwidth]{images/unmasking/unmasking.svg}
    \caption{Workflow of Generalized Unmasking~\citep{bevendorff_generalizing_2019}: (1) Create chunks by oversampling words of disputed and candidate texts and represent them using word frequencies. (2) Obtain \ac{svm} accuracy. (3) Eliminate most discriminative features. (4) Repeat from (3)-(4).}
    \label{fig:unmasking}
\end{figure}



  
\subsubsection{\imp{} Method}
\label{sec:impostor_method_theory}


\begin{definition}
    [\imp{} method]
    This method extends the ngram-unmasking method, i.e. iteratively omitting most influencely features (repeated feature subsampling \citep{koppel_determining_2014})
    from a trained classifier and classifying the accuracy drop.
    It takes score of how often an author is predicted after each feature-elimination step.
    The final prediction is made based on this score \citep{tyo_state_2022}.
\end{definition}


\begin{definition}
    [Hard Negative Mining]
    This method updates the model during training only with the most difficult examples in each batch.
    In the \ac{aa} context, difficult is defined as the most similar two texts from different authors, 
    which makes the decision the most difficult.
    \citet{tyo_state_2022} claim that the \ac{av} setting is strictly easier since 
    it most compare to only a single text.
    Due to the fact, that the most difficult example is model-dependent, \ac{av} problems can be made harder 
    but they can not exist of exactly the hardest negatives.
\end{definition}


\begin{definition}
    [Domain]
    The domain include topic, genre, register, idiolect, time period etc. \citep{bischoff_importance_2020}.
\end{definition}
  
\begin{definition}
    [Domain variables]
    These include topic, genre and language \citep{bischoff_importance_2020}.
\end{definition}

\begin{definition}
    [within-domain]
    Experiments with P=Q.
    Hence, it is necessary to ensure all texts are mutually from the same domain \citep{bischoff_importance_2020}.
    \begin{table}[tbp]
        \centering
        \caption{Typical scheme $S_1$ for \ac{aa} problem instances, where A, B, are authors and P, Q domains and 
        the vertical mapping denotes which author has written in which domain. 
        For training, texts from A and B take turn; for testing, previously unseen texts from A and B are used \citep{bischoff_importance_2020}.}
        \label{tab:within_domain_aa}
        \begin{tabular}{|l|ll|ll|}
        \hline
        \textbf{Scheme $S_1$} & \multicolumn{2}{l|}{\textbf{training}} & \multicolumn{2}{l|}{\textbf{testing}} \\ \hline
        \textbf{authors} & \multicolumn{1}{l|}{A} & B & \multicolumn{1}{l|}{A} & B \\ \hline
        \textbf{domains} & \multicolumn{1}{l|}{P} & Q & \multicolumn{1}{l|}{P} & Q \\ \hline
        \end{tabular}%
    \end{table}
\end{definition}


The \impAppr{} leverages random projections to lower dimensional spaces (i.e. random set of features set to zero is a projection).
\begin{figure}[htbp]
    \centering
    \includesvg[width=\textwidth]{images/imposter/imposter.svg}
    \caption{\imp{}.}
    \label{fig:impostor}
\end{figure}

% AV -> open-set
\ac{av} is an open-set problem, meaning that the author of an anonymous document 
may or may be not be part of the set of candidate authors.

% AA -> closed-set
\ac{aa} is a closed-set problem, meaning that the author of an anonymous document
is part of the set of candidate authors.
For each candidate author, writing samples are available.
The task is to determine the author of the anonymous document from the set of candidate authors.

% reduction: closed-set AA -> open-set AV
\citet{koppel_determining_2014} state that all closed-set \ac{aa} problems are reducible to the \ac{av} problem.
The reverse is not true.
To reduce the \ac{aa} problem to the \ac{av} problem, we solve a \ac{av} problem, i.e. if text was written by a candidate author, 
for each of the respective candidates.
Ideally, we receive one positive answer for the correct candidate author and negative answers for all other candidates.

% complexity
\citet{koppel_determining_2014} explain that the \ac{av} problem is more complex than the \ac{aa} problem.
They claim that the ability to solve a closed-set \ac{aa} problem does not imply the ability to solve an open-set \ac{av} problem.

% open-set identification/ AA = many candidates problem
\citet{koppel_determining_2014} define the many-candidates problem, or the so-called open-set identification problem:
Given a large set of candidate authors, determine which, if any, of them wrote a given anonymous document.
According to \citet{koppel_determining_2014}, the many-candidates problem can be solved reasonably well: \autoref{lst:many_candidate_algo}.
\section{Impact of \acs{llm} on Authorship Verification}

With the advances in \ac{gai} come risks and opportunities.
Similarly, we can frame \acp{llm} as authors weakening our trust in the authencity of digital texts, but also see them as valuable tools to enhance our regulatory methods.

\subsection{\acsp{llm} as authors}
There is no general definition of when a text is \ac{llm} generated rather than co-created by humans with \ac{llm} assistance.
% Obviously, fully generated texts should be marked as \ac{llm} generated.
Minor human edits of \ac{llm} generated texts do not change the fact that the core content was \ac{llm} generated.
If \acp{llm} are used for grammar checking, polishing, and editing suggestion the primary substantial contribution was human.
One could denote these texts "\ac{ai}-revised Human-Written Text"~\citep{wang_stumbling_2024}.

With advances of generative models with regard to mimicking human writing, we have to face the fact that \acp{llm} will play a crucial role in any authorship analysis related tasks from now on.
\citet{llm_detection_av_2025} claim that \ac{llm} detection is not an \ac{aa} task, i.e., a closed-set binary classification where both classes are sufficiently discriminative, but an \ac{av} task, i.e., an open-set one-class classification problem. 

% differences: also more in ~\citep{wang_stumbling_2024}
Despite the advances, there are still some statistical differences on \ac{llm} generated and human authored texts.
\ac{llm} generated texts lack lexical diversity, overuses certain adjectives (e.g. "innovative") and produces longer, more complex sentences.
Moreover, \acp{llm} possess stylistic fingerprints and memorize patterns from the training data.
% lengths
Furthermore, word length averages and distributions across genre differ for \acp{llm} and humans~\citep{llm_detection_av_2025}.

% future of LLMs
However, as \acp{llm} progress, basic heuristics applied by human detectors no longer suffice.
\acp{llm} will become more human-like and thus, \ac{llm} detection will increasingly resemble a human authorship classification task~\citep{llm_detection_av_2025}.


\subsection{\acsp{llm} as Discriminator}
\label{sec:llm_discriminator}

While \acp{llm} can generate coherent text, their performance as direct discriminators in \ac{aa} tasks varies. 
Words with similar meaning, such as "color" and "colour", are mapped to similar vector representations~\citep{altakrori_topic_2021}.
This reduces sensitivity to subtle language system differences, which are often critical for identifying an author. 
Consequently, relying on \acp{llm} as the primary discriminator in \ac{aa} or \ac{av} tasks may be suboptimal.

Rather than serving as standalone discriminators, \acp{llm} are more appropriately employed as supporting tools. 
For instance, they can generate cross-domain training data to improve model robustness. 
Moreover, prior methods assessed the extent to which text changes under \ac{llm}-based paraphrasing, since machine generated text tends to undergo minimal alteration, whereas human-authored text exhibits greater variation~\citep{mao_raidar_2024}.
Conversely, in contexts where privacy is a concern, \acp{llm} can facilitate author obfuscation through controlled paraphrasing. 







\section{\acl{av}}

\section{\ac{av} as One-Class categorization}
\label{sec:av_one_class}

% \citet{koppel_authorship_2004} claim research had shown that linear separators work well for text categorization.
% Linear models include Naive Bayes (linear separator for two classes), 
% Window and Exponential Gradient and linear \acp{svm} \citep{koppel_authorship_2004}. 

% \subsection{\ac{aa} framework}
% \citet{koppel_authorship_2004} state that the following framework solve a number of real world \ac{aa} problems:
% \begin{enumerate}
%     \item Construction of appropriate feature vectors
%     \item Construction of a distinguishing model via a learning algorithm
%     \item Assessment of effectiveness of methods using k-fold cross-validation or bootstrapping
% \end{enumerate}

\subsection{Chunks}
\citet{koppel_authorship_2004} propose chunking texts such that each chunk is of approximately equal length, 
and at least 500 words without breaking paragraphs. 

\subsection{Definitions}
For author $A$ and book $X$, \citet{koppel_authorship_2004} define the following:
If $A$ is not the author of $X$, $A_X$ is the set of all works by author $A$.
If $A$ is the author of $X$, $A_X$ is the set of all works by author $A$ except $X$.
A pair of $A_X$ and $X$ is called \emph{same-author} if X was authored by $A$.
A pair of $A_X$ and $X$ is called \emph{different-author} if $X$ was not authored by $A$.

\subsection{Initial feature set}
The initial feature set consists of the 250 words with the highest average (over $X$ and $X_A$) frequency \citep{koppel_authorship_2004}.

\subsection{Features for meta-classifier}
\citet{koppel_authorship_2004} propose the following features for the meta-classifier 
(where $i$ is the number of elimination steps):
\begin{itemize}
    \item accuracy after $i$ elimination steps
    \item accuracy difference between round $i$ and round $i+1$
    \item accuracy difference between round $i$ and round $i+2$
    \item $i^{th}$ highest accuracy drop in one iteration
    \item $i^{th}$ highest accuracy drop in two iterations
\end{itemize}

The vectors are grouped by \emph{same-author} and \emph{different-author} pairs and thus, 
used to train a meta-learning scheme.

\subsection{Negative examples for Elimination method}
\citet{koppel_authorship_2004} state that negative examples are neither exhaustive nor representative.
They propose using words of several authors $A_1, ..., A_n$ roughly filling the same profile as candidate $A$ 
in terms of geography, chronology, culture and genre.
$A_1, ..., A_n$ are said to collectively represent class not-$A$.

\subsection{Elimination method}

The elimination method is only used to overrule positive predictions.
Hence, it can eliminate \acp{fp}.
One can frame it as a filter which is applied after or before unmasking.

% training
\citet{koppel_authorship_2004} learn a model for $A$ and against not-$A$, 
and multiple models for $A_i$ and against not-$A_i$.
% inference
Then, $X$ is tested against all of these models.
$A(X)$ is the percentage of examples of $X$ classed as $A$ rather than not-$A$ 
(i.e., $A_i(X)$ analogously).
If $A(X)$ is not larger than all $A_i(X)$, $A$ is not the author of $X$.
If $A(X)$ is larger than all $A_i(X)$, conclude nothing.




 % before related work
    \cleardoublepage 
    \chapter{Related Work}
\label{chap:related_work}

This work is different to the work of \citet{koppel_determining_2014} and \citet{kocher_unine_2015} 
in that it uses \acp{llm} to generate imposter texts.

% AV

% LLM detection using generative models
%% AA against LLMs
With the recent advances of \ac{nlg} come new challenges in text authorship.
The new technologies may be misused for fraudulent activities to scam naive or inexperienced users~\citep{uchendu_authorship_2020,bhattacharjee_fighting_2024}.
\citet{uchendu_authorship_2020} identified three authorship tasks essential for fighting fraudulent activities:
(1) Given two texts $t_1$ and $t_2$, determine whether they were produced by the same method (i.e. human author or a specific \ac{nlg} method).
(2) Given a text $t$, determine whether it was human authored or machine generated (Turing Test).
(3) Given a text $t$, find its author among $k+1$ candidates, which consists of one human and $k$ machines.
They compare classical \ac{ml} models, neural models and state-of-the-art \ac{aa} models as classifiers 
for these single- (Problem 1 and 2) and multi-class (Problem 3) tasks.
Their findings include, that as of 2020, most \ac{nlg} methods were distinguishable from human authors, 
but some \acp{llm} proved difficult to detect.
%%% compared to our work
In the following, we consider (1) \ac{av}, (2) classical \ac{llm} detection, and (3) closed-set \ac{aa}.
Our approach differs from the work of \citet{uchendu_authorship_2020} in that our candidates (i.e. imposters) do not include a human author (3), 
but only \acp{llm}.
Moreover, we use different classifiers originally designed for \ac{av}, rather than \ac{aa}.

%% LLM (gpt-3.5, GPT-4) as detector
\citet{bhattacharjee_fighting_2024} evaluate using an \ac{llm} as classifier for \ac{llm} detection.
They use \ac{gpt}-3.5 and \ac{gpt}-4 to classify texts as human or machine generated.
They find that \ac{gpt}-3.5 performs better when being fed simple instructions, rather than constrained prompts.
They find that \ac{gpt}-4 predicts almost exclusively \ac{ai} generated texts, 
while \ac{gpt}-3.5 predictions are more reliable (especially for actually human authored texts).
%%% compared to our work
Our work differs from theirs in that we use \acp{llm} to generate imposter texts specific to the candidate text, 
rather than using the publicly available dataset TuringBench with previously generated texts.

%% DetectGPT: Perturb (Mask), score, compare (unsupervised)
\citet{mitchell_detectgpt_2023} propose DetectGPT, a method that is threefold:
(1) They perturb the input text by (1.1) masking out random 2-word spans until 15 \% of the text is masked. 
Masked spans are replaced (1.2) with words from an off-the-shelf (i.e. not finetuned to target domain) \ac{llm} (e.g. T5-3B). 
These perturbations are semantically similar paraphrases of the original text.
(2) They score (in terms of log probability) each perturbed text using a scoring \ac{llm} 
(ideally their candidate \ac{llm}, but it works also with any \ac{llm}, though scores deteriorate). 
(3) The difference of the score of the original text and the average score of the perturbed texts is denoted perturbation discrepancy $d$. 
(4) Normalize $d$ by the standard deviation of the scores of the perturbed texts.
(5) Based on a threshold $\epsilon=0.1$, classify the original text as human authored or machine generated 
(formally Local Perturbation Discrepancy Gap hypothesis).
If $d$ is positive, the original text is likely machine generated.
If $d$ is near zero, i.e. $d < \epsilon$, the original text is likely human authored.
\citet{mitchell_detectgpt_2023} motivate their method by the observation that generated texts tend to occupy 
negative curvature regions of the model's log probability function (i.e. they lie on the local maximum of the manifold).
When the text is machine generated, it lies on a local maximum, 
and perturbing it will lead to lower log probabilities of perturbed texts.
When the text is human authored, it does not lie on a local maximum to begin with, 
rendering log probabilities of perturbed texts similar either bigger or smaller than the original text.
Averaging the log probabilities of perturbed human texts leads to a value that is 
close to the original text's log probability (i.e. a perturbation discrepancy $d$ near zero).
Even though, DetectGPT works best when the source (i.e. generating) \ac{llm} and the scoring \ac{llm} are the same 
(requires white-box access to the \ac{llm}), 
it works also with different \acp{llm} as surrogate for the source model when scoring (in a black-box case).
%%% compared to our work
We can not supply a white box setting, because we do know the source \ac{llm} that generated the imposter texts.
%%%% imposters and perturbations
However, this approach is similar to our approach, because perturbing texts can be seen as a 
form of imposter generation (especially as we use paraphrases). 
%%%% sample from the source model
Both approaches try to sample from the probability distribution of the source model either 
by using imposters (via prompting an \ac{llm}) or by perturbing the original text (using an \ac{llm}).
%%%% input
While the imposter approach is an \ac{av} task (i.e. input is a disputed and a candidate text), 
DetectGPT receives a disputed text and a candidate \ac{llm} as input.
%%%% similarity measure
While we use a similarity measure on traditional n-gram frequency vectors, 
\citet{mitchell_detectgpt_2023} require a scoring \ac{llm} to compute the perturbation discrepancy $d$.
Hence, our approach is easier in terms of computational resources and requirements.

%% LLM rewrite LLM texts less than human texts (no AA, but edit distance hypothesis)
RAIDAR~\citep{mao_raidar_2024} builds upon the invariance property of \acp{llm}, 
which states that prompting an \ac{llm} to rewrite a machine generated text will introduce little change.
They motivate this by the observation that (different) autoregressive models produce similar patterns and thus, 
consider texts generated by (different) \acp{llm} as high quality that do not require rewriting.
Change is measured by the edit distance between the original text and the rewritten text. 
\citet{mao_raidar_2024} propose using an edit distance based on the Levenshtein distance or \ac{bow} representations.
RAIDAR operates on character level rather than using deep neural network features, and it does not require the original generating model for classification. 
RAIDAR fails to detect \ac{llm} generated texts in out-of-distribution scenarios (i.e. different domains than training), 
or when \ac{llm} were explicitly instructed to produce text prone to heavy \ac{llm} modification when being asked to rewrite the text \citep{li_learning_2025}.
Based on RAIDAR (\citep{mao_raidar_2024}), \citet{li_learning_2025} propose fine-tuning an \ac{llm} to rewrite human authored texts more than machine generated text.
Classification is carried out by comparing the edit distance of the original text and the rewritten text to a threshold.
\citet{li_learning_2025} admit that their approach is slow in inference time, 
since a candidate text has to be rewritten multiple times (about 200 different prompts) to obtain a reliable score.
\citet{mao_raidar_2024} find that the quality of perturbation based models (i.e. rewriting) for \ac{llm} detection correlates with the perturbation model size.
\citet{mitchell_detectgpt_2023} find a negative correlation (\textcolor{red}{TODO: chapter 2 vorletzter Absatz}) between the size of the perturbation model and the performance of DetectGPT.
%%% compared to our work
%%%% generation of texts during inference
Both approaches are similar to our work in that they use \acp{llm} to generate texts during inference.
We do not fine-tune an \ac{llm} for paraphrasing but use off-the-shelf models (like RAIDAR).
%%%% similarity measure
All these approaches compute the similarity of the original text and the generated text.
However, we do not use edit distance (i.e. Levenshtein distance) as similarity measure.
%%%% limitations
This approach is unable to detect which \ac{llm} generated the text.

%% LLMDet: Proxy to perplexity (problem: requires access to the LLM to build the dictionary)
Perplexity is a reliable statistical metric for attributing texts to \acp{llm}~\citep{zhang_llmdet_2023}.
Unfortunately, perplexity requires access to \acp{llm}' parameters (i.e., white-box detection).
\citet{zhang_llmdet_2023} propose LLMDet, a method that uses a proxy to perplexity, 
where a dictionary of frequent n-gram (frequent among $n$ randomly prompted generated texts per \ac{llm}) 
next token probabilities is pre-computed (i.e. requiring access to the \ac{llm}), 
and is subsequently used during inference to approximate perplexity by replacing $x_{<i}$ in $p(x_i | x_{<i})$ with an n-gram.
Since the construction of the dictionary requires access to the \ac{llm}, LLMDet requires contribution of the closed-source model owners.
The disputed text is tokenized and the proxy perplexity is calculated for each model and thus, constructing a proxy perplexity vector.
This vector is input to a trained classifier.
%%% compared to our work
Proxy perplexity could be used as a baseline for our approach, though it requires access to the \ac{llm} and is thus not applicable in our case.

%% Mirror Minds: extract query, genrate two paraphrases, compare & classify via threshold (very similar to our work)
\citet{baradia_mirror_2025} propose (1) extracting a query from the disputed text, which captures the essence of the text, 
(2) generating two paraphrases of the original text using the query as input prompt to two \acp{llm}, 
and (3) comparing the paraphrases to the original text via the BLEU and the METEOR score.
Both score capture syntactic similarity, even though \citet{baradia_mirror_2025} argue they also capture semantic similarity.
They use the maximum across the two models per similarity measure as a final score pair.
Classification of the resemblance to \ac{ai} generated content requires a threshold.
%%% compared to our work
%%%% same approach
This approach is similar to our approach in that it uses \acp{llm} to generate paraphrases of the original text.
Moreover, it compares the original text to the generated paraphrases as in a \ac{aa} problem. % rather AI detection?????
%%%% similarity measure
We do not use BLEU or METEOR as similarity measure, nor do we compare directly on paraphrase-level (i.e. BLEU calculates n-gram overlap) 
but construct our own frequency based n-gram vectors input vector similarity metrics.
%%%% they discard information, solve another problem
However, this approach discards the information which \ac{llm} produced the most similar paraphrase. 
While our goal is to solve an \ac{aa} problem (i.e. multiclass classification), 
\citet{baradia_mirror_2025} solve a binary classification problem (i.e. human vs. \ac{ai} generated text). % teil: background ML etc.
    \cleardoublepage 
    \chapter{\acs{llm}-based Impostors Method}
\label{chap:llm_impostor_method}

A central challenge in authorship analysis is the presence of confounding factors such as register, genre, topic, and target audience. 
These dimensions restrict linguistic options, thereby obscuring the stylometric features relevant for authorship-related tasks. 
To address this issue, we propose an extension of the \impAppr{} originally proposed by \citet{koppel_determining_2014}.
We intend to leverage \acp{llm} to systematically control for such confounders. 
By generating texts under carefully constrained conditions, we obtain \imps{} that preserve the domain variables of the original reference text, thereby serving as controlled proxies for replicating the actual generation process of the reference text.



\section{Impact of Confounders on Authorial Style}
\label{sec:contextual_factors}

State-of-the-art models for authorship analysis exhibit strong sensitivity to domain shifts. 
Performance often deteriorates sharply when models are applied in out-of-distribution scenarios, i.e.\ in cross-domain settings, a phenomenon largely attributable to the influence of confounders such as topic, genre, and register~\citep{Sundararajan_style_18,bischoff_importance_2020}

Confounders are problematic because they influence the very features used to characterise authorial style. 
Topical vocabulary, for example, can dominate lexical distributions, while genre-specific conventions shape syntax. 
As these factors cannot be cleanly separated from genuine stylistic markers~\citep{bischoff_importance_2020}, they obscure the style markers.

If topics can be represented by characteristic word distributions, then a document can be seen as a subset of words selected by the author, reflecting individual preferences in synonym choice~\citep{altakrori_topic_2021}. 
Consequently, texts by the same author on different topics may appear unrelated, whereas texts by different authors on the same topic may seem deceptively similar.

Empirical evidence confirms the severity of this problem.
Both contemporary \acp{av} methods~\citep{Thomas_cross_topic_24}, and \ac{llm} detection approaches, such as DetectGPT~\citep{mitchell_detectgpt_2023,Wu_ODD_challenges_2025} suffer significant effectiveness degradation in out-of-distribution scenarios.

As a result, authorship research has diverged into two main directions. 
One line of work aims to identify domain-invariant features, a challenge that remains largely unresolved~\citep{bischoff_importance_2020}. 
The other focuses on in-domain scenarios, in which confounding factors are deliberately controlled. % \ac{id}
While restricting tasks to a single topic or genre does not fully eliminate the entanglement between content and style, it reduces its impact sufficiently to produce stable and interpretable results. 
Moreover, some studies indicate that simply using domain or topic labels is insufficient to control for topic similarity in corpora, as this approach ignores semantic relationships between topics~\citep{sawatphol_cross_topic_av_24}.


\section{\acs{llm}-based \Imp{} Generation}
\label{sec:impostor_generation}

% good \imps{}: hard negatives
Following the notion introduced by \citet{koppel_determining_2014}, ideal \imp{} texts can be understood as hard negatives, i.e.\ documents not written by the candidate author, yet sufficiently similar in style to be difficult to distinguish from the candidate's own writing. 
The quality of \imps{} directly affects model effectiveness.
If they are too different from the candidate text, models are trained on trivial contrasts, yielding \acp{fp}. 
Conversely, if \imps{} are too similar to the candidate, the risk of \acp{fn} increases.
This manifests as increased recall in the former case and increased precision in the latter case.

% obstacles for \imp{} selection in the past
Traditional \imp{} selection techniques struggled to simultaneously control multiple confounders:
The original \texttt{fixed} approach samples from a pool of unrelated texts, while the \texttt{on-the-fly} approach fails to align genre with the candidate text~\citep{koppel_determining_2014}. 
Since authorial style is tightly entangled with domain variables, \imps{} produced by these methods differ systematically from the candidate text, weakening their utility. 
Ideally, \imp{} generation should replicate the conditions of reference text production, including task, topic, register, target audience and century. 
In practice, however, this information is rarely available.
In \ac{av}, the candidate author may be deceased, unavailable, or unwilling to cooperate.

% heuristics: paraphrase
\acp{llm} make it possible to approximate this ideal. 
By conditioning generation on the candidate text, domain variables such as topic, genre, and register can be more tightly controlled. 
In particular, paraphrasing offers a heuristic for simulating the text generation process. 
The model produces stylistic variants of the candidate's writing while maintaining semantic alignment.


\subsection{Hard Negative Mining via Paraphrasing}

The lack of a universally accepted definition of paraphrases complicates their generation and evaluation~\citep{gohsen_task_oriented_2024}. 
In this work, paraphrases qualify as suitable \imps{} if they preserve the topic, genre, tone, and length of the reference text, while introducing enough lexical and syntactic variation to avoid near-duplication. 
Limited semantic divergence is permitted, including mild hallucinations, which we define as added details that remain broadly consistent with the original topic.
We propose two approaches to paraphrasing, namely one-step and two-step paraphrasing, as illustrated in \Cref{fig:paraphrasing_approaches}.
We expect the recall scores of the \ac{llm}-based \impAppr{} to decrease while precision increases, i.e.\ yielding higher confidence in positive labelled text pairs, due to harder negatives leading to fewer positive predictions.

\begin{figure}[h]
    \centering
    \includesvg{images/paraphrasing/idea/Paraphraser_architectures.svg}
    \caption[Different paraphrasing approaches]{Proposed one-step and two-step paraphrasing approaches.}
    \label{fig:paraphrasing_approaches}
\end{figure}


Paraphrasing is often analysed along two main dimensions. 
At the lexical level, it involves word-level substitutions such as synonym replacement, typically achieved through rule-based or thesaurus-driven methods. 
At the syntactic level, paraphrasing entails changes to sentence structure, which can be realised through monolingual machine translation~\citep{zhou_paraphrase_2021} or by explicitly prompting \acp{llm}~\citep{kurt_pehlivanoglu_comparative_2024}.
% While more fine-grained taxonomies exist~\citep{zhou_paraphrase_2025}, they are not central here, as this study does not involve training or constructing paraphrase models directly.
Paraphrase quality is largely determined by prompt design and model selection~\citep{Wu_ODD_challenges_2025}.


\subsection{One-step Paraphrasing}

In our one-step approach, \imps{} are generated directly via \ac{llm} prompting. 
Effective prompting proved crucial. 
We found that specifying constraints was vital for valid outputs. 
Moreover, placing task instructions at the end reduced instruction neglect in long-context settings (e.g. \dataGutenberg{} corpus). 
We evaluated three prompt variants.

The first prompt explicitly decomposes the sentence into subject, verb, and object before reconstruction, encouraging lexical diversity. 
\begin{quote}
    \textit{Paraphrase the given sentence by identifying the main subject, verb, and object. Replace each with synonyms or closely related words, adjusting grammar naturally. Keep the new sentence close in length to the original. Output only the final paraphrased sentence.}
\end{quote}

The second prompt instructs the model to preserve meaning while varying wording and structure. 
Both prompts emphasise concise outputs restricted to the paraphrase itself.
\begin{quote}
    \textit{Paraphrase the sentence above without changing its meaning. Use different words and vary the sentence structure while keeping the tone consistent. Keep the new sentence similar in length to the original. Output only the paraphrased sentence, with no explanations or extra text.}
\end{quote}

The third prompt extends the second prompt by explicitly instructing the \ac{llm} to produce paraphrases three times as long as the reference text.
\begin{quote}
  \textit{Paraphrase the text above without changing its meaning. Use different words and vary the sentence structure while maintaining a consistent tone. Your paraphrase should be three times as long than the original. Output only the paraphrased sentence, with NO explanations or extra text.}
\end{quote}

Initial experiments with T5-based HuggingFace models showed poor instruction adherence and frequent malformed outputs. 
Our final experiments therefore employed the models listed in \Cref{tab:base_llms}, accessed via the \ac{gwdg} infrastructure\footnote{\url{https://docs.hpc.gwdg.de/} (August 31, 2025)}.
Further details on these models are provided in \Cref{app:language_models}.

\begin{table}[h]
\centering
\caption{\acp{llm} used for paraphrasing.}
\label{tab:base_llms}
% \resizebox{\textwidth}{!}{%
\begin{tabular}{@{}ll@{}}
\toprule
\textbf{Model ID}                    & \textbf{Host} \\
\midrule
qwen3-32b                            & \ac{gwdg}    \\
mistral-large-instruct               & \ac{gwdg}    \\
openai-gpt-oss-120b                  & \ac{gwdg}    \\
meta-llama-3.1-8b-instruct           & \ac{gwdg}    \\
\bottomrule   
\end{tabular}%
% }
\end{table}


\subsection{Two-step Paraphrasing}

The two-step approaches are inspired by prior literature~\citep{bevendorff_overview_2024, ayele_overview_2024}, and differ in how they extract and utilize auxiliary information (i.e.\ metadata) from the source text before generating the paraphrase. 
The two steps enforce a clearer disentanglement of content and style at the cost of increased computational overhead.

As shown in \Cref{fig:two_step_paraphraser}, the two-step paraphrasing decomposes the task into information extraction and text generation. 
In the first stage, the model identifies domain variables of the input text via a \pextractor{}.
These variables include tone, register, time period, target audience, and genre.
Different approaches extract an additional feature specific to their approach.
The additional features are a bullet point summary, the authorial task, topic, or title. 
The extractor prompts can be derived from \Cref{app:extractor_prompts}.
In the second stage, a paraphrase is generated from this metadata. 
Metadata can be derived automatically from the reference text or provided as ground truth when calling the paraphraser function.
All generator prompts can be found in \Cref{app:generator_prompts}.

\begin{figure}[ht]
  \centering
\resizebox{\textwidth}{!}{%
\begin{tikzpicture}[node distance=2cm]
% Dataset node
\node[dataset] (dataset) {Dataset};

% Anchor for Reference drawing
\node[draw=none, right=of dataset] (refpos) {};

% Draw Reference as an outer rectangle + snakes inside
\node[draw, thick, rounded corners=6pt, minimum width=1.8cm, minimum height=2cm,
      right=of dataset, inner sep=4pt] (reference) {};

% Add multiple snakes inside reference (smaller, stacked)
\foreach \y in {0.8,0.4, 0.0,-0.4,-0.8}{
  \draw[black, very thick, decorate, decoration={snake,amplitude=1pt,segment length=5pt}]
    ($(reference.west)+(0.3,\y)$) -- ($(reference.east)+(-0.3,\y)$);
}

% Caption
\node[below=0.2cm of reference] {Reference};


% Other blocks
\node[above=0.8cm of reference.north] (meta) {Ground Truth Metadata};
\node[block, right=of reference.east] (extractor) {Extractor};
\node[block, right=of extractor] (generator) {Generator};

% paraphrase
\node[draw, thick, rounded corners=6pt, minimum width=1.8cm, minimum height=2cm,
      right=of generator, inner sep=4pt, teal!80!black] (paraphrase) {};

% Add multiple snakes inside reference (smaller, stacked)
\foreach \y in {0.8,0.4, 0.0,-0.4,-0.8}{
  \draw[teal!80!black, very thick, decorate, decoration={snake,amplitude=2pt,segment length=5pt}]
    ($(paraphrase.west)+(0.3,\y)$) -- ($(paraphrase.east)+(-0.3,\y)$);
}
\node[below=0.2cm of paraphrase] {Paraphrase};

% Arrows
\draw[arrow] (dataset) -- (reference.west);
\draw[arrow] (dataset) -- node[pos=0.4, above left=2pt and -12.5pt] {$\{0,1\}$} (meta);
\draw[arrow] (reference.east) -- (extractor);
\draw[dashedarrow] (extractor) -- (generator);
\draw[arrow] (generator) -- (paraphrase);
\draw[dashedarrow] (meta) -- (generator);


\end{tikzpicture}
}
  \caption[Two-step paraphraser]{Visual description of the two-step paraphraser.
  The \pgenerator{} is fed metadata information from the \pextractor{} or from ground truth data.}
  \label{fig:two_step_paraphraser}
\end{figure}


The translation-based paraphraser is another two-step approach in which a text is translated into another language and then back into English. 
We use French as second language.
While conceptually simple, it has been shown to produce only limited stylistic variation~\citep{zhou_paraphrase_2025}.
 
    \cleardoublepage 
    \chapter{Experimental Setup}
\label{chap:experimental_setup}

In the following, we will outline the experimental setup for the experiments we ran.
This includes not only allocation, preprocessing and pair selection of each dataset, but also a description and motivation of the experiments carried out.


\section{Dataset}
\label{sec:dataset}

Since our method extends the original \impAppr{} proposed by \citet{koppel_determining_2014}, we first obtained the datasets used in their study to validate our implementation and reproduce their results. 
The original experiments were carried out on the \dataBlog{} and \dataStudent{} datasets, described in detail in \Cref{subsec:original_data}.
In addition to these, we incorporated two supplementary datasets, \dataPan{} and \dataGutenberg{}, presented in \Cref{subsec:additional_data}. 
Following the general description of all datasets, we outline our preprocessing pipeline in \Cref{subsec:dataset_preprocessing} and conclude with the text-pair selection procedure in \Cref{subsec:dataset_text_pair_selection}.


\subsection{Original Data}
\label{subsec:original_data}

% Blog
The \dataBlog{} corpus~\citep{blog_dataset_2006} consists of blog posts collected from \textit{blogger.com} on or before August 2004, with each blog authored by a single user.
According to the Kaggle repository~\footnote{\href{https://www.kaggle.com/datasets/rtatman/blog-authorship-corpus?resource=download}{Kaggle dataset \texttt{rtatman/blog-authorship-corpus}} (26.07.2025)}, the dataset contains \num{681288}~posts from \num{19320}~bloggers, averaging approximately 35~posts and \num{7250}~words per author.
Each record includes the following metadata: \texttt{id}, \texttt{gender}, \texttt{age}, \texttt{topic}, 
\texttt{sign} (referring to the author's zodiac sign), \texttt{date}, and \texttt{text}.

% student essays
The \dataStudent{} dataset is not publicly available due to the presence of sensitive student information. 
Access may be requested from J. W. Pennebaker, the official custodian.
The dataset comprises \num{7052}~student essays written for five assignments by a cohort of 950 university students in 2006~\citep{koppel_determining_2014}.
Its restricted availability makes it a particularly valuable testbed for evaluation, as it is highly unlikely to have been incorporated into the training data of \acp{llm}.

The assignments include (1) a stream-of-consciousness task, (2) a reflections on childhood, (3) a self-assessment of personality, (4) a thematic apperception test, and (5) four examples of four different theories.
% Most files are named solely by the author ID, whereas those from the first assignment follow the format \texttt{2006\_authorID}.
Following \citet{koppel_determining_2014}, our dataset is built from the first four assignments. 
The dataset provides metadata collected from structured columns in the \texttt{.dat} files and derived from file names, which we combined into a unified resource. 
Metadata includes year, author ID, author name, political orientation, task, sex, ethnicity, and teacher.


\subsection{Additional Data}
\label{subsec:additional_data}
To broaden the evaluation scope of the \impAppr{}, we incorporated additional datasets that both control for confounding factors such as genre and topic and consist of texts with verified, undisputed authorship. 
Both the \dataPan{} and \dataGutenberg{} datasets meet these criteria.

% PAN20: Fanfiction
The \dataPan{} corpus~\citep{bischoff_importance_2020} comprises fanfiction texts sourced from \textit{fanfiction.net}.
Each text belongs exclusively to one fandom (i.e.\ thematic category), with no crossovers between fandoms.
According to \href{https://pan.webis.de/clef20/pan20-web/author-identification.html}{the official \acs{pan} website}, 
train and test set originate from two different fanfictions.
Dataset features include \texttt{id}, \texttt{fandoms}, and \texttt{pair}, where the latter contains the paired texts.
An additional \texttt{jsonl} file provides the ground truth for each pair, specifying \texttt{id}, \texttt{same} (ground truth label for \ac{av}), and \texttt{authors}.

% Gutenberg
The \dataGutenberg{} dataset~\footnote{\url{https://www.gutenberg.org/} (26.07.2025)} contains a curated selection of literary works from Project Gutenberg, a digital library dedicated primarily to older works whose U.S. copyrights have expired.
As of this writing, the collection contains over \num{75000}~digitized and proofread e-books contributed by volunteers according to their website.
% For our experiments, we selected 19 works authored by 7 writers from the 16th to 19th centuries, the distribution of genres is given in \Cref{tab:genre_counts_gutenberg}.
For our experiments, we selected 19~works authored by 7~writers from the 16th to 19th centuries, covering nine dramas, nine fiction texts and one poetry work.
Metadata for these works was manually extracted from the \href{https://www.gutenberg.org/}{Project Gutenberg website} and Wikipedia.

% \begin{table}[]
% \centering
% \caption{Unique value count of genres of \dataGutenberg{} dataset.}
% \label{tab:genre_counts_gutenberg}
% % \resizebox{\textwidth}{!}{%
% \begin{tabular}{@{}ll@{}}
%     \toprule
% \textbf{Genre} & \textbf{Count} \\
% \midrule
% Drama          & 9              \\
% Fiction        & 9              \\
% Poetry         & 1    \\
% \bottomrule         
% \end{tabular}%
% % }
% \end{table}

% \textcolor{orange}{We augment the \dataStudent{} dataset with artificial generated texts and denote this dataset \dataArtificialStudent{}.
% It contains pairs of student essays written in response to different academic assignments. 
% The dataset includes both human-written essays and artificially generated paraphrases created by \acp{llm} instructed to simulate student writing. 
% Artificially generated essays are produced using \acp{llm} prompted to emulate an 18-year-old college freshman's voice, taking into account demographic features like sex, ethnicity, and political orientation.
% Each pair is labelled as same-author or different-author indicating whether the text was generated artificially. 
% }

\subsection{Dataset Preprocessing}
\label{subsec:dataset_preprocessing}

To control confounding factors that influence authorial style, we preprocess each dataset twice:
(1) Prior to generating the arrow dataset file and (2) before using the \impAppr{}.
This two-stage approach addresses both experimental scenarios, where all material is prepared in advance, and inference scenarios in which the \impAppr{} is applied directly to texts.
The preprocessing process was designed to meet the following requirements:
\begin{enumerate}
    \item Removal of all formatting and layout information to produce plain text
    \item Cropping texts to match the length of the shorter text in each pair
    \item Removal of texts with less than 700 words
\end{enumerate}
For a controlled evaluation environment in our \impAppr{}, we opted to work with relatively small, curated datasets rather than scaling to larger collections.  
Text-length filtering is essential to the creation of optimal testbeds for \ac{av}.
To determine the necessary preprocessing steps, we examined our datasets and analysed their respective artefacts.

Since both the \dataBlog{} and the \dataPan{} dataset originate from the Internet, some of their documents contain HTML fragments such as \texttt{< >} enclosed tags.
Upon inspection of the \dataGutenberg{} dataset, we found that certain patterns reappear due to the presences of many plays.
Based on these findings, we collected eight preprocessing steps we deemed potentially useful. 
Notably, we do not consider actor instructions (e.g. character cues) or structural elements (e.g. chapter headings) as part of authorial style.
Hence, we remove these patterns using regular expressions.
You may find the regular expression attached in \Cref{app:regex_preproc}. % of \Cref{ch:appendix}.
Note that the perceived utility of these steps should be reevaluated for other datasets.
Next, we analysed the effect of individual preprocessing steps on vocabulary size.  
We define the vocabulary of a dataset as unique tokens across all texts (regardless of their text length or split) of that dataset.
In line with \citep{koppel_determining_2014}, tokens are space-free character 4-grams.
We leave the capitalisation of tokens unchanged. 
The results are displayed in \Cref{fig:preprocesing_impact_vocab_size}.

We found that HTML-specific preprocessing steps had minimal impact on the size of vocabularies.
Similarly, the removal of artefacts specific to theatre plays had little effect on the overall vocabulary size.
The results aligned with expectations, as the vocabularies are defined over space-free 4-gram units that, due to their limited length, exhibit a high likelihood of redundancy.
Consequently, repetitive patterns are mapped on the same vocabulary item.
Once the repetitive patterns are filtered out, the vocabulary lacks their respective items, which are not many.

Based on these considerations, our preprocessing steps include removing HTML artefacts, play artefacts, newlines, converting UTF-8 to ASCII, and stripping leading and trailing whitespace.
We opted to forgo lowercasing the texts, as our preliminary analysis indicated that lowercasing had no meaningful effect on any dataset while potentially discarding deliberate authorial capitalisation choices.
Due to preprocessing, our \dataPan{} version differs from those applied elsewhere.

\begin{figure}[htbp]
    \centering
    \includesvg[width=\textwidth]{images/dataset/impact_preprocessing_steps.svg}
    \caption[Effect of preprocessing steps on vocabulary size.]{Effect of preprocessing steps on vocabulary size. The vocabulary contains unique space-free character 4-grams.
    For this experiment, there was no minimum text length.
    For \dataPan{}, train and test split were combined.}
    \label{fig:preprocesing_impact_vocab_size}
\end{figure}


\subsection{Selection of Text Pairs}
\label{subsec:dataset_text_pair_selection}

For the \dataBlog{}, \dataStudent{}, and \dataGutenberg{} datasets, we selected pairs of texts according to specific criteria to control potential confounding factors.
Only texts with a minimum length of \num{700}~words were considered eligible. 
For the \dataPan{} dataset, we retained the existing pairs in the arrow dataset, but filtered out texts with less than \num{700}~words. 
All datasets include both same-author and different-author pairs. 

For the \dataBlog{} dataset, we matched text pairs on topic, year, gender, and author age. The training set contains 80\% of the data and the test set 20\%, with distinct topics in each split.

For the \dataStudent{} dataset, following \citet{koppel_determining_2014}, we drew all text pairs from different tasks, with authors matched by sex, ethnicity, and political orientation. Pairs were split into training (70\%) and test (30\%) sets.
The test portion is larger than in the \dataBlog{} dataset because each author typically contributes only one essay per task, so including only one of four tasks in the test set would prevent pair formation.

For the \dataGutenberg{} dataset, we selected text pairs that share the same genre and century and split the authors into training (80\%) and test (20\%) sets.

% minimum length necessesary for AV/ AA
The choice of minimum text length was informed by literature research.
\citet{bevendorff_generalizing_2019}\ used text chunks of at least 700 words for an unmasking approach, while \citet{koppel_authorship_2004}\ set the minimum to \num{500}~words.
Recent work~\citep{llm_detection_av_2025} identifies \num{2500}-\num{4000}~characters to be sufficient for effective \ac{llm} detection framed as \ac{av}.

Regardless of the selection criteria, the final datasets contain only three columns: \texttt{authors}, \texttt{pair}, and \texttt{same}.
The \texttt{pair} column contains the texts of the pair as a list of strings,
the \texttt{authors} column contains the authors of the texts as a list of strings,
and the \texttt{same} column indicates whether the texts originate from the same author (\texttt{True}) or from different authors (\texttt{False}).
Descriptive statistics for all preprocessed datasets are provided in Table~\ref{tab:data_stats}.

% The \dataArtificialStudent{} dataset is split into training and test sets using a stratified approach, ensuring that all combinations of author type (human vs. \ac{llm}), pair type (same- vs. different-author), and artificial generation (True vs. False) are proportionally represented in both splits. 
% Since the \dataArtificialStudent{} dataset is artificially created, its feature differ from the other datasets. 
% Each record contains the assignment names and descriptions, the paired texts, the authors, and metadata flags indicating author sameness and artificial generation.

% The presence of short artefacts in the \dataArtificialStudent{} dataset cannot be attributed to human-authored texts, as these have been filtered to meet a minimum length.
% Hence, the artificially generated texts exhibit shorter average lengths compared to other datasets.

\begin{table}[H]
% \begin{sidewaystable}
\centering\small
\caption[Statistics of preprocessed datasets.]{Statistics of preprocessed datasets \dataBlog{}, \dataGutenberg{}, \dataPan{}, and \dataStudent{}. %, and \dataArtificialStudent{}.
$p_s$, $p_{\neg s}$ denote same-author and different-authors pairs, while $l_w$, $l_c$ denote text length in words and characters, respectively.
}
\label{tab:data_stats}
\resizebox{\textwidth}{!}{%
\begin{tabular}{@{}lrrrrrrrrr@{}}   % numbers should be right aligned, text left aligned
\toprule
dataset & \# pairs & \# authors & \# $p_s$ & \# $p_{\neg s}$ & \diameter $l_w$ ($l_c$) & max $l_w$ & $\sigma_{l_w}$ & median $l_w$ \\
\midrule
\dataBlog{}            & 11565 & 5997  & 6204 & 5361  & 6249.94 (1154.25)     & 115365 & 1493.97 & 913 \\
\dataGutenberg{}       & 12    & 7     & 6     & 6     & 437870.75 (78698.79) & 297704 & 68329.91 & 60282 \\
\dataPan{}           & 66905 & 52771 & 35616 & 31289 & 21418.76 (3914.76)   & 55413 & 512.19 & 3889 \\
\dataStudent{} & 224  & 163   & 112   & 112  & 4403.73 (851.45)     & 1520 & 138.15 & 807  \\
% \dataArtificialStudent{} & 110 & 32 & 50 & 60 & 661.39 (3581.46) & 1769 & 267.08 & 703 \\
\bottomrule
\end{tabular}%
}
\end{table}
% \end{sidewaystable}


% regardless of experimental design
\section{Evaluation Measures}
\label{sec:evaluation_measures}

In the following sections, we review state-of-the-art quantitative evaluation metrics for \ac{av} in \Cref{subsec:av_quality_measures} and for paraphrase generation in \Cref{subsec:paraphrase_evaluation}.
While human judgment is subjective, quantitive metrics are designed to be comparable and reproducible, providing a more objective basis for evaluation.


\subsection{Authorship Verification Quality Measures}
\label{subsec:av_quality_measures}

Since \ac{av} forms the core of \ac{aa}, and because every \ac{aa} task can be reduced to \ac{av}, this section focuses primarily on \ac{av}. 
Evaluation relies on standard classification metrics, each with distinct advantages and limitations.

% \paragraph{Accuracy}
% The most straightforward metric is accuracy (\Cref{eq:accuracy}), which measures the proportion of correctly classified cases across all samples. 
% While intuitive, accuracy can be misleading in scenarios with class imbalance. 

% \begin{equation}\label{eq:accuracy}
%     \text{Accuracy} = \frac{\text{\acs{tp}} + \acs{tn}}{\acs{tp} + \acs{tn} + \acs{fp} + \acs{fn}}
% \end{equation}

% \paragraph{Precision, Recall \& $\operatorname{F_{1}}$}
Precision $\operatorname{P}$ and recall $\operatorname{R}$ address the limitations of accuracy in class imbalance scenarios by focusing on the effectiveness for the positive class, that is, the correct identification of same-author pairs. 
Precision measures the proportion of positive predictions that are correct, while recall quantifies the proportion of \ac{tp} cases that are successfully detected. 
Because these metrics often trade off against each other, their harmonic mean, the $\operatorname{F_{1}}$ score, is commonly used to provide a balanced assessment of effectiveness. 
\Cref{eq:f1} shows the computation of the $\operatorname{F_{1}}$ score~\citep{neal_surveying_2018}.

\begin{equation}\label{eq:f1}
     \operatorname{F_{1}} = \frac{2\mathrm{P}  \mathrm{R}}{\mathrm{P} + \mathrm{R}}
\end{equation}

% \paragraph{$\operatorname{F_{0.5u}}$ \& $\operatorname{c@1}$}
% Beyond these basic measures, \ac{av} research has introduced metrics that explicitly account for difficult borderline cases. 
% The modified $\operatorname{F_{0.5u}}$ score penalizes non-answers by treating them as \acp{fn}~\citep{bevendorff_overview_2024}. 
% This places additional weight on correctly deciding same-author cases~\citep{weerasinghe_feature_vector_difference_2021}, thereby evaluating the ability of \ac{av} methods to abstain from hard samples~\citep{tyo_state_2022}. 
% The $\operatorname{c@1}$ metric, by contrast, was designed to reward abstention from particularly ambiguous cases. 
% It does so by granting unanswered problems partial credit, equal to the average accuracy on the remaining cases~\citep{bevendorff_overview_2024}, thus encouraging systems to remain silent when uncertain.

% However, in the present work, $\operatorname{c@1}$ and $\operatorname{F_{0.5u}}$ are not appropriate measures. 
Although we also used $\operatorname{F_{1}}$ and accuracy in our work, we opted to exclude them in this thesis to keep results simple and comparable to the original work by \citet{koppel_determining_2014}.
More advanced metrics, such as $\operatorname{F_{0.5u}}$ and $\operatorname{c@1}$, assume that models can explicitly abstain by assigning a score of exactly 0.5, a convention used in \acs{pan}'s shared tasks~\citep{tyo_state_2022,bevendorff_overview_2024,kocher_unine_2015}. 
In our setting, the model produces only the binary outcomes "same author" or "don't know". 
Since there is no natural mechanism for producing a calibrated 0.5 score, abstention cannot be meaningfully represented. 
One could introduce a second threshold to create an artificial abstention region, but this would not align with the open-set nature of \ac{av}. 
The different-author class is inherently ill-defined, as the set of possible authors is unbounded and cannot be exhaustively represented. 

% Moreover, we do not report average effectiveness scores across different recall values, since we do not need good results across different thresholds, but only high effectiveness for one threshold.
% For this reason, the focus in this thesis remains on precision, recall, and $\operatorname{F_{1}}$. 




\subsection{Paraphrase Evaluation}
\label{subsec:paraphrase_evaluation}

There is no universal definition of what constitutes a paraphrase. 
Definitions vary in degree of semantic equivalence required. 
This conceptual ambiguity makes the task of evaluation especially challenging, since different applications may prioritize different aspects such as fidelity to meaning, stylistic variation, or grammatical well-formedness.
Because of this, paraphrase evaluation must account both for syntactic diversity and for the extent to which semantic content is preserved. 

Existing approaches can broadly be grouped into automatic and human-based methods. 
Automatic measures attempt to quantify the similarity between a candidate and a reference paraphrase using algorithmic techniques. 
These methods can be further distinguished by the linguistic level at which they operate. 
Some focus on syntactic structure, while others evaluate semantic preservation~\citep{gohsen_captions_2023}. 
Human evaluation, in contrast, remains the gold standard, as it naturally incorporates all of these dimensions.
In the following, we focus on both automatic evaluation strategies. 

\input{chapter/section-05/quantitative_evaluation_metrics.tex}


% \subsubsection{Qualitative Evaluation}
% \label{subsec:qualitative_evaluation}

% Human qualitative evaluation can combine syntactic and semantic dimensions more reliable than any automatic metric proposed.
% Naturally, when being asked to evaluate the quality of a paraphrase, individuals will score syntactic difference from the reference text, the readability from the paraphrase and semantic similarity to the reference text.
% Evaluation is usually formalized via a Likert scale~\citep{gohsen_captions_2023}.


\section{Experimental Setup}
\label{sec:experimental_setup}
% for each: Question to answer, experiment design, same language (unified description), when duplicate: short description and reference to other occurence (but avoid sole references)
% questions can be specific, should be related to research question(s)

The following experiments are designed to systematically evaluate the extension of the \impAppr{}~\citep{koppel_determining_2014} with \ac{llm} generated impostors. 

We begin by reproducing the original experiments by \citet{koppel_determining_2014} to ensure comparability between our implementation and previously reported results. 
Next, we investigate the quality of paraphrases used in \imp{} generation. 
In particular, we compare our one-step paraphrasing strategy with our two-step approach based on the paraphrasing measures from \autoref{subsec:traditional_quantitative_evaluation_measures}. 
Then, we evaluate paraphrase quality across different numbers of text chunks.
We then analyse the effect of syntactic similarity between paraphrases and original texts on \imp{} scores. 
Then, we compare different \ac{av} methods including the \impAppr{} with several strategies for \imp{} generation for varying sources of input text pairs. 
This allows us to examine the relative strengths of different \ac{av} approaches for specific \ac{av} tasks.


\subsection{Exp.\ 1: Reproduction of Original Work}

To assess the validity of our extension to the traditional \impAppr{}, we first verified the correctness of our implementation. 
For this purpose, we designed two experiments, which we ran on a subset of 100~pairs from the training and test sets of the \dataBlog{} and \dataStudent{} dataset respectively. 
Half of the selected samples belong to the same-author class.

\begin{table}[h]
\centering\small
\caption{Exp.\ 1: \impAppr{} configuration.}
\label{tab:config_exp1}
\begin{tabular}{@{}rrlrrl@{}}   % numbers should be right aligned, text left aligned
\toprule
Experiment & \# Impostors & Generation & Rounds & Top $n$ & Upsample \\
\midrule
1(a) & \textit{Variable} & Fixed & 100 & \num{100000} & False \\
1(b) & 50 & \textit{Variable} & 100 & \num{100000} & False \\
\bottomrule
\end{tabular}%
\end{table}

\paragraph{Exp.\ 1(a): Varying number of \imps{}.}
The first experiment evaluates the effect of varying the number of \imps{} while setting the \imp{} generation method to \texttt{fixed}.
All other hyperparameter values are set to the default values reported by \citet{koppel_determining_2014}\ (cf.~Table~\ref{tab:config_exp1}). 
Adhering \citet{koppel_determining_2014}, we compute precision and recall scores across different thresholds.
\textcolor{orange}{For comparison, reference precision-recall points reported by \citet{koppel_determining_2014}\ are included in our visualization.} 
Based on their description, we deduced that their reported scores were obtained using the \dataBlog{} dataset.

\paragraph{Exp.\ 1(b): Varying \imp{} generation.}
The second experiment evaluates different \imp{} generation methods while keeping the number of \imps{} fixed.
Again, all other hyperparameter values are set to the default values reported by \citet{koppel_determining_2014}\ (cf.~Table~\ref{tab:config_exp1}). 
Following \citet{koppel_determining_2014}, we compare the \texttt{fixed} and \texttt{on-the-fly} \imp{} generation methods with the baseline approaches unsupervised min-max similarity, unsupervised cosine similarity, and supervised linear \ac{svm}.

As in the first experiment, precision and recall are used as the primary evaluation metrics. 
Consistent with \citet{koppel_determining_2014}, we calculate precision and recall with respect to both the same-author and different-author class, alternately treating each as the reference class.
We note that the different-author class is ill-defined, as \ac{av} constitutes a one-class classification scenario in which covering representative instances of this class is infeasible.

\subsection{Exp. 2: Comparison of different Paraphrasers}
\label{subsec:comp_paraphrasers_setup}

Next, we wanted to assess our paraphrasing approaches.
We hence designed two experiments.
The first experiment computes state-of-the-art paraphrasing measures for all paraphrasers on different datasets.
The second experiments aims to evaluate the ability of our two-step models to adhere to instructions.
We tested their proficiency extracting metadata and generating paraphrases of similar length as the reference text.

\paragraph{Exp. 2(a): Quantitative evaluation.}

We select one text from the \dataBlog{}, the \dataGutenberg{} and the \dataStudent{} dataset, respectively.
For the \dataGutenberg{} dataset, we load ground truth metadata. % Student Essay: not used, even though existent
The paraphraser configurations contain two different temperatures for two-step paraphrasers, and two different prompts for one-step paraphrasers.
We create one paraphrase for each text configuration pair.
Evaluation measures include BLEU, ROUGE1, ROUGE2, ROUGEL, ROUGELsum, METEOR, BERTScore Precision, BERTScore Recall, BERTScore F1, SBERT \ac{wms}, SBERT cosine similarity.
Based on these we also compute syntactic and semantic similarity, as well as Gohsen Delta~\citep{gohsen_captions_2023}.
We save the extremest (min, max) paraphrases per metric.
The scores are subsequently visualized via syntactic-semantic scatters, score distributions, and radar plots per paraphraser and per prompt. 

\paragraph{Exp. 2(b): Evaluation of prompt adherence.}
For the second experiment, we select five samples from the \dataBlog{}, \dataGutenberg{} and the \dataStudent{} datasets. 
Our extractor extracts the genre, the topic, and the century of each input text.
Extracted and ground truth values are lowercased and stripped from leading and trailing whitespaces. 
For single genre and topic values, we compute the cosine similarity on their respective SBERT embedding.
For the genre extraction, we split the extractors' result by comma and use the maximum similarity.
Since the ground truth topic usually consists of multiple topics separated by comma, we split them into a list and use the maximum similarity.
For century match, we processed the result of the extractor by mapping \textit{present}, \textit{current}, and \textit{now} to 21, then extracting digits and omitting the last two digits from any numbers with at least three digits and finally adding one if the original digit was not divisible by 100.
We then use the ground truth as a baseline $b$ for extracted century $a$ and compute $\frac{a}{b}$.
\textcolor{red}{TODO: examples}
While the \dataBlog{} metadata comes its csv dataset, the \dataStudent{} metadata is derived from existing information about and in the dataset, the \dataGutenberg{} metadata was completely manually curated.
Additional to the previously mentioned processing steps for century values, ground truth century date values were casted to dates.
Moreover, we obtain the relative length difference of the paraphrase and original text for every selected sample. 
This concludes in an evaluation of our paraphrasing (extractors) in terms of genre, topic, century and length similarity.

\subsection{Exp.\ 3: Paraphrasing Chunks}
\label{subsec:paraphrasing_chunks}

We designed this experiment to evaluate whether chunk-to-chunk paraphrases exhibit better control than text-to-text paraphrases, since chunks contain fewer topic changes than whole texts in theory.
We use one text from the \dataBlog{}, \dataGutenberg{}, and the \dataStudent{} dataset, respectively.


\begin{figure}[htbp]
  \centering
  \includesvg[width=\linewidth]{images/paraphrasing/experiments/chunks/setup/chunk_api_calls.svg}
  \caption[Paraphrase configuration hyperparameters.]{Breakdown of individual hyperparameters in the paraphrase configuration.
  We use one document per dataset, chunked into one to five sections and paraphrased with all nine paraphrasers in two variance inducing settings (i.e.\ prompt for one-step, temperature for two-step).
  This amounts to a total of 936 API calls. 
  }
  \label{fig:chunks_api_calls}
\end{figure}


First, texts are chunked preserving sentences.
Chunks are filled with sentences in sentence order such that each chunk roughly contains the same number of words.
Second, paraphrase configurations are defined.
Each one-step paraphraser is paired with two prompts, while each two-step paraphraser is paired with two temperatures.
Third, each chunk is paraphrased with all configurations.
These steps account for a minimum of 936 API calls for paraphrasing.
Each component of the configuration is displayed in \autoref{fig:chunks_api_calls}.
Finally, for each paraphrase, we compute \ac{bleu}, \ac{rouge}-1, \ac{rouge}-2, \ac{rouge}-L, \ac{rouge}-Lsum, METEOR, \ac{bert}\-Score Precision, \ac{bert}\-Score Recall, \ac{bert}\-Score F1, \ac{sbert} \ac{wms}, \ac{sbert} cosine similarity.
Final scores per metric for each text-configuration pair are computed by averaging the scores of its constituent text chunks.
The adequate formula is given in \autoref{eq:avg_chunks} and an example is illustrated in \autoref{fig:mean-bleu}.

\begin{equation}
    score(t) = \frac{1}{\#\text{ chunks}}\sum_{i=1}^{\#\text{ chunks}}score(c_i)\text{, for chunk }c_i \in \text{text }t
\label{eq:avg_chunks}
\end{equation}

\begin{figure}[ht]
  \centering
\resizebox{0.9\textwidth}{!}{%
\begin{tikzpicture}[line join=round,line cap=round, >=latex, font=\sffamily]

% --- Left black container with three chunks ---
\draw[black, very thick, rounded corners=6pt]
  (-0.2,3.6) rectangle (3.0,-0.4);

% three inner rounded rectangles
\foreach \y in {2.8,1.6,0.4}{
  \draw[black, thick, rounded corners=5pt] (0.15,\y+0.45) rectangle (2.65,\y-0.45);
  % squiggle inside
  \draw[black, thick, decorate, decoration={snake,amplitude=1.2pt,segment length=6pt}]
    (0.45,\y-0.15) -- (2.35,\y-0.15);
  \draw[black, thick, decorate, decoration={snake,amplitude=1.2pt,segment length=6pt}]
  (0.45,\y+0.15) -- (2.35,\y+0.15);
}

% --- Colored BLEU labels next to each chunk ---
\node[anchor=west, text=teal!80!black, scale=1.2]  at (3.6,2.8) {BLEU $=\,0.1$};
\node[anchor=west, text=orange!85!black, scale=1.2] at (3.6,1.6) {BLEU $=\,0.5$};
\node[anchor=west, text=violet, scale=1.2]         at (3.6,0.4) {BLEU $=\,0.3$};

% --- n_chunks = 3 (black) ---
\node[anchor=west, text=black, scale=1.2] at (0.0,-1.0) {$\#\text{ chunks}=3$};

% --- Arrow to the right and mean BLEU expression ---
\draw[black, very thick, ->, >=latex] (7.7,1.6) -- (9.1,1.6);

\node[anchor=west, text=black, scale=1.4] at (9.3,1.6)
  {$\varnothing\ \text{BLEU} \;=\; \displaystyle
   \frac{\textcolor{teal!80!black}{0.1}+\textcolor{orange!85!black}{0.5}+\textcolor{violet}{0.3}}{3}$};

\end{tikzpicture}
}
  \caption[Computation of the mean \ac{bleu} score over chunks.]{Computation of the mean \ac{bleu} score over three text chunks of a text.}
  \label{fig:mean-bleu}
\end{figure}


\subsection{Exp.\ 4: Impact of Syntactic Similarity on \impApprTitle{} Performance}
\label{sec:syn_sim_impact_}

We designed this experiment in order to assess whether the syntactic similarity of generated paraphrases, i.e.\ the difficulty of hard negatives, influences the effectiveness of the \impAppr{}.
We conducted this experiment on both the \dataBlog{} and \dataStudent{} datasets, selecting 15 samples each from the training and test splits. 
The detector was configured according to Table~\ref{tab:imp_syn_sim_config}.

\begin{table}[h]
\centering\small
\caption{Exp.\ 4: \impAppr{} configuration.}
\label{tab:imp_syn_sim_config}
\begin{tabular}{@{}rlrrl@{}}   % numbers should be right aligned, text left aligned
\toprule
\# Impostors & Generation & Rounds & Top $n$ & Upsample \\
\midrule
50 & LLM & 100 & \num{100000} & False \\
\bottomrule
\end{tabular}%
\end{table}

For generation, we loop through all \ac{llm}-based paraphrasers until we successfully created 50 \imps{}.
One-step paraphrasers are used with both prompts.
Predictions on the test set were obtained by thresholding the detector’s scores with a decision threshold was determined using Youden’s J statistic on the training set.
We computed the average syntactic similarity on the test set. 
Following \citet{gohsen_captions_2023}, we define average syntactic similarity $\diameter_{syn}$ as the mean of the \ac{bleu}, \ac{rouge}-1, and \ac{rouge}-L scores. 
For each input pair in the test set, we calculated
(1) the average syntactic similarity between the two texts in the pair, (2) the mean average syntactic similarity between the candidate reference text and its paraphrases, and (3) the mean average syntactic similarity between the disputed text and the paraphrases.

We further grouped samples based on (1), (2), (3) and the difference (2)–(1). 
For each group, we computed accuracy, precision, recall, and F1 score of the detector’s predictions. 
The average values for each metric in a bin are presented in a bar chart.



\subsection{Exp. 5: Comparing \acs{av} Methods in Traditional Human-Human Scenario}
\label{subsec:imp_gen}

We want to answer the question of how our \ac{llm}-based \imp{} generation performs compared to (a) traditional \imp{} generation methods in the \impAppr{}~\citep{koppel_determining_2014}, and compared to (b) \acl{sota} \ac{av} methods in the traditional \ac{av} scenario.
We thus, create 10 same- and 10 different-author pairs from the \dataStudent{}. % (llm_detection_scenarios.py)
% We thus, create 100 same- and 100 different-author pairs from the \dataStudent{} (comp_av.py)
% 1 each on and the \dataBlog{} datasets % 1 for Blog, 100 for Student Essays
It is noteworthy, that an approach predicting only one output will obtain an accuracy of $0.5$.
The \impAppr{} and \unmasking{} detector configuration are shown in \autoref{tab:exp5_imp_config} and \autoref{tab:exp5_unmasking_config}, respectively.

\begin{table}[h]
\centering\small
\caption{Exp. 5: \impAppr{} configurations.}
\label{tab:exp5_imp_config}
\begin{tabular}{@{}rlrrl@{}}   % numbers should be right aligned, text left aligned
\toprule
\# Impostors & Generation & Rounds & Top $n$ & Upsample \\
\midrule
50 & \textit{Variable} & 100 & \num{100000} & False \\
\bottomrule
\end{tabular}%
\end{table}

\begin{table}[h]
\centering\small
\caption[Exp. 5: Unmasking configurations.]{Exp. 5: Unmasking configurations. CV denotes cross-validation.}
\label{tab:exp5_unmasking_config}
\begin{tabular}{@{}rrrrl@{}}   % numbers should be right aligned, text left aligned
\toprule
\# CV Folds & \# Chunks & Rounds & Top $n$ & Upsample \\
\midrule
3 & 60 & 30 & \num{250} & False \\
\bottomrule
\end{tabular}%
\end{table}

For each impostor generation method, we computed accuracy, precision, recall, and F1 score for different thresholds. 


\subsection{Comparing \ac{av} Methods}

This experiment evaluates the performance of the \impAppr{} in comparison to established \ac{av} methods.
As baselines, we employ generalized unmasking~\citep{bevendorff_generalizing_2019} and the compression-based approach PPMD approach.
All methods share the common characteristic of operating on lower-dimensional representations of the input text pair to determine whether both texts originate from the same author.

Following the experimental setup described in \autoref{subsec:imp_gen}, we assess performance using  accuracy, precision, recall, and the F1 score. 





 % ggf experimente setup extra/ oder extra + datensätze + exp+ analyse (Tabelle)
    \cleardoublepage 
    \chapter{Experimental Results}
\label{chap:experimental_results}

This chapter presents the experimental results. 
To begin, we reproduce original experiments to validate our approach, after which we compare different paraphrasing techniques. 
We then analyse paraphrase quality across text chunk sizes and prompts, and finally contrast traditional \ac{av} methods with the \ac{llm}-based \impAppr{}.


\section{Exp.\ 1: Reproduction of Original Work}

Our first experiment covers the reproduction of the original results~\citep{koppel_determining_2014}.
It is noteworthy, that we do not expect exact replication since we could only reimplement the approach to our best knowledge.
Both Mr. Koppel and Mr. Winter were very forthcoming when we contacted them regarding implementation details that could help to improve our implementation.
Unfortunately, the code was no longer traceable and neither author could recall the preprocessing steps.
After consultation with Mr. Winter, he assured that our preprocessing steps seem reasonable.
This, however, certainly is one of the reasons why our implementation diverges from the original work.

\paragraph{Exp.\ 1(a): Varying number of \imps{}.}

For this experiment, we generate \imps{} using the \texttt{fixed} method while varying only the number of \imps{}.
\citet{koppel_determining_2014}\ report results of this experiment exclusively on the \dataBlog{} dataset.
While we compare their findings with ours, it is important to note that direct comparability is limited due to differences in the underlying datasets, as our results are based on the \dataStudent{} corpus.
In contrast to the original study, our \impAppr{} recall–precision curves intersect, suggesting that the results reported by \citet{koppel_determining_2014}\ exhibit stronger separation.
In our experiments, the number of \imps{} exerts minimal influence on precision and recall.
Nevertheless, similar to the original findings, we observe that using 50 \imps{} yields marginally superior performance.


\begin{figure}[htbp]
    \centering
    \includesvg[width=\textwidth]{images/imposter/reproduction_koppel_figures/fig2/student_essays/student_roc_prec_recall_curve_r100_top100000_dif_n_imp.svg}
    \caption[Recall-precision curves for the various sized \imp{} set sizes.]{Recall-precision curves for the various sized \texttt{fixed} \imp{} set sizes on the \dataStudent{} dataset.}
    \label{fig:student_essays_dif_n}
\end{figure}

% \begin{figure}[htbp]
%   \centering
%   \begin{subfigure}[b]{0.48\textwidth}
%     \centering
%     \includesvg[width=\linewidth]{images/imposter/reproduction_koppel_figures/fig2/student_essays/student_roc_prec_recall_curve_r100_top100000_dif_n_imp.svg}
%     \caption{\dataBlog{} \textcolor{red}{TODO: needs to run (15.09.2025)}}
%     \label{fig:blog_dif_n}
%   \end{subfigure}
%   \hfill
%   \begin{subfigure}[b]{0.48\textwidth}
%     \centering
%     \includesvg[width=\linewidth]{images/imposter/reproduction_koppel_figures/fig2/student_essays/student_roc_prec_recall_curve_r100_top100000_dif_n_imp.svg}
%     \caption{\dataStudent{}}
%     \label{fig:student_essays_dif_n}
%   \end{subfigure}
%   \caption{Recall-precision curves for the various sized \imp{} set sizes.}
%   \label{fig:repr_diff_n_imps_fixed}
% \end{figure}


\paragraph{Exp.\ 1(b): Varying \imp{} generation.}

Adhering to \citet{koppel_determining_2014}, we compare multiple \ac{av} methods with variants of the \impAppr{} that employ different \imp{} generation strategies, evaluating them in the \ac{av} scenario with human-authored text pairs.
The \imp{} generation strategies include sampling from a \texttt{fixed} set of potential \imp{} candidates, and \texttt{on-the-fly} \imp{} generation. 
The baselines introduced by \citet{koppel_determining_2014} are \ac{svm}, and unsupervised similarity based approaches.
Each class is treated as the positive (i.e. reference) class once, with the other class serving as the negative class. 
This choice affects how the evaluation scores are computed, but not how the thresholds are fit during the training.
Although not explicitly stated, it is likely that \citet{koppel_determining_2014} have trained each approach for both reference class scenarios once.
The original work reports the results for the \dataBlog{} dataset only.

% original results
The recall–precision curves reported by \citet{koppel_determining_2014}\ indicate that \texttt{fixed} \imp{} generation outperforms \texttt{on-the-fly} \imp{} generation.
In their study, the minimum recall values for the different-author class were consistently higher (approximately $0.625$ for \texttt{fixed} and $0.58$ for \texttt{on-the-fly}) than for the same-author class (approximately $0.38$ for \texttt{fixed} and $0.22$ for \texttt{on-the-fly}).
In contrast, precision values exhibited the opposite trend where all approaches achieved higher precision in the same-author setting compared to the different-author setting.

Our interpretation of the results reported by \citet{koppel_determining_2014}\ is presented in \autoref{fig:findings_original_work}.
We argue that these findings are a consequence of the nature of \ac{av} as an open-set, one-class classification problem.
Because the different-author class lacks representative samples, it remains inherently ill-defined.
As a consequence, both \imp{} generation methods tend to yield more different-author predictions than same-author predictions, producing higher recall for the different-author class and higher precision for the same-author class.

\begin{figure}[htbp]
    \centering
    \includesvg[width=\textwidth]{images/imposter/reproduction_koppel_figures/fig2/student_essays/fig2_original_findings.svg}
    \caption[Aggregating original \impAppr{} experiment results.]{Results reported by \citet{koppel_determining_2014} suggest that the different\-author class is more difficult to model, leading to more different-author predictions and consequently higher recall for that class.}
    \label{fig:findings_original_work}
\end{figure}

% our results
Both \citet{koppel_determining_2014}\ and our results indicate that \texttt{fixed} \imp{} generation performs best in the same-author scenario.
Beyond this agreement, however, our recall–precision curves deviate from the original findings.
In the different-author scenario, all methods except \texttt{on-the-fly} generation exhibit low precision, with \texttt{fixed} \imps{} performing worst.
Low precision corresponds to a high number of \acp{fp}, i.e., many same-author pairs incorrectly classified as different authors.
This outcome suggests that training on the same-author class produces systematically poor performance in a different-author class test scenario, as the model is not optimized for it.

Opposed to the original work of \citet{koppel_determining_2014}, our \texttt{on-the-fly} \imps{} lack difficulty explaining the low precision in the same-author and the high precision in the different-author scenario.
Since we had to reduce the number of test samples for \texttt{on-the-fly} \imp{} generation due to API call limits, we suggest analysing this approaches' results with a grain of salt.
However, Mr. Winter noted that our \texttt{on-the-fly} \imp{} generation implementation seems insufficient, due to the rise of bot prevention blocking web scraping and the very limited number of API calls per month.

% \textcolor{red}{TODO: figures SVC to SVM}
\begin{figure}[htbp]
  \centering
  \begin{subfigure}[b]{0.49\textwidth}
    \centering
    \includesvg[width=\linewidth]{images/imposter/reproduction_koppel_figures/fig4/blog/blog_roc_prec_recall_curve_r100_top100000_Same_Author_dif_imp_gen.svg}
    \caption{Same-author reference class. }
    \label{fig:blog_same_author}
  \end{subfigure}
  \hfill
  \begin{subfigure}[b]{0.49\textwidth}
    \centering
    \includesvg[width=\linewidth]{images/imposter/reproduction_koppel_figures/fig4/blog/blog_roc_prec_recall_curve_r100_top100000_Different_Author_dif_imp_gen.svg}
    \caption{Different-author reference class.}
    \label{fig:blog_diff_author}
  \end{subfigure}
  \caption[Recall-precision curves for the \dataBlog{} dataset. ]{Recall-precision curves for the \dataBlog{} dataset. 
  (B) indicates the original baseline approaches from~\citep{koppel_determining_2014}.
  Due to API limit restrictions, the test set for \texttt{on-the-fly} was smaller which is visible in the respective curves.
  Classifiers were not retrained for the different-author reference class scenario explaining the poor results.}
  \label{fig:diff_imp_gen_blog}
\end{figure}


\section{Exp.\ 2: Comparison of Paraphrasers}
\label{sec:comp_paraphrases}

To evaluate the quality of generated paraphrases, we conducted two experiments. 
In Exp.\ 2(a), we assessed paraphrasing using standard quantitative metrics, while in Exp.\ 2(b), we compared the extracted information and text lengths of the generated paraphrases to the ground truth metadata and original texts, respectively.

\paragraph{Exp.\ 2(a): Quantitative evaluation.}

Paraphrasing scores were computed separately for the \dataBlog{}, \dataGutenberg{}, and \dataStudent{} datasets. 
\autoref{fig:sem_syn_blog} presents aggregated semantic and syntactic measurement scores for the \dataBlog{} dataset, while results for \dataGutenberg{} are provided in the Appendix in \autoref{sec:app_paraphrases}.

Syntactic similarity was quantified by averaging \ac{rouge}-1, \ac{rouge}-L, and BLEU scores. 
Semantic similarity was assessed using BERTScore, cosine similarity of BERT-based embeddings, and SBERT \ac{wmd}. 
It is important to note that for our purposes, syntactic diversity is desirable.
High syntactic similarity values may reflect near-identical paraphrases due to n-gram overlap. 
Semantic similarity measures the content overlap between the paraphrase and the original text, often via vector-based cosine similarity. 
% The precise interpretation of these metrics remains somewhat unclear.

\begin{figure}[htbp]
    \centering
    \includesvg[width=\textwidth]{images/paraphrasing/experiments/sem_syn_scatter/Blog_sem_syn_scatter_grouped_by_Paraphraser.svg}
    \caption[Comparison of paraphrasers on the \dataBlog{} dataset.]{Average semantic $\diameter_{sem}$ and syntactic similarity $\diameter_{syn}$ for different paraphraser on the \dataBlog{}.}
    \label{fig:sem_syn_blog}
\end{figure}


Analysis reveals two distinct clusters corresponding to one-step and two-step paraphrasers. 
Most one-step paraphrasers achieve lower syntactic and semantic similarity than two-step paraphrasers. 
The exception is the one-step paraphraser using \texttt{qwen3-32b}, which exhibits higher syntactic similarity than most two-step paraphrasers. 
The translation-based approach emerges as an outlier in terms of both syntactic and semantic similarity. 
Different prompt formulations appear to have minimal impact on these similarity scores (cf. \autoref{sec:app_paraphrases}). 
Overall, most paraphrases fall within the desired quadrant of high semantic similarity with low syntactic similarity.


\paragraph{Exp.\ 2(b): Evaluation of prompt adherence.}

Our two-step paraphrasing approach relies on extracting valid information from the source text. 
To evaluate the quality of extraction by the \pextractor{}, we compared the topic, genre, and century to the ground truth metadata for the \dataBlog{}, \dataGutenberg{}, and \dataStudent{} datasets. 
Genre and topic were evaluated in terms of semantic similarity, while century was assessed via the percentage deviation from the ground truth.

We observed that instructions for the \pextractor{} must follow the input text, as \acp{llm} tend to focus attention toward the end of the input. 
Otherwise, the \pextractor{} failed to produce the requested JSON format for long texts from the \dataGutenberg{} dataset. 
Results across datasets are summarized in \autoref{tab:extraction_eval_stats}, with $\diameter$ and $\sigma$ denoting the mean and standard deviation across the five selected samples. 
The \dataBlog{} dataset proved most challenging for genre and topic extraction. 
While differences in text length between reference and paraphrase were maximal among all datasets, the \pextractor{} performed best on the \dataGutenberg{} dataset.


\begin{table}[h]
\centering
\caption[Extraction performance and length matching for different datasets.]{Extraction performance and length matching for different datasets. Five documents per dataset were processed using the \pextractor{} to obtain genre, century, and topic.}
\label{tab:extraction_eval_stats}
\begin{tabular}{lrrrrrrrr} % numbers should be right aligned, text left aligned
\toprule
 &
  \multicolumn{2}{l}{\textbf{Genre}} &
  \multicolumn{2}{l}{\textbf{Century}} &
  \multicolumn{2}{l}{\textbf{Topic}} &
  \multicolumn{2}{l}{\textbf{Length}} \\
  \textbf{Dataset}
 &
  \textbf{\diameter} &
  \textbf{$\sigma$} &
  \textbf{\diameter} &
  \textbf{$\sigma$} &
  \textbf{\diameter} &
  \textbf{$\sigma$} &
  \textbf{\diameter} &
  \textbf{$\sigma$} \\
  \midrule
\dataBlog{}            & 0.38 & 0.06  & 0.99 & 0.02 & 0.04  & 0.05  & -0.10 & 0.73 \\
\dataGutenberg{}       & 0.58 & 0.14  & 1.00 & 0.04 & 0.3 & 0.15 & -1.00 & 0.00  \\
\dataStudent{} & 0.53 & 0.26 & 0.60 & 0.55 & 0.25 & 0.05  & 0.34 & 0.20 \\
  \bottomrule
\end{tabular}%
\end{table}

Notably, paraphrases generated in other experiments were often substantially shorter than the reference texts. 
In multiple cases, the \pgenerator{} returned placeholders such as \texttt{I’m sorry, but I can’t help with that}.


\section{Exp.\ 3: Paraphrasing Chunks}
\label{sec:results_chunks}

This experiment tests the hypothesis that dividing a text into smaller chunks improves paraphrasing effectiveness, as individual chunks typically contain fewer topics than the full text. 
We computed several paraphrasing metrics for each chunk and averaged the results over the chunks of a text.

\begin{table}[t]
\centering
\caption[Impact of the number of chunks on paraphrase measures]{Impact of the number of chunks on syntactic and semantic paraphrase measures. 
Impact is reported as the absolute change between a single-chunk paraphrase and the maximum number of chunks (i.e.\ 5). 
Bold values indicate the largest observed changes. 
Ideally, syntactic measures should be minimised, while semantic measures are maximised.}
\label{tab:impact_chunks_dataset_paraphraser}
\resizebox{\textwidth}{!}{%
\begin{tabular}{@{}llrrrr@{}} % numbers should be right aligned, text left aligned
    \toprule
\textbf{}         & \textbf{}            & \multicolumn{2}{l}{\textbf{Syntactic Measure} $\downarrow$} & \multicolumn{2}{l}{\textbf{Semantic Measure} $\uparrow$} \\
\textbf{Dataset} & \textbf{Model Type} & \textbf{\diameter}          & \textbf{$\sigma$}          & \textbf{\diameter}          & \textbf{$\sigma$}         \\
\midrule
\dataBlog{}        & one-step & 0.01  & 0.01 & -0.03 & 0.02 \\
                                & two-step & \textbf{-0.12} & 0.08 & -0.03 & 0.04 \\
\dataGutenberg{}    & one-step & 0.0   & 0.0  & -0.04 & 0.03 \\
                                & two-step & 0.0   & 0.0  & -0.01 & 0.05 \\
\dataStudent{} & one-step & 0.03  & 0.04 & 0.03  & 0.07 \\
                                & two-step & \textbf{-0.15} & 0.08 & -0.05 & 0.02 \\
                                \bottomrule
\end{tabular}%
}
\end{table}

\Cref{tab:impact_chunks_dataset_paraphraser} summarises the effect of chunking on syntactic and semantic measures. 
For two-step paraphrasers, the difference between semantic and syntactic scores increases with the number of chunks, driven by decreasing mean average syntactic similarity for the \dataBlog{} and \dataStudent{} datasets. 
\Cref{fig:abl_chunks_blog_translation} illustrates the effect for the translation-based paraphraser on the \dataBlog{} dataset. 
On the \dataGutenberg{} dataset, chunking reduces the semantic similarity of translation-based paraphrases, while other two-step paraphrasers remain largely unaffected, resulting in minimal change to the overall mean semantic similarity. 
As shown in \Cref{fig:abl_chunks_student_essays_llama}, chunking had negligible impact on one-step paraphrasing approaches. 
Additional visualisations are provided in the Appendix (cf.~\Cref{sec:app_chunks}).

Since \ac{rouge} scores are computed on individual reference-candidate pairs, the union of sentences in candidate contains only a single text per comparison. 
Consequently, \ac{rouge}-L and \ac{rouge}-Lsum yield identical results in this setting.

In summary, increasing the number of chunks decreases syntactic diversity for two-step approaches, with only minor reductions in semantic similarity for some datasets. 
Given that processing $n$ chunks with a two-step approach requires $2n$ API calls in the best case, and that chunking has no effect on one-step paraphrases, we excluded chunking from our paraphrasing pipeline.

% \begin{figure}[htbp]
%     \centering
%     \includesvg[width=\textwidth]{images/paraphrasing/experiments/chunks/setup/results/Blog/Translation_metrics_plot_category_Blog.svg}
%     \caption[Impact of the number of chunks on \dataBlog{} dataset]{
%     Average syntactic and semantic measures (shaded areas indicate standard deviation) for the translation-based paraphraser on the \dataBlog{} dataset. 
%     Increasing the number of chunks reduces syntactic similarity.    
%     }
%     \label{fig:abl_chunks_blog_translation}
% \end{figure}

% \begin{figure}[htbp]
%     \centering
%     \includesvg[width=\textwidth]{images/paraphrasing/experiments/chunks/setup/results/Student_Essays/meta-llama-3.1-8b-instruct_metrics_plot_category_Student Essays.svg}
%     \caption[Impact of the number of chunks on \dataStudent{} dataset]{Average paraphrasing measures (shaded areas indicate standard deviation) for a Llama-based paraphraser on the \dataStudent{} dataset. 
%     One-step paraphrasing is unaffected by the number of chunks.
%     }
%     \label{fig:abl_chunks_student_essays_llama}
% \end{figure}

\begin{figure}[htbp]
  \centering
  \begin{subfigure}[b]{\textwidth}
    \centering
    \includesvg[width=\textwidth]{images/paraphrasing/experiments/chunks/setup/results/Blog/Translation_metrics_plot_category_Blog.svg}
    \caption[Translation-based paraphraser on \dataBlog{}]{
    Translation-based paraphraser on the \dataBlog{} dataset.    
    }
    \label{fig:abl_chunks_blog_translation}
  \end{subfigure}
  \hfill
  \begin{subfigure}[b]{\textwidth}
    \centering
    \includesvg[width=\textwidth]{images/paraphrasing/experiments/chunks/setup/results/Student_Essays/meta-llama-3.1-8b-instruct_metrics_plot_category_Student Essays.svg}
    \caption[Llama-based paraphraser on \dataStudent{}]{Llama-based paraphraser on the \dataStudent{} dataset.
    }
    \label{fig:abl_chunks_student_essays_llama}
  \end{subfigure}
  \caption[Effect of chunking on syntactic and semantic measures]{Average syntactic and semantic similarity measures (shaded areas indicate standard deviation) for different number of chunks.
  \Cref{fig:abl_chunks_blog_translation} shows that increasing the number of chunks reduces syntactic similarity for the translation-based paraphrases, whereas \Cref{fig:abl_chunks_student_essays_llama} demonstrates that one-step paraphrasing is largely unaffected by the number of chunks.
  }
  \label{fig:abl_chunks}
\end{figure}

\section{Exp.\ 4: Comparing Prompts}%Assessing the Impact of the Prompt on Paraphrasing}
\label{sec:prompt_impact_res}

In this experiment, we investigate how different prompting strategies influence the quality of paraphrases generated by \acp{llm}. 
To this end, we measured the relative length difference between reference and paraphrase pairs across different \ac{llm}–prompt combinations. 
A subset of pairs was also manually inspected to assess semantic fidelity and readability.

Post-processing was required to remove reasoning traces present in some model outputs, particularly in generations from models such as \texttt{qwen3-32b}. 
These traces, typically delimited by \texttt{</think>}, consist of repeated fragments of the input prompt and do not contribute to the semantic content of the paraphrase. 
We excluded them to retain only the task-relevant text produced by the \ac{llm}.

After post-processing, we computed the relative length difference between each reference and its corresponding paraphrase for all model–prompt combinations. 
The distribution of these differences is presented in \Cref{fig:prompt_impact_post_processed}. 
Because our objective in \imp{} generation is to control for confounding variables, a paraphrase length close to that of the reference is interpreted as an indicator of higher paraphrase quality. 
We additionally performed a manual assessment of content quality that focused on paraphrases both extremely long and length-balanced relative to the reference.

\begin{figure}[H]
    \centering
    \includesvg[width=\textwidth]{images/prompt_impact/paraphraser_length_distribution_post_process_len_perc(qwen)_linear.svg}
    \caption[Impact of different prompts on paraphrases]{
    Box plots of relative paraphrase lengths after post-processing across different prompts.    
    The dotted gray line marks the optimal paraphrase length.
    \texttt{prompt2} consistently generates paraphrases whose lengths are more comparable to the reference than those produced by the other two prompts.
    }
    \label{fig:prompt_impact_post_processed}
\end{figure}

Our results show that the relative length difference of paraphrases strongly depends on the prompt used to instruct the \ac{llm}. 
Notably, the third prompt, i.e.\ \texttt{prompt2}, explicitly instructed models to generate paraphrases three times longer than the reference. 
While this instruction might seem extreme, it consistently generated paraphrases whose lengths were more comparable to the reference than those produced by the other two prompts, across different models, as summarised in \Cref{tab:impact_prompts_paraphrases_lengths}.

\begin{table}[h]
\centering
\caption[Impact of different prompts on paraphrase lengths]{Impact of the prompts on paraphrase lengths. 
Relative length difference is defined as $\frac{\mathrm{len}(paraphrase)}{\mathrm{len}(reference)}\times 100\%$ and denoted $d$. 
Optimal paraphrases are expected to approximate the reference length, i.e.\ $\diameter d \approx 100$. 
Subscript $pp$ indicates post-processed outputs (with reasoning traces removed). 
``Count'' denotes the number of paraphrases considered for each setting. 
For \texttt{prompt2}, only post-processed results are reported.
Bold \diameter $r_{pp}$ values are those closest to the optimal paraphrase length.
}
\label{tab:impact_prompts_paraphrases_lengths}
\resizebox{\textwidth}{!}{%
\begin{tabular}{@{}llrrrrr@{}}
\toprule
Paraphraser & Prompt  & \diameter $d$ & $\sigma d$ & \diameter $d_{pp}$ & $\sigma d_{pp}$ & Count \\
\midrule
meta-llama-3.1-8b-instruct & prompt0 & 39.93 & 52.64 & 39.93 & 52.64  & 135   \\
                            & prompt1 & 40.27  & 24.21 & 40.27  & 24.21 & 124 \\
                            & prompt2 & - & - & \textbf{98.15} & 24.97 & 639  \\
mistral-large-instruct & prompt0 & 1.89   & 1.0   & 1.89   & 1.0   & 138 \\
                        & prompt1 & 13.09  & 17.96 & 13.09  & 17.96 & 129 \\
                        & prompt2 & - & - & \textbf{79.70}  & 11.40 & 449  \\
openai-gpt-oss-120b   & prompt0 & 5.53   & 13.47 & 5.53   & 13.47 & 139 \\
                        & prompt1 & 19.21  & 25.0  & 19.21  & 25.0  & 129 \\
                        & prompt2 & - & - & \textbf{147.54} & 48.45 & 590  \\
qwen3-32b           & prompt0 & 88.36  & 70.02 & 18.68  & 24.09 & 134 \\
                        & prompt1 & 95.73  & 47.72 & 38.34  & 15.64 & 123 \\
                        & prompt2 & - & - & \textbf{81.42}  & 13.46 & 532 \\
                                \bottomrule
\end{tabular}%
}
\end{table}

Manual inspection indicated that paraphrases generated with \texttt{prompt2} exhibited only mild hallucinations and generally remained on topic. 
Moreover, paraphrases generated with \texttt{prompt2} outperformed paraphrases from other prompts in terms of semantic preservation across all \acp{llm}.

Based on these findings, subsequent experiments adopted the following design choices: 
(1) exclude paraphrases generated with \texttt{prompt0} and \texttt{prompt1}, 
(2) remove all \texttt{</think>}–delimited reasoning traces via post-processing, and 
(3) discard paraphrases shorter than $60\%$ of the reference length.


% \section{Exp. 4: Impact of Syntactic Similarity on \impApprTitle{} Performance}
\label{sec:res_syn_sim_impact}

\textcolor{red}{TODO: still runs on 23.08.2025}

% \subsection{\dataStudent{}}
% \begin{figure}[htbp]
%     \centering
%     \includesvg[width=\textwidth]{images/paraphrasing/experiments/syntactic_similarity_impact/student_essay/student_essays_syn_sim_Syntactic_Similarity_Difference_accuracy.svg}
%     \caption{.}
%     \label{fig:impact_syn_student_diff}
% \end{figure}

% \begin{figure}[htbp]
%     \centering
%     \includesvg[width=\textwidth]{images/paraphrasing/experiments/syntactic_similarity_impact/student_essay/student_essays_syn_sim_Syntactic_Similarity_of_Disputed_&_Candidate_accuracy.svg}
%     \caption{.}
%     \label{fig:impact_syn_student_disp_cand}
% \end{figure}

% \begin{figure}[htbp]
%     \centering
%     \includesvg[width=\textwidth]{images/paraphrasing/experiments/syntactic_similarity_impact/student_essay/student_essays_syn_sim_Syntactic_Similarity_of_Reference_&_Paraphrases_accuracy.svg}
%     \caption{.}
%     \label{fig:impact_syn_student_ref_paraph}
% \end{figure}

% \subsection{\dataBlog{}}
% \begin{figure}[htbp]
%     \centering
%     \includesvg[width=\textwidth]{images/paraphrasing/experiments/syntactic_similarity_impact/blog/blog_syn_sim_Syntactic_Similarity_Difference_accuracy.svg}
%     \caption{.}
%     \label{fig:impact_syn_student_diff_acc}
% \end{figure}

% \begin{figure}[htbp]
%     \centering
%     \includesvg[width=\textwidth]{images/paraphrasing/experiments/syntactic_similarity_impact/blog/blog_syn_sim_Syntactic_Similarity_Difference_precision.svg}
%     \caption{.}
%     \label{fig:impact_syn_student_diff_prec}
% \end{figure}

% \begin{figure}[htbp]
%     \centering
%     \includesvg[width=\textwidth]{images/paraphrasing/experiments/syntactic_similarity_impact/blog/blog_syn_sim_Syntactic_Similarity_Difference_f1.svg}
%     \caption{.}
%     \label{fig:impact_syn_student_diff_f1}
% \end{figure}

% \section{Exp.\ 5: Comparing \acs{av} Methods in Traditional Human-Human Scenario}
\label{sec:results_trad_av}


This experiment investigates how our \ac{llm}-based \imp{} generation performs relative to (a) traditional \imp{} generation within the \impAppr{}, and (b) \acl{sota} \ac{av} methods in a conventional human-human \ac{av} scenario. 
To this end, we construct \textcolor{orange}{10} same-author and \textcolor{orange}{10} different-author pairs from the \dataStudent{} dataset and evaluate the approaches across varying thresholds.

The \mirrorMinds{} approach assigns nearly every input pair to the same-author class, yielding a precision of approximately $0.5$. 
This behaviour is due to the nature of its paraphrases, which consist of single words. 
The discriminator interprets any candidate text as stylistically more similar than \mirrorMinds{}' minimal \imps{}, leading to a bias toward same-author predictions. 
In contrast, naive \ac{llm}-based \imp{} generation achieves the highest $F_1$ scores, although its precision remains between $0.6$ and $0.7$. 
This suggests that the generated \imps{} are still relatively easy, producing \acp{fp} in some cases.

\begin{figure}[h]
\centering
    \includesvg[width=\linewidth]{images/AV_comparison/detection_scenarios/f1/student_essays_Human-Human_threshold_f1s_curves_all_incl_baselines.svg}
  \caption[Traditional \ac{av} $F_1$ scores.]{$F_1$ scores for the same-author class across different thresholds on the \dataStudent{} dataset. 
\ac{llm}-based approaches achieve slightly better effectiveness than baselines across thresholds.}
  \label{fig:human-human_f1}
\end{figure}

$F_1$ scores generally decline for most approaches as the threshold increases. 
\ac{llm}-based methods maintain relatively stable effectiveness across thresholds due to consistently high recall values. 

Among the original baselines~\citep{koppel_determining_2014}, the fixed \imp{} generation performs best, with its text-length-informed variant showing slightly improved results. 
Both approaches exhibit high precision but low recall. 
The length-based \imp{} generation consistently achieves higher recall than the fixed variant. 
Incorporating content similarity into \imp{} selection does not significantly affect effectiveness, as both content-based and fixed \imp{} approaches yield similar outcomes. 
Recall curves across thresholds are provided in the Appendix (cf.~\autoref{sec:app_detection_scenarios}).


% % results
\section{Exp.\ 6: Comparing \acs{av} Methods in \acs{llm} Author Scenarios}
\label{sec:results_llm_av}

This experiment evaluates the performance of the \impAppr{} against established \ac{av} methods on the \dataArtificialStudent{} dataset where each input pair contains at least one \ac{llm}-generated text.  
The baseline \mirrorMinds{} performs comparatively well due to its consistently high recall.  
The recall curves across different thresholds and scenarios can be found in the Appendix in \autoref{sec:app_detection_scenarios}.  
However, \mirrorMinds{}' strength is a misconception due to its oversimplified paraphrases, which consist of single words.  
This triviality biases the discriminator toward predicting same-author, since any candidate text necessarily contains more stylistic information than the single-word \imps{}.  
As a result, this method produces a large number of \acp{fp}.


\paragraph{\ac{llm} detection}

In the \ac{llm} detection setting, the traditional baselines \ac{ppmd}, Unmasking, and the supervised \ac{svm} perform best in terms of precision.  
Nevertheless, it becomes evident that no method achieves a favourable balance between high precision or high recall since none can attain both simultaneously.  

Among the \imp{}-based approaches, random sampling variants (content, text length, and fixed) outperform naive \ac{llm}-based \imps{}, although their absolute performance remains unsatisfactory.  
\mirrorMinds{} and naive \ac{llm} \imps{} are overly simplistic, leading to consistently low precision and inflated recall, since the majority of input pairs are incorrectly classified as same-author.  
Consequently, these approaches are particularly ill-suited for detecting \acp{llm}.

\begin{figure}[h]
  \includesvg[width=\linewidth]{images/AV_comparison/detection_scenarios/precision/student_essays_LLM-Detection_threshold_precisions_curves_all_incl_baselines.svg}
\caption[\ac{llm} detection precision curves.]{Precision curves for the class same-author. 
The candidate text is \ac{llm} generated.
In these scenarios, ground truth True corresponds to the disputed text being \ac{llm} generated irrespective of the responsible model.
}
\label{fig:llm_detection_prec}
\end{figure}

\paragraph{General \ac{llm} \ac{av}}

Contrary to initial intuition, neither \mirrorMinds{} nor naive \ac{llm} approaches are suitable for this task, as their high recall results in excessive \acp{fp} and low precision.  
Although \ac{ppmd} and Unmasking achieve relatively high precision, their recall is consistently low, yielding poor overall $F_1$ performance.  
Both methods predominantly predict different-author, thereby underestimating same-author cases.  

The best-performing baseline is the supervised model, which reaches a maximum precision of approximately $0.7$ for recall $\leq 0.55$.  
However, overall performance remains unsatisfactory across all methods, particularly in terms of precision, which hovers slightly above $0.2$.


\begin{figure}[h]
  \centering
  \includesvg[width=\linewidth]{images/AV_comparison/detection_scenarios/f1/student_essays_LLM-AV_threshold_f1s_curves_all_incl_baselines.svg}
  \caption[\ac{llm} \ac{av} $F_1$ scores.]{$F_1$ scores for the class same-author. 
The candidate text is \ac{llm} generated.
In these scenarios, ground truth True corresponds to the disputed text being generated by the same \ac{llm} as the candidate text.
}
  \label{fig:llm_av_prec}
\end{figure}



\paragraph{\ac{llm}-\ac{llm} \ac{av}}

Overall $F_1$ scores for \ac{av} in the \ac{llm}-\ac{llm} scenario are higher than in the general \ac{llm} \ac{av} setting, suggesting that the presence of human-authored texts complicates attribution.  

At first glance, \mirrorMinds{} appears to perform best. 
However, this impression is misleading.  
Its trivial \imps{} inflate recall without offering genuine discriminative power.  
Interestingly, the Unmasking baseline exhibits similar recall behaviour for thresholds below $0.4$.  

Fixed, content-based, and text-length-driven \imp{} generation methods achieve precision between $0.4$ and $0.7$, yet their $F_1$ scores remain low due to poor recall, i.e.\ most pairs are incorrectly assigned to the different-author class.  
Similarly, \ac{ppmd} and the unsupervised min-max approach yield high precision but equally low recall, again resulting in poor $F_1$ values.  

The overall best-performing approach is naive \ac{llm} generation, which maintains a precision of approximately $0.6$ across thresholds.  
However, with increasing threshold its recall decreases monotonically indicating both \acp{tp} and \acp{fp} are discarded in roughly equal measure, implying that the generated \imps{} fail to meaningfully distinguish between same- and different-author pairs.


  \begin{figure}[h]
    \centering
    \includesvg[width=\linewidth]{images/AV_comparison/detection_scenarios/f1/student_essays_LLM-AV-(only-LLMs)_threshold_f1s_curves_all_incl_baselines.svg}
    \caption[\ac{llm}-\ac{llm} \ac{av} $F_1$ scores.]{$F_1$ scores for the class same-author.
  All texts are \ac{llm} generated.
  The overall $F_1$ scores for \ac{av} in \ac{llm}-\ac{llm} are better than those of the general \ac{llm} \ac{av} scenario.
  }
    \label{fig:llm-llm_f1}
  \end{figure}



\section{Exp.\ 5: Comparing \acs{av} Methods}% in Traditional Human-Human Scenario}
\label{subsec:imp_gen_res}

We evaluate precision–recall values across different thresholds for both the baseline methods and the fixed approach proposed by \citet{koppel_determining_2014}.
Our results indicate that the supervised baseline and our one-step paraphrasing approach (\textcolor{red}{Naive \ac{llm}} in \autoref{fig:sem_syn_blog}) produce identical outcomes.
An examination of the individual values confirms this observation, revealing identical precision–recall pairs where respective thresholds are different.
We attribute this effect to the relatively small sample size of only \textcolor{red}{10} text pairs.

With respect to optimizing the precision–recall trade-off, the supervised baseline and our \ac{llm}-based extension of the \impAppr{} achieve the most favorable performance at a threshold of $0.075$, and $0.32$ respectively.
When precision is prioritized over recall, however, all remaining methods outperform these two approaches, as they achieve higher recall while maintaining the same maximum precision.
Although the definition of an optimal balance between precision and recall is application-dependent, we argue that a recall of $0.4$ at perfect precision is generally less informative than jointly optimizing both metrics.

\begin{figure}[htbp]
    \centering
    \includesvg[width=\textwidth]{images/imposter/our_contribution/roc_prec_recall_curve_r100_top100000_Same_Author_dif_imp_gen.svg}
    \caption[Recall-precision curves for the \dataStudent{}.]{Recall-precision curves for the \textcolor{red}{10} samples of the \dataStudent{}. 
    One-step paraphrasers are denoted \textcolor{red}{Naive \ac{llm}}.
    (B)~indicates the original baseline approaches from~\citep{koppel_determining_2014}.
    }
    \label{fig:sem_syn_blog}
\end{figure}

These findings contrast with our initial expectation that \ac{llm}-based \imp{} generation would achieve higher precision, but indicate that our approach performs best when precision and recall are considered equally important.

    \cleardoublepage 
    \chapter{Discussion}
\label{chap:discussion}
    \cleardoublepage 
    \chapter{Conclusion}
\label{chap:conclusion}

reproduction not entirely possible

syntactic and semantic scores of paraphrases (+ impact?),
paraphrases and chunks,
extraction of metadata,
quality of paraphrases?

does not work for llm detection,
works better in av without any humans,





\section{Future Work}
\subsection{Perplexity}
\label{subsec:perplexity}

% Perplexity is measure to assess how surprised a language model is by a text.
Perplexity $PPL$ can be employed to compute the likelihood of a \ac{lm} generating a text.
A low perplexity indicates that the sequence aligns with model's predictions, 
while a high perplexity indicates that the sequence is unexpected or unlikely according to the model.
Perplexity is computed as follows:
\begin{equation}
    PPL = \exp\left(-\frac{1}{t}\sum_{i=1}^{t}\log P(w_i|w_{<i})\right)
\end{equation}
where $t$ is the number of words or tokens in the sequence, 
$w_i$ is the $i$-th word/ token, and $P(w_i|w_{<i})$ is the probability of the $i$-th word/ token given all previous words/ tokens in the sequence.
The exponent is the cross-entropy loss between the model's predictions and the actual sequence.
The cross-entropy can be refactored to the sum of the entropy of the model's predictions and the KL divergence of the prediction and the data.
While Python libraries such as \texttt{PyTorch} and \texttt{TensorFlow} use the natural logarithm $\log$ for perplexity calculations,
traditional information theory uses the logarithm to base 2. 
Note, that different bases differ only by a constant factor.
For sequences longer than the context window of the model, 
perplexity is computed on the windows of $n$ tokens, where $n$ is the context window size.
% strides: not good
Depending on the tokenizer, perplexity can be computed on the word or sub-word level, 
where sub-word level perplexity is often smaller due to higher likelihoods of smaller character sequences.
Since larger vocabulary lead to lower likelihoods per token, perplexity is generally higher for larger vocabularies.
% disadvantages
Due to the lack of comparability across different tokenizers or models and 
the requirement for access to the model's probabilities $P(w_i|w_{<i})$, which are often not available, 
we decided to refrain from using perplexity for \ac{llm} detection.


experiment llm detection scenario: Add llm pair of same architecture or trained on same data and see if scores influenced (related llms)



Ausblick: LLM Detection
% AV
% LLM detection using generative models
%% AA against LLMs
With the recent advances of \ac{nlg} come new challenges in text authorship.
The new technologies may be misused for fraudulent activities to scam naive or inexperienced users~\citep{uchendu_authorship_2020,bhattacharjee_fighting_2024}.
\citet{uchendu_authorship_2020} identified three authorship tasks essential for fighting fraudulent activities:
(1) Given two texts $t_1$ and $t_2$, determine whether they were produced by the same method (i.e. human author or a specific \ac{nlg} method).
(2) Given a text $t$, determine whether it was human authored or machine generated (Turing Test).
(3) Given a text $t$, find its author among $k+1$ candidates, which consists of one human and $k$ machines.
They compare classical \ac{ml} models, neural models and state-of-the-art \ac{aa} models as classifiers 
for these single- (Problem 1 and 2) and multi-class (Problem 3) tasks.
Their findings include, that as of 2020, most \ac{nlg} methods were distinguishable from human authors, 
but some \acp{llm} proved difficult to detect.
%%% compared to our work
In the following, we consider (1) \ac{av}, (2) classical \ac{llm} detection, and (3) closed-set \ac{aa}.
Our approach differs from the work of \citet{uchendu_authorship_2020} in that our candidates (i.e. \imps{}) do not include a human author (3), 
but only \acp{llm}.
Moreover, we use different classifiers originally designed for \ac{av}, rather than \ac{aa}.

There are different categories of \ac{llm} detectors-
Metric-based detectors classify based on a threshold and the inferred log-probability from the generator \ac{llm}.
Examples inlcude GLTR, Range, LogRank and DetectGPT.
Fine-tuned detectors are pretrained \acp{lm} in a supervised scenario.
Examples include the OpenAI Detector.
Watermark-based detectors add algorithmically detectable signatures into the text during generation~\citep{wang_stumbling_2024}.

%% LLM (gpt-3.5, GPT-4) as detector
\citet{bhattacharjee_fighting_2024} evaluate using an \ac{llm} as classifier for \ac{llm} detection.
They use \ac{gpt}-3.5 and \ac{gpt}-4 to classify texts as human or machine generated.
They find that \ac{gpt}-3.5 performs better when being fed simple instructions, rather than constrained prompts.
They find that \ac{gpt}-4 predicts almost exclusively \ac{ai} generated texts, 
while \ac{gpt}-3.5 predictions are more reliable (especially for actually human authored texts).
%%% compared to our work
Our work differs from theirs in that we use \acp{llm} to generate \imp{} texts specific to the candidate text, 
rather than using the publicly available dataset TuringBench with previously generated texts.

%% DetectGPT: Perturb (Mask), score, compare (unsupervised)
\citet{mitchell_detectgpt_2023} propose DetectGPT, a method that is threefold:
(1) They perturb the input text by (1.1) masking out random 2-word spans until 15 \% of the text is masked. 
Masked spans are replaced (1.2) with words from an off-the-shelf (i.e. not finetuned to target domain) \ac{llm} (e.g. T5-3B). 
These perturbations are semantically similar paraphrases of the original text.
(2) They score (in terms of log probability) each perturbed text using a scoring \ac{llm} 
(ideally their candidate \ac{llm}, but it works also with any \ac{llm}, though scores deteriorate). 
(3) The difference of the score of the original text and the average score of the perturbed texts is denoted perturbation discrepancy $d$. 
(4) Normalize $d$ by the standard deviation of the scores of the perturbed texts.
(5) Based on a threshold $\epsilon=0.1$, classify the original text as human authored or machine generated 
(formally Local Perturbation Discrepancy Gap hypothesis).
If $d$ is positive, the original text is likely machine generated.
If $d$ is near zero, i.e. $d < \epsilon$, the original text is likely human authored.
\citet{mitchell_detectgpt_2023} motivate their method by the observation that generated texts tend to occupy 
negative curvature regions of the model's log probability function (i.e. they lie on the local maximum of the manifold).
When the text is machine generated, it lies on a local maximum, 
and perturbing it will lead to lower log probabilities of perturbed texts.
When the text is human authored, it does not lie on a local maximum to begin with, 
rendering log probabilities of perturbed texts similar either bigger or smaller than the original text.
Averaging the log probabilities of perturbed human texts leads to a value that is 
close to the original text's log probability (i.e. a perturbation discrepancy $d$ near zero).
Even though, DetectGPT works best when the source (i.e. generating) \ac{llm} and the scoring \ac{llm} are the same 
(requires white-box access to the \ac{llm}), 
it works also with different \acp{llm} as surrogate for the source model when scoring (in a black-box case).
%%% compared to our work
We can not supply a white box setting, because we do know the source \ac{llm} that generated the \imp{} texts.
%%%% \imps{} and perturbations
However, this approach is similar to our approach, because perturbing texts can be seen as a 
form of \imp{} generation (especially as we use paraphrases). 
%%%% sample from the source model
Both approaches try to sample from the probability distribution of the source model either 
by using \imps{} (via prompting an \ac{llm}) or by perturbing the original text (using an \ac{llm}).
%%%% input
While the \impAppr{} is an \ac{av} task (i.e. input is a disputed and a candidate text), 
DetectGPT receives a disputed text and a candidate \ac{llm} as input.
%%%% similarity measure
While we use a similarity measure on traditional n-gram frequency vectors, 
\citet{mitchell_detectgpt_2023} require a scoring \ac{llm} to compute the perturbation discrepancy $d$.
Hence, our approach is easier in terms of computational resources and requirements.
Research building on DetectGPt finds that the whitebox approach is vulnerable to unknown models, especiall GPT-3.5-Turbo~\citep{Wu_ODD_challenges_2025}.

%% LLM rewrite LLM texts less than human texts (no AA, but edit distance hypothesis)
RAIDAR~\citep{mao_raidar_2024} builds upon the invariance property of \acp{llm}, 
which states that prompting an \ac{llm} to rewrite a machine generated text will introduce little change.
They motivate this by the observation that (different) autoregressive models produce similar patterns and thus, 
consider texts generated by (different) \acp{llm} as high quality that do not require rewriting.
Change is measured by the edit distance between the original text and the rewritten text. 
\citet{mao_raidar_2024} propose using an edit distance based on the Levenshtein distance or \ac{bow} representations.
RAIDAR operates on character level rather than using deep neural network features, and it does not require the original generating model for classification. 
RAIDAR fails to detect \ac{llm} generated texts in out-of-distribution scenarios (i.e. different domains than training), 
or when \ac{llm} were explicitly instructed to produce text prone to heavy \ac{llm} modification when being asked to rewrite the text \citep{li_learning_2025}.
Based on RAIDAR (\citep{mao_raidar_2024}), \citet{li_learning_2025} propose fine-tuning an \ac{llm} to rewrite human authored texts more than machine generated text.
Classification is carried out by comparing the edit distance of the original text and the rewritten text to a threshold.
\citet{li_learning_2025} admit that their approach is slow in inference time, 
since a candidate text has to be rewritten multiple times (about 200 different prompts) to obtain a reliable score.
\citet{mao_raidar_2024} find that the quality of perturbation based models (i.e. rewriting) for \ac{llm} detection correlates with the perturbation model size.
\citet{mitchell_detectgpt_2023} find a negative correlation (\textcolor{red}{TODO: chapter 2 vorletzter Absatz}) between the size of the perturbation model and the performance of DetectGPT.
%%% compared to our work
%%%% generation of texts during inference
Both approaches are similar to our work in that they use \acp{llm} to generate texts during inference.
We do not fine-tune an \ac{llm} for paraphrasing but use off-the-shelf models (like RAIDAR).
%%%% similarity measure
All these approaches compute the similarity of the original text and the generated text.
However, we do not use edit distance (i.e. Levenshtein distance) as similarity measure.
%%%% limitations
This approach is unable to detect which \ac{llm} generated the text.

%% LLMDet: Proxy to perplexity (problem: requires access to the LLM to build the dictionary)
Perplexity is a reliable statistical metric for attributing texts to \acp{llm}~\citep{zhang_llmdet_2023}.
Unfortunately, perplexity requires access to \acp{llm}' parameters (i.e., white-box detection).
\citet{wu_llmdet_2023} propose LLMDet, a method that uses a proxy to perplexity, 
where a dictionary of frequent n-gram (frequent among $n$ randomly prompted generated texts per \ac{llm}) 
next token probabilities is pre-computed (i.e. requiring access to the \ac{llm}), 
and is subsequently used during inference to approximate perplexity by replacing $x_{<i}$ in $p(x_i | x_{<i})$ with an n-gram.
Since the construction of the dictionary requires access to the \ac{llm}, LLMDet requires contribution of the closed-source model owners.
The disputed text is tokenized and the proxy perplexity is calculated for each model and thus, constructing a proxy perplexity vector.
This vector is input to a trained classifier.
%%% compared to our work
Proxy perplexity could be used as a baseline for our approach, though it requires access to the \ac{llm} and is thus not applicable in our case.

%% Mirror Minds: extract query, genrate two paraphrases, compare & classify via threshold (very similar to our work)
\citet{baradia_mirror_2025} propose (1) extracting a query from the disputed text, which captures the essence of the text, 
(2) generating two paraphrases of the original text using the query as input prompt to two \acp{llm}, 
and (3) comparing the paraphrases to the original text via the BLEU and the METEOR score.
Both score capture syntactic similarity, even though \citet{baradia_mirror_2025} argue they also capture semantic similarity.
They use the maximum across the two models per similarity measure as a final score pair.
Classification of the resemblance to \ac{ai} generated content requires a threshold.
%%% compared to our work
%%%% same approach
This approach is similar to our approach in that it uses \acp{llm} to generate paraphrases of the original text.
Moreover, it compares the original text to the generated paraphrases as in a \ac{aa} problem. % rather AI detection?????
%%%% similarity measure
We do not use BLEU or METEOR as similarity measure, nor do we compare directly on paraphrase-level (i.e. BLEU calculates n-gram overlap) 
but construct our own frequency based n-gram vectors input vector similarity metrics.
%%%% they discard information, solve another problem
However, this approach discards the information which \ac{llm} produced the most similar paraphrase. 
While our goal is to solve an \ac{aa} problem (i.e. multiclass classification), 
\citet{baradia_mirror_2025} solve a binary classification problem (i.e. human vs. \ac{ai} generated text).

    % \cleardoublepage 
    % \section*{Acknowledgements}

The authors gratefully acknowledge the computing time granted by the KISSKI project. 
The calculations for this research were conducted with computing resources under the project \textcolor{red}{<ID of your project>}.

We also thank J.~W.~Pennebaker for granting access to the original data used by \citet{koppel_determining_2014}, as well as Moshe Koppel and Mr.~Winter for their valuable and forthcoming communication, which informed the development of this work.
    \pagenumbering{roman}
    \addtocounter{page}{6} % Dies ist die Anzahl der Seiten vor der Einleitung, muss möglicherweise angepasst werden, wenn das Inhaltsverzeichnis mehrere Seiten umfasst.
    \listoffigures % optional, usually not needed
    \listoftables % optional, usually not needed
    \appendix
\chapter{First Appendix}
\label{ch:appendix}


\section{Extractor prompts used for Two-Step Paraphrasing}
\label{app:extractor_prompts}
% bullet point
\begin{quote}
    \textit{Summarize the text above in five to six short bullet points. Respond ONLY with a JSON object in the following format: $\{$"bullet\_points":"<list of bullet points>","tone":"<tone>","time\_period":<time\_period>,"language\_register":<register>,"target\_audience":"<target\_audience>","genre":"<genre>"$\}$. Do not use direct quotes.}
\end{quote}

% task
\begin{quote}
    \textit{Act as the author of the text above. From that perspective, infer your role or identity, the topic being addressed, and the purpose or instruction behind writing the text. Combine these elements into a concise task prompt that you would give to an LLM to reproduce the text. Respond ONLY with a JSON object in the following format: \{"task":"<task>","tone":"<tone>","time\_period":<time\_period>,"language\_register":<register>,"target\_audience":"<target\_audience>","genre":"<genre>"$\}$. Do not use direct quotes.}
\end{quote}

% topic
\begin{quote}
    \textit{Extract the topic, tone, time period, register, target audience, and genre from the text above. Respond ONLY with a JSON object in the following format: \{"topic":"<topic>","tone":"<tone>","time\_period":<time\_period>,"language\_register":<register>,"target\_audience":"<target\_audience>","genre":"<genre>"$\}$. Do not use direct quotes.}
\end{quote}

% title
\begin{quote}
    \textit{Find a concise title for the text, extract the tone, time period, register, target audience and genre from the text above. Respond ONLY with a JSON object in the following format: \{"title":"<title>","tone":"<tone>","time\_period":<time\_period>,"language\_register":<register>,"target\_audience":"<target\_audience>","genre":"<genre>"$\}$. Do not use direct quotes.}
\end{quote}

\section{Generator prompts used for Two-Step Paraphrasing}
\label{app:generator_prompts}
% bullet point
\begin{minted}{python}
prompt = "Write a text which covers the following items:\n" 
    + "\n".join(f"- {bp}" for bp in bullet_points)
\end{minted}

% task
\begin{minted}{python}
generator_prompt = "Write a text of about {l} words with a {tone} tone, a {genre} genre, in the {register} register for the target audience of {target_audience} and in the {time_period} time period, covering the following task:\n{task}. Do not use asterisks. Only output the text without any additional commentary.".format(
            l=len(text.split()),
            tone=tone,
            genre=genre,
            task=task,
            time_period=time_period,
            register=register,
            target_audience=target_audience,
        )
\end{minted}

% topic
\begin{minted}{python}
generator_prompt = "Write a text of about {l} words with a {topic} topic, {tone} tone, a {genre} genre, in the {register} register for the target audience of {target_audience} and in the {time_period} time period. Do not use asterisks. Only output the text without any additional commentary.".format(
            l=len(text.split()),
            tone=tone,
            genre=genre,
            topic=topic,
            time_period=time_period,
            register=register,
            target_audience=target_audience,
        )
\end{minted}

% title
\begin{minted}{python}
generator_prompt = "Write a text of about {l} words with a {title} title, {tone} tone, a {genre} genre, in the {register} register for the target audience of {target_audience} and in the {time_period} time period. Do not use asterisks. Only output the text without any additional commentary.".format(
            l=len(text.split()),
            tone=tone,
            genre=genre,
            title=title,
            time_period=time_period,
            register=register,
            target_audience=target_audience,
        )
\end{minted}

\section{Preprocessing Regular Expressions}
\label{app:regex_preproc}

\textcolor{red}{TODO}

\section{Exp. 2: Comparison of Different Paraphrasers}
\label{sec:app_paraphrases}

\begin{figure}[H]
    \centering
    \includesvg[width=0.9\textwidth]{images/paraphrasing/experiments/radar/Blog_paraphrasing_metrics_grouped_by_Paraphraser_radar_chart.svg}
    \caption{Radar chart of syntactic and semantic paraphrase evaluation measures different paraphrasers on the \dataBlog{} dataset.}
    \label{fig:radar_blog}
\end{figure}


\begin{figure}[H]
    \centering
    \includesvg[width=0.9\textwidth]{images/paraphrasing/experiments/sem_syn_scatter/Gutenberg_sem_syn_scatter_grouped_by_Paraphraser.svg}
    \caption{Average semantic and syntactic similarity for different paraphraser on the \dataGutenberg{}.}
    \label{fig:sem_syn_gutenberg}
\end{figure}

\begin{figure}[H]
    \centering
    \includesvg[width=0.9\textwidth]{images/paraphrasing/experiments/radar/Gutenberg_paraphrasing_metrics_grouped_by_Paraphraser_radar_chart.svg}
    \caption{Radar chart of syntactic and semantic paraphrase evaluation measures different paraphrasers on the \dataGutenberg{} dataset.}
    \label{fig:radar_gutenberg}
\end{figure}


% appendix
\section{Exp. 6: Comparing \ac{av} Methods in different scenarios}
\label{sec:app_detection_scenarios}

% \begin{figure}[htbp]
% \centering
%     \includesvg[width=\linewidth]{}
%   \caption{Accuracy curves for the class same-author across different threshold on artificially augmented \dataStudent{} dataset. 
%   }
%   \label{fig:human-human_acc}
% \end{figure}


\begin{figure}[htbp]
  \centering
  \begin{subfigure}[b]{0.52\textwidth}
    \centering
    \includesvg[width=\linewidth]{images/AV_comparison/detection_scenarios/recall/student_essays_Human-Human_threshold_recalls_curves_all_incl_baselines.svg}
    \caption{Human-Human}
    \label{fig:detec_scen_human-human_recall}
  \end{subfigure}
  \hfill
  \begin{subfigure}[b]{0.52\textwidth}
    \centering
    \includesvg[width=\linewidth]{images/AV_comparison/detection_scenarios/recall/student_essays_Human-LLM_threshold_recalls_curves_all_incl_baselines.svg}
    \caption{Human-\ac{llm}}
    \label{fig:detec_scen_human-llm_recall}
  \end{subfigure}
  \hfill
  \begin{subfigure}[b]{0.52\textwidth}
    \centering
    \includesvg[width=\linewidth]{images/AV_comparison/detection_scenarios/recall/student_essays_LLM-LLM_same_threshold_recalls_curves_all_incl_baselines.svg}
    \caption{\ac{llm}-\ac{llm}}
    \label{fig:detec_scen_llm-llm_recall}
    \end{subfigure}
  \caption{Recall curves for the class same-author across different threshold on artificially augmented \dataStudent{} dataset.}
  \label{fig:detec_scen_recall}
\end{figure}


\section{Libraries used}
\label{app:libraries}

\textcolor{red}{TODO}
e.g. sentence tokens: \texttt{nltk}'s \texttt{sent\_tokenize}
    

    % Die nächsten zwei Zeilen sind optional, sie sorgen dafür dass alles nach dem Inhalt wieder mit römischen Zahlen nummeriert wird.
    
    % \nocite{*} % uncomment to print all references
    \bibliographystyle{plainnat} % requires package natbib. An alternative is apalike
    \bibliography{
        bibliography/author_identification
    }

    % \chapter*{Eidesstattliche Erklärung}

\markboth{Eidesstattliche Erklärung}{Eidesstattliche Erklärung}

Hiermit erkläre ich, \thesisauthorname, dass ich die vorliegende Arbeit mit dem Titel "\thesistitle" selbstständig 
und nur mit den nach der Prüfungsordnung der Universität Kassel zulässigen Hilfsmitteln angefertigt habe.
Die verwendete Literatur ist im Literaturverzeichnis angegeben.
Wörtlich oder sinngemäß übernommene Inhalte habe ich als solche kenntlich gemacht.

\vspace{1cm}

Kassel, \today

\begin{flushright}
  \underline{\hspace{7cm}} \\
  \thesisauthorname
\end{flushright}

\end{document}
