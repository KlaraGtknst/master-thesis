\documentclass[
    %draft, % Mit % kommentieren, um Bilder sichtbar zu machen und Links zu aktivieren
    english,
    master,
    ks, % Kassel University
    % pdftex,
    % a4paper,
    % twoside,
    % parskip=half,
    % numbers=noenddot,
    % listof=totoc,
    bibliography=totoc,
    % hyperfootnotes=false,
    % english,
    % openright
]{webisthesis}

% FIXME: this doesn't work: https://blog.rtwilson.com/how-to-add-simple-new-commands-to-latex-to-help-with-writing-papers/ (11.04.2025)
\newcommand{\todo}[1] {\textbf{\textcolor{red}{TODO:}} #1}
% \newcommand{\thesistype}{M A S T E R \space \space T H E S I S}
\newcommand{\thesistypedesc}{Department of Electrical Engineering and Computer Science \\
    University of Kassel}
\newcommand{\thesisauthorname}{Klara Maximiliane Gutekunst}
\newcommand{\thesisauthorhomestreet}{\todo}
\newcommand{\thesisauthorhometown}{34119 Kassel}
\newcommand{\thesisauthormatrikelnumber}{35677772}
\newcommand{\thesisauthoremail}{klara.gutekunst@uni-kassel.de}
\newcommand{\thesisdepartment}{Chair Deep Semantic Learning}
\newcommand{\thesisfirstreviewer}{Prof.\ Dr.\ Martin Potthast}
\newcommand{\thesissecondreviewer}{Prof.\ Dr.\ Gerd Stumme}
\newcommand{\thesissupervisor}{\todo}
\newcommand{\thesisdate}{\today}

% \ThesisSetTitle{Towards LLM-Resilient Authorship Verification: Revisiting the Impostors Method}
\ThesisSetTitle{Leveraging \acs{llm}-Generated Impostors for the \impApprTitle{}}
\ThesisSetKeywords{These, are, my, Keywords} % only for PDF meta attributes
\ThesisSetLocation{\thesisauthorhometown{}} 

\ThesisSetAuthor{\thesisauthorname{}}
\ThesisSetStudentNumber{\thesisauthormatrikelnumber{}}
\ThesisSetDateOfBirth{25}{09}{2001}
\ThesisSetPlaceOfBirth{Brilon, Germany}

% Supervisors should usually be Professors from the candidate's university. A second supervisor is not always needed. 
\ThesisSetSupervisors{\thesisfirstreviewer{},\thesissecondreviewer{}}

\ThesisSetSubmissionDate{26}{09}{2025}


% content-specific commands
\newcommand{\tira}{TIRA}
\newcommand{\pextractor}{Paraphrase Extractor}
\newcommand{\pgenerator}{Paraphrase Generator}
\newcommand{\dataGutenberg}{Gutenberg}
\newcommand{\dataBlog}{Blog Posts}
\newcommand{\dataPan}{PAN20}
\newcommand{\dataStudent}{Student Essay}
\newcommand{\dataArtificialStudent}{Artificial Student Essay}
\newcommand{\dataCustom}{Custom CNN News}
\newcommand{\bluert}{\href{https://github.com/google-research/bleurt}{BLUERT}}

\usepackage{booktabs}
%    For tables ``looking the right way''.
% \usepackage{tabularx}
%    Enables tables with columns that automatically fill the page width.
%
% \usepackage[ruled,algochapter]{algorithm2e}
%    A package for pseudo code algorithms.
%
% \usepackage{amsmath}
%    For tabular-style formatting of mathematical environments.
%

\usepackage{fontawesome}
%    For lots of awesome glyphs: https://mirror.physik.tu-berlin.de/pub/CTAN/fonts/fontawesome/doc/fontawesome.pdf
% \usepackage[printonlyused]{acronym}
\pagenumbering{roman} % lowercase roman numerals
% Select input encoding, usually utf8 is the best choice, on windows, \usepackage[latin1]{inputenc} maybe required
\usepackage[utf8]{inputenc}
\usepackage[T1]{fontenc}
\usepackage[english]{babel}
\usepackage{fvextra}
\usepackage{csquotes}
\usepackage{xcolor}

\MakeOuterQuote{"} % Damit ist es möglich, " " zu verwenden ohne Umlaut zu erzeugen
\defaulthyphenchar=127 % Dadurch werden auch Wörter mit Bindestrich getrennt, die schon Bindestriche enthalten.

% geometry
% \usepackage[bindingoffset=1cm, left=2.5cm, right=2.5cm, top=2.5cm, bottom=2.5cm]{geometry}
\usepackage{rotating}
% Headline
\usepackage{fancyhdr}
\pagestyle{fancy}
\renewcommand{\chaptermark}[1]{\markboth{\thechapter\ #1}{}}
\lhead{\leftmark} \rhead{\thepage}
\cfoot{}
\fancypagestyle{plain}{}

% \RedeclareSectionCommand[beforeskip=1.5cm,afterskip=1cm]{chapter}

% Colors
\usepackage{color}
\usepackage{colortbl}

% Tables
\usepackage{tabularx}
\usepackage{multirow}
\setlength{\tabcolsep}{4pt}

% Drawing graphs etc.
\usepackage{pgf}
\usepackage{tikz}
\usetikzlibrary{arrows,automata,decorations.pathmorphing,arrows.meta, positioning,shapes.misc, shapes.symbols,positioning, shapes.geometric,calc}

\tikzset{
  block/.style = {draw, thick, rounded corners, minimum width=2.5cm, minimum height=1cm, align=center},
  smallblock/.style = {draw, thick, minimum width=0.8cm, minimum height=0.8cm},
  arrow/.style = {thick, -{Stealth[length=3mm]}},
  person/.style = {draw, thick, circle, minimum size=0.7cm},
  dataset/.style = {cylinder, draw, shape border rotate=90, aspect=0.25,
                    minimum height=2.2cm, minimum width=1cm, align=center,
                    cylinder uses custom fill, cylinder body fill=white, cylinder end fill=white},
}


% Footnotes
\usepackage{footmisc}
\usepackage{xspace}
\newcommand{\sic}{[\acs{sic}]\xspace}

% math
\usepackage{amsmath}
\usepackage{amssymb}
\usepackage{pifont}

\newcommand{\cmark}{\ding{51}} % ✓
\newcommand{\xmark}{\ding{55}} % ✗
\usepackage{wasysym}

\usepackage{siunitx}

% lists
\usepackage{paralist}

% Figures
\usepackage{graphicx, wrapfig}
\usepackage{placeins}

% Hyperlinks
\usepackage[hyphens]{url}
\usepackage{hyperref}
\hypersetup{colorlinks, citecolor=black, linkcolor=black, urlcolor=black}
% \providecommand*{\listingautorefname}{Listing}  % TODO: doesn't solve problem of missing autoref

% Minted
\usepackage[chapter]{minted}
\usepackage{algorithm}
\usepackage[noend]{algpseudocode}
%\usemintedstyle{xcode}
\setminted{frame=single,tabsize=2,linenos,autogobble}

\newmintinline[code]{text}{breaklines}

\newminted[mdcodeblock]{md}{autogobble,frame=none,linenos=false,breaklines}

% definitions
\newtheorem{definition}{Definition}


% list of abbreviations
\usepackage[printonlyused]{acronym}

% Set line pitch
\usepackage{setspace}
\onehalfspacing              % anderthalbzeilig (oder auch \doublespace)

%fancyBox
%\usepackage{fancybox}

% Layout corrections (Schusterjungen)
\clubpenalty = 10000
% Layout corrections (Hurenkinder)
\widowpenalty = 10000
\displaywidowpenalty = 10000

\usepackage[super]{nth}

% Figures
\usepackage{caption}
\usepackage[hypcap=true,labelformat=simple]{subcaption}
\renewcommand{\thesubfigure}{(\alph{subfigure})}
\usepackage[inkscapelatex=false]{svg}

% Tables
\usepackage{booktabs}
\usepackage{graphicx}
%\usepackage[table,xcdraw]{xcolor}

% enumerate
\usepackage{enumitem}
\newlist{questions}{enumerate}{2}
\setlist[questions,1]{label=RQ\arabic*.,ref=RQ\arabic*}
\setlist{nolistsep}

% Bibliography
\usepackage[sort&compress]{natbib}
%   Allows citing in different ways (e.g., only the authors if you use the
%   citation again within a short time).
%

% Frequently used column types
\newcolumntype{C}[1]{>{\centering\arraybackslash}p{#1}} % centering column type with fixed width
\newcolumntype{R}[1]{>{\raggedleft\arraybackslash}p{#1}} % right aligned column type with fixed width
\newcolumntype{L}[1]{>{\raggedright\arraybackslash}p{#1}} % left aligned column type with fixed width

% Shortcuts for referencing floats:
\newcommand{\fig}[1]{\figurename~\ref{#1}} %shortcut for a figure reference
\newcommand{\tab}[1]{Table~\ref{#1}} %shortcut for a table reference
\newcommand{\eq}[1]{(\ref{#1})} %shortcut for an equation reference
\newcommand{\lst}[1]{Listing~\ref{#1}} %shortcut for a listing reference
\newcommand{\sect}[1]{Section~\ref{#1}} %shortcut for a Section reference
\addto\extrasenglish{%
  \renewcommand{\sectionautorefname}{Section}%
  \renewcommand{\subsectionautorefname}{Subsection}%
  \renewcommand{\chapterautorefname}{Chapter}%
  % \renewcommand{\algorithmautorefname}{Algorithm}%
  % \renewcommand{\listingautorefname}{Listing}%
}

% Shortcut for terms
\newcommand{\localMaschineStats}{Apple M2 Pro MNW83D/A with 16 \ac{gb} RAM and 12 cores}
\newcommand{\slurm}{Slurm}
\newcommand{\impAppr}{Impostors method}  % can or can not be capitalized, plural acc. to original paper
\newcommand{\impApprTitle}{Impostors Method}
\newcommand{\imp}{impostor} % not capitalized
\newcommand{\Imp}{Impostor}
\newcommand{\imps}{impostors} % not capitalized
\newcommand{\ai}{Authorship Identification}
\newcommand{\unmasking}{Unmasking}
\newcommand{\mirrorMinds}{Mirror Minds}




\begin{document}
\begin{frontmatter}
    % \pagenumbering{roman} not used at webis
    % \include{titlepage} not used at webis
    \chapter*{Abstract}
\markboth{Abstract}{Abstract}

\Acl{av} seeks to determine whether two texts share the same author. 
Existing approaches often perform poorly in cross-domain scenarios. 
The influential \impAppr{} of \citet{koppel_determining_2014} introduced hard negative sampling to sharpen predictions, yet it struggles to fix confounding variables during inference. 
This thesis investigates whether \aclp{llm} can help address these limitations. 
Specifically, we employ \acs{llm}-generated paraphrases as hard negatives. 
Since these paraphrases aim to preserve the confounding variables of the original text, they mitigate domain-related biases during inference. 
Moreover, by constructing a tailored case for each input text pair, the approach eliminates \acl{ood} issues, ensuring that comparisons remain within the distribution defined by the pair itself.
We evaluate the approach in the traditional human-authored \acl{av} setting and find that the \acs{llm}-based extension outperforms the original baselines. 
At the same time, our results reveal the practical and conceptual challenges of integrating \acsp{llm} into \acl{av}, including issues of reliability, hallucination, and control over paraphrase quality.



% The original \impAppr{} compares a disputed text to a candidate text and hard negatives, considering disputed and candidate text to share an author if the candidate is consistently the most similar across random feature projections. 
    % \chapter*{Zusammenfassung}
\markboth{Zusammenfassung}{Zusammenfassung}

 not used at webis

    \tableofcontents
    % optional: list of symbols/notation (e.g., using the nomencl package) but usually not needed
    \chapter*{List of Abbreviations}
\markboth{List of Abbreviations}{List of Abbreviations}
\addcontentsline{toc}{chapter}{List of Abbreviations}

\begin{acronym}[XXXXXXXXX]
    \acro{aa}[AA]{Authorship Attribution}
    \acro{ai}[AI]{Artificial Intelligence}
    \acro{ap}[AP]{Author Profile}
    \acro{auc}[AUC]{Area Under the Curve}
    \acro{av}[AV]{Authorship Verification}
    \acro{bert}[BERT]{Bidirectional Encoder Representations from Transformers}
    \acro{bleu}[BLEU]{Bilingual Evaluation Understudy}
    \acro{bow}[BoW]{Bag-of-Words}
    \acro{clef}[CLEF]{Conference and Labs of the Evaluation Forum}
    \acro{clm}[CLM]{Causal Language Model}
    \acro{cng}[CNG]{common n-grams}
    \acro{eloquent}[ELOQUENT]{Evaluation of generative Language Model Quality and Usefulness}
    \acro{f1}[F1]{F1-Score}
    \acro{fn}[FN]{False Negative}
    \acro{fpr}[FPR]{False Positive Rate}
    \acro{fp}[FP]{False Positive}
    \acro{gai}[GenAI]{Generative Artificial Intelligence}
    \acro{glove}[GloVe]{Global Vectors for Word Representation}
    \acro{gpt}[GPT]{Generative Pre-trained Transformer}
    \acro{gwdg}[GWDG]{Gesellschaft für wissenschaftliche Datenverarbeitung mbH Göttingen}
    \acro{id}[ID]{in-domain}
    \acro{ir}[IR]{Information Retrieval}
    \acro{jsd}[JS divergence]{Jensen-Shannon divergence}
    \acro{kld}[KL divergence]{Kullback-Leibler divergence}
    \acro{lcs}[LCS]{Longest Common Subsequence}
    \acro{llm}[LLM]{Large Language Model}
    \acro{lm}[LM]{Language Model}
    \acro{meteor}[METEOR]{Metric for Evaluation of Translation with Explicit ORdering}
    \acro{mlm}[MLM]{Masked Language Model}
    \acro{ml}[ML]{Machine Learning}
    \acro{nb}[NB]{Naive Bayes}
    \acro{nlg}[NLG]{Natural Language Generation}
    \acro{nlp}[NLP]{Natural Language Processing}
    \acro{nltk}[NLTK]{Natural Language Toolkit}
    \acro{nn}[NN]{Neural Network}
    \acro{ood}[OOD]{out-of-distribution}
    \acro{pan}[PAN]{Plagiarism Analysis and Authorship Mining} % TODO: find out if correct
    \acro{pca}[PCA]{Principal Component Analysis}
    \acro{pg}[PG]{Paraphrase Generation}
    \acro{pi}[PI]{Paraphrase Identification}
    \acro{pos}[POS]{Part-of-Speech}
    \acro{ppm}[PPM]{Partial Matching}
    \acro{ppmd}[PPMd]{Prediction by Partial Matching}
    \acro{rar}[RAR]{Roshal ARchive}
    \acro{rl}[RL]{Reinforcement learning}
    \acro{roc-auc}[AUROC]{Area Under the Receiver Operating Characteristic Curve}
    \acro{roc}[ROC]{Receiver Operating Characteristic}
    \acro{rouge}[ROUGE]{Recall-Oriented Understudy for Gisting Evaluation}
    \acro{saia}[SAIA]{Scalable Artificial Intelligence Accelerator}
    \acro{sbert}[SBERT]{Sentence-BERT}
    \acro{sota}[SOTA]{state-of-the-art}
    \acro{spi}[SPI]{Simplified Profile Intersection}
    \acro{svc}[SVC]{Support Vector Classifier}
    \acro{svm}[SVM]{Support Vector Machine}
    \acro{t5}[T5]{Text-to-Text Transfer Transformer}
    \acro{tfidf}[TF-IDF]{Term Frequency-Inverse Document Frequency}
    \acro{tn}[TN]{True Negative}
    \acro{tpr}[TPR]{True Positive Rate}
    \acro{tp}[TP]{True Positive}
    \acro{vae}[VAE]{Variational Autoencoder}
    \acro{wmd}[WMD]{Word Mover Distance}
    \acro{wms}[WMS]{Word Mover-based Similarity score}
    % \acro{}[]{}
\end{acronym}

    \end{frontmatter}


    

    % \pagebreak
    \pagenumbering{arabic}

    \chapter{Introduction}
\label{chap:introduction}



% motivation
\ac{av} denotes the task of determining whether two texts were written by the same author. 
It forms the foundation of broader authorship-related problems such as authorship attribution. 
Historically, \ac{av} has been applied to literary disputes, but contemporary concerns have shifted. 
In an era where large amounts of text can be copied, paraphrased, or fabricated with ease, \ac{av} has gained renewed importance in practical contexts such as plagiarism detection in digital communication.

The emergence of \acp{llm} adds an additional layer of complexity. 
While these models are widely embraced for beneficial applications such as summarization, code generation, and customer service automation, their ability to convincingly imitate human writing creates new risks. 
\acp{llm} can be exploited to generate misinformation, fabricate academic submissions, or impersonate individuals, thereby undermining trust in digital communication. 
Since \acp{llm} can be conceptualized as authors their detection naturally falls within the scope of \ac{av} rather than requiring a separate methodological framework. 
Thus, instead of treating \ac{llm} detection as an isolated task, it is more consistent to frame it as a specialized case of \ac{av}.

% specificity rather than generality
Despite significant advances in \ac{av}, many existing approaches pursue generalized solutions, training a model once and applying it across domains. 
Evidence shows that such models struggle in \ac{ood} settings, where topic or genre diverges from the training data. 
This shortcoming motivates a shift toward scenario-specific solutions, where models are trained anew for narrowly defined cases. 
Such single-case approaches enable more precise control over contextual factors and place greater emphasis on stylistic idiosyncrasies rather than domain-level variation.

% AV
Among the techniques developed for \ac{av}, the \impAppr{} by \citet{koppel_determining_2014} represents a particularly influential approach. 
It introduces the idea of generating \imp{} texts, i.e. hard negatives used to sharpen the discrimination between genuine and false authorship matches. 
However, the method's effectiveness is limited by the quality and contextual adequacy of these \imp{} texts. 
Previous work did not fully address how to construct challenging \imps{} via controlled contextual variables.

This thesis extends the \impAppr{} by leveraging \acl{sota} \acp{llm} to generate hard negative examples in a controlled scenario. 
By integrating \ac{llm}-based \imp{} generation, we bridge the gap between traditional \ac{av} and modern challenges involving \acp{llm}. 
The contribution lies not in proposing a new detection paradigm, but in enhancing an established verification method by improving its treatment of contextual factors and strengthening its ability to discriminate between genuine and artificial authorship.


% \section*{Research Questions}
% \label{sec:research_questions}
To guide this objective, we formulate the following research questions:
\begin{questions}
    \item \textbf{How can we instruct a \ac{llm} to paraphrase the text of a candidate author such that it captures the \ac{llm}'s stylistic properties?} \label{enum:rq1} \hfill \\
    The goal is to create hard negatives for the Impostor method by controlling contextual factors.
    By controlling genre, topic and other factors, similarity measures primarily focus on differences in authorial style rather than the impact of content on style.
    We obtain this controlled environment by utilizing \acp{llm} to paraphrase the original text.
    There are different approaches to paraphrasing text using \acp{llm}.
    They include (a) directly asking the \ac{llm} to paraphrase the text, 
    (b) first extracting specific information from the original text and subsequently generating a paraphrase based on the information.
    This thesis compares both approaches on \dataStudent{}, \dataBlog{}, \dataGutenberg{} and \dataPan{}.

    \item \textbf{How do we evaluate the quality of paraphrases?} \label{enum:rq2} \hfill \\
    Paraphrase evaluation is inherently challenging, as there is no universally agreed-upon definition of what constitutes a paraphrase. 
    Prior research often adapts evaluation metrics from related \ac{nlp} tasks such as machine translation or summarization. 
    Two key dimensions are typically considered: semantic similarity and syntactic similarity.
    Contrary to initial intuition, high syntactic similarity is not necessarily desirable, as it may indicate that the \ac{llm} has merely copied the original text with minimal changes. 
    Instead, our focus lies on achieving high semantic similarity while maintaining syntactic diversity to ensure genuine rephrasing.
    Furthermore, we acknowledge that relatively low automatic scores can still be acceptable if qualitative human evaluation confirms the paraphrase’s adequacy.

    % \item \textbf{Which features are used for the \ac{av} problem?} \label{enum:rq3} \hfill \\
    % Traditional features include character tri-gram features, while newer research has proposed using \ac{llm} such as BERT.

    \item \textbf{How does the \ac{llm}-based impostor approach perform compared to state-of-the-art models?} \label{enum:rq4} \hfill \\
    Though our approach is computationally expensive, we argue that it is not a general purpose \ac{llm} detection method, but rather a single case solution tailored to specific detection tasks.
    We evaluate its performance in scenarios where (a) the disputed text is human generated,
    (b) the disputed text is \ac{llm} generated and the candidate is the same \ac{llm}, and
    (c) the disputed text is \ac{llm} generated, but the candidate is a different \ac{llm}.
    In terms of performance, we compare our method to other \ac{av} approaches on the \dataStudent{}, \dataBlog{}, \dataGutenberg{} and \dataPan{} datasets.
    
\end{questions}

% \section*{Idea}
% \label{sec:idea}

% Given a text of unknown authorship (i.e., human or \ac{llm}), 
% construct a set of impostor texts using state-of-the-art \acp{llm} based on the original text.
% Obtain the author by \ac{aa}/ \ac{av} methods, such as unmasking, to \textit{confidently}, i.e. high precision, identify \ac{llm} generated texts
% (and possibly which \ac{llm}).



% \section*{Contributions}
% \label{sec:contributions}
The contributions of this thesis are:
\begin{enumerate}
    \item Reimplementation of the traditional Impostor approach (cf. \autoref{chap:implementation}).
    \item Extension of the impostor approach with \ac{llm} generated impostors for line-up of difficult opponents (cf. \autoref{chap:methodology}). 
    \item Frame \ac{llm} detection as a \ac{av} problem and use \ac{llm} generated text as candidate for text of "unknown" authorship.
\end{enumerate}

    \chapter{Background on Authorship Identification}
\label{chap:authorship_identification}
   AI
    - Woher kommt es?
        - initial statistische analyse
        - Dylo hypethese???
        - stylometry
        - Use case 1/ original use case: Literatur Forschung
        - Use case 2: Digital text forensics
    - AA
    - AV als Kernproblem von allem
        - Fokus hier, und LLM detection istr ein Plus, eher im Anhang
    - technischer Hintergrund: open/ closed set, one-class classification
    - state of the art models
    - \imp{} method

This section gives background on ...

\textcolor{red}{TODO}






\section{\acl{av}}

\subsection{Compression-based???}
\textcolor{red}{TODO}


\subsection{Unmasking Method}
\textcolor{red}{TODO}


\begin{figure}[htbp]
    \centering
    \includesvg[width=\textwidth]{images/unmasking/unmasking.svg}
    \caption{Unmasking.}
    \label{fig:unmasking}
\end{figure}
Our method is an extension of the original \impAppr{} by \citet{koppel_determining_2014}.
By varying the seed and temperature, we can generate as many texts as we want.
  
\subsection{\imp{} Method}
\label{sec:impostor_method_theory}

The \impAppr{} leverages random projections to lower dimensional spaces (i.e. random set of features set to zero is a projection).
\begin{figure}[htbp]
    \centering
    \includesvg[width=\textwidth]{images/imposter/imposter.svg}
    \caption{\imp{}.}
    \label{fig:impostor}
\end{figure}


 % before related work
    \chapter{Related Work}
\label{chap:related_work}

This work is different to the work of \cite{koppel_determining_2014} and \cite{kocher_unine_2015} 
in that it uses \acp{llm} to generate imposter texts.

% LLM detection using generative models
%% AA against LLMs
With the recent advances of \ac{nlg} come new challenges in text authorship.
The new technologies may be misused for fraudulent activities to scam naive users.
\citet{uchendu_authorship_2020} identified three authorship tasks essential for fighting fraudulent activities:
(1) Given two texts $t_1$ and $t_2$, determine whether they were produced by the same method (i.e. human author or a specific \ac{nlg} method).
(2) Given a text $t$, determine whether it was human authored or machine generated.
(3) Given a text $t$, find its author among $k+1$ candidates, which consists of one human and $k$ machines.
They compare classical \ac{ml} models, neural models and state-of-the-art \ac{aa} models as classifiers 
for these single- (Problem 1 and 2) and multi-class (Problem 3) tasks.
Their findings include, that as of 2020, most \ac{nlg} methods were distinguishable from human authors, 
but some \acp{llm} proved difficult to detect.
%%% compared to our work
In the following, we consider (1) \ac{av}, (2) classical \ac{llm} detection, and (3) closed-set \ac{aa}.
Our approach differs from the work of \citet{uchendu_authorship_2020} in that our candidates (i.e. imposters) do not include a human author (3), 
but only \acp{llm}.
Moreover, we use different classifiers originally designed for \ac{av}, rather than \ac{aa}. % teil: background ML etc.
    \chapter{\acs{llm}-based \imp{} Method}
\label{chap:llm_impostor_method}

\section{Introduction}

A central challenge in \ai{} is the presence of confounding factors such as register, genre, topic, and target audience. 
These dimensions influence linguistic choices independently of authorial style, thereby obscuring the stylometric features relevant for authorship-related tasks. 
To address this issue, we propose an extension of the \impAppr{} originally proposed by \citet{koppel_determining_2014} that leverages \acp{llm} to systematically control for such confounders. 
By generating texts under carefully constrained conditions, we obtain \imps{} that preserve the semantic content of the original while capturing the \ac{llm}'s authorial style, thereby serving as controlled proxies for modeling the actual generation process of the reference text.

The generation of \imps{} relies on the paraphrasing capabilities of \acp{llm}. 
An \imp{} can be obtained either in a single step, where the model directly produces a paraphrase, or in two steps, where the model first extracts information from the reference text and subsequently generates a paraphrase based on the extracted information. 
The latter allows for more rigorous disentanglement of content and style, but at the cost of additional computational overhead. 
In both cases, \imp{} generation functions as a proxy re-generating the reference text, yielding text variants that systematically neutralize confounding stylistic factors while retaining the core meaning of the original document.

\section{Impact of Confounders on Authorial Style}
\label{sec:contextual_factors}

\Acl{sota} models for authorship analysis exhibit strong sensitivity to domain shifts. 
Performance often deteriorates sharply when models are applied in \ac{ood} scenarios, i.e.\ in cross-domain settings, a phenomenon largely attributable to the influence of confounders such as topic, genre, and register~\citep{Sundararajan_style_18,bischoff_importance_2020}

Confounders are problematic because they influence the very features used to characterize authorial style. 
Topical vocabulary, for example, can dominate lexical distributions, while genre-specific conventions shape syntax. 
As these factors cannot be cleanly separated from genuine stylistic markers~\citep{bischoff_importance_2020}, they obscure the style markers.

If topics can be represented by characteristic word distributions, then a document can be seen as a subset of words selected by the author, reflecting individual preferences in synonym choice~\citep{altakrori_topic_2021}. 
Consequently, texts by the same author on different topics may appear unrelated, whereas texts by different authors on the same topic may seem deceptively similar.

Empirical evidence confirms the severity of this problem.
Both contemporary \acp{av} methods~\citep{Thomas_cross_topic_24}, and \ac{llm} detection approaches, such as DetectGPT~\citep{mitchell_detectgpt_2023,Wu_ODD_challenges_2025} suffer significant effectiveness degradation in \ac{ood} scenarios.

As a result, authorship research has diverged into two main directions. 
One line of work aims to identify domain-invariant features, a challenge that remains largely unresolved~\citep{bischoff_importance_2020}. 
The other focuses on \ac{id} scenarios, in which confounding factors are deliberately controlled. 
While restricting tasks to a single topic or genre does not fully eliminate the entanglement between content and style, it reduces its impact sufficiently to produce stable and interpretable results. 
Moreover, some studies indicate that simply using domain or topic labels is insufficient to control for topic similarity in corpora, as this approach ignores semantic relationships between topics~\citep{sawatphol_cross_topic_av_24}.


\section{\acs{llm}-based \Imp{} Generation}
\label{sec:impostor_generation}

% good \imps{}: hard negatives
Following the notion introduced by \citet{koppel_determining_2014}, ideal \imp{} texts can be understood as hard negatives: 
Documents not written by the candidate author, yet sufficiently similar in style to be difficult to distinguish from the candidate's own writing. 
The quality of \imps{} directly affects model performance.
If they are too different from the candidate text, models are trained on trivial contrasts, yielding \acp{fp}. 
Conversely, if \imps{} are too similar to the candidate, the risk of \acp{fn} increases.

% obstacles for \imp{} generation in the past
Traditional \imp{} generation techniques struggled to control \imp{} similarity. 
The fixed approach samples from a fixed pool of unrelated texts, while the on-the-fly approach fails to align genre with the candidate text~\citep{koppel_determining_2014}. 
Since authorial style is tightly entangled with such contextual factors, \imps{} produced by these methods differ systematically from the candidate text, weakening their utility. 
Ideally, \imp{} generation should replicate the conditions of original text production, including task, topic, register, target audience and century. 
In practice, however, this information is rarely available.
In \ac{av}, the candidate author may be deceased, unavailable, or unwilling to cooperate.

% heuristics: paraphrase
\acp{llm} make it possible to approximate this ideal. 
By conditioning generation on the candidate text itself, external factors such as topic, genre, and register can be more tightly controlled. 
In particular, paraphrasing offers a heuristic for simulating the text generation process. 
The model produces stylistic variants of the candidate's writing while maintaining semantic alignment.


\subsection{Paraphrasing as a Basis for \Imp{} Generation}

The lack of a universally accepted definition of paraphrase complicates evaluation~\citep{gohsen_task_oriented_2024}. 
For the present work, paraphrases are considered suitable impostors if they satisfy three conditions: 
They remain consistent with the original text's topic, genre, and tone; 
they exhibit sufficient lexical and syntactic divergence to avoid near-duplication; 
and they may diverge semantically, including mild hallucinations, provided the stylistic resemblance is preserved.
Paraphrase quality is largely determined by prompt design and model selection~\citep{Wu_ODD_challenges_2025}.

Paraphrasing is often analysed along two main dimensions. 
At the lexical level, it involves word-level substitutions such as synonym replacement, typically achieved through rule-based methods or thesaurus-driven resources. 
At the syntactic level, paraphrasing entails changes to sentence structure, which can be realized through monolingual machine translation~\citep{zhou_paraphrase_2021} or by explicitly prompting \acp{llm}~\citep{kurt_pehlivanoglu_comparative_2024}.
% While more fine-grained taxonomies exist~\citep{zhou_paraphrase_2025}, they are not central here, as this study does not involve training or constructing paraphrase models directly.


\subsection{One-step Paraphrasing}

In the one-step approach, an \imp{} is produced directly through prompting an \ac{llm}. 
Effective prompting proved essential.
We observed that specification of what to avoid was vital to obtain valid results. 
Moreover, task instructions placed at the end of the prompt reduced cases where models ignored them in long-context settings, such as the \dataGutenberg{} corpus.

After iterative refinement, the following two prompts yielded consistent results.
The first explicitly decomposes the sentence into subject, verb, and object before reconstruction, encouraging lexical diversity. 
The second instructs the model to preserve meaning while varying wording and structure. 
Both emphasize concise outputs restricted to the paraphrase itself.

\begin{quote}
    \textit{For the text above: Paraphrase the sentence by first identifying the main subject, verb, and object. Then find synonyms for each and construct a new sentence. Only output the final paraphrased sentence.}
\end{quote}

\begin{quote}
    \textit{For the text above: Paraphrase this sentence. Do not change the meaning, but use different words and structure. Output only the paraphrased sentence.}
\end{quote}

We initially experimented with T5-based models from HuggingFace, but these demonstrated poor instruction following and often produced malformed outputs. 
Our final experiments used the models listed in \autoref{tab:base_llms}, all accessed through GWDG/SAIA or IONOS infrastructure.

\begin{table}[]
\centering
\caption{Base \acp{llm} used for paraphrasing.}
\label{tab:base_llms}
% \resizebox{\textwidth}{!}{%
\begin{tabular}{ll}
\toprule
\textbf{Model ID}                    & \textbf{Host} \\
\midrule
qwen3-32b                            & GWDG/ SAIA    \\
mistral-large-instruct               & GWDG/ SAIA    \\
openai-gpt-oss-120b                  & GWDG/ SAIA    \\
meta-llama-3.1-8b-instruct           & GWDG/ SAIA    \\
meta-llama/Llama-3.3-70B-Instruct    & IONOS         \\
mistralai/Mixtral-8x7B-Instruct-v0.1 & IONOS     \\
\bottomrule   
\end{tabular}%
% }
\end{table}


\subsection{Two-step Paraphrasing}

The two-step approaches are inspired by prior literature~\citep{bevendorff_overview_2024, ayele_overview_2024}, and differ in how they extract and utilize auxiliary information from the source text before generating the paraphrase. 
The two steps enforce a clearer disentanglement of content and style at the cost of increased computational overhead.

Two-step paraphrasing decomposes the task into information extraction and text generation. 
In the first stage, the model identifies contextual features of the input text.
These feature include tone, register, time period, target audience, and genre.
Different approaches extract an additional feature specific to their approach.
The additional features are a bullet point summary, the authorial task, topic, or title. 
The bullet point extractor prompt is specified in the following, while all others can be derived from the Appendix.

\begin{quote}
    \textit{Summarize the following text in five to six short bullet points and give an overall description
    of the genre and tone of the text.}
\end{quote}

In the second stage, a paraphrase is generated based on the extracted information. 
Again, all methods other than the bullet point based one slightly differ from the one given below in Listing~\ref{lst:two_step_bullet_point_prompt} and can be found in the Appendix.

\begin{listing}[ht]
\begin{minted}{python}
prompt = "Write a text which covers the following items:\n" 
    + "\n".join(f"- {bp}" for bp in bullet_points)
\end{minted}
\caption{Two-step paraphrase generation prompt.}
\label{lst:two_step_bullet_point_prompt}
\end{listing}

Translation-based paraphrasing is a related two-step variant, where a text is first translated into another language and then back into English. 
Although conceptually simple, this approach has been shown to yield limited stylistic diversity~\citep{zhou_paraphrase_2025}.
 
    \chapter{Experimental Setup}
\label{chap:experimental_setup}

In the following, we will outline the experimental setup for the experiments we ran.
This includes not only allocation, preprocessing and pair selection of each dataset, but also a description and motivation of the experiments carried out.


\section{Dataset}
\label{sec:dataset}

Since our method extends the original \impAppr{} proposed by \citet{koppel_determining_2014}, we first obtained the datasets used in their study to validate our implementation and reproduce their results. 
The original experiments were conducted on the \dataBlog{} and \dataStudent{} datasets, which are described in detail in \autoref{subsec:original_data}. 
In addition to these, we incorporated two supplementary datasets, \dataPan{} and \dataGutenberg{}, presented in \autoref{subsec:additional_data}. 
Following the general description of all datasets, we outline our preprocessing pipeline in \autoref{subsec:dataset_preprocessing} and conclude with the text-pair selection procedure in \autoref{subsec:dataset_text_pair_selection}.


\subsection{Original Data}
\label{subsec:original_data}

% Blog
The \dataBlog{} corpus~\citep{blog_dataset_2006} consists of blog posts collected from \textit{blogger.com} on or before August 2004, with each blog authored by a single user.
According to the Kaggle repository~\footnote{\href{https://www.kaggle.com/datasets/rtatman/blog-authorship-corpus?resource=download}{Kaggle dataset \texttt{rtatman/blog-authorship-corpus}} (26.07.2025)}, the dataset contains \num{681288} posts from \num{19320} bloggers, averaging approximately 35 posts and \num{7250} words per author.
Users are grouped into three age categories: 13-17, 23-27, and 33-47.
Each record includes the following metadata: \texttt{id}, \texttt{gender}, \texttt{age}, \texttt{topic}, 
\texttt{sign} (referring to the author's astrological/zodiac sign), \texttt{date}, and \texttt{text}.

% student essays
The \dataStudent{} dataset is not publicly available due to the presence of sensitive student information. 
We gratefully acknowledge J. W. Pennebaker for granting access to the original data used by \citet{koppel_determining_2014} in their study. 
The dataset comprises \num{7052} student essays written for five assignments by a cohort of 950 university students in 2006~\citep{koppel_determining_2014}.

The assignments include (1) a stream-of-consciousness task, (2) a reflections on childhood, (3) a self-assessment of personality, (4) a thematic apperception test, and (5) four examples of four different theories.
An inconsistency in the file naming convention is notable: 
Most files are named solely by the author ID, whereas those from the first assignment follow the format \texttt{2006\_authorID}.
Following \citet{koppel_determining_2014}, our \impAppr{} experiments employ only the first four assignments. 
Consistent with \citet{koppel_determining_2014}, disputed–candidate pairs in our setup are drawn from different assignments, irrespective of their class label (\texttt{same-author} or \texttt{different-author}).
In contrast to the original \impAppr{}, which truncated each essay to the first 500 words, we crop texts to the minimum length of the text pair.

Due to privacy restrictions, researchers seeking access to the \dataStudent{} dataset should contact J. W. Pennebaker, the official data custodian. 
For establishing a baseline in our \imp{} experiments, we use both the \dataBlog{} and \dataStudent{} datasets.


\subsection{Additional Data}
\label{subsec:additional_data}
To broaden the evaluation scope of the \impAppr{}, we incorporated additional datasets selected according to two criteria:
(1) control over confounding factors such as genre and topic, and 
(2) verified, undisputed authorship.
Both the \dataPan{} and \dataGutenberg{} datasets satisfy these conditions.

% PAN20: Fanfiction
The \dataPan{} corpus~\citep{bischoff_importance_2020} comprises fanfiction texts sourced from \textit{fanfiction.net}.
Each text belongs exclusively to one fandom (i.e. thematic category), with no crossovers between fandoms.
According to \href{https://pan.webis.de/clef20/pan20-web/author-identification.html}{the official \ac{pan} website}, 
train and test set originate from two different fanfictions and approximate the (long-tail) distribution of the fandoms in the original dataset.
Dataset features include \texttt{id}, \texttt{fandoms}, and \texttt{pair}, where the latter contains the paired texts.
An additional \texttt{jsonl} file provides the ground truth for each pair, specifying \texttt{id}, \texttt{same} (authorship label), and \texttt{authors}.

% Gutenberg
The \dataGutenberg{} dataset~\footnote{\href{https://www.gutenberg.org/}{Project Gutenberg} (26.07.2025)} contains a curated selection of literary works from Project Gutenberg, a digital library dedicated primarily to older works whose U.S. copyrights have expired.
As of this writing, the collection contains over \num{75000} digitized and proofread e-books contributed by volunteers.
For our experiments, we selected 19 works authored by 7 writers from the 16th to 19th centuries, covering genres such as drama, fiction, and poetry.
Metadata for these works was manually extracted from the Project Gutenberg website and Wikipedia.


\subsection{Dataset Preprocessing}
\label{subsec:dataset_preprocessing}

To control confounding factors that influence authorial style, we preprocess each dataset twice:
(1) Prior to generating arrow dataset file and (2) before using the \impAppr{}.
This two-stage approach addresses both experimental scenarios, where all material is prepared in advance, and inference scenarios in which the \impAppr{} model is applied directly to texts.
The preprocessing process was designed to meet the following requirements:
\begin{itemize}
    \item Removal of all formatting and layout information to produce plain text.
    \item Cropping texts to match the length of the shorter text in each pair.
\end{itemize}
It is important to note that text-length adjustment is performed exclusively within the \impAppr{} implementation and is not applied to the Arrow dataset itself.
For a controlled evaluation environment in our \impAppr{}, we opted to work with relatively small, curated datasets rather than scaling to larger collections.  
The effect of individual preprocessing steps on vocabulary size was analysed, with results presented in \autoref{fig:preprocesing_impact_vocab_size}.
In line with \citep{koppel_determining_2014}, the vocabulary consists of space-free character 4-grams.
Removing formatting and layout information includes removing HTML artefacts, play artefacts, newlines, 
converting UTF-8 to ASCII, and stripping leading and trailing whitespace.
Since both \dataBlog{} and \dataPan{} originate from the Internet, we applied HTML-specific preprocessing steps, which had minimal impact on their respective vocabularies.
Upon inspection of the \dataGutenberg{}, we found that certain patterns reappear due to the \textit{play} genre. 
Notably, since we do not consider actor instructions (e.g., character cues) or structural elements (e.g., chapter headings) as part of authorial style, these patterns were removed using regular expressions.
We opted to forgo lowercasing the texts, as our preliminary analysis indicated that lowercasing had no meaningful effect on any dataset while potentially discarding deliberate authorial capitalization choices.

\begin{figure}[htbp]
    \centering
    \includesvg[width=\textwidth]{images/dataset/impact_preprocessing_steps.svg}
    \caption{Effect of preprocessing steps on vocabulary size (space-free character 4-grams).}
    \label{fig:preprocesing_impact_vocab_size}
\end{figure}


\subsection{Selection of Text Pairs}
\label{subsec:dataset_text_pair_selection}

For the \dataBlog{}, \dataStudent{}, and \dataGutenberg{} datasets, we selected pairs of texts according to specific criteria to control potential confounding factors.
Only texts with a minimum length of \num{700} words were considered eligible. 
For the \dataPan{} dataset, we retained the existing pairs in the Arrow dataset to ensure comparability with prior work. 
All datasets include both same-author and different-author pairs. 

For \dataBlog{}, pairs were constructed so that the two texts share the same topic, year, gender, and age (the latter referencing the author). 
The training set comprises 80\% of the data and the test set 20\%, with different topics assigned to each split.

For \dataStudent{}, texts were assigned to either the training (70\%) or test (30\%) set. 
The test set is larger because each author typically contributes only one essay per task and if only a single task were included in the test set, no same-author pairs could be formed. 
Same-author pairs were selected such that authors share the same sex, ethnicity, and political orientation. 
As per \citep{koppel_determining_2014}, all text pairs are drawn from the different tasks.



% Please add the following required packages to your document preamble:
% \usepackage{graphicx}
\begin{table}[tbp]
    \centering
    \caption{\ac{aa} scenarios with author $i$ is shortened with $A_i$ \citep{altakrori_topic_2021}.}
    \label{tab:aa_same_topic}
    \begin{tabular}{|l|l|l|}
    \hline
    \textbf{} & \textbf{Train} & \textbf{Test} \\ \hline
    \textbf{Topic $T_1$} & $A_1, A_2$ & $A_1, A_2$ \\ \hline
    \textbf{Topic $T_2$} & $A_1, A_2$ & $A_1, A_2$ \\ \hline
    \end{tabular}%
\end{table}

% Please add the following required packages to your document preamble:
% \usepackage{graphicx}
\begin{table}[tbp]
    \centering
    \caption{\ac{aa} scenarios with author $i$ is shortened with $A_i$ \citep{altakrori_topic_2021}.}
    \label{tab:aa_cross_topic}
    \begin{tabular}{|l|l|l|}
    \hline
    \textbf{} & \textbf{Train} & \textbf{Test} \\ \hline
    \textbf{Topic $T_1$} & $A_1, A_2$ &  \\ \hline
    \textbf{Topic $T_2$} &  & $A_1, A_2$ \\ \hline
    \end{tabular}%
\end{table}

% Please add the following required packages to your document preamble:
% \usepackage{graphicx}
\begin{table}[tbp]
    \centering
    \caption{\ac{aa} scenarios with author $i$ is shortened with $A_i$ \citep{altakrori_topic_2021}.}
    \label{tab:aa_topic_confusion}
    \begin{tabular}{|l|l|l|}
    \hline
    \textbf{} & \textbf{Train} & \textbf{Test} \\ \hline
    \textbf{Topic $T_1$} & $A_1$ & $A_2$ \\ \hline
    \textbf{Topic $T_2$} & $A_2$ & $A_1$ \\ \hline
    \end{tabular}%
\end{table}


For the \dataGutenberg{} dataset, pairs were selected such that texts share the same genre and century. 
Authors were split into training (80\%) and test (20\%) sets.

Regardless of the selection criteria, the final datasets contain only three columns: \texttt{authors}, \texttt{pair}, and \texttt{same}.
The \texttt{pair} column contains the texts of the pair as a list of strings,
the \texttt{authors} column contains the authors of the texts as a list of strings,
and the \texttt{same} column indicates whether the texts are from the same author (\texttt{True}) or from different authors (\texttt{False}).
Descriptive statistics for all preprocessed datasets are provided in Table~\ref{tab:data_stats}.

% minimum length necessesary for AV/ AA
The choice of minimum text length was informed by related research in \ac{av} and \ac{aa}.
\citet{bevendorff_generalizing_2019}used text chunks of at least 700 words for an unmasking approach, while \citet{koppel_authorship_2004} set the minimum at 500 words.
Recent work~\citep{llm_detection_av_2025} identifies \num{2500}-\num{4000} characters to be sufficient for effective \ac{llm} detection framed as \ac{av} or \ac{aa}, leading those authors to adopt a \num{3000}-character minimum for their datasets.

% \begin{table}[h]
\begin{sidewaystable}
\centering\small
\caption{Statistics of preprocessed datasets \dataPan{}, \dataBlog{}, \dataGutenberg{}, and \dataStudent{}.}
\label{tab:data_stats}
\resizebox{\textwidth}{!}{%
\begin{tabular}{@{}lrrrrrrrrr@{}}   % numbers should be right aligned, text left aligned
\toprule
dataset & num\_pairs & num\_authors & num\_same\_pairs & num\_different\_pairs & avg\_text\_len & max\_text\_len\_words & std\_text\_len\_words & median\_text\_len\_words \\
\midrule
pan20           & 66905 & 52771 & 35616 & 31289 & 21418.76 (3914.76)   & 55413 & 512.19 & 3889 \\
blog            & 11565 & 5997  & 6204 & 5361  & 6249.94 (1154.25)     & 115365 & 1493.97 & 913 \\
gutenberg       & 12    & 7     & 6     & 6     & 437870.75 (78698.79) & 297704 & 68329.91 & 60282 \\
student\_essays & 224  & 222   & 112   & 112  & 4459.32 (865.90)     & 1634 & 157.41 & 815  \\
\bottomrule
\end{tabular}%
}
% \end{table}
\end{sidewaystable}


% regardless of experimental design
\section{Evaluation measures}
\label{sec:evaluation_measures}

In this chapter, we introduce state-of-the-art quantitative evaluation metrics for \ac{aa}, \ac{av}, and paraphrase generation. 
Unlike subjective human judgments, these metrics are designed to be comparable and reproducible, providing a more objective basis for evaluation.

\textcolor{red}{
c@1 does not make sense here bc (1) output is "same author" or "i do not know" (standard 0.5 from PAN) and (2) output is 0 or 1, that is why we have threshold -> never 0.5. Maybe we could train two thresholds and could return 0.5 if neither is triggered, but does not make sense bc "different author" class ill-defined (open set -> no representative/exhaustive samples)
}


We will present state-of-the-art quantitative evaluation measures for \ac{aa}, \ac{av} and paraphrase generation.
Ideally, quantitative measures are comparable and reproducible since they do not arise from human biased judgment.

\section{\ac{av} Quality Measures}
\label{sec:av_quality_measures}

Common metrics for evaluating the \ac{av} performance are:
\begin{itemize}
    \item $Accuracy = \frac{TP + TN}{TP + TN + FP + FN}$ \citep{elmanarelbouanani_authorship_2014,neal_surveying_2018} 
    measures the percentage of classified correctly over all test cases \citep{neal_surveying_2018}.

    \item $Precision = \frac{TP}{TP + FP}$ \citep{elmanarelbouanani_authorship_2014,neal_surveying_2018,chen_web_2008} 
    measures how often a system gets positive classification correctly \citep{neal_surveying_2018}.

    \item $Recall = \frac{TP}{TP + FN}$ \citep{elmanarelbouanani_authorship_2014,neal_surveying_2018,chen_web_2008} 
    measures how often a system correctly classifies positive samples when it encounters them \citep{neal_surveying_2018}.
    Recall is also called sensitivity or \acl{tp} rate \citep{palivela_optimization_2021}

    \item $F-measure = \frac{2 \cdot precision \cdot recall}{precision + recall}$~\citep{chen_web_2008,abbasi_writeprints_2008}

    % PAN
    \item \ac{roc-auc} \citep{bevendorff_overview_2024,weerasinghe_feature_vector_difference_2021,kocher_unine_2015}
    where \ac{roc} plots \ac{fpr} $= \frac{FP}{FP+TN}$ on the x-axis against the \ac{tpr} $=\frac{TP}{TP+FN}$ on the y-axis 
    for varying thresholds \citep{kocher_unine_2015,neal_surveying_2018}.
    The maximum \ac{roc} value of 1.0 indicated a perfect performance \citep{kocher_unine_2015}.
    Greater \ac{auc} indicates a better performance \citep{neal_surveying_2018}.
    % The \ac{auc} of the \ac{roc} is biased since the \ac{roc} gives more emphasis 
    % on the first position and therefore increases the total \ac{auc}.
    % A misclassification with a lower probability is less penalized with \ac{roc-auc} \citep{kocher_unine_2015}.
    \citet{kocher_unine_2015} claim that both \ac{roc} and \ac{auc} are difficult to interpret.
    % LLM detection as AV task
    \citet{llm_detection_av_2025} argue that the reduction from the \ac{fpr}-\ac{tpr} curve of \ac{roc} to a single \ac{roc-auc} number 
    comes with information loss due to the absence of a fixed threshold and trade-off.
    Moreover, \ac{roc}'s \ac{fpr} and \ac{tpr} are independent of class prevalence, which is desirable.
    However, in highly imbalanced class scenarios \ac{roc} can be misleading (overly optimistic or pessimistic).
    
    \item BRIER: Complement of the Brier score \citep{bevendorff_overview_2024,weerasinghe_feature_vector_difference_2021}, 
    in \citet{bevendorff_overview_2024}'s case equivalent to the mean squared loss.
    The Brier score is used to evaluate the ability of \ac{av} methods to abstain from hard samples \citep{tyo_state_2022}.
    
    \item $C@1 = \frac{nc}{np}(1+\frac{nu}{np})$ where $np$ is the number of problems, $nc$ the number of correct answers, 
    and $nu$ the number of unanswered problems \citep{kocher_unine_2015}. 
    Modified version of the accuracy \citep{bevendorff_overview_2024}/ F1-score \citep{weerasinghe_feature_vector_difference_2021} score, 
    where the non-answers (abstained) \citep{llm_detection_av_2025} are assigned the average accuracy of the remaining cases \citep{bevendorff_overview_2024}. 
    It rewards systems that leave difficult problems unanswered \citep{weerasinghe_feature_vector_difference_2021}.
    
    \item $F_1 = 2 \cdot \frac{precision \cdot recall}{precision + recall}$~\citep{neal_surveying_2018}: Harmonic mean of precision and recall \citep{bevendorff_overview_2024,weerasinghe_feature_vector_difference_2021}.
    A higher value indicates a better performance \citep{neal_surveying_2018}.
    
    \item $F_{0.5u}$: Modified version of the $F_{0.5}$ score, where the non-answers are considered \acp{fn} \citep{bevendorff_overview_2024}. 
    A measure that puts more emphasis on deciding same-author cases correctly \citep{weerasinghe_feature_vector_difference_2021}.
    Used to evaluate the ability of \ac{av} methods to abstain from hard samples \citep{tyo_state_2022}.
\end{itemize}
$TP$ is the number of \aclp{tp}, $FP$ is the number of \aclp{fp} 
and $FN$ is the number of \aclp{fn}~\citep{chen_web_2008}.



% PAN AA & AV metrics: Abstaining from hard samples
% The \todo{F0.5u, C@1, and Brier Score metrics} are used to evaluate the ability of \ac{av} methods 
% to abstain from hard samples \citep{tyo_state_2022}.
% For each sample, a score $\in [0, 1]$ is assigned to the sample.
% A score of exactly 0.5 means the model abstains from the sample \citep{tyo_state_2022,bevendorff_overview_2024,kocher_unine_2015}.

% The AUC metric is used to evaluate the ability of methods to rank predictions.
% No threshold is required.
% \ac{pan} ignores any abstained samples when calculating the AUC metric \citep{tyo_state_2022}.

% != PAN
% \citet{tyo_state_2022} chose to adopt the macro-averaged accuracy metric, so-called macro-accuracy, for \ac{aa}, 
% and AUC \ac{av} tasks.

\subsection{Paraphrase evaluation}
\label{subsec:paraphrase_evaluation}

Evaluating paraphrases is a central problem in \ac{nlp}, yet it is complicated by the fact that there is no universal definition of what constitutes a paraphrase. 
Definitions vary in degree of semantic equivalence required. 
This conceptual ambiguity makes the task of evaluation especially challenging, since different applications may prioritize different aspects such as fidelity to meaning, stylistic variation, or grammatical well-formedness.

Because of this, paraphrase evaluation cannot be reduced to syntactic variance alone. 
A meaningful assessment must account both for syntactic diversity and for the extent to which semantic content is preserved. 
At the same time, paraphrases must be judged for their fluency and acceptability, as grammatical errors or stylistic awkwardness may render an otherwise accurate reformulation unsuitable in practice.

Existing approaches can broadly be grouped into automatic and human-based methods. 
Automatic measures attempt to quantify the similarity or equivalence between a candidate and a reference paraphrase using algorithmic techniques. 
These methods can be further distinguished by the linguistic level at which they operate. 
Some focus on syntactic structure and word overlap, while others rely on semantic representations to evaluate meaning preservation beyond surface form. 
Human evaluation, in contrast, remains the gold standard, as it incorporates judgments about adequacy, grammaticality, and contextual appropriateness that automatic measures cannot fully capture.

% Syntactic (\ac{bleu}, \ac{rouge}-1, \ac{rouge}-L), semantic (BERTScore, cosine similarity of SBERT vectors, WMS), human evaluation (TODO)

% There is manual (by humans) evaluation and automatic evaluation for paraphrase generation \citep{fu_learning_2024,zhou_paraphrase_2021}.
% According to \citet{zhou_paraphrase_2021}, automatic evaluation metrics mainly focus on the n-gram overlaps instead of meaning, 
% and hence, human evaluation is more accurate and has a higher quality.
% In the following, we focus on automatic evaluation.

In the following, we examine both automatic and human evaluation strategies. 
The automatic measures are divided into syntactic and semantic approaches, reflecting the different dimensions along which paraphrases can be compared. 
This is followed by a discussion of human evaluation, which complements automatic measures by providing a more comprehensive assessment of paraphrase quality.


\subsection{Traditional Quantitative Paraphrase Evaluation Measures}
\label{subsec:traditional_quantitative_evaluation_measures}

Evaluating paraphrases can be reduced to summarization or translation evaluation.
The evaluation of paraphrases can be divided into syntactic and semantic approaches. 
% \citet{gohsen_captions_2023} normalized all metrics and averaged the semantic and syntactic scores separately.

\subsubsection{Syntactic Measures}
Syntactic evaluation metrics mainly focus on the n-gram overlaps~\citep{zhou_paraphrase_2021}. 
Common syntactic evaluation metrics include \acs{bleu}, \acs{rouge}-1, and \acs{rouge}-L.

\input{chapter/section-05/metrics/BLEU.tex}
\input{chapter/section-05/metrics/ROUGE.tex}
\input{chapter/section-05/metrics/METEOR.tex}


\subsubsection{Semantic Measures}
Syntactic measures are inadequate when the goal is to evaluate paraphrases that prioritize semantic preservation over lexical similarity. 
To address this limitation, semantic metrics leverage distributed representations of words or sentences.
We compute semantic similarity between transformer based models~\citep{gohsen_captions_2023}.

BERTScore~\citep{hanna_fine_grained_2021} computes similarity between contextual BERT embeddings of candidate and reference texts. 
For reference vectors $r$ and candidate vectors $c$, precision and recall are defined as \autoref{eq:bert_p} and \autoref{eq:bert_r}, respectively.

\begin{equation}
    P_{BERT} = \frac{1}{|c|} \sum_{c_i \in c} \max_{z_j \in r} r_j\top c_i
\label{eq:bert_p}
\end{equation}
\begin{equation}
    R_{BERT} = \frac{1}{|r|} \sum_{r_i \in r} \max_{c_j \in c} r_i\top c_j
\label{eq:bert_r}
\end{equation}

% \begin{equation}
%     F_1 = \frac{2 P_{BERT} R_{BERT}}{P_{BERT} + R_{BERT}} 
% \label{eq:bert_f1}
% \end{equation}
% Since $F_1 \in \left[-1,1\right]$ it can be rescaled to $[0,1]$ by modifying the precision and recall calculation 
% to $\hat{P}_{BERT} = \frac{P_{BERT} - a}{1 - a}$ ($R_{BERT}$ analogous), where $a$ is the empirical lower bound on the BERTScore \citep{hanna_fine_grained_2021}.


BERTScore correlates with human judgment at the semantic level \citep{kurt_pehlivanoglu_comparative_2024}, although it may struggle when lexically overlapping but semantically incorrect candidates are present \citep{hanna_fine_grained_2021}.

\ac{wmd} measures the minimal transport cost of aligning word embeddings from one text to another \citep{gohsen_captions_2023}. 
\textcolor{red}{TODO: Rechnung}

\textcolor{red}{TODO: SBERT cosine similarity}
The cosine similarity between dense vector representations of a SBERT model~\citep{gohsen_captions_2023}.

\subsubsection{Gohsen Delta $\Delta_{sem,syn}$}
First, all syntactic and semantic measures are normalized to a scale from zero to one.
Then, the average syntactic similarity $\diameter_{syn}$ and the average semantic similarity $\diameter_{sem}$ is calculated.
Syntactic metrics include \ac{rouge}-1, \ac{rouge}-L, and \ac{bleu}.
Semantic measures include \ac{wms}, BERT, and cosine similarity of the SBERT embeddings.
Finally, $\Delta_{sem,syn}$ is defined as in \autoref{eq:gohsen_delta}, i.e. the difference of semantic and syntactic average distance~\citep{gohsen_captions_2023}.
\begin{equation}
    \Delta_{sem,syn}=\diameter_{sem}-\diameter_{syn}
    \label{eq:gohsen_delta}
\end{equation}
Hence, high $\Delta_{sem,syn}$ values indicate structurally and lexically diverse and semantically similar text pairs.


\subsubsection{Bespoke Quantitative Evaluation}
\label{subsec:custom_quantitative_evaluation}

% make subsubsub
\subsubsection{Text Extraction}
\label{subsec:text_extraction}

In order to evaluate the quality of the information extracted by the \pextractor{}, 
we decided to compare the genre, century, and the paraphrase-specifc topic to the 
ground truth available for the \dataBlog{}, \dataGutenberg{} and the \dataCustom{} dataset.

We found that the instructions for the \pextractor{} have to be positioned after the text to be extracted, 
due to the inability of the \pextractor{} to return the extracted information in the specified JSON format 
when the prompt was at the beginning of the input for long texts such as those from the \dataGutenberg{} dataset.

\textcolor{red}{TODO: insert table with results}

Irrespective of the quality of the text extraction, we hypothesize that the quality of the final result of the \pgenerator{} will be good irrespective of the quality of the \pextractor{}.
We motivate this by the fact that both the \pextractor{} and the \pgenerator{} are \acp{llm} and therefore generate text similar.
% Attention: Causal vs. masked language model work different

% make subsubsub
\subsubsection{Paraphrase Generation}
\label{subsec:paraphrase_generation}
To evaluate the quality of the paraphrases generated by the \pgenerator{}, 
we not only computed different paraphrase quality metrics, 
but also compared the text lengths of the generated paraphrases and the original text.

\textcolor{red}{TODO: insert table with results}

% shortcomings of paraphrasing metrics and need for human evaluation
Though easier to reproduce, it is somehow unclear what paraphrase metrics actually measure beyond what their formula states.
While high n-gram overlap might not be the indicator of a good paraphrase in the sense of high syntactic diversity, 
it is not clear if high cosine similarity between the embedding of two texts is a good indicator of a good paraphrase.
Moreover, for all metrics, threshold values for good paraphrases are not well-defined.
It remains to be found whether the worst performing paraphrases are still good enough in terms of human evaluation.
We therefore also employed qualitative evaluation of the paraphrases.

% make subsubsub
\subsubsection{Measures and Findings}
\label{subsec:measures_and_findings}

% shortcoming of traditional quantitative paraphrase metrics
We used state-of-the art measures for the quantitative evaluation of paraphrases. 
Unfortunately, these measures can be misleading since it is unclear what they actually measure.
Generally high scores in BLEU, ROUGE, METEOR mean nearly identical paraphrase (high n-gram overlap).
In this case, we want value syntactic diversity, rendering these measures unintuitive 
for high values do not necessarily correspond to good paraphrases.
Semantic similarity measurements compare the content of the paraphrase to the original text, 
where the interpretation of the cosine of a vector is not clear either.

% findings
Non-naive paraphrasers generally lower syntactic scores than naive paraphrasers,
supposedly because they have a weaker influence on the \pgenerator{} 
(i.e. disclosing extracted content rather than the original text)
leaving more room for variance in texts.
Consequently, \enquote{bad} scored non-naive paraphrases are good in terms of syntactic diversity.

\subsubsection{Qualitative Evaluation}
\label{subsec:qualitative_evaluation}

In addition to the quantitative evaluation, we qualitatively evaluated the paraphrases generated by the \pgenerator{}.
Prior to the evaluation we specified a list of criteria that a good paraphrase should fulfill.
\textcolor{red}{nicht specifc to work here, but general}


\section{Experimental Setup}
\label{sec:experimental_setup}
% for each: Question to answer, experiment design, same language (unified description), when duplicate: short description and reference to other occurence (but avoid sole references)
% questions can be specific, should be related to research question(s)

The following experiments are designed to systematically evaluate the extension of the \impAppr{}~\citep{koppel_determining_2014} with \ac{llm} generated impostors. 
The evaluation proceeds in several stages in order to both reproduce baseline results and to assess the proposed modifications in a controlled and comparative manner.

We begin by reproducing the original \impAppr{} experiments of \citet{koppel_determining_2014} to establish a reliable baseline and to ensure comparability between our implementation and previously reported results. 
Next, we investigate the quality of paraphrases used in \imp{} generation. 
In particular, we compare a one-step paraphrasing strategy with a two-step approach. 
We compare paraphrase metrics across different numbers of text chunks.
We then analyse the effect of syntactic similarity between paraphrases and original texts on \imp{} scores. 
Then, we compare different strategies for \imp{} generation, ranging from traditional baselines to \ac{llm}-based paraphrasing methods. 
We evaluate both the raw \imp{} scores and their behaviour under different experimental scenarios, such as varying sources of input text pairs. 
This allows us to examine not only the overall effectiveness of different \imp{} generation approaches, but also their relative strengths under specific \ac{av} settings.

Together, these experiments provide a comprehensive analysis of the impact of paraphrase quality, syntactic similarity, and \imp{} generation strategy on the performance of the \impAppr{}, thereby offering insights into the role of \acp{llm} in enhancing authorship verification.


\subsection{Exp. 1: Reproduction of Original Work}

To assess the validity of our extension to the traditional \impAppr{}, we first verified the correctness of our implementation. 
For this purpose, we designed two experiments, which we ran on a subset of 100 pairs from the training and test sets of the \dataBlog{} and \dataStudent{} dataset respectively. 
Half of the selected samples belong to the same-author class.

\paragraph{Exp. 1(a): Varying number of \imps{}.}
The first experiment evaluates the effect of varying the number of \imps{} while setting the \imp{} generation method to \texttt{fixed}.
All other hyperparameter values are set to the default values reported by \citet{koppel_determining_2014} (cf. Table~\ref{tab:repr_exp1}). 
Adhering \citet{koppel_determining_2014}, we compute precision and recall scores across different thresholds.
\textcolor{orange}{For comparison, reference precision-recall points reported by \citet{koppel_determining_2014} are included in our visualization.} 
Based on their description, we deduced that their reported scores were obtained using the \dataBlog{} dataset.


\begin{table}[h]
\centering\small
\caption{Exp. 1(a): \impAppr{} configuration.}
\label{tab:repr_exp1}
\begin{tabular}{@{}llrrl@{}}   % numbers should be right aligned, text left aligned
\toprule
\# Impostors & Generation & Rounds & Top $n$ & Upsample \\
\midrule
\textit{Variable} & Fixed & 100 & \num{100000} & False \\
\bottomrule
\end{tabular}%
\end{table}

% Exp 1c: find best threshold via different metrics
% A detector instance was trained on each training set, and the optimal decision threshold was determined using Youden's J statistic. 
% This threshold was then applied to the 15 test set pairs to generate final predictions, which were summarized in a confusion matrix.

% We also considered using thresholds that maximized alternative metrics, such as the F1 score, but rejected this approach because it produced imbalanced detector classifications. 

\paragraph{Exp. 1(b): Varying impostor generation.}
The second experiment evaluates different \imp{} generation methods while keeping the number of \imps{} fixed.
Again, all other hyperparameter values are set to the default values reported by \citet{koppel_determining_2014} (cf. Table~\ref{tab:repr_exp2}). 
Following \citet{koppel_determining_2014}, we compare the \texttt{fixed} and \texttt{on-the-fly} \imp{} generation methods with the baseline approaches unsupervised min-max similarity, unsupervised cosine similarity, and supervised linear \ac{svm}.

\begin{table}[h]
\centering\small
\caption{Exp. 1(b): \impAppr{} configuration.}
\label{tab:repr_exp2}
\begin{tabular}{@{}rlrrl@{}}   % numbers should be right aligned, text left aligned
\toprule
\# Impostors & Generation & Rounds & Top $n$ & Upsample \\
\midrule
50 & \textit{Variable} & 100 & \num{100000} & False \\
\bottomrule
\end{tabular}%
\end{table}

As in the first experiment, precision and recall are used as the primary evaluation metrics. 
Consistent with \citet{koppel_determining_2014}, we calculate precision and recall with respect to both the same-author and different-author class, alternately treating each as the reference class.
We note that the different-author class if ill-defined as \ac{av} is a one-class classification problem.


\subsection{Exp.\ 3: Paraphrasing Chunks}
\label{subsec:paraphrasing_chunks_setup}

We designed this experiment to evaluate whether chunk-to-chunk paraphrases exhibit better control than text-to-text paraphrases, since chunks contain fewer topic changes than whole texts in theory.
We use one text from the \dataBlog{}, \dataGutenberg{}, and the \dataStudent{} dataset, respectively.


\begin{figure}[htbp]
  \centering
  \includesvg[width=\linewidth]{images/paraphrasing/experiments/chunks/setup/chunk_api_calls.svg}
  \caption[Paraphrase configuration hyperparameters]{Breakdown of individual hyperparameters in the paraphrase configuration.
  We use one document per dataset, chunked into one to five sections and paraphrased with all nine paraphrasers in two variance inducing settings (i.e.\ prompt for one-step, temperature for two-step).
  This amounts to a total of 936 API calls. 
  }
  \label{fig:chunks_api_calls}
\end{figure}


First, texts are chunked preserving sentences.
Chunks are filled with sentences in sentence order such that each chunk roughly contains the same number of words.
Second, paraphrase configurations are defined.
Each one-step paraphraser is paired with two prompts (i.e.\ \texttt{prompt0} and \texttt{prompt1} from \Cref{subsec:one_step_paraphrasing_prompts}), while each two-step paraphraser is paired with two temperatures (i.e.\ 0 and 1).
Third, each chunk is paraphrased with all configurations.
These steps account for a minimum of 936 API calls for paraphrasing.
Each component of the configuration is displayed in \Cref{fig:chunks_api_calls}.
Finally, for each paraphrase, we compute \ac{bleu}, \ac{rouge}-1, \ac{rouge}-2, \ac{rouge}-L, \ac{rouge}-Lsum, \ac{meteor}, \ac{bert}\-Score Precision, \ac{bert}\-Score Recall, \ac{bert}\-Score F1, \ac{sbert} \ac{wms}, and \ac{sbert} cosine similarity.
Final scores per metric for each text-configuration pair are computed by averaging the scores of its constituent text chunks.
The adequate formula is given in \Cref{eq:avg_chunks} and an example is illustrated in \Cref{fig:mean-bleu}.

\begin{equation}
    score(t) = \frac{1}{\#\text{ chunks}}\sum_{i=1}^{\#\text{ chunks}}score(c_i)\text{, for chunk }c_i \in \text{text }t
\label{eq:avg_chunks}
\end{equation}

\begin{figure}[ht]
  \centering
\resizebox{0.9\textwidth}{!}{%
\begin{tikzpicture}[line join=round,line cap=round, >=latex, font=\sffamily]

% --- Left black container with three chunks ---
\draw[black, very thick, rounded corners=6pt]
  (-0.2,3.6) rectangle (3.0,-0.4);

% three inner rounded rectangles
\foreach \y in {2.8,1.6,0.4}{
  \draw[black, thick, rounded corners=5pt] (0.15,\y+0.45) rectangle (2.65,\y-0.45);
  % squiggle inside
  \draw[black, thick, decorate, decoration={snake,amplitude=1.2pt,segment length=6pt}]
    (0.45,\y-0.15) -- (2.35,\y-0.15);
  \draw[black, thick, decorate, decoration={snake,amplitude=1.2pt,segment length=6pt}]
  (0.45,\y+0.15) -- (2.35,\y+0.15);
}

% --- Colored BLEU labels next to each chunk ---
\node[anchor=west, text=teal!80!black, scale=1.2]  at (3.6,2.8) {BLEU $=\,0.1$};
\node[anchor=west, text=orange!85!black, scale=1.2] at (3.6,1.6) {BLEU $=\,0.5$};
\node[anchor=west, text=violet, scale=1.2]         at (3.6,0.4) {BLEU $=\,0.3$};

% --- n_chunks = 3 (black) ---
\node[anchor=west, text=black, scale=1.2] at (0.0,-1.0) {$\#\text{ chunks}=3$};

% --- Arrow to the right and mean BLEU expression ---
\draw[black, very thick, ->, >=latex] (7.7,1.6) -- (9.1,1.6);

\node[anchor=west, text=black, scale=1.4] at (9.3,1.6)
  {$\varnothing\ \text{BLEU} \;=\; \displaystyle
   \frac{\textcolor{teal!80!black}{0.1}+\textcolor{orange!85!black}{0.5}+\textcolor{violet}{0.3}}{3}$};

\end{tikzpicture}
}
  \caption[Computation of the mean \ac{bleu} score over chunks]{Computation of the mean \ac{bleu} score over three text chunks of a text.}
  \label{fig:mean-bleu}
\end{figure}


\subsection{Exp.\ 2: Comparison of Paraphrasers}
\label{subsec:comp_paraphrasers_setup}

Next, we wanted to evaluate our paraphrasing approaches.
We hence designed two experiments.
The first experiment computes state-of-the-art paraphrasing measures for all paraphrasers on different datasets.
The second experiment aims to evaluate the ability of our two-step models to adhere to instructions.
We tested their proficiency extracting metadata and generating paraphrases of similar length as the reference text.
While the \dataBlog{} ground truth metadata comes with its CSV dataset, the \dataStudent{} metadata is derived from existing information about and in the dataset, and the \dataGutenberg{} metadata is manually curated.
We omit \dataPan{} from this experiment due to infeasible manual metadata curation.

\paragraph{Exp.\ 2(a): Quantitative evaluation.}

To assess the quality of generated paraphrases and the factors influencing it, we designed this first experiment.
We selected one text from the \dataBlog{}, the \dataGutenberg{} and the \dataStudent{} dataset, respectively.
The paraphraser configurations contain two different temperatures (i.e.\ 0 and 1) for two-step paraphrasers, and two different prompts (i.e.\ \texttt{prompt0} and \texttt{prompt1} from \Cref{subsec:one_step_paraphrasing_prompts}) for one-step paraphrasers.
We create one paraphrase for each text configuration pair.
Evaluation measures include \ac{bleu}, \ac{rouge}-1, \ac{rouge}-2, \ac{rouge}-L, \ac{rouge}\-Lsum, \ac{meteor}, \ac{bert}\-Score Precision, \ac{bert}\-Score Recall, \ac{bert}\-Score F1, \ac{sbert} \ac{wms}, \ac{sbert} cosine similarity.
Based on these we also compute average syntactic and average semantic similarity, as well as Gohsen Delta $\Delta_{sem,syn}$~\citep{gohsen_captions_2023}.
While the syntactic similarity $\diameter_{syn}$ is computed by averaging \ac{rouge}-1, \ac{rouge}-L, and BLEU scores, the semantic similarity $\diameter_{sem}$ is calculated averaging \ac{bert}Score, cosine similarity of \ac{bert}-based embeddings, and \acs{glove} \ac{wmd}. 

We save the most extreme (min, max) paraphrases per metric.
The scores are subsequently visualised via syntactic-semantic scatters, score distributions, and radar plots per paraphraser and per prompt. 

\paragraph{Exp.\ 2(b): Evaluation of prompt adherence.}

Our \pextractor{} extracts the genre, topic, and century for each input text. 
Since the two-step paraphrasing approach relies on accurate metadata for subsequent paraphrase generation, we evaluated the quality of the initial extraction step. 
To this end, we conducted a second experiment comparing the extracted metadata with ground truth values. 
For this experiment, we selected five samples each from the \dataBlog{}, \dataGutenberg{}, and \dataStudent{} datasets.
The prompts used for the extraction are provided in \autoref{app:extractor_prompts}.

Both extracted and ground truth values are lowercased and stripped of leading and trailing whitespace. 
Genre and topic values may consist of multiple items separated by commas. 
We split the \pextractor{}'s output for genre by commas to obtain individual genre values. 
Similarly, ground truth topics, which often include multiple items separated by commas, are split into a list. 
Cosine similarities between the \ac{sbert} embeddings of extracted and ground truth values are computed for each genre and topic, with the final score given by the maximum similarity.

For century matching, we preprocess the \pextractor{} output by mapping terms such as \textit{present}, \textit{current}, and \textit{now} to 21, and, for numbers with three or more digits, dropping the last two digits and adding one. 
Examples are shown in \Cref{tab:examples_extract_century}. 
Ground truth century values are first cast to dates before being preprocessed in the same manner.
We compute the ratio $\frac{a}{b}$ where $a$ is the extracted century and $b$ is the ground truth.

\begin{table}[h]
\centering
\caption[Examples for century processing]{Examples for century processing.}
\label{tab:examples_extract_century}
% \resizebox{\textwidth}{!}{%
\begin{tabular}{@{}ll@{}}
    \toprule
\textbf{Original} & \textbf{Processeed} \\
\midrule
1964              & 20                  \\
now               & 21                  \\
190               & 2     \\
\bottomrule             
\end{tabular}%
% }
\end{table}

Moreover, we obtain the relative length difference $d = \frac{p - r}{r}$, where $p$ and $r$ denote the paraphrase and reference text length, respectively. 
Values of $d > 0$ indicate that the paraphrase is longer than the reference, whereas $d < 0$ indicates the opposite.
Paraphrases were generated using the \pgenerator{} with ground truth metadata.
We chose to use ground truth metadata rather than extracted metadata to evaluate the generator's ability to match reference length under ideal conditions without errors from the extraction step.
The prompts used for the generation are provided in \Cref{app:generator_prompts}.


\subsection{Exp.\ 4: Impact of Syntactic Similarity on \impApprTitle{} Effectivity}
\label{sec:syn_sim_impact_}

We designed this experiment in order to assess whether the syntactic similarity of generated paraphrases, i.e.\ the difficulty of hard negatives, influences the effectiveness of the \impAppr{}.
We conducted this experiment on both the \dataBlog{} and \dataStudent{} datasets, selecting 15~samples each from the training and test splits. 
The detector was configured according to Table~\ref{tab:imp_syn_sim_config}.

\begin{table}[h]
\centering\small
\caption{Exp.\ 4: \impAppr{} configuration.}
\label{tab:imp_syn_sim_config}
\begin{tabular}{@{}rlrrl@{}}   % numbers should be right aligned, text left aligned
\toprule
\# Impostors & Generation & Rounds & Top $n$ & Upsample \\
\midrule
50 & LLM & 100 & \num{100000} & False \\
\bottomrule
\end{tabular}%
\end{table}

For generation, we loop through all \ac{llm}-based paraphrasers until we successfully created 50~\imps{}.
One-step paraphrasers are used with both prompts.
Predictions on the test set were obtained by thresholding the detector’s scores with a decision threshold was determined using Youden’s J statistic on the training set.
We computed the average syntactic similarity on the test set. 
Following \citet{gohsen_captions_2023}, we define average syntactic similarity $\diameter_{syn}$ as the mean of the \ac{bleu}, \ac{rouge}-1, and \ac{rouge}-L scores. 
For each input pair in the test set, we calculated
(1) the average syntactic similarity between the two texts in the pair, (2) the mean average syntactic similarity between the candidate reference text and its paraphrases, and (3) the mean average syntactic similarity between the disputed text and the paraphrases.

We further grouped samples based on (1), (2), (3) and the difference (2)–(1). 
For each group, we computed accuracy, precision, recall, and F1 score of the detector’s predictions. 
The average values for each metric in a bin are presented in a bar chart.



\subsection{Exp. 5: Comparing \acs{av} methods in traditional Human-Human scenario}
\label{subsec:imp_gen}

We want to answer the question of how our \ac{llm}-based \imp{} generation performs compared to (a) traditional \imp{} generation in the \impAppr{}~\citep{koppel_determining_2014}, and compared to (b) \ac{sota} \ac{av} methods in the traditional \ac{av} scenario.
We thus, create 100 same- and 100 different-author pairs from the \dataStudent{} 
% 1 each on and the \dataBlog{} datasets % 1 for Blog, 100 for Student Essays
dataset and evaluate the performance of the approaches for different thresholds.
It is noteworthy, that the dataset contains equally many naive same- and different-author pairs and hence, an approach predicting only one output will obtain an accuracy of $0.5$.
The \impAppr{} and unmasking detector configuration are shown in \autoref{tab:exp5_imp_config} and \autoref{tab:exp5_unmasking_config}, respectively.

\begin{table}[h]
\centering\small
\caption{Exp. 5: \impAppr{} configurations.}
\label{tab:exp5_imp_config}
\begin{tabular}{@{}rlrrl@{}}   % numbers should be right aligned, text left aligned
\toprule
\# Impostors & Generation & Rounds & Top $n$ & Upsample \\
\midrule
50 & \textit{Variable} & 100 & \num{100000} & False \\
\bottomrule
\end{tabular}%
\end{table}

\begin{table}[h]
\centering\small
\caption{Exp. 5: Unmasking configurations.}
\label{tab:exp5_unmasking_config}
\begin{tabular}{@{}rrrrl@{}}   % numbers should be right aligned, text left aligned
\toprule
\# CV Folds & \# Chunks & Rounds & Top $n$ & Upsample \\
\midrule
3 & 60 & 30 & \num{250} & False \\
\bottomrule
\end{tabular}%
\end{table}

For each impostor generation method, we computed accuracy, precision, recall, and F1 score for different thresholds. 


\subsection{Exp. 6: Comparing \ac{av} Methods in \acs{llm} author scenarios}

This experiment evaluates the performance of the \impAppr{} in comparison to established \ac{av} methods.
As baselines, we employ generalized unmasking~\citep{bevendorff_generalizing_2019} and the compression-based approach PPMD approach and baselines from the original work~\citep{koppel_determining_2014}.
Different to experiment 5, each input pair contains at least one \ac{llm} generated text.
We compare three scenarios.
The first one simulates \ac{llm} detection, i.e. output true if the disputed text is generated by an \ac{llm}.
We design our dataset, such that candidate texts are \ac{llm} generated and the disputed texts can be either \ac{llm} generated or human authored.
A pair receives the truth label if the disputed text is \ac{llm} generated irrespective of the model used.
The second group is a \ac{llm} \ac{av} scenario where the dataset is the same as the one before, but we only label same author pairs with true.
The third scenario contains only \ac{llm} generated texts.
We denote this a more difficult \ac{llm} \ac{av} scenario.


Following the experimental setup described in \autoref{tab:av_comp} and \autoref{tab:exp6_unmasking_config} for the \impAppr{} and unmasking respectively, we assess performance using  accuracy, precision, recall, and the F1 score. 

\begin{table}[h]
\centering\small
\caption{Exp. 6: \impAppr{} configurations.}
\label{tab:av_comp}
\begin{tabular}{@{}rlrrl@{}}   % numbers should be right aligned, text left aligned
\toprule
\# Impostors & Generation & Rounds & Top $n$ & Upsample \\
\midrule
50 & \textit{Variable} & 100 & \num{100000} & False \\
\bottomrule
\end{tabular}%
\end{table}

\begin{table}[h]
\centering\small
\caption{Exp. 6: Unmasking configurations.}
\label{tab:exp6_unmasking_config}
\begin{tabular}{@{}rrrrl@{}}   % numbers should be right aligned, text left aligned
\toprule
\# CV Folds & \# Chunks & Rounds & Top $n$ & Upsample \\
\midrule
3 & 60 & 30 & \num{250} & False \\
\bottomrule
\end{tabular}%
\end{table}





 % ggf experimente setup extra/ oder extra + datensätze + exp+ analyse (Tabelle)
    \chapter{Experimental Results}
\label{chap:experimental_results}

This chapter presents the experimental results. 
In \Cref{sec:reproduction_res}, we begin by reproducing the original experiments to validate our implementation. 
We then compare different paraphrasing techniques in \Cref{sec:comp_paraphrases}, followed by an analysis of paraphrase quality across varying text chunk sizes in \Cref{sec:results_chunks} and across prompt designs in \Cref{sec:prompt_impact_res}. 
Finally, \Cref{sec:imp_gen_res} contrasts traditional \ac{av} methods with the \ac{llm}-based \impAppr{}.


\section{Exp. 1: Reproduction of Original Work}

Our first experiment covers the reproduction of the original results attached in \citet{koppel_determining_2014}'s paper.
It is noteworthy, that we do not expect exact replication since we could only reimplement the approach to our best knowledge.
Both Mr. Koppel and Mr. Winter were very forthcoming when we contacted them regarding implementation details that could help to improve our implementation.
Unfortunately, the code was no longer traceable and neither author could recall exactly what preprocessing steps included.
After consultation with Mr. Winter, he assured that our preprocessing steps seem reasonable.
This, however, certainly is one of the reasons why our implementation diverges from the original work.

\paragraph{Exp. 1(a): Varying number of \imps{}}

For this experiment, we set the \imp{} generation method to \texttt{fix} and varied only the number of \imps{}.
We only have reference results of this experiment on \dataBlog{} from the original paper.
Fortunately, though the form of the original precision-recall plot slightly differs, we obtain the same result:
In terms of recall, 50 \imps{} achieve the best results.

\begin{figure}[htbp]
  \centering
  \begin{subfigure}[b]{0.48\textwidth}
    \centering
    \includesvg[width=\linewidth]{images/imposter/reproduction_koppel_figures/fig2/student_essays/student_roc_prec_recall_curve_r100_top100000_dif_n_imp.svg}
    \caption{\dataBlog{} \textcolor{red}{TODO: runs (22.08.2025)}}
    \label{fig:blog_dif_n}
  \end{subfigure}
  \hfill
  \begin{subfigure}[b]{0.48\textwidth}
    \centering
    \includesvg[width=\linewidth]{images/imposter/reproduction_koppel_figures/fig2/student_essays/student_roc_prec_recall_curve_r100_top100000_dif_n_imp.svg}
    \caption{\dataStudent{}}
    \label{fig:student_essays_dif_n}
  \end{subfigure}
  \caption{Recall-precision curves for the various sized \imp{} set sizes.}
  \label{fig:repr_diff_n_imps_fixed}
\end{figure}


\paragraph{Exp. 1(b): Varying \imp{} generation}

Adhereing to \citet{koppel_determining_2014}, we compare the \impAppr{} for fixed, and on-th-fly \imp{} generation with \ac{svc}, and similarity based baselines.
They consider both the same-author and the different-authors class the reference class once.
The original work included only the results for the \dataBlog{} dataset.

\begin{figure}[htbp]
  \centering
  \begin{subfigure}[b]{0.495\textwidth}
    \centering
    \includesvg[width=\linewidth]{images/imposter/reproduction_koppel_figures/fig4/blog/blog_roc_prec_recall_curve_r100_top100000_Same_Author_dif_imp_gen.svg}
    \caption{\dataBlog{}}
    \label{fig:blog_same_author}
  \end{subfigure}
  \hfill
  \begin{subfigure}[b]{0.495\textwidth}
    \centering
    \includesvg[width=\linewidth]{images/imposter/reproduction_koppel_figures/fig4/student_essays/student_roc_prec_recall_curve_r100_top100000_Same_Author_dif_imp_gen.svg}
    \caption{\dataStudent{}}
    \label{fig:student_essays_same_author}
  \end{subfigure}
  \caption{Recall-precision curves for the class same-author. Due to API limit restrictions, the test set for on-the-fly was smaller which is visible in the respective curves.}
  \label{fig:same_authors}
\end{figure}

For the same-author reference class, both \citet{koppel_determining_2014} and us find that the fixed \imp{} generation approach generally performs best across different threshold.
While in the original work, the \ac{svc} baseline performed best from the three baselines, we find the opposite to be true.
Generally, our results perform worse than the original. 
While \citet{koppel_determining_2014} obtain precision higher than $0.9$ for recall values up to $0.7$, we drop to sub $0.8$ precision fairly quickly.
Except for the on-the-fly results, the overall form resembles the original work roughly.

\begin{figure}[htbp]
  \centering
  \begin{subfigure}[b]{0.495\textwidth}
    \centering
    \includesvg[width=\linewidth]{images/imposter/reproduction_koppel_figures/fig4/blog/blog_roc_prec_recall_curve_r100_top100000_Different_Author_dif_imp_gen.svg}
    \caption{\dataBlog{}}
    \label{fig:blog_different_author}
  \end{subfigure}
  \hfill
  \begin{subfigure}[b]{0.495\textwidth}
    \centering
    \includesvg[width=\linewidth]{images/imposter/reproduction_koppel_figures/fig4/student_essays/student_roc_prec_recall_curve_r100_top100000_Different_Author_dif_imp_gen.svg}
    \caption{\dataStudent{}}
    \label{fig:student_essays_different_author}
  \end{subfigure}
  \caption{Recall-precision curves for the class different-author. Due to API limit restrictions, the test set for on-the-fly was smaller which is visible in the respective curves.}
  \label{fig:different_authors}
\end{figure}

For different authors as reference class, our results greatly differ from the original work.
Even though, \citet{koppel_determining_2014}'s precision scores for fixed and on-the-fly \imp{} generation are generally lower than those for the same-author reference class, our scores are way lower.
Notably, in our case, the on-the-fly method seems to perform best in this open-set scenario.


\section{Exp.\ 2: Comparison of Different Paraphrasers}
\label{sec:comp_paraphrases}

To evaluate the quality of generated paraphrases, we conducted two experiments. 
In Exp.\ 2(a), we assessed paraphrasing using standard quantitative metrics, while in Exp.\ 2(b), we compared the extracted information and text lengths of the generated paraphrases to the ground truth metadata and original texts, respectively.

\paragraph{Exp.\ 2(a): Quantitative evaluation.}

Paraphrasing scores were computed separately for the \dataBlog{}, \dataGutenberg{}, and \dataStudent{} datasets. 
\autoref{fig:sem_syn_blog} presents aggregated semantic and syntactic measurement scores for the \dataBlog{} dataset, while results for \dataGutenberg{} are provided in the Appendix in \autoref{sec:app_paraphrases}.

Syntactic similarity was quantified by averaging \ac{rouge}-1, \ac{rouge}-L, and BLEU scores. 
Semantic similarity was assessed using BERTScore, cosine similarity of BERT-based embeddings, and SBERT \ac{wmd}. 
It is important to note that for our purposes, syntactic diversity is desirable.
High syntactic similarity values may reflect near-identical paraphrases due to n-gram overlap. 
Semantic similarity measures the content overlap between the paraphrase and the original text, often via vector-based cosine similarity. 
The precise interpretation of these metrics remains somewhat unclear.

\begin{figure}[htbp]
    \centering
    \includesvg[width=\textwidth]{images/paraphrasing/experiments/sem_syn_scatter/Blog_sem_syn_scatter_grouped_by_Paraphraser.svg}
    \caption[Comparison of paraphrasers on the \dataBlog{} dataset.]{Average semantic $\diameter_{sem}$ and syntactic similarity $\diameter_{syn}$ for different paraphraser on the \dataBlog{}.}
    \label{fig:sem_syn_blog}
\end{figure}


Analysis reveals two distinct clusters corresponding to one-step and two-step paraphrasers. 
Most one-step paraphrasers achieve lower syntactic and semantic similarity than two-step paraphrasers. 
The exception is the one-step paraphraser using \texttt{qwen3-32b}, which exhibits higher syntactic similarity than most two-step paraphrasers. 
The translation-based approach emerges as an outlier in terms of both syntactic and semantic similarity. 
Different prompt formulations appear to have minimal impact on these similarity scores (cf. \autoref{sec:app_paraphrases}). 
Overall, most paraphrases fall within the desired quadrant of high semantic similarity with low syntactic similarity.


\paragraph{Exp.\ 2(b): Evaluation of prompt adherence.}

Our two-step paraphrasing approach relies on extracting valid information from the source text. 
To evaluate the quality of extraction by the \pextractor{}, we compared the topic, genre, and century to the ground truth metadata for the \dataBlog{}, \dataGutenberg{}, and \dataStudent{} datasets. 
Genre and topic were evaluated in terms of semantic similarity, while century was assessed via the percentage deviation from the ground truth.

We observed that instructions for the \pextractor{} must follow the input text, as \acp{llm} tend to focus attention toward the end of the input. 
Otherwise, the \pextractor{} failed to produce the requested JSON format for long texts from the \dataGutenberg{} dataset. 
Results across datasets are summarized in \autoref{tab:extraction_eval_stats}, with $\diameter$ and $\sigma$ denoting the mean and standard deviation across the five selected samples. 
The \dataBlog{} dataset proved most challenging for genre and topic extraction. 
While differences in text length between reference and paraphrase were maximal among all datasets, the \pextractor{} performed best on the \dataGutenberg{} dataset.


\begin{table}[h]
\centering
\caption[Extraction performance and length matching for different datasets.]{Extraction performance and length matching for different datasets. Five documents per dataset were processed using the \pextractor{} to obtain genre, century, and topic.}
\label{tab:extraction_eval_stats}
\begin{tabular}{lllllllll}
\toprule
 &
  \multicolumn{2}{l}{\textbf{Genre}} &
  \multicolumn{2}{l}{\textbf{Century}} &
  \multicolumn{2}{l}{\textbf{Topic}} &
  \multicolumn{2}{l}{\textbf{Length}} \\
  \textbf{Dataset}
 &
  \textbf{\diameter} &
  \textbf{$\sigma$} &
  \textbf{\diameter} &
  \textbf{$\sigma$} &
  \textbf{\diameter} &
  \textbf{$\sigma$} &
  \textbf{\diameter} &
  \textbf{$\sigma$} \\
  \midrule
\dataBlog{}            & 0.38 & 0.06  & 0.99 & 0.02 & 0.04  & 0.05  & -0.10 & 0.73 \\
\dataGutenberg{}       & 0.58 & 0.14  & 1.00 & 0.04 & 0.3 & 0.15 & -1.00 & 0.00  \\
\dataStudent{} & 0.53 & 0.26 & 0.60 & 0.55 & 0.25 & 0.05  & 0.34 & 0.20 \\
  \bottomrule
\end{tabular}%
\end{table}

Notably, paraphrases generated in other experiments were often substantially shorter than the reference texts. 
In multiple cases, the \pgenerator{} returned placeholders such as \texttt{I’m sorry, but I can’t help with that}.


\section{Exp. 3: Paraphrasing Chunks}

Initially, we hypothesized that smaller chunks of text would lead to better paraphrases, 
since smaller chunks are easer to process and control in terms of topic,
suggesting that the separation of text into smaller chunks would be beneficial for the paraphrasing process.
We therefore designed an ablation study to test this hypothesis.
We computed several paraphrasing measurements for the same input texts averaged over the number of chunks.
As visualized in \autoref{fig:abl_chunks_T5_Google_PAWS} and \autoref{fig:abl_chunks_BulletPoint},
the syntactic scores of naive paraphrasers (here: \ac{t5} model) increase with the number of chunks,
while the semantic scores remain stable.
This leads to a decreasing Gohsen Delta score, which is the difference between the semantic and syntactic scores.
In contrast, the non-naive paraphrasers (here: BulletPoint model) do not show any significant change in the paraphrasing scores with the number of chunks.
This suggests that the naive paraphrasers are more sensitive to the number of chunks, 
while the non-naive model is not affected by it.
Since a large Gohsen Delta score indicates a good paraphrase,
the results suggest that the naive paraphrasers perform better with fewer chunks, 
while the non-naive paraphrasers are more robust to the number of chunks.
This finding indicates that more chunks allow for better syntactic control, which is not necessarily beneficial for the quality of the paraphrase.
As we are interested in syntactically diverse paraphrases,
we will not use chunks for the paraphrasing process but rather stick to text-to-text paraphrases.
 
\begin{figure}[htbp]
    \centering
    \includegraphics[width=\textwidth]{images/paraphrasing/experiments/T5_Google_PAWS_metrics_plot.png}
    \caption{Average score over different prompts (standard deviation shaded) for different paraphrasing scores for the \ac{t5} model.
    The syntactic scores rise with the number of chunks, while the semantic scores is stable.
    Consequently, the Gohsen Delta score is decreasing with the number of chunks.}
    \label{fig:abl_chunks_T5_Google_PAWS}
\end{figure}

\begin{figure}[htbp]
    \centering
    \includegraphics[width=\textwidth]{images/paraphrasing/experiments/BulletPoint_metrics_plot.png}
    \caption{Different paraphrasing scores for the BulletPoint model. 
    This model is not affected by the number of chunks.}
    \label{fig:abl_chunks_BulletPoint}
\end{figure}

\section{Exp.\ 4: Comparing Prompts}%Assessing the Impact of the Prompt on Paraphrasing}
\label{sec:prompt_impact_res}

In this experiment, we investigate how different prompting strategies influence the quality of paraphrases generated by \acp{llm}. 
To this end, we measured the relative length difference between reference and paraphrase pairs across different \ac{llm}–prompt combinations. 
A subset of pairs was also manually inspected to assess semantic fidelity and readability.

Post-processing was required to remove reasoning traces present in some model outputs, particularly in generations from models such as \texttt{qwen3-32b}. 
These traces, typically delimited by \texttt{</think>}, consist of repeated fragments of the input prompt and do not contribute to the semantic content of the paraphrase. 
We excluded them to retain only the task-relevant text produced by the \ac{llm}.

After post-processing, we computed the relative length difference between each reference and its corresponding paraphrase for all model–prompt combinations. 
The distribution of these differences is presented in \Cref{fig:prompt_impact_post_processed}. 
Because our objective in \imp{} generation is to control for confounding variables, a paraphrase length close to that of the reference is interpreted as an indicator of higher paraphrase quality. 
We additionally performed a manual assessment of content quality that focused on paraphrases both extremely long and length-balanced relative to the reference.

\begin{figure}[H]
    \centering
    \includesvg[width=\textwidth]{images/prompt_impact/paraphraser_length_distribution_post_process_len_perc(qwen)_linear.svg}
    \caption[Impact of different prompts on paraphrases]{
    Box plots of relative paraphrase lengths after post-processing across different prompts.    
    The dotted gray line marks the optimal paraphrase length.
    \texttt{prompt2} consistently generates paraphrases whose lengths are more comparable to the reference than those produced by the other two prompts.
    }
    \label{fig:prompt_impact_post_processed}
\end{figure}

Our results show that the relative length difference of paraphrases strongly depends on the prompt used to instruct the \ac{llm}. 
Notably, the third prompt, i.e.\ \texttt{prompt2}, explicitly instructed models to generate paraphrases three times longer than the reference. 
While this instruction might seem extreme, it consistently generated paraphrases whose lengths were more comparable to the reference than those produced by the other two prompts, across different models, as summarised in \Cref{tab:impact_prompts_paraphrases_lengths}.

\begin{table}[h]
\centering
\caption[Impact of different prompts on paraphrase lengths]{Impact of the prompts on paraphrase lengths. 
Relative length difference is defined as $\frac{\mathrm{len}(paraphrase)}{\mathrm{len}(reference)}\times 100\%$ and denoted $d$. 
Optimal paraphrases are expected to approximate the reference length, i.e.\ $\diameter d \approx 100$. 
Subscript $pp$ indicates post-processed outputs (with reasoning traces removed). 
``Count'' denotes the number of paraphrases considered for each setting. 
For \texttt{prompt2}, only post-processed results are reported.
Bold \diameter $r_{pp}$ values are those closest to the optimal paraphrase length.
}
\label{tab:impact_prompts_paraphrases_lengths}
\resizebox{\textwidth}{!}{%
\begin{tabular}{@{}llrrrrr@{}}
\toprule
Paraphraser & Prompt  & \diameter $d$ & $\sigma d$ & \diameter $d_{pp}$ & $\sigma d_{pp}$ & Count \\
\midrule
meta-llama-3.1-8b-instruct & prompt0 & 39.93 & 52.64 & 39.93 & 52.64  & 135   \\
                            & prompt1 & 40.27  & 24.21 & 40.27  & 24.21 & 124 \\
                            & prompt2 & - & - & \textbf{98.15} & 24.97 & 639  \\
mistral-large-instruct & prompt0 & 1.89   & 1.0   & 1.89   & 1.0   & 138 \\
                        & prompt1 & 13.09  & 17.96 & 13.09  & 17.96 & 129 \\
                        & prompt2 & - & - & \textbf{79.70}  & 11.40 & 449  \\
openai-gpt-oss-120b   & prompt0 & 5.53   & 13.47 & 5.53   & 13.47 & 139 \\
                        & prompt1 & 19.21  & 25.0  & 19.21  & 25.0  & 129 \\
                        & prompt2 & - & - & \textbf{147.54} & 48.45 & 590  \\
qwen3-32b           & prompt0 & 88.36  & 70.02 & 18.68  & 24.09 & 134 \\
                        & prompt1 & 95.73  & 47.72 & 38.34  & 15.64 & 123 \\
                        & prompt2 & - & - & \textbf{81.42}  & 13.46 & 532 \\
                                \bottomrule
\end{tabular}%
}
\end{table}

Manual inspection indicated that paraphrases generated with \texttt{prompt2} exhibited only mild hallucinations and generally remained on topic. 
Moreover, paraphrases generated with \texttt{prompt2} outperformed paraphrases from other prompts in terms of semantic preservation across all \acp{llm}.

Based on these findings, subsequent experiments adopted the following design choices: 
(1) exclude paraphrases generated with \texttt{prompt0} and \texttt{prompt1}, 
(2) remove all \texttt{</think>}–delimited reasoning traces via post-processing, and 
(3) discard paraphrases shorter than $60\%$ of the reference length.


\section{Exp.\ 5: Comparing Authorship Verification Methods}% in Traditional Human-Human Scenario}
\label{sec:imp_gen_res}

We evaluate precision–recall performance across multiple thresholds for the original baseline methods, the traditional \impAppr{} using the \texttt{fixed} \imp{} selection strategy proposed by \citet{koppel_determining_2014}, \unmasking{}, \ac{ppmd}, and the \ac{llm}-based \impAppr{}.
Results are obtained on a subset of 40 text pairs from the \dataStudent{} dataset.

\begin{figure}[htbp]
    \centering
    \includesvg[width=0.9\textwidth]{images/imposter/our_contribution/roc_prec_recall_curve_r100_top100000_Same_Author_dif_imp_gen.svg}
    \caption[Recall-precision curves for the \dataStudent{}]{
        Recall-precision curves for 20 samples per class of the \dataStudent{}. 
        (B)~indicates the original baseline approaches from~\citep{koppel_determining_2014}.
    }
    \label{fig:comp_naive_student}
\end{figure}

In terms of optimising the precision–recall trade-off, the \ac{llm}-based \impAppr{} performs best. 
At a threshold of $0.105$, it achieves a precision of $0.73$ and recall of $0.8$, outperforming other approaches for certain operating points. 
However, when prioritising precision over recall, the traditional \impAppr{} employing \texttt{fixed} \imp{} sampling, the unsupervised min-max similarity baseline~\citep{koppel_determining_2014}, and \ac{ppmd} achieve higher recall while maintaining maximum precision. These results contrast with our initial expectation that \ac{llm}-based \imp{} generation would yield higher precision but lower recall than the traditional \texttt{fixed} strategy.

While the optimal balance between precision and recall is application-dependent, a recall of at most $0.4$ at perfect precision generally provides limited information compared to jointly optimising both metrics. Considering \ac{auc}-PR values, all approaches achieve scores between $0.49$ and $0.69$, with our approach positioned in the middle.

\begin{table}[h]
\centering
\label{tab:auc_pr}
\caption[\ac{auc} Precision-Recall results]{\ac{auc} Precision-Recall scores of different \ac{av} approaches on the \dataStudent{} dataset. 
Our approach is highlighted in bold.
}
% \resizebox{\textwidth}{!}{%
\begin{tabular}{lr}
\toprule
\ac{av} approach           & \ac{auc}-PR \\
\midrule
\acs{ppmd}                       & 0.69   \\
Fixed                      & 0.68   \\
Unsup. Min-Max (B)         & 0.68   \\
One-Step Paraphraser (\ac{llm}) & \textbf{0.64}   \\
Unsup. Cosine (B)          & 0.52   \\
\unmasking{}                  & 0.51   \\
Sup. \ac{svm} (B)               & 0.49  \\
\bottomrule
\end{tabular}%
% }
\end{table}



    \chapter{Discussion}
\label{chap:discussion}

Our baseline and reproduction experiments yielded results lower than previously reported by \citet{koppel_determining_2014}, which limits the extent to which we can fully validate the original \impAppr{}. 
While Mr.~Winter considered our preprocessing strategy reasonable, he noted that our on-the-fly \imp{} generation was constrained by (1) bot-prevention mechanisms restricting web scraping, and (2) severe limits on API calls.  

% Interpretation of results
% What do the findings mean in relation to your research questions or hypotheses?
The results suggest that controlled \ac{llm}-generated \imps{} can, in principle, serve as more effective hard negatives than traditional methods in \ac{av} scenarios restricted to human-authored texts. 
Our one-step \imp{} generation outperformed the baselines when precision and recall were weighted equally.  
However, employing \ac{llm}-generated paraphrases as hard negatives requires extensive exploration of prompting strategies.  
We observed that paraphrase length is highly sensitive to prompt design. 
Although paraphrase scores indicate that outputs are semantically faithful and syntactically diverse, insufficient manual inspection can result in overly short paraphrases, which in turn reduce \impAppr{} effectiveness.  
Additionally, the ordering of prompt and text affects prompt adherence, particularly for longer texts, as \acp{llm} tend to prioritize the end of the input.  
Finally, artefacts specific to the chosen \ac{llm} must be filtered from outputs to maintain quality. 

As a consequence, this approach is neither off-the-shelf nor general-purpose. 
It requires task-specific solutions and careful management of internal parameters, making it computationally and operationally expensive.


Concerning the second research question (\autoref{enum:rq2}), we found that a large variety of syntactic and semantic paraphrase metrics used in the literature complicates interpretation. 
Following \citet{gohsen_captions_2023}, we aggregated syntactic and semantic scores to enable a clearer two-dimensional comparison.  
Interestingly, low syntactic similarity emerged as an unintuitive predictor of paraphrase quality. 
Measures of syntactic dissimilarity may better align with human judgment.  
Even when syntactic scores suggested good paraphrases, this sometimes reflected overly short outputs, which we consider poor quality since paraphrase length is a key confounding variable in hard negative mining.  
In our implementation, \ac{rouge}-Lsum was redundant, producing results identical to \ac{rouge}-L.  

We further introduced metrics for \pextractor{} effectiveness and text length preservation. 
Metadata extraction for genre and topic performed poorly, whereas century extraction was reliable. 
Paraphrases of long \dataGutenberg{} texts deviated substantially from the original in terms of length.  


Addressing the third research question (\autoref{enum:rq3}), we compared \ac{av} methods in the traditional human-authored text scenario, using \unmasking{} and \ac{ppmd} as baselines to evaluate the extended \impAppr{}.
Performance proved highly sensitive to prompt design, highlighting a key challenge when using \acp{llm}. 
Prompt engineering remains largely ad-hoc, shifting focus from framework development to instruction crafting. 
Importantly, there is no single prompt that works universally for all \acp{llm}. 
Effective prompts depend, among other factors, on the size of the model. 
Larger \acp{llm} can leverage intrinsic knowledge under more permissive prompts, whereas smaller models require carefully tailored instructions~\citep{schmidt_llm_av_latin_24}.
Once an effective prompt generating sufficiently long paraphrases was identified for the specific \ac{llm}, our approach achieved competitive results relative to the baselines proposed by \citet{koppel_determining_2014}, demonstrating that prompt tuning is a critical factor in practical performance.

Theoretically, extending the \impAppr{} should increase precision at the cost of recall, as harder negatives reduce same-author predictions. 
Interestingly, this trade-off did not manifest as expected, suggesting that we have not yet comprehended the effect of artificial generated hard negatives for the \impAppr{}.
    \chapter{Conclusion}
\label{chap:conclusion}

We explored extending the traditional \impAppr{} with \ac{llm}-generated \imps{}, addressing (1) whether \acp{llm} can produce more effective hard negatives, (2) which measures best evaluate paraphrases, and (3) how the extended approach compares to traditional baselines in the standard \ac{av} setting.

Our findings indicate that \acp{llm} are capable of generating paraphrases that can serve as strong hard negatives. 
However, the effectiveness of \ac{av} critically depends on prompt design. 
Constructing effective prompts is non-trivial, making the approach neither off-the-shelf nor general-purpose. 
Poorly designed prompts may produce paraphrases that appear favourable according to low syntactic similarity metrics but are overly short, undermining their utility as hard negatives and increasing \acp{fp}. 
In addition, the approach is computationally intensive, limiting its practical applicability. 
Despite these challenges, the \ac{llm}-based \impAppr{} outperformed the other \ac{av} approaches at specific operating points in the precision–recall space.

We further found that aggregated syntactic and semantic similarity measures facilitate interpretable evaluation. Moreover, low syntactic similarity alone is not a reliable indicator of paraphrase quality. 
% Overall, our results suggest that \ac{llm}-based \imp{} generation is not yet a practical or robust extension for general-purpose \ac{av}, though it may hold promise in carefully constrained, task-specific scenarios.  



% \section{Future Work}

Several directions emerge from the limitations identified in this study. 
Improving paraphrase quality remains a central priority. 
Future work should focus on designing more effective and potentially generalisable prompting strategies to ensure that paraphrases maintain semantic fidelity, controlled length, and sufficient syntactic variation.  
%
Furthermore, the potential of two-step paraphrase generation should be explored further. 
Although implemented in this study, time constraints prevented an assessment of its impact on \impAppr{} predictions. 

Reproducibility and data collection also require attention. Standardised datasets, transparent preprocessing pipelines, and mitigation of practical obstacles such as bot-prevention mechanisms and API limitations are essential for consistent experimentation.

Expanding the range of text types to shorter texts and more diverse domains such as \dataPan{}, and \ac{llm}-generated pairs would provide a more comprehensive assessment and demonstrate potential for \ac{llm} detection.

Finally, incorporating abstention mechanisms to withhold predictions under high uncertainty could enhance the reliability of \ac{av} results.

    \section*{Acknowledgements}

The authors gratefully acknowledge the computing time granted by the KISSKI project. 
The calculations for this research were conducted with computing resources under the project \textcolor{red}{<ID of your project>}.

We also thank Mr.~James W.~Pennebaker for granting access to the original data of the \impAppr{}, as well as Mr.~Moshe Koppel and Mr.~Yaron Winter for their valuable and forthcoming communication, which informed the development of this work.
    \appendix
\chapter{First Appendix}
\label{ch:appendix}


\section{Extractor prompts used for Two-Step Paraphrasing}
\label{app:extractor_prompts}
% bullet point
\paragraph{Bullet point approach}
\begin{quote}
\textit{
Summarize the text above in five to six short bullet points. 
Respond ONLY with a JSON object in the following format: \\
$\{$ \\
\hspace{1em}"bullet\_points": "<list of bullet points>", \\
\hspace{1em}"tone": "<tone>", \\
\hspace{1em}"time\_period": "<time\_period>", \\
\hspace{1em}"language\_register": "<register>", \\
\hspace{1em}"target\_audience": "<target\_audience>", \\
\hspace{1em}"genre": "<genre>" \\
$\}$. 
Do not use direct quotes.
}
\end{quote}


% task
\paragraph{Task approach}
\begin{quote}
\textit{
Act as the author of the text above. From that perspective, infer your role or identity, the topic being addressed, and the purpose or instruction behind writing the text. 
Combine these elements into a concise task prompt that you would give to an LLM to reproduce the text. 
Respond ONLY with a JSON object in the following format: \\
$\{$ \\
\hspace{1em}"task": "<task>", \\
\hspace{1em}"tone": "<tone>", \\
\hspace{1em}"time\_period": "<time\_period>", \\
\hspace{1em}"language\_register": "<register>", \\
\hspace{1em}"target\_audience": "<target\_audience>", \\
\hspace{1em}"genre": "<genre>" \\
$\}$. 
Do not use direct quotes.
}
\end{quote}


% topic
\paragraph{Topic approach}
\begin{quote}
\textit{
Extract the topic, tone, time period, register, target audience, and genre from the text above. 
Respond ONLY with a JSON object in the following format: \\
$\{$ \\
\hspace{1em}"topic": "<topic>", \\
\hspace{1em}"tone": "<tone>", \\
\hspace{1em}"time\_period": "<time\_period>", \\
\hspace{1em}"language\_register": "<register>", \\
\hspace{1em}"target\_audience": "<target\_audience>", \\
\hspace{1em}"genre": "<genre>" \\
$\}$. 
Do not use direct quotes.
}
\end{quote}


% title
\paragraph{Title approach}
\begin{quote}
\textit{
Find a concise title for the text and extract the tone, time period, register, target audience, and genre from the text above. 
Respond ONLY with a JSON object in the following format: \\
$\{$ \\
\hspace{1em}"title": "<title>", \\
\hspace{1em}"tone": "<tone>", \\
\hspace{1em}"time\_period": "<time\_period>", \\
\hspace{1em}"language\_register": "<register>", \\
\hspace{1em}"target\_audience": "<target\_audience>", \\
\hspace{1em}"genre": "<genre>" \\
$\}$. 
Do not use direct quotes.
}
\end{quote}


\section{Generator prompts used for Two-Step Paraphrasing}
\label{app:generator_prompts}

% bullet point
\paragraph{Bullet point approach}
\begin{minted}{python}
prompt = "Write a text which covers the following items:\n" 
    + "\n".join(f"- {bp}" for bp in bullet_points)
\end{minted}

% task
\paragraph{Task approach}
\begin{minted}[breaklines]{python}
generator_prompt = "Write a text of about {l} words with a {tone} tone, a {genre} genre, in the {register} register for the target audience of {target_audience} and in the {time_period} time period, covering the following task:\n{task}. Do not use asterisks. Only output the text without any additional commentary.".format(
            l=len(text.split()),
            tone=tone,
            genre=genre,
            task=task,
            time_period=time_period,
            register=register,
            target_audience=target_audience,
        )
\end{minted}

% topic
\paragraph{Topic approach}
\begin{minted}[breaklines]{python}
generator_prompt = "Write a text of about {l} words with a {topic} topic, {tone} tone, a {genre} genre, in the {register} register for the target audience of {target_audience} and in the {time_period} time period. Do not use asterisks. Only output the text without any additional commentary.".format(
            l=len(text.split()),
            tone=tone,
            genre=genre,
            topic=topic,
            time_period=time_period,
            register=register,
            target_audience=target_audience,
        )
\end{minted}

% title
\paragraph{Title approach}
\begin{minted}[breaklines]{python}
generator_prompt = "Write a text of about {l} words with a {title} title, {tone} tone, a {genre} genre, in the {register} register for the target audience of {target_audience} and in the {time_period} time period. Do not use asterisks. Only output the text without any additional commentary.".format(
            l=len(text.split()),
            tone=tone,
            genre=genre,
            title=title,
            time_period=time_period,
            register=register,
            target_audience=target_audience,
        )
\end{minted}

\section{Preprocessing Regular Expressions}
\label{app:regex_preproc}

TODO
% \begin{table}[]
% \centering
% \caption{Regular Expressions used for preprocessing}
% \label{tab:preproc_regex}
% \resizebox{\textwidth}{!}{%
% \begin{tabular}{ll}
%     \toprule
% \textbf{Description} & \textbf{Regular Expression}      
% \midrule                                                                                                        \\
% html tags            & r"\textless{}{\[}\textasciicircum{}\textgreater{}{\]}+\textgreater{}"                                                                      \\
% play artifacts       & r"\textasciicircum{}{\[}A-Z\textbackslash{}s{\]}+{\[}.:{\]}?\$"                                                                              \\
% chapter artifacts    & r"\textasciicircum{}\textbackslash{}s*Chapter\textbackslash{}s+\textbackslash{}w+.*\textbackslash{}n\textbackslash{}s*\textbackslash{}n" \\
% play artifacts &
%   \begin{tabular}[c]{@{}l@{}}r"\textbackslash{}s*ACT\textbackslash{}s+\textbackslash{}w+\textbackslash{}b\textbackslash{}.?"\\ r"\textbackslash{}s*SCENE\textbackslash{}s+\textbackslash{}w+\textbackslash{}b\textbackslash{}.?"\\ r"\[(.*?)\]" -\textgreater r"\textbackslash{}1"\\ r"\_+"r"\textbackslash{}s+\textbackslash{}d+\textbackslash{}s*\$"\end{tabular} \\
% whitespace           & r"\textbackslash{}s+"  \\
% \bottomrule                                                                                                                 
% \end{tabular}%
% }
% \end{table}



\section{Exp. 1(b): \impAppr{} configuration.}

\begin{figure}[H]
  \centering
  \begin{subfigure}{0.75\textwidth}
    \centering
    \includesvg[width=\linewidth]{images/imposter/reproduction_koppel_figures/fig4/student_essays/student_roc_prec_recall_curve_r100_top100000_Same_Author_dif_imp_gen.svg}
    \caption{Same author reference class}
    \label{fig:student_essays_same_author}
  \end{subfigure}
  \hfill
  \begin{subfigure}{0.75\textwidth}
    \centering
    \includesvg[width=\linewidth]{images/imposter/reproduction_koppel_figures/fig4/student_essays/student_roc_prec_recall_curve_r100_top100000_Different_Author_dif_imp_gen.svg}
    \caption{Different author reference class}
    \label{fig:student_essays_different_author}
  \end{subfigure}
  \caption{Recall-precision curves for the \dataStudent{}. 
Due to API limit restrictions, the test set for on-the-fly was smaller which is visible in the respective curves.
  }
  \label{fig:diff_imp_gen_student_essays}
\end{figure}

\section{Exp.\ 2: Comparison of Paraphrasers}
\label{sec:app_paraphrases}

\begin{figure}[H]
    \centering
    \includesvg[width=\textwidth]{images/paraphrasing/experiments/radar/Blog_paraphrasing_metrics_grouped_by_Paraphraser_radar_chart_altered.svg}
    \caption[Paraphrase evaluation radar chart on the \dataBlog{} dataset]{Radar chart of syntactic and semantic paraphrase scores for different paraphrasers on the \dataBlog{} dataset.}
    \label{fig:radar_blog}
\end{figure}


\begin{figure}[H]
    \centering
    \includesvg[width=\textwidth]{images/paraphrasing/experiments/sem_syn_scatter/Gutenberg_sem_syn_scatter_grouped_by_Paraphraser_altered.svg}
    \caption[Comparison of paraphrasers on the \dataGutenberg{} dataset]{Average semantic and syntactic similarity for different paraphraser on the \dataGutenberg{}.}
    \label{fig:sem_syn_gutenberg}
\end{figure}

\begin{figure}[H]
    \centering
    \includesvg[width=\textwidth]{images/paraphrasing/experiments/radar/Gutenberg_paraphrasing_metrics_grouped_by_Paraphraser_radar_chart_altered.svg}
    \caption[Paraphrase evaluation radar chart on the \dataGutenberg{} dataset]{Radar chart of syntactic and semantic paraphrase scores for different paraphrasers on the \dataGutenberg{} dataset.}
    \label{fig:radar_gutenberg}
\end{figure}


\section{Exp.\ 3: Paraphrasing Chunks}
\label{sec:app_chunks}

\begin{figure}[H]
    \centering
    \includesvg[width=0.9\textwidth]{images/paraphrasing/experiments/chunks/setup/results/Gutenberg/Translation_metrics_plot_category_Gutenberg.svg}
    \caption[Paraphrasing scores for the translation-based paraphraser.]{Comparing different paraphrasing scores for the translation-based paraphraser on the \dataGutenberg{} dataset for different number of chunks. 
    Changing the number of chunks has little effect on the model scores.}
    \label{fig:abl_chunks_gutenberg_translation}
\end{figure}

\begin{figure}[H]
    \centering
    \includesvg[width=0.9\textwidth]{images/paraphrasing/experiments/chunks/setup/results/Student_Essays/Task_metrics_plot_category_Student Essays.svg}
    \caption[Paraphrasing scores for the Bullet point model.]{Comparing different paraphrasing scores for the Bullet point model on the \dataStudent{} dataset for different number of chunks. 
    \ac{rouge}-1 and METEOR decrease, while other scores are not affected by increasing the number of chunks.}
    \label{fig:abl_chunks_student_essays_task}
\end{figure}

% appendix
\section{Exp.\ 6: Comparing \acs{av} Methods in Different Scenarios}
\label{sec:app_detection_scenarios}
\FloatBarrier
\begin{figure}[H]
  \centering
  \begin{subfigure}[b]{0.82\textwidth}
    \centering
    \includesvg[width=\linewidth]{images/AV_comparison/detection_scenarios/recall/student_essays_Human-Human_threshold_recalls_curves_all_incl_baselines.svg}
    \caption{Human-Human}
    \label{fig:detec_scen_human-human_recall}
  \end{subfigure}
  \hfill
  \begin{subfigure}[b]{0.82\textwidth}
    \centering
    \includesvg[width=\linewidth]{images/AV_comparison/detection_scenarios/recall/student_essays_Human-LLM_threshold_recalls_curves_all_incl_baselines.svg}
    \caption{Human-\ac{llm}}
    \label{fig:detec_scen_human-llm_recall}
  \end{subfigure}
  \hfill
  \begin{subfigure}[b]{0.82\textwidth}
    \centering
    \includesvg[width=\linewidth]{images/AV_comparison/detection_scenarios/recall/student_essays_LLM-LLM_same_threshold_recalls_curves_all_incl_baselines.svg}
    \caption{\ac{llm}-\ac{llm}}
    \label{fig:detec_scen_llm-llm_recall}
    \end{subfigure}
  \caption[Recall curves on \dataArtificialStudent{} dataset.]{Recall curves for the class same-author across different threshold on \dataArtificialStudent{} dataset.}
  \label{fig:detec_scen_recall}
\end{figure}


\section{Libraries used}
\label{app:libraries}

\ac{rouge} is a wrapper for Google Research\footnote{https://huggingface.co/spaces/evaluate-metric/rouge}.


\textcolor{red}{TODO}
e.g. sentence tokens: \texttt{nltk}'s \texttt{sent\_tokenize}
    \listoffigures % optional, usually not needed

    \listoftables % optional, usually not needed

    % Die nächsten zwei Zeilen sind optional, sie sorgen dafür dass alles nach dem Inhalt wieder mit römischen Zahlen nummeriert wird.
    \pagenumbering{roman}
    \addtocounter{page}{10} % Dies ist die Anzahl der Seiten vor der Einleitung, muss möglicherweise angepasst werden, wenn das Inhaltsverzeichnis mehrere Seiten umfasst.

    % \nocite{*} % uncomment to print all references
    \bibliographystyle{plainnat} % requires package natbib. An alternative is apalike
    \bibliography{
        bibliography/author_identification
    }

    % \include{declaration}
\end{document}
