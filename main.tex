\documentclass[
    %draft, % Mit % kommentieren, um Bilder sichtbar zu machen und Links zu aktivieren
    english,
    master,
    ks, % Kassel University
    % pdftex,
    % a4paper,
    % twoside,
    % parskip=half,
    % numbers=noenddot,
    % listof=totoc,
    bibliography=totoc,
    % hyperfootnotes=false,
    % english,
    % openright
]{webisthesis}

% FIXME: this doesn't work: https://blog.rtwilson.com/how-to-add-simple-new-commands-to-latex-to-help-with-writing-papers/ (11.04.2025)
\newcommand{\todo}[1] {\textbf{\textcolor{red}{TODO:}} #1}
% \newcommand{\thesistype}{M A S T E R \space \space T H E S I S}
\newcommand{\thesistypedesc}{Department of Electrical Engineering and Computer Science \\
    University of Kassel}
\newcommand{\thesisauthorname}{Klara Maximiliane Gutekunst}
\newcommand{\thesisauthorhomestreet}{\todo}
\newcommand{\thesisauthorhometown}{34119 Kassel}
\newcommand{\thesisauthormatrikelnumber}{35677772}
\newcommand{\thesisauthoremail}{klara.gutekunst@uni-kassel.de}
\newcommand{\thesisdepartment}{Chair Deep Semantic Learning}
\newcommand{\thesisfirstreviewer}{Prof.\ Dr.\ Martin Potthast}
\newcommand{\thesissecondreviewer}{Prof.\ Dr.\ Gerd Stumme}
\newcommand{\thesissupervisor}{\todo}
\newcommand{\thesisdate}{\today}

% \ThesisSetTitle{Towards LLM-Resilient Authorship Verification: Revisiting the Impostors Method}
\ThesisSetTitle{Leveraging \acs{llm}-Generated Impostors for the Impostors Method}
\ThesisSetKeywords{These, are, my, Keywords} % only for PDF meta attributes
\ThesisSetLocation{\thesisauthorhometown{}} 

\ThesisSetAuthor{\thesisauthorname{}}
\ThesisSetStudentNumber{\thesisauthormatrikelnumber{}}
\ThesisSetDateOfBirth{25}{09}{2001}
\ThesisSetPlaceOfBirth{Brilon, Germany}

% Supervisors should usually be Professors from the candidate's university. A second supervisor is not always needed. 
\ThesisSetSupervisors{\thesisfirstreviewer{},\thesissecondreviewer{}}

\ThesisSetSubmissionDate{26}{09}{2025}


% content-specific commands
\newcommand{\tira}{TIRA}
\newcommand{\pextractor}{Paraphrase Extractor}
\newcommand{\pgenerator}{Paraphrase Generator}
\newcommand{\dataGutenberg}{Gutenberg}
\newcommand{\dataBlog}{Blog Posts}
\newcommand{\dataPan}{PAN20}
\newcommand{\dataStudent}{Student Essay}
\newcommand{\dataCustom}{Custom CNN News}
\newcommand{\bluert}{\href{https://github.com/google-research/bleurt}{BLUERT}}

\usepackage{booktabs}
%    For tables ``looking the right way''.
% \usepackage{tabularx}
%    Enables tables with columns that automatically fill the page width.
%
% \usepackage[ruled,algochapter]{algorithm2e}
%    A package for pseudo code algorithms.
%
% \usepackage{amsmath}
%    For tabular-style formatting of mathematical environments.
%

\usepackage{fontawesome}
%    For lots of awesome glyphs: https://mirror.physik.tu-berlin.de/pub/CTAN/fonts/fontawesome/doc/fontawesome.pdf
% \usepackage[printonlyused]{acronym}

% Select input encoding, usually utf8 is the best choice, on windows, \usepackage[latin1]{inputenc} maybe required
\usepackage[utf8]{inputenc}
\usepackage[T1]{fontenc}
\usepackage[english]{babel}
\usepackage{csquotes}
\usepackage{xcolor}

\MakeOuterQuote{"} % Damit ist es möglich, " " zu verwenden ohne Umlaut zu erzeugen
\defaulthyphenchar=127 % Dadurch werden auch Wörter mit Bindestrich getrennt, die schon Bindestriche enthalten.

% geometry
\usepackage[bindingoffset=1cm, left=2.5cm, right=2.5cm, top=2.5cm, bottom=2.5cm]{geometry}

% Headline
\usepackage{fancyhdr}
\pagestyle{fancy}
\renewcommand{\chaptermark}[1]{\markboth{\thechapter\ #1}{}}
\lhead{\leftmark} \rhead{\thepage}
\cfoot{}
\fancypagestyle{plain}{}

\RedeclareSectionCommand[beforeskip=1.5cm,afterskip=1cm]{chapter}

% Colors
\usepackage{color}
\usepackage{colortbl}

% Tables
\usepackage{tabularx}
\usepackage{multirow}
\setlength{\tabcolsep}{4pt}

% Drawing graphs etc.
\usepackage{pgf}
\usepackage{tikz}
\usetikzlibrary{arrows,automata}

% Footnotes
\usepackage{footmisc}
\usepackage{xspace}
\newcommand{\sic}{[\acs{sic}]\xspace}

% math
\usepackage{amsmath}
\usepackage{amssymb}

\usepackage{siunitx}

% lists
\usepackage{paralist}

% Figures
\usepackage{graphicx, wrapfig}

% Hyperlinks
\usepackage[hyphens]{url}
\usepackage{hyperref}
\hypersetup{colorlinks, citecolor=black, linkcolor=black, urlcolor=black}

% Minted
\usepackage[chapter]{minted}
%\usemintedstyle{xcode}
\setminted{frame=single,tabsize=2,linenos,autogobble}

\newmintinline[code]{text}{breaklines}

\newminted[mdcodeblock]{md}{autogobble,frame=none,linenos=false,breaklines}

% list of abbreviations
\usepackage[printonlyused]{acronym}

% Set line pitch
\usepackage{setspace}
\onehalfspacing              % anderthalbzeilig (oder auch \doublespace)

%fancyBox
%\usepackage{fancybox}

% Layout corrections (Schusterjungen)
\clubpenalty = 10000
% Layout corrections (Hurenkinder)
\widowpenalty = 10000
\displaywidowpenalty = 10000

% Figures
\usepackage{caption}
\usepackage[hypcap=true,labelformat=simple]{subcaption}
\renewcommand{\thesubfigure}{(\alph{subfigure})}
\usepackage[inkscapelatex=false]{svg}

% Tables
\usepackage{booktabs}
%\usepackage[table,xcdraw]{xcolor}

% enumerate
\usepackage{enumitem}
\newlist{questions}{enumerate}{2}
\setlist[questions,1]{label=RQ\arabic*.,ref=RQ\arabic*}
\setlist{nolistsep}

% Bibliography
\usepackage[square,numbers]{natbib}
\bibliographystyle{plainnat} % or plainnat abbrvnat unsrtnat

% Frequently used column types
\newcolumntype{C}[1]{>{\centering\arraybackslash}p{#1}} % centering column type with fixed width
\newcolumntype{R}[1]{>{\raggedleft\arraybackslash}p{#1}} % right aligned column type with fixed width
\newcolumntype{L}[1]{>{\raggedright\arraybackslash}p{#1}} % left aligned column type with fixed width

% Shortcuts for referencing floats:
\newcommand{\fig}[1]{\figurename~\ref{#1}} %shortcut for a figure reference
\newcommand{\tab}[1]{Table~\ref{#1}} %shortcut for a table reference
\newcommand{\eq}[1]{(\ref{#1})} %shortcut for an equation reference
\newcommand{\lst}[1]{Listing~\ref{#1}} %shortcut for a listing reference
\newcommand{\sect}[1]{Section~\ref{#1}} %shortcut for a Section reference
\addto\extrasenglish{%
  \renewcommand{\sectionautorefname}{Section}%
  \renewcommand{\subsectionautorefname}{Subsection}%
  \renewcommand{\chapterautorefname}{Chapter}%
}

% Shortcut for terms
\newcommand{\databaseName}{Elasticsearch}
\newcommand{\flask}{Flask}
\newcommand{\angular}{Angular}
\newcommand{\infersent}{InferSent}
\newcommand{\wordcloud}{word cloud}
\newcommand{\localMaschineStats}{Apple M2 Pro MNW83D/A with 16 \ac{gb} RAM and 12 cores}
\newcommand{\slurm}{Slurm}




\begin{document}
\begin{frontmatter}
    % \pagenumbering{roman} not used at webis
    % \include{titlepage} not used at webis
    % \chapter*{Abstract}
\markboth{Abstract}{Abstract}

\Acl{av} seeks to determine whether two texts share the same author, a task critical for ensuring the integrity of academic submissions or online content.
Existing approaches typically exhibit poor generalisation across domains.
The \impAppr{} introduced hard negative sampling to create input-pair-specific settings to improve cross-domain generalisation. 
However, traditional sampling strategies are unable to simultaneously control for multiple confounding variables, such as genre and topic.
%, highlighting the need for improved sampling strategies.
This thesis investigates whether \aclp{llm} can help address these limitations. 
Specifically, we employ \acs{llm}-generated paraphrases as hard negatives. 
% Since these paraphrases aim to preserve the confounding variables of the original text, they mitigate domain-related biases during inference. 
% Moreover, by constructing a tailored case for each input text pair, the approach eliminates out-of-distribution issues, ensuring that comparisons remain within the distribution defined by the pair itself.
Evaluation on the \dataStudent{} dataset from the original study shows that the \acs{llm}-based extension surpasses the original baselines by \citet{koppel_determining_2014}, \acs{ppmd}, and \unmasking{} in terms of precision and recall.
At the same time, our results reveal the practical and conceptual challenges of integrating \acsp{llm} into \acl{av}, including issues of reliability, hallucination, and control over paraphrase quality.



% The original \impAppr{} compares a disputed text to a candidate text and hard negatives, considering disputed and candidate text to share an author if the candidate is consistently the most similar across random feature projections. 
    % \chapter*{Zusammenfassung}
\markboth{Zusammenfassung}{Zusammenfassung}

 not used at webis

    \tableofcontents

    % \chapter*{Acknowledgements} % optional
    % I thank the authors of the webisthesis template for their excellent work!

    % \listoffigures % optional, usually not needed

    % \listoftables % optional, usually not needed

    % \listofalgorithms % optional, usually not needed
    %    requires package algorithm2e

    % optional: list of symbols/notation (e.g., using the nomencl package) but usually not needed
    \end{frontmatter}


    \chapter*{List of abbreviations}
\markboth{List of abbreviations}{List of abbreviations}

\begin{acronym}[XXXXXXXXX]
    \acro{ai}[AI]{Artificial Intelligence}
    \acro{ir}[IR]{Information Retrieval}
    \acro{nlp}[NLP]{Natural Language Processing}
    \acro{llm}[LLM]{Large Language Model}
    \acro{roc-auc}[ROC-AUC]{Area Under the Receiver Operating Characteristic Curve}
    \acro{pan}[PAN]{Plagiarism Analysis and Authorship Mining} % TODO: find out if correct
    \acro{bert}[BERT]{Bidirectional Encoder Representations from Transformers}
    \acro{bow}[BoW]{Bag-of-Words}

    % \acro{}[]{}
\end{acronym}


    % \pagebreak
    \pagenumbering{arabic}

    \chapter{Introduction}
\label{chap:introduction}



% motivation
Historically, authorship analysis focused on literary disputes~\citep{neal_surveying_2018,stamatatos_survey_2009}, but contemporary concerns have shifted towards practical applications.
In an era where large amounts of text can be copied, paraphrased, or fabricated with ease, determining the true author of a text is crucial for maintaining trust in communication. 
Scenarios include detecting plagiarised passages of texts~\citep{stein_intrinsic_2011}, and verifying the authenticity of online content or student submissions. 
Formally, we refer to these problems as \acf{av} or \acf{aa}, where every \ac{aa} task can be formulated as a sequence of \ac{av} problems~\citep{tyo_state_2022,barlas_cross_domain_2020}.

The emergence of \acp{llm} adds an additional layer of complexity. 
While these models are widely embraced for beneficial applications such as summarisation, information seeking, and assistive writing~\citep{wang_stumbling_2024}, their ability to convincingly imitate human writing creates new risks. 
\acp{llm} can be used to generate misinformation, impair academic honesty, or impersonate individuals, thereby inflicting harm on individuals who fall victim to these schemes~\citep{mitchell_detectgpt_2023,li_learning_2025,wang_stumbling_2024,bhattacharjee_fighting_2024}. 
Since \acp{llm} can be conceptualised as authors, their detection naturally falls within the scope of \ac{av}. 
Thus, instead of treating \ac{llm} detection as an isolated task, it is more consistent to frame it as a specialised case of \ac{av}~\citep{llm_detection_av_2025}.

% specificity rather than generality
Existing approaches to generalisation typically train a single model and apply it across domains.
Despite significant advances in \ac{av}, prior work finds that such models struggle in \ac{ood} settings, where the topic or genre diverges from the training data~\citep{Sundararajan_style_18,bischoff_importance_2020,li_learning_2025}. 
This shortcoming motivates a shift towards scenario-specific solutions, i.e.\ models are trained anew for narrowly defined cases. 
Such single-case approaches enable more precise control over contextual factors and place greater emphasis on stylistic idiosyncrasies rather than domain-level variation.

% AV
The \impAppr{} by \citet{koppel_determining_2014}\ introduces the idea of generating \imp{} texts, i.e.\ hard negatives, used to sharpen the discrimination between genuine and false authorship matches. 
However, the method's effectiveness is limited by the quality and contextual adequacy of these \imp{} texts. 
Previous work did not fully address how to construct challenging \imps{} via controlled contextual variables.

The thesis extends the \impAppr{} by leveraging \acl{sota} \acp{llm} to generate paraphrases as \imps{}, enabling control over multiple confounding factors such as genre, topic, and target audience. 
In doing so, the approach shifts the focus towards authorial style rather than domain differences, yielding improved precision–recall on the \dataStudent{} dataset compared with the original sampling strategies, \unmasking{} and \acs{ppmd}.


\section{Research Questions}
\label{sec:research_questions}
To guide this objective, we formulate the following research questions:
\begin{questions}
    \item \textbf{How can we instruct a \ac{llm} to paraphrase the text of a candidate author such that it captures the \ac{llm}'s stylistic properties?} \label{enum:rq1} \hfill \\
    The goal is to create hard negatives for the Impostor method by controlling contextual factors.
    By controlling genre, topic and other factors, similarity measures primarily focus on differences in authorial style rather than the impact of content on style.
    We obtain this controlled environment by utilizing \acp{llm} to paraphrase the original text.
    There are different approaches to paraphrasing text using \acp{llm}.
    They include (a) directly asking the \ac{llm} to paraphrase the text, 
    (b) first extracting specific information from the original text and subsequently generating a paraphrase based on the information.
    This thesis compares both approaches on \dataStudent{}, \dataBlog{}, \dataGutenberg{} and \dataPan{}.

    \item \textbf{How do we evaluate the quality of paraphrases?} \label{enum:rq2} \hfill \\
    Paraphrase evaluation is inherently challenging, as there is no universally agreed-upon definition of what constitutes a paraphrase. 
    Prior research often adapts evaluation metrics from related \ac{nlp} tasks such as machine translation or summarization. 
    Two key dimensions are typically considered: semantic similarity and syntactic similarity.
    Contrary to initial intuition, high syntactic similarity is not necessarily desirable, as it may indicate that the \ac{llm} has merely copied the original text with minimal changes. 
    Instead, our focus lies on achieving high semantic similarity while maintaining syntactic diversity to ensure genuine rephrasing.
    Furthermore, we acknowledge that relatively low automatic scores can still be acceptable if qualitative human evaluation confirms the paraphrase’s adequacy.

    % \item \textbf{Which features are used for the \ac{av} problem?} \label{enum:rq3} \hfill \\
    % Traditional features include character tri-gram features, while newer research has proposed using \ac{llm} such as BERT.

    \item \textbf{How does the \ac{llm}-based impostor approach perform compared to state-of-the-art models?} \label{enum:rq4} \hfill \\
    Though our approach is computationally expensive, we argue that it is not a general purpose \ac{llm} detection method, but rather a single case solution tailored to specific detection tasks.
    We evaluate its performance in scenarios where (a) the disputed text is human generated,
    (b) the disputed text is \ac{llm} generated and the candidate is the same \ac{llm}, and
    (c) the disputed text is \ac{llm} generated, but the candidate is a different \ac{llm}.
    In terms of performance, we compare our method to other \ac{av} approaches on the \dataStudent{}, \dataBlog{}, \dataGutenberg{} and \dataPan{} datasets.
    
\end{questions}

% \section*{Idea}
% \label{sec:idea}

% Given a text of unknown authorship (i.e., human or \ac{llm}), 
% construct a set of impostor texts using state-of-the-art \acp{llm} based on the original text.
% Obtain the author by \ac{aa}/ \ac{av} methods, such as unmasking, to \textit{confidently}, i.e. high precision, identify \ac{llm} generated texts
% (and possibly which \ac{llm}).



\section{Contributions}
\label{sec:contributions}
The contributions of this thesis are:
\begin{enumerate}
    \item Reimplementation of the traditional Impostor approach (cf. \autoref{chap:implementation}).
    \item Extension of the impostor approach with \ac{llm} generated impostors for line-up of difficult opponents (cf. \autoref{chap:methodology}). 
    \item Frame \ac{llm} detection as a \ac{av} problem and use \ac{llm} generated text as candidate for text of "unknown" authorship.
\end{enumerate}


\section{Thesis Structure}
\label{sec:thesis_structure}
The thesis is structured as follows:
    \chapter{Background on \ai{}}
\label{chap:authorship_identification}
%    AI
%     - Woher kommt es?
%         - initial statistische analyse
%         - Dylo hypethese???
%         - stylometry
%         - Use case 1/ original use case: Literatur Forschung
%         - Use case 2: Digital text forensics
%     - AA
%     - AV als Kernproblem von allem
%         - Fokus hier, und LLM detection istr ein Plus, eher im Anhang
%     - technischer Hintergrund: open/ closed set, one-class classification
%     - state of the art models
%     - \imp{} method

This section offers background information on \ai{}.
Starting exploring with the origin of \ai{} in \ac{aa} of historical texts, we outline the initial statistical approaches and touch upon stylometry.
Next, we highlight why \ai{} is still relevant today, and break down some essential tasks of \ai{}.
We proceed with the technical background and \ac{sota} approaches.
We conclude with an outlook on the impact of \acp{llm} on \ai{}.

\textcolor{red}{TODO}

\section{Historical Background and Early Approaches}
\label{sec:origin}

Research on authorship originated in the 19th century as an attempt to resolve disputes over the authorship of literary works. 
Augustus de Morgan was among the first to propose a quantitative approach in 1851, using word-length frequencies. 
Building on this idea, in 1887 Thomas C. Mendenhall carried out the first systematic analysis of word-length distributions in the works of Bacon, Marlowe, and Shakespeare, aiming to shed light on the authorship of Shakespearean plays~\citep{neal_surveying_2018,stamatatos_survey_2009}.
Word-length distributions quantify the number of words on the vertical axis by their length in characters on the horizontal axis. 
\citet{wordlengths_mendenhall_1887} concluded that differences between the curves of two disputed texts may indicate different authorship, whereas similarity in the curves is less reliable as evidence of shared authorship.

% Popular examples of \ai{} include a collection of Hebrew-Aramaic letters supposedly by a rabbinic scholar in Baghdad in the late 19th century~\citep{koppel_authorship_2004}.
% The latter denied having authored the text collection, which was contested by \citet{koppel_authorship_2004}.

% \section{First Approaches}

% Approaches of the \nth{19} century were limited to the analysis of word lengths.
In the early \nth{20}, statistical measures such as Zipf's law (1932) and Yule's characteristic (1944) were introduced. 
These measures captured word frequency distributions and vocabulary richness, laying the foundation for more formalized analyses of writing style~\citep{neal_surveying_2018,stamatatos_survey_2009}.

A decisive milestone came with the work of Mosteller and Wallace (1964), who applied Bayesian statistical analysis to the "The Federalist Papers".
Their computer-assisted approach, based on the frequencies of a subset of common function words (e.g., "and", "to"), demonstrated the potential of rigorous quantitative methods for \ac{aa}. 
This work is often regarded as the beginning of modern stylometry, the systematic study of quantifiable features of writing style.

Between the 1960s and the late 1990s, stylometric research flourished.
Around \num{1000} features were proposed, including sentence and word lengths, character and word frequencies, and vocabulary richness. 
However, the field faced two major limitations: the absence of benchmark datasets prevented systematic comparison of methods, and evaluation often relied on subjective inspection of visualizations (e.g., scatterplots) rather than standardized metrics~\citep{stamatatos_survey_2009}.

The late 1990s marked the transition into a new era. 
The growing availability of electronic text collections enabled the construction of benchmark datasets and more thorough evaluation. 
At the same time, \ac{ml} algorithms facilitated more expressive text representations, moving beyond simple frequency counts toward feature-rich models. 
As a result, the scope of \ai{} expanded from resolving historical literary questions to addressing practical, real-world applications such as \ac{av} in digital forensics~\citep{stamatatos_survey_2009}. 
Nevertheless, as \citet{abbasi_writeprints_2008} note, even by 2008 stylometric methods struggled with scalability across large author sets, diverse genres, and open-world scenarios.

% \subsection{Stylometry}

\begin{definition}
    [Stylometry]
    Stylometry refers to a linguistic research area, where literary style is quantified by statistical features.
\end{definition}
% In other words, stylometry is the statistical analysis of literary style between one writer or genre and another \citep{tyo_state_2022}.
Researchers working on stylometric features believe that subconscious authorial syntactic idiosyncrasies are quantifiably measurable and sufficient to define an author's unique style~\citep{neal_surveying_2018}. 
These stylometric features are also referred to as style markers, or stylistic features if they are the most effective discriminators of authorship. 
Static features are context-free such as function words, word-length distributions, vocabulary richness measures, and dynamic features context-dependent attributes and include $n$-grams and misspelled words~\citep{abbasi_writeprints_2008}.
% Stylometric features include lexical, syntactic and structural features \citep{stein_intrinsic_2011}.
% Research includes five subtasks \citep{neal_surveying_2018}:
% \begin{itemize}
%     \item \ac{aa}
%     \item \ac{av}
%     \item Author profiling
%     \item Stylochronometry
%     \item adversarial stylometry
% \end{itemize}
We outline prominent style marker taxonomy categories in \autoref{tab:stylometric_features} in the following. 
Generally speaking, the more detailed the style marker extraction process, the more noisy are the produced measures~\citep{stamatatos_survey_2009}.

% lexical
Lexical features treat text as a mere sequence tokens~\citep{stamatatos_survey_2009}.
Token units include character, word, and sentence.
We \autoref{tab:comp_lexical} we present characteristics of the character and word feature units.
Character (n-gram) counts, average word~\citep{stein_intrinsic_2011}, sentence length~\citep{stein_intrinsic_2011,abbasi_writeprints_2008}, are examples for character, sentence, and sentence unit-based lexical features.
% line length~\citep{abbasi_writeprints_2008}, word length distribution~\citep{abbasi_writeprints_2008}, 
% vocabulary richness~\citep{abbasi_writeprints_2008,neal_surveying_2018} ...) 
Errors in \autoref{tab:stylometric_features} refers to idiosyncratic features include misspellings, grammatical mistakes, and other usage anomalies~\citep{abbasi_writeprints_2008,neal_surveying_2018}.


\begin{table}[]
\centering
\caption{Comparison of a subset of lexical features~\citep{stamatatos_survey_2009}. Requirements refer to computational requirements such as a tokenizer.}
\label{tab:comp_lexical}
\resizebox{\textwidth}{!}{%
\begin{tabular}{lllll}
    \toprule
    \textbf{unit} & \textbf{complexity} & \textbf{language-independent} & \textbf{affected by noise} & \textbf{requirements} \\
    \midrule
    character & low & \checkmark & low & \xmark  \\
    word & high & \xmark  & high  & \checkmark    \\
    \bottomrule
    \end{tabular}%
}
\end{table}

% syntactic
Syntax or syntactic structure is the structural organization of sentences \citep{kurt_pehlivanoglu_comparative_2024}.
Given robust \ac{nlp} tools, syntactic features are considered more reliable than lexical features.
Well-established syntactic features built on function words, or on syntactic errors such as mismatched tense or sentence fragments, or use morpho-syntactic \ac{pos} tags for each word token for \ac{pos} tag (n-gram) frequencies~\citep{abbasi_writeprints_2008,stamatatos_survey_2009}.
It is evident, that every syntactic feature requires parsing or processing of natural language and is thus, language-dependent~\citep{neal_surveying_2018,stamatatos_survey_2009}.

% semantic
Semantic features capture meaning behind words, phrases, and sentences, such as through analysis of synonyms and semantic dependencies \citep{neal_surveying_2018}.
Semantic similarities between words, synonym or hyponym relationships are derived using WordNet, any thesaurus or latent semantic analysis.
Semantic dependencies such as the specification of a person with location, can boost classification performance when combined with lexical and syntactic information~\cite{stamatatos_survey_2009}.


% application specific (structural + content)
Structural features include text organization, layout, file extensions, font, sizes, colours, 
use of braces and comments (for analysing computer programs)~\citep{abbasi_writeprints_2008,neal_surveying_2018}.
Content-specific features include important keywords and phrases on certain topics such as word $n$-grams~\citep{abbasi_writeprints_2008}.
Domain-specific features include ratios of quoted words and external links, number of paragraphs, and paragraphs average length for the news article domain~\citep{potthast_stylometric_2018}


\begin{table}[]
    \centering
    \caption{Incomplete taxonomy of style markers from \citep{stamatatos_survey_2009}.}
    \label{tab:stylometric_features}
 
    \begin{tabular}{@{}ll@{}} % numbers should be right aligned, text left aligned
    \toprule
    \textbf{Category} & \textbf{Features} \\ 
    \midrule
    Lexical & Token-based \\ %(word/ sentence length, ...) \\
     & Vocabulary richness  \\
     & Word frequencies  \\
     & Word n-grams  \\
     & Errors \\
    %  \midrule
    % Character & Character types (letters, digits, ...)\\
    %  & Character n-grams (fixed length)  \\
    %  & Character n-grams (variable length) \\
    %  & Compression methods \\
    %  \midrule
    Syntactic & Part-of-Speech (POS)  \\
    %  & Chunks \\
    %  & Sentence and phrase structure  \\
    %  & Rewrite rule frequencies \\
     & Errors  \\
    %  \midrule
    Semantic & Synonyms \\
     & Semantic dependencies \\
    %  & Functional  \\
    %  \midrule
    Application-specific & Structural  \\
     & Content-specific\\
    %  & Language-specific \\
     \bottomrule
    \end{tabular}%

\end{table}
\section{Contemporary Applications} % Contemporary Relevance/ Perspective

While early work in \ai{} focused on resolving historical literary disputes, its present-day relevance lies primarily in digital forensics. % forensic linguistics: sapkota_cross_topic_2014
\ac{aa} provides critical tools for addressing cybercrime, where anonymity enables identity deception, harassment, and financial fraud~\citep{abbasi_writeprints_2008,chendu_authorship_2020,bhattacharjee_fighting_2024}. 
Concrete applications include the detection of fraudulent e-mails, impersonation on social media, and the identification of fake product reviews that undermine trust in e-commerce platforms. 

Beyond forensic contexts, plagiarism detection constitutes another significant application area. 
Here, authorship analysis assists in identifying unacknowledged reuse of material across domains ranging from literature to academia~\citep{neal_surveying_2018}. 
Together, these examples highlight the contemporary relevance of \ai{} in safeguarding digital integrity, supporting legal investigation, and upholding standards of authorship in both professional and academic contexts.

\section{Tasks}

In this section we outline the main tasks of authorship analysis. 
We begin with the foundational problems of \ac{aa} and \ac{av}, before turning to derivative tasks such as author profiling,  author obfuscation, and plagiarism detection.

\subsection{\Acl{aa}}
\ac{aa} is the classical multiclass, single-label text classification task in which, given a set of candidate authors with known writings, the goal is to determine which author wrote a disputed text~\citep{koppel_authorship_2004}. 
Formally, let $A$ be the set of authors, $K=\bigcup_{a\in A} K_a$ the set of known texts, and $U$ the set of unknown texts. 
In the closed-set setting, each $d \in U$ must be attributed to exactly one $a \in A$. 
Variants such as cross-topic or cross-genre \ac{aa} introduce distributional shifts between training and test data, complicating the task~\citep{barlas_cross_domain_2020}.

\subsection{\Acl{av}}
\ac{av} addresses the problem of establishing whether a given text $t$ was written by a candidate author $a$, using a set of the author’s known writings $K_a$ as reference~\citep{koppel_authorship_2004}.
Disputed-candidate pair denotes the input of the text of unknown authorship and the candidate author's text in the following.
In contrast to \ac{aa}, \ac{av} does not have reliable negative examples, since assembling a representative sample of all texts not authored by 
$a$ is impossible. 
This limitation makes \ac{av} a more challenging classification problem than \ac{aa}~\citep{llm_detection_av_2025,neal_surveying_2018,koppel_authorship_2004}.
\ac{av} is framed as a one-class, a binary, or a similarity detection task depending on the methodological perspective~\citep{neal_surveying_2018,koppel_authorship_2004}.  
\ac{aa} can be reduced to a series of \ac{av} problems, where the other direction is typically not true~\citep{barlas_cross_domain_2020,tyo_state_2022}.
% Gespräch Martin Potthast 19.05.2025: problem formulation 2 is less common and in the context of very sparse (metadata) information:
% This task can also be formulated as whether two texts $t_1$ and $t_2$ are written by the same author 
% \citep{bevendorff_generalizing_2019,bevendorff_divergence_based_2020,embarcadero_ruiz_graph_based_2022,rivera_soto_learning_2021,ordonez_will_2020,futrzynski_pairwise_2021,weerasinghe_feature_vector_difference_2021,llm_detection_av_2025}.


% \subsection{Author Profiling}
% Author profiling infers sociolinguistic attributes of an author, such as age, gender, education, or mental health, from a set of texts. 
% It is grounded in the assumption that subconscious idiosyncrasies of writing style encode personal traits~\citep{emmery_adversarial_2021,stamatatos_survey_2009,elmanarelbouanani_authorship_2014}. 
% Profiling raises substantial ethical and privacy concerns due to the sensitivity of the inferred attributes.

% \subsection{Author Obfuscation}
% Author obfuscation is an adversarial task in which a text is deliberately modified to conceal the author's identity while preserving its semantic content. 
% It directly opposes \ac{aa} and \ac{av} by aiming to neutralize stylometric features~\citep{bischoff_importance_2020,bevendorff_divergence_based_2020,gohsen_task_oriented_2024}. 

% \subsection{Intrinsic Plagiarism Detection}   % style change detection
% Plagiarism detection is the task of identifying reused or unattributed content in texts~\citep{stein_intrinsic_2011,gohsen_task_oriented_2024}. 
% Unlike \ac{aa}, the goal is not to establish authorship but to uncover overlap between documents, regardless of author identity~\citep{elmanarelbouanani_authorship_2014}.
% Hence, plagiarism detection is an application of \ac{av}~\citep{rivera_soto_learning_2021}.


\subsection{\acs{llm} detection}
Given the ability of \acp{llm} to generate text closely resembling human writing, we can conceptualize \acp{llm} as one or multiple distinct authors.
This perspective allows us to frame \ac{llm} detection as an \ac{av} task, where given two texts (i.e. one of unknown authorship and one known to be generated by an \ac{llm} author) the objective is to determine whether they share the same author~\citep{llm_detection_av_2025}.



\section{Technical Background}

This section outlines the \ac{ml} principles and paradigms that underpin modern authorship analysis. 
We first introduce the main classification concepts relevant to \ac{av}. 
We then discuss training and evaluation practices, including domain shift scenarios. 
Finally, we present the principal categories of authorship analysis methods.

\subsection{\acl{ml} Principles}

\ac{aa} tasks are conventionally formulated as classification problems, where the objective is to attribute an anonymous text to one of a set of candidate authors. 
Unlike regression, which estimates a continuous value, classification yields a discrete label corresponding to the author’s identity.

\paragraph{Closed- vs. open-set classification.} 
In a closed-set scenario, the true author is guaranteed to be among the candidate set~\citep{stamatatos_survey_2009,koppel_authorship_2011,barlas_cross_domain_2020,boenninghoff_o2d2_2021,neal_surveying_2018}. 
In contrast, open-set classification acknowledges that the author of a disputed document may not belong to the candidate set~\citep{stamatatos_survey_2009,barlas_cross_domain_2020,neal_surveying_2018}. 

\paragraph{One-class classification.} 
In some cases, training data is available for only a single class, and the task is to decide whether a new sample belongs to this class.
If counterexamples, i.e. so-called outliers, are available, they are usually not considered to be representative of non-target class. 
This is formalized as one-class classification, where the model learns the characteristics of the target class without reliable counterexamples~\citep{stein_intrinsic_2011,koppel_authorship_2004}.

\paragraph{Multi-class classification.} 
The standard formulation involves multiple authors, each represented by several training texts. 
Here, the challenge is to discriminate among a large and often imbalanced set of classes~\citep{stamatatos_survey_2009,koppel_authorship_2004,elmanarelbouanani_authorship_2014} .


\subsection{Training and Testing}

Models are typically trained on one portion of the data (training set), tuned on another (validation set), and evaluated on a disjoint partition (test set). 
Any overlap between these partitions constitutes data leakage and invalidates the results~\citep{bischoff_importance_2020,altakrori_topic_2021,boenninghoff_o2d2_2021}. 

A major challenge in stylometry is covariate shift, i.e., a mismatch between the distribution of training and test data. 
This often arises from topic variability~\citep{boenninghoff_o2d2_2021}. 
Two common evaluation settings are:
\begin{itemize}
    \item \textbf{Cross-topic attribution}, where models trained on one set of topics are tested on previously unseen topics~\citep{altakrori_topic_2021}.  
    \item \textbf{Cross-domain attribution}, where training and test texts differ in topic or genre~\citep{barlas_cross_domain_2020}.  
\end{itemize}

\paragraph{Supervised vs. unsupervised learning.}  
Supervised methods require labelled training data and include classifiers such as \acp{svm}, decision trees, \acp{nn}, and linear discriminant analysis. 
\acp{svm} are particularly common in authorship analysis due to their robustness. 
Unsupervised methods do not rely on labels.
Clustering techniques or \ac{pca} have been used to uncover latent stylistic patterns or to reduce feature dimensionality~\citep{abbasi_writeprints_2008}.


\subsection{Authorship Analysis Methods}
\label{subsec:attribution_methods}

Approaches to authorship analysis can be grouped into three families~\citep{stamatatos_survey_2009}:

\paragraph{Profile-based methods.} 
All training texts of an author are concatenated into a single profile, from which a cumulative feature representation is extracted. 
This approach is effective when only short texts are available.
Profile-based methods ignore intra-author variation~\citep{stamatatos_survey_2009,elmanarelbouanani_authorship_2014,neal_surveying_2018}.  

\paragraph{Instance-based methods.} 
Here, each training text is treated as a separate instance of the author's style. 
This allows models to capture intra-author variability~\citep{stamatatos_survey_2009,altakrori_topic_2021,elmanarelbouanani_authorship_2014,neal_surveying_2018}.  

\paragraph{Hybrid methods.} 
Hybrid approaches combine both paradigms by representing texts individually while aggregating author profiles through feature-wise averages computed over an author’s texts~\citep{stamatatos_survey_2009}. 

\section{Canonical Methods}

The following sections introduce \ac{sota} approaches to \ac{av}.
We begin with compression-based methods, continue with traditional and generalized Unmasking and end with the traditional \impAppr{}.

\subsubsection{Compression-based}
% compression models, e.g. RAR or GZIP 
This category of \ac{aa} approaches is based on general-purpose compression models such as RAR or PPMD. %(i.e. a variant of \ac{ppm}~\citep{tyo_state_2022}), LZW, GZIP, BZIP2 and 7ZIP.
Such models capture textual characteristics by exploiting repeated character sequences~\citep{stamatatos_survey_2009,neal_surveying_2018}. 
Natural language, in particular, allows for high compression ratios due to its strong predictability (English has an entropy of at most 1.75 bits per character). 
For example, PPMD employs finite-order Markov language models for compression, which are highly effective in predicting character sequences in natural text but are also sensitive to increased entropy caused by text obfuscation~\citep{bevendorff_divergence_based_2020}.
Accordingly, compression-based \ac{aa} methods are considered character-based approaches.

They are further classified as profile-based methods. In this framework, an author profile is first constructed for each candidate author by concatenating all texts attributed to them and then compressing the resulting sequence. 
The disputed text is subsequently concatenated with each author profile and compressed as well. 
The difference in compression size between (i) the concatenated profile with the disputed text and (ii) the profile alone is then calculated~\citep{stamatatos_survey_2009,elmanarelbouanani_authorship_2014,neal_surveying_2018}. 
The author whose profile yields the smallest difference is selected as the most likely author~\citep{stamatatos_survey_2009,elmanarelbouanani_authorship_2014}.

The rationale behind this approach is that texts written by the same author can typically be compressed more efficiently than texts produced by different authors~\citep{stamatatos_survey_2009,elmanarelbouanani_authorship_2014}.

% RAR is the most accurate one \citep{elmanarelbouanani_authorship_2014}.
% \citet{elmanarelbouanani_authorship_2014} include the Normalized Compressor Distance (NCD) as a distance measure for compression-based methods. % Chap. 4.2
% \citet{stamatatos_survey_2009} claim that probabilistic approaches are faster in comparison to compression models.
% \citet{neal_surveying_2018} state that LZ77 is a lossless data compression algorithm that is used to compress data by detecting duplicates.

% Compression-based models can also be considered similarity-based measures which are slow 
% since the compression algorithm is called for each training text \citep{stamatatos_survey_2009,neal_surveying_2018}.



\subsubsection{Unmasking Method}
\label{subsec:unmasking}

The meta-learning approach Unmasking algorithm was first proposed by \citet{koppel_authorship_2004}.
Meta learning is a technique where the system learns to learn based on learning successes and failures.
It is based on the idea that omitting discriminant features and the consequent drop in accuracy of the classifier can be used for inference of the author of the unseen text.
For Unmasking, (1) an unseen text is chunked, such that the non-overlapping chunks compose multiple samples belonging either to the author or to a different author.
Next, one \ac{svm} is trained for each candidate author to discriminate the disputed texts' chunks from the candidate author's texts.
The \acp{svm}' features are usually frequencies over the $n=250$ highest average frequency words.
(2) The 10-fold cross validation accuracy for the trained model are obtained.
(3) For the next iteration, omit the most discriminating features among those left.
(4) Repeat steps (3) and (4).
(5) Another linear \ac{svm} classifier is trained on the accuracy curve, its central-difference gradients (first- and second order), 
and its gradients sorted by magnitude.
This classifier is used to predict the whether the texts originate from the same author.

After a few iterations, the classifier is no longer able to discriminate between the unseen text and the texts of the true author~\citep{stein_intrinsic_2011,tyo_state_2022,bevendorff_divergence_based_2020,koppel_authorship_2004,stamatatos_survey_2009} 
Two texts are probably written by different authors if the differences between are robust to changes in the underlying feature set used to represent the documents.

To operationalize this idea, differences are measured using classification via cross-validation accuracy~\citep{koppel_authorship_2011,bevendorff_generalizing_2019,bevendorff_divergence_based_2020,potthast_stylometric_2018,koppel_authorship_2004}, 
creating a performance degradation curve~\citep{tyo_state_2022,koppel_authorship_2004}.
An \ac{svm} is trained to classify the degradation curve to determine whether two text originated from the same author~\citep{tyo_state_2022,bevendorff_generalizing_2019,koppel_authorship_2004}.
Steep decrease in the curve indicates that the two texts are similar, and thus, written by the same authors~\citep{potthast_stylometric_2018,koppel_authorship_2004}.
% Provided that the unseen text is very large, this method can handle small open candidate sets \citep{koppel_authorship_2011}.
% koppel_determining_2014, pg. 1 + bevendorff_generalizing_2019 chap. 3.1 incl. algo: based on text chunks of length >= 500 words each
% \citet{koppel_determining_2014,bevendorff_generalizing_2019} claim that effective unmasking requires input documents to be large 
% (i.e. > 10000 words~\citep{koppel_determining_2014}, book-length~\citep{bevendorff_generalizing_2019}, 
% $\geq$ 5000 words (500 words per chunk) \citep{bevendorff_divergence_based_2020}).
% Otherwise the training set becomes too sparse and no descriptive curves can be generated 
% \citep{bevendorff_generalizing_2019,bevendorff_divergence_based_2020}.

% generalized unmasking
\citet{bevendorff_generalizing_2019,bevendorff_divergence_based_2020} propose creating chunks by oversampling words in a bootstrap aggregating manner. 
Each text is a pool of words, from which words are sampled without replacement.
The pool is replenished if it is exhausted before the chunk has sufficiently many words.
Since the random sampling of unmasking features introduces variance, unmasking is performed multiple times and the curves are averaged.
The algorithm is displayed in \autoref{alg:generalized_unmasking}.
The content of the while loop is, except the number of removed features (\citep{koppel_authorship_2004}: 6 total), similar to the original unmasking algorithm \citep{koppel_authorship_2004}.

\begin{algorithm}
    \caption{Generalized Unmasking Algorithm \citep{bevendorff_generalizing_2019,bevendorff_divergence_based_2020}}
    \label{alg:generalized_unmasking}
    \begin{algorithmic}[1]
    \Procedure{Unmasking}{$A$, $B$}
        \Comment{$A$, $B$: input documents}
    
        \State $\mathcal{C}_A \gets \text{RandomChunks}(A, 30, 700)$ \Comment{30 chunks, 700 words each}
        \State $\mathcal{C}_B \gets \text{RandomChunks}(B, 30, 700)$
        \State $\mathcal{F} \gets \text{TopFreqWords}(A, B, 250)$
        \State $\mathcal{C} \gets \mathcal{C}_A \cup \mathcal{C}_B$

        
        \While{$|\mathcal{F}| \geq 0$}
        \State $a \gets \text{CVAcc}(\mathcal{C}_A, \mathcal{C}_B, \mathcal{F}, linSVM)$ \Comment{Append $10$-fold cross-validation accuracy}
        \State $\mathcal{F} \gets \mathcal{F} \setminus \mathcal{F}_{\text{top}}^{\pm}$ \Comment{Remove top $5$ most significant positive and negative features}
    
        \EndWhile
    
        \State \Return List of recorded accuracies $a$
    \EndProcedure
    \end{algorithmic}
\end{algorithm}

% hyperparameters
% The most important hyperparameters are the number of chunks, the number of words per chunk, the size of feature vectors, 
% the number of word removals per round, and the number of averaged unmasking runs.
% More chunks result in generally shallower curves while shorter features vectors or more removals produce steep curves.
% Ideally, curves are not too steep and granular enough to allow distinguishing between different same and different author pairs.
% \citet{bevendorff_bias_2019} recommend 25 to 50 chunks, vector sizes of 250 to 400 features, not fewer than 5, yet not more than 20 removals per round, 
% between 500 and 700 words per chunk and about 10 runs to average for a curve.
% They increase the minimal distance between the \ac{svm} hyperplane and the decision boundary, i.e. their confidence parameter $c$, to increase precision.
% In a medium- to high-assurance scenario (where \acp{fp} should be avoided, but are not entirely critical), they recommend $c \geq 0.6$.
% If \acp{fp} should be avoided at all costs, they recommend $c \geq 0.7$.
% \citet{bevendorff_bias_2019} claim that, for this approach, hyperparameter tuning is simpler than for black box approaches.

% % metric results
% \citet{bevendorff_bias_2019} report that the generalized unmasking approach heavily prioritizes precision 
% opposed to compression-based approaches that balance precision and recall.
    






\begin{figure}[htbp]
    \centering
    \includesvg[width=\textwidth]{images/unmasking/unmasking.svg}
    \caption{Workflow of Generalized Unmasking~\citep{bevendorff_generalizing_2019}: (1) Create chunks by oversampling words of disputed and candidate texts and represent them using word frequencies. (2) Obtain \ac{svm} accuracy. (3) Eliminate most discriminative features. (4) Repeat from (3)-(4).}
    \label{fig:unmasking}
\end{figure}



  
\subsubsection{\imp{} Method}
\label{sec:impostor_method_theory}


\begin{definition}
    [\imp{} method]
    This method extends the ngram-unmasking method, i.e. iteratively omitting most influencely features (repeated feature subsampling \citep{koppel_determining_2014})
    from a trained classifier and classifying the accuracy drop.
    It takes score of how often an author is predicted after each feature-elimination step.
    The final prediction is made based on this score \citep{tyo_state_2022}.
\end{definition}


\begin{definition}
    [Hard Negative Mining]
    This method updates the model during training only with the most difficult examples in each batch.
    In the \ac{aa} context, difficult is defined as the most similar two texts from different authors, 
    which makes the decision the most difficult.
    \citet{tyo_state_2022} claim that the \ac{av} setting is strictly easier since 
    it most compare to only a single text.
    Due to the fact, that the most difficult example is model-dependent, \ac{av} problems can be made harder 
    but they can not exist of exactly the hardest negatives.
\end{definition}


\begin{definition}
    [Domain]
    The domain include topic, genre, register, idiolect, time period etc. \citep{bischoff_importance_2020}.
\end{definition}
  
\begin{definition}
    [Domain variables]
    These include topic, genre and language \citep{bischoff_importance_2020}.
\end{definition}

\begin{definition}
    [within-domain]
    Experiments with P=Q.
    Hence, it is necessary to ensure all texts are mutually from the same domain \citep{bischoff_importance_2020}.
    \begin{table}[tbp]
        \centering
        \caption{Typical scheme $S_1$ for \ac{aa} problem instances, where A, B, are authors and P, Q domains and 
        the vertical mapping denotes which author has written in which domain. 
        For training, texts from A and B take turn; for testing, previously unseen texts from A and B are used \citep{bischoff_importance_2020}.}
        \label{tab:within_domain_aa}
        \begin{tabular}{|l|ll|ll|}
        \hline
        \textbf{Scheme $S_1$} & \multicolumn{2}{l|}{\textbf{training}} & \multicolumn{2}{l|}{\textbf{testing}} \\ \hline
        \textbf{authors} & \multicolumn{1}{l|}{A} & B & \multicolumn{1}{l|}{A} & B \\ \hline
        \textbf{domains} & \multicolumn{1}{l|}{P} & Q & \multicolumn{1}{l|}{P} & Q \\ \hline
        \end{tabular}%
    \end{table}
\end{definition}


The \impAppr{} leverages random projections to lower dimensional spaces (i.e. random set of features set to zero is a projection).
\begin{figure}[htbp]
    \centering
    \includesvg[width=\textwidth]{images/imposter/imposter.svg}
    \caption{\imp{}.}
    \label{fig:impostor}
\end{figure}

% AV -> open-set
\ac{av} is an open-set problem, meaning that the author of an anonymous document 
may or may be not be part of the set of candidate authors.

% AA -> closed-set
\ac{aa} is a closed-set problem, meaning that the author of an anonymous document
is part of the set of candidate authors.
For each candidate author, writing samples are available.
The task is to determine the author of the anonymous document from the set of candidate authors.

% reduction: closed-set AA -> open-set AV
\citet{koppel_determining_2014} state that all closed-set \ac{aa} problems are reducible to the \ac{av} problem.
The reverse is not true.
To reduce the \ac{aa} problem to the \ac{av} problem, we solve a \ac{av} problem, i.e. if text was written by a candidate author, 
for each of the respective candidates.
Ideally, we receive one positive answer for the correct candidate author and negative answers for all other candidates.

% complexity
\citet{koppel_determining_2014} explain that the \ac{av} problem is more complex than the \ac{aa} problem.
They claim that the ability to solve a closed-set \ac{aa} problem does not imply the ability to solve an open-set \ac{av} problem.

% open-set identification/ AA = many candidates problem
\citet{koppel_determining_2014} define the many-candidates problem, or the so-called open-set identification problem:
Given a large set of candidate authors, determine which, if any, of them wrote a given anonymous document.
According to \citet{koppel_determining_2014}, the many-candidates problem can be solved reasonably well: \autoref{lst:many_candidate_algo}.
\section{Impact of \acs{llm} on Authorship Verification}

With the advances in \ac{gai} come risks and opportunities.
Similarly, we can frame \acp{llm} as authors weakening our trust in the authencity of digital texts, but also see them as valuable tools to enhance our regulatory methods.

\subsection{\acsp{llm} as authors}
There is no general definition of when a text is \ac{llm} generated rather than co-created by humans with \ac{llm} assistance.
% Obviously, fully generated texts should be marked as \ac{llm} generated.
Minor human edits of \ac{llm} generated texts do not change the fact that the core content was \ac{llm} generated.
If \acp{llm} are used for grammar checking, polishing, and editing suggestion the primary substantial contribution was human.
One could denote these texts "\ac{ai}-revised Human-Written Text"~\citep{wang_stumbling_2024}.

With advances of generative models with regard to mimicking human writing, we have to face the fact that \acp{llm} will play a crucial role in any authorship analysis related tasks from now on.
\citet{llm_detection_av_2025} claim that \ac{llm} detection is not an \ac{aa} task, i.e., a closed-set binary classification where both classes are sufficiently discriminative, but an \ac{av} task, i.e., an open-set one-class classification problem. 

% differences: also more in ~\citep{wang_stumbling_2024}
Despite the advances, there are still some statistical differences on \ac{llm} generated and human authored texts.
\ac{llm} generated texts lack lexical diversity, overuses certain adjectives (e.g. "innovative") and produces longer, more complex sentences.
Moreover, \acp{llm} possess stylistic fingerprints and memorize patterns from the training data.
% lengths
Furthermore, word length averages and distributions across genre differ for \acp{llm} and humans~\citep{llm_detection_av_2025}.

% future of LLMs
However, as \acp{llm} progress, basic heuristics applied by human detectors no longer suffice.
\acp{llm} will become more human-like and thus, \ac{llm} detection will increasingly resemble a human authorship classification task~\citep{llm_detection_av_2025}.


\subsection{\acsp{llm} as Discriminator}
\label{sec:llm_discriminator}

While \acp{llm} can generate coherent text, their performance as direct discriminators in \ac{aa} tasks varies. 
Words with similar meaning, such as "color" and "colour", are mapped to similar vector representations~\citep{altakrori_topic_2021}.
This reduces sensitivity to subtle language system differences, which are often critical for identifying an author. 
Consequently, relying on \acp{llm} as the primary discriminator in \ac{aa} or \ac{av} tasks may be suboptimal.

Rather than serving as standalone discriminators, \acp{llm} are more appropriately employed as supporting tools. 
For instance, they can generate cross-domain training data to improve model robustness. 
Moreover, prior methods assessed the extent to which text changes under \ac{llm}-based paraphrasing, since machine generated text tends to undergo minimal alteration, whereas human-authored text exhibits greater variation~\citep{mao_raidar_2024}.
Conversely, in contexts where privacy is a concern, \acp{llm} can facilitate author obfuscation through controlled paraphrasing. 







\section{\acl{av}}

\section{\ac{av} as One-Class categorization}
\label{sec:av_one_class}

% \citet{koppel_authorship_2004} claim research had shown that linear separators work well for text categorization.
% Linear models include Naive Bayes (linear separator for two classes), 
% Window and Exponential Gradient and linear \acp{svm} \citep{koppel_authorship_2004}. 

% \subsection{\ac{aa} framework}
% \citet{koppel_authorship_2004} state that the following framework solve a number of real world \ac{aa} problems:
% \begin{enumerate}
%     \item Construction of appropriate feature vectors
%     \item Construction of a distinguishing model via a learning algorithm
%     \item Assessment of effectiveness of methods using k-fold cross-validation or bootstrapping
% \end{enumerate}

\subsection{Chunks}
\citet{koppel_authorship_2004} propose chunking texts such that each chunk is of approximately equal length, 
and at least 500 words without breaking paragraphs. 

\subsection{Definitions}
For author $A$ and book $X$, \citet{koppel_authorship_2004} define the following:
If $A$ is not the author of $X$, $A_X$ is the set of all works by author $A$.
If $A$ is the author of $X$, $A_X$ is the set of all works by author $A$ except $X$.
A pair of $A_X$ and $X$ is called \emph{same-author} if X was authored by $A$.
A pair of $A_X$ and $X$ is called \emph{different-author} if $X$ was not authored by $A$.

\subsection{Initial feature set}
The initial feature set consists of the 250 words with the highest average (over $X$ and $X_A$) frequency \citep{koppel_authorship_2004}.

\subsection{Features for meta-classifier}
\citet{koppel_authorship_2004} propose the following features for the meta-classifier 
(where $i$ is the number of elimination steps):
\begin{itemize}
    \item accuracy after $i$ elimination steps
    \item accuracy difference between round $i$ and round $i+1$
    \item accuracy difference between round $i$ and round $i+2$
    \item $i^{th}$ highest accuracy drop in one iteration
    \item $i^{th}$ highest accuracy drop in two iterations
\end{itemize}

The vectors are grouped by \emph{same-author} and \emph{different-author} pairs and thus, 
used to train a meta-learning scheme.

\subsection{Negative examples for Elimination method}
\citet{koppel_authorship_2004} state that negative examples are neither exhaustive nor representative.
They propose using words of several authors $A_1, ..., A_n$ roughly filling the same profile as candidate $A$ 
in terms of geography, chronology, culture and genre.
$A_1, ..., A_n$ are said to collectively represent class not-$A$.

\subsection{Elimination method}

The elimination method is only used to overrule positive predictions.
Hence, it can eliminate \acp{fp}.
One can frame it as a filter which is applied after or before unmasking.

% training
\citet{koppel_authorship_2004} learn a model for $A$ and against not-$A$, 
and multiple models for $A_i$ and against not-$A_i$.
% inference
Then, $X$ is tested against all of these models.
$A(X)$ is the percentage of examples of $X$ classed as $A$ rather than not-$A$ 
(i.e., $A_i(X)$ analogously).
If $A(X)$ is not larger than all $A_i(X)$, $A$ is not the author of $X$.
If $A(X)$ is larger than all $A_i(X)$, conclude nothing.




 % before related work
    \chapter{Related Work}
\label{chap:related_work}

This work is different to the work of \citet{koppel_determining_2014} and \citet{kocher_unine_2015} 
in that it uses \acp{llm} to generate imposter texts.

% AV

% LLM detection using generative models
%% AA against LLMs
With the recent advances of \ac{nlg} come new challenges in text authorship.
The new technologies may be misused for fraudulent activities to scam naive or inexperienced users~\citep{uchendu_authorship_2020,bhattacharjee_fighting_2024}.
\citet{uchendu_authorship_2020} identified three authorship tasks essential for fighting fraudulent activities:
(1) Given two texts $t_1$ and $t_2$, determine whether they were produced by the same method (i.e. human author or a specific \ac{nlg} method).
(2) Given a text $t$, determine whether it was human authored or machine generated (Turing Test).
(3) Given a text $t$, find its author among $k+1$ candidates, which consists of one human and $k$ machines.
They compare classical \ac{ml} models, neural models and state-of-the-art \ac{aa} models as classifiers 
for these single- (Problem 1 and 2) and multi-class (Problem 3) tasks.
Their findings include, that as of 2020, most \ac{nlg} methods were distinguishable from human authors, 
but some \acp{llm} proved difficult to detect.
%%% compared to our work
In the following, we consider (1) \ac{av}, (2) classical \ac{llm} detection, and (3) closed-set \ac{aa}.
Our approach differs from the work of \citet{uchendu_authorship_2020} in that our candidates (i.e. imposters) do not include a human author (3), 
but only \acp{llm}.
Moreover, we use different classifiers originally designed for \ac{av}, rather than \ac{aa}.

%% LLM (gpt-3.5, GPT-4) as detector
\citet{bhattacharjee_fighting_2024} evaluate using an \ac{llm} as classifier for \ac{llm} detection.
They use \ac{gpt}-3.5 and \ac{gpt}-4 to classify texts as human or machine generated.
They find that \ac{gpt}-3.5 performs better when being fed simple instructions, rather than constrained prompts.
They find that \ac{gpt}-4 predicts almost exclusively \ac{ai} generated texts, 
while \ac{gpt}-3.5 predictions are more reliable (especially for actually human authored texts).
%%% compared to our work
Our work differs from theirs in that we use \acp{llm} to generate imposter texts specific to the candidate text, 
rather than using the publicly available dataset TuringBench with previously generated texts.

%% DetectGPT: Perturb (Mask), score, compare (unsupervised)
\citet{mitchell_detectgpt_2023} propose DetectGPT, a method that is threefold:
(1) They perturb the input text by (1.1) masking out random 2-word spans until 15 \% of the text is masked. 
Masked spans are replaced (1.2) with words from an off-the-shelf (i.e. not finetuned to target domain) \ac{llm} (e.g. T5-3B). 
These perturbations are semantically similar paraphrases of the original text.
(2) They score (in terms of log probability) each perturbed text using a scoring \ac{llm} 
(ideally their candidate \ac{llm}, but it works also with any \ac{llm}, though scores deteriorate). 
(3) The difference of the score of the original text and the average score of the perturbed texts is denoted perturbation discrepancy $d$. 
(4) Normalize $d$ by the standard deviation of the scores of the perturbed texts.
(5) Based on a threshold $\epsilon=0.1$, classify the original text as human authored or machine generated 
(formally Local Perturbation Discrepancy Gap hypothesis).
If $d$ is positive, the original text is likely machine generated.
If $d$ is near zero, i.e. $d < \epsilon$, the original text is likely human authored.
\citet{mitchell_detectgpt_2023} motivate their method by the observation that generated texts tend to occupy 
negative curvature regions of the model's log probability function (i.e. they lie on the local maximum of the manifold).
When the text is machine generated, it lies on a local maximum, 
and perturbing it will lead to lower log probabilities of perturbed texts.
When the text is human authored, it does not lie on a local maximum to begin with, 
rendering log probabilities of perturbed texts similar either bigger or smaller than the original text.
Averaging the log probabilities of perturbed human texts leads to a value that is 
close to the original text's log probability (i.e. a perturbation discrepancy $d$ near zero).
Even though, DetectGPT works best when the source (i.e. generating) \ac{llm} and the scoring \ac{llm} are the same 
(requires white-box access to the \ac{llm}), 
it works also with different \acp{llm} as surrogate for the source model when scoring (in a black-box case).
%%% compared to our work
We can not supply a white box setting, because we do know the source \ac{llm} that generated the imposter texts.
%%%% imposters and perturbations
However, this approach is similar to our approach, because perturbing texts can be seen as a 
form of imposter generation (especially as we use paraphrases). 
%%%% sample from the source model
Both approaches try to sample from the probability distribution of the source model either 
by using imposters (via prompting an \ac{llm}) or by perturbing the original text (using an \ac{llm}).
%%%% input
While the imposter approach is an \ac{av} task (i.e. input is a disputed and a candidate text), 
DetectGPT receives a disputed text and a candidate \ac{llm} as input.
%%%% similarity measure
While we use a similarity measure on traditional n-gram frequency vectors, 
\citet{mitchell_detectgpt_2023} require a scoring \ac{llm} to compute the perturbation discrepancy $d$.
Hence, our approach is easier in terms of computational resources and requirements.

%% LLM rewrite LLM texts less than human texts (no AA, but edit distance hypothesis)
RAIDAR~\citep{mao_raidar_2024} builds upon the invariance property of \acp{llm}, 
which states that prompting an \ac{llm} to rewrite a machine generated text will introduce little change.
They motivate this by the observation that (different) autoregressive models produce similar patterns and thus, 
consider texts generated by (different) \acp{llm} as high quality that do not require rewriting.
Change is measured by the edit distance between the original text and the rewritten text. 
\citet{mao_raidar_2024} propose using an edit distance based on the Levenshtein distance or \ac{bow} representations.
RAIDAR operates on character level rather than using deep neural network features, and it does not require the original generating model for classification. 
RAIDAR fails to detect \ac{llm} generated texts in out-of-distribution scenarios (i.e. different domains than training), 
or when \ac{llm} were explicitly instructed to produce text prone to heavy \ac{llm} modification when being asked to rewrite the text \citep{li_learning_2025}.
Based on RAIDAR (\citep{mao_raidar_2024}), \citet{li_learning_2025} propose fine-tuning an \ac{llm} to rewrite human authored texts more than machine generated text.
Classification is carried out by comparing the edit distance of the original text and the rewritten text to a threshold.
\citet{li_learning_2025} admit that their approach is slow in inference time, 
since a candidate text has to be rewritten multiple times (about 200 different prompts) to obtain a reliable score.
\citet{mao_raidar_2024} find that the quality of perturbation based models (i.e. rewriting) for \ac{llm} detection correlates with the perturbation model size.
\citet{mitchell_detectgpt_2023} find a negative correlation (\textcolor{red}{TODO: chapter 2 vorletzter Absatz}) between the size of the perturbation model and the performance of DetectGPT.
%%% compared to our work
%%%% generation of texts during inference
Both approaches are similar to our work in that they use \acp{llm} to generate texts during inference.
We do not fine-tune an \ac{llm} for paraphrasing but use off-the-shelf models (like RAIDAR).
%%%% similarity measure
All these approaches compute the similarity of the original text and the generated text.
However, we do not use edit distance (i.e. Levenshtein distance) as similarity measure.
%%%% limitations
This approach is unable to detect which \ac{llm} generated the text.

%% LLMDet: Proxy to perplexity (problem: requires access to the LLM to build the dictionary)
Perplexity is a reliable statistical metric for attributing texts to \acp{llm}~\citep{zhang_llmdet_2023}.
Unfortunately, perplexity requires access to \acp{llm}' parameters (i.e., white-box detection).
\citet{zhang_llmdet_2023} propose LLMDet, a method that uses a proxy to perplexity, 
where a dictionary of frequent n-gram (frequent among $n$ randomly prompted generated texts per \ac{llm}) 
next token probabilities is pre-computed (i.e. requiring access to the \ac{llm}), 
and is subsequently used during inference to approximate perplexity by replacing $x_{<i}$ in $p(x_i | x_{<i})$ with an n-gram.
Since the construction of the dictionary requires access to the \ac{llm}, LLMDet requires contribution of the closed-source model owners.
The disputed text is tokenized and the proxy perplexity is calculated for each model and thus, constructing a proxy perplexity vector.
This vector is input to a trained classifier.
%%% compared to our work
Proxy perplexity could be used as a baseline for our approach, though it requires access to the \ac{llm} and is thus not applicable in our case.

%% Mirror Minds: extract query, genrate two paraphrases, compare & classify via threshold (very similar to our work)
\citet{baradia_mirror_2025} propose (1) extracting a query from the disputed text, which captures the essence of the text, 
(2) generating two paraphrases of the original text using the query as input prompt to two \acp{llm}, 
and (3) comparing the paraphrases to the original text via the BLEU and the METEOR score.
Both score capture syntactic similarity, even though \citet{baradia_mirror_2025} argue they also capture semantic similarity.
They use the maximum across the two models per similarity measure as a final score pair.
Classification of the resemblance to \ac{ai} generated content requires a threshold.
%%% compared to our work
%%%% same approach
This approach is similar to our approach in that it uses \acp{llm} to generate paraphrases of the original text.
Moreover, it compares the original text to the generated paraphrases as in a \ac{aa} problem. % rather AI detection?????
%%%% similarity measure
We do not use BLEU or METEOR as similarity measure, nor do we compare directly on paraphrase-level (i.e. BLEU calculates n-gram overlap) 
but construct our own frequency based n-gram vectors input vector similarity metrics.
%%%% they discard information, solve another problem
However, this approach discards the information which \ac{llm} produced the most similar paraphrase. 
While our goal is to solve an \ac{aa} problem (i.e. multiclass classification), 
\citet{baradia_mirror_2025} solve a binary classification problem (i.e. human vs. \ac{ai} generated text). % teil: background ML etc.
    \chapter{\acs{llm}-based Impostors Method}
\label{chap:llm_impostor_method}

A central challenge in authorship analysis is the presence of confounding factors such as register, genre, topic, and target audience. 
These dimensions restrict linguistic options, thereby obscuring the stylometric features relevant for authorship-related tasks. 
To address this issue, we propose an extension of the \impAppr{} originally proposed by \citet{koppel_determining_2014}.
We intend to leverage \acp{llm} to systematically control for such confounders. 
By generating texts under carefully constrained conditions, we obtain \imps{} that preserve the domain variables of the original reference text, thereby serving as controlled proxies for replicating the actual generation process of the reference text.



\section{Impact of Confounders on Authorial Style}
\label{sec:contextual_factors}

State-of-the-art models for authorship analysis exhibit strong sensitivity to domain shifts. 
Performance often deteriorates sharply when models are applied in out-of-distribution scenarios, i.e.\ in cross-domain settings, a phenomenon largely attributable to the influence of confounders such as topic, genre, and register~\citep{Sundararajan_style_18,bischoff_importance_2020}

Confounders are problematic because they influence the very features used to characterise authorial style. 
Topical vocabulary, for example, can dominate lexical distributions, while genre-specific conventions shape syntax. 
As these factors cannot be cleanly separated from genuine stylistic markers~\citep{bischoff_importance_2020}, they obscure the style markers.

If topics can be represented by characteristic word distributions, then a document can be seen as a subset of words selected by the author, reflecting individual preferences in synonym choice~\citep{altakrori_topic_2021}. 
Consequently, texts by the same author on different topics may appear unrelated, whereas texts by different authors on the same topic may seem deceptively similar.

Empirical evidence confirms the severity of this problem.
Both contemporary \acp{av} methods~\citep{Thomas_cross_topic_24}, and \ac{llm} detection approaches, such as DetectGPT~\citep{mitchell_detectgpt_2023,Wu_ODD_challenges_2025} suffer significant effectiveness degradation in out-of-distribution scenarios.

As a result, authorship research has diverged into two main directions. 
One line of work aims to identify domain-invariant features, a challenge that remains largely unresolved~\citep{bischoff_importance_2020}. 
The other focuses on in-domain scenarios, in which confounding factors are deliberately controlled. % \ac{id}
While restricting tasks to a single topic or genre does not fully eliminate the entanglement between content and style, it reduces its impact sufficiently to produce stable and interpretable results. 
Moreover, some studies indicate that simply using domain or topic labels is insufficient to control for topic similarity in corpora, as this approach ignores semantic relationships between topics~\citep{sawatphol_cross_topic_av_24}.


\section{\acs{llm}-based \Imp{} Generation}
\label{sec:impostor_generation}

% good \imps{}: hard negatives
Following the notion introduced by \citet{koppel_determining_2014}, ideal \imp{} texts can be understood as hard negatives, i.e.\ documents not written by the candidate author, yet sufficiently similar in style to be difficult to distinguish from the candidate's own writing. 
The quality of \imps{} directly affects model effectiveness.
If they are too different from the candidate text, models are trained on trivial contrasts, yielding \acp{fp}. 
Conversely, if \imps{} are too similar to the candidate, the risk of \acp{fn} increases.
This manifests as increased recall in the former case and increased precision in the latter case.

% obstacles for \imp{} selection in the past
Traditional \imp{} selection techniques struggled to simultaneously control multiple confounders:
The original \texttt{fixed} approach samples from a pool of unrelated texts, while the \texttt{on-the-fly} approach fails to align genre with the candidate text~\citep{koppel_determining_2014}. 
Since authorial style is tightly entangled with domain variables, \imps{} produced by these methods differ systematically from the candidate text, weakening their utility. 
Ideally, \imp{} generation should replicate the conditions of reference text production, including task, topic, register, target audience and century. 
In practice, however, this information is rarely available.
In \ac{av}, the candidate author may be deceased, unavailable, or unwilling to cooperate.

% heuristics: paraphrase
\acp{llm} make it possible to approximate this ideal. 
By conditioning generation on the candidate text, domain variables such as topic, genre, and register can be more tightly controlled. 
In particular, paraphrasing offers a heuristic for simulating the text generation process. 
The model produces stylistic variants of the candidate's writing while maintaining semantic alignment.


\subsection{Hard Negative Mining via Paraphrasing}

The lack of a universally accepted definition of paraphrases complicates their generation and evaluation~\citep{gohsen_task_oriented_2024}. 
In this work, paraphrases qualify as suitable \imps{} if they preserve the topic, genre, tone, and length of the reference text, while introducing enough lexical and syntactic variation to avoid near-duplication. 
Limited semantic divergence is permitted, including mild hallucinations, which we define as added details that remain broadly consistent with the original topic.
We propose two approaches to paraphrasing, namely one-step and two-step paraphrasing, as illustrated in \Cref{fig:paraphrasing_approaches}.
We expect the recall scores of the \ac{llm}-based \impAppr{} to decrease while precision increases, i.e.\ yielding higher confidence in positive labelled text pairs, due to harder negatives leading to fewer positive predictions.

\begin{figure}[h]
    \centering
    \includesvg{images/paraphrasing/idea/Paraphraser_architectures.svg}
    \caption[Different paraphrasing approaches]{Proposed one-step and two-step paraphrasing approaches.}
    \label{fig:paraphrasing_approaches}
\end{figure}


Paraphrasing is often analysed along two main dimensions. 
At the lexical level, it involves word-level substitutions such as synonym replacement, typically achieved through rule-based or thesaurus-driven methods. 
At the syntactic level, paraphrasing entails changes to sentence structure, which can be realised through monolingual machine translation~\citep{zhou_paraphrase_2021} or by explicitly prompting \acp{llm}~\citep{kurt_pehlivanoglu_comparative_2024}.
% While more fine-grained taxonomies exist~\citep{zhou_paraphrase_2025}, they are not central here, as this study does not involve training or constructing paraphrase models directly.
Paraphrase quality is largely determined by prompt design and model selection~\citep{Wu_ODD_challenges_2025}.


\subsection{One-step Paraphrasing}

In our one-step approach, \imps{} are generated directly via \ac{llm} prompting. 
Effective prompting proved crucial. 
We found that specifying constraints was vital for valid outputs. 
Moreover, placing task instructions at the end reduced instruction neglect in long-context settings (e.g. \dataGutenberg{} corpus). 
We evaluated three prompt variants.

The first prompt explicitly decomposes the sentence into subject, verb, and object before reconstruction, encouraging lexical diversity. 
\begin{quote}
    \textit{Paraphrase the given sentence by identifying the main subject, verb, and object. Replace each with synonyms or closely related words, adjusting grammar naturally. Keep the new sentence close in length to the original. Output only the final paraphrased sentence.}
\end{quote}

The second prompt instructs the model to preserve meaning while varying wording and structure. 
Both prompts emphasise concise outputs restricted to the paraphrase itself.
\begin{quote}
    \textit{Paraphrase the sentence above without changing its meaning. Use different words and vary the sentence structure while keeping the tone consistent. Keep the new sentence similar in length to the original. Output only the paraphrased sentence, with no explanations or extra text.}
\end{quote}

The third prompt extends the second prompt by explicitly instructing the \ac{llm} to produce paraphrases three times as long as the reference text.
\begin{quote}
  \textit{Paraphrase the text above without changing its meaning. Use different words and vary the sentence structure while maintaining a consistent tone. Your paraphrase should be three times as long than the original. Output only the paraphrased sentence, with NO explanations or extra text.}
\end{quote}

Initial experiments with T5-based HuggingFace models showed poor instruction adherence and frequent malformed outputs. 
Our final experiments therefore employed the models listed in \Cref{tab:base_llms}, accessed via the \ac{gwdg} infrastructure\footnote{\url{https://docs.hpc.gwdg.de/} (August 31, 2025)}.
Further details on these models are provided in \Cref{app:language_models}.

\begin{table}[h]
\centering
\caption{\acp{llm} used for paraphrasing.}
\label{tab:base_llms}
% \resizebox{\textwidth}{!}{%
\begin{tabular}{@{}ll@{}}
\toprule
\textbf{Model ID}                    & \textbf{Host} \\
\midrule
qwen3-32b                            & \ac{gwdg}    \\
mistral-large-instruct               & \ac{gwdg}    \\
openai-gpt-oss-120b                  & \ac{gwdg}    \\
meta-llama-3.1-8b-instruct           & \ac{gwdg}    \\
\bottomrule   
\end{tabular}%
% }
\end{table}


\subsection{Two-step Paraphrasing}

The two-step approaches are inspired by prior literature~\citep{bevendorff_overview_2024, ayele_overview_2024}, and differ in how they extract and utilize auxiliary information (i.e.\ metadata) from the source text before generating the paraphrase. 
The two steps enforce a clearer disentanglement of content and style at the cost of increased computational overhead.

As shown in \Cref{fig:two_step_paraphraser}, the two-step paraphrasing decomposes the task into information extraction and text generation. 
In the first stage, the model identifies domain variables of the input text via a \pextractor{}.
These variables include tone, register, time period, target audience, and genre.
Different approaches extract an additional feature specific to their approach.
The additional features are a bullet point summary, the authorial task, topic, or title. 
The extractor prompts can be derived from \Cref{app:extractor_prompts}.
In the second stage, a paraphrase is generated from this metadata. 
Metadata can be derived automatically from the reference text or provided as ground truth when calling the paraphraser function.
All generator prompts can be found in \Cref{app:generator_prompts}.

\begin{figure}[ht]
  \centering
\resizebox{\textwidth}{!}{%
\begin{tikzpicture}[node distance=2cm]
% Dataset node
\node[dataset] (dataset) {Dataset};

% Anchor for Reference drawing
\node[draw=none, right=of dataset] (refpos) {};

% Draw Reference as an outer rectangle + snakes inside
\node[draw, thick, rounded corners=6pt, minimum width=1.8cm, minimum height=2cm,
      right=of dataset, inner sep=4pt] (reference) {};

% Add multiple snakes inside reference (smaller, stacked)
\foreach \y in {0.8,0.4, 0.0,-0.4,-0.8}{
  \draw[black, very thick, decorate, decoration={snake,amplitude=1pt,segment length=5pt}]
    ($(reference.west)+(0.3,\y)$) -- ($(reference.east)+(-0.3,\y)$);
}

% Caption
\node[below=0.2cm of reference] {Reference};


% Other blocks
\node[above=0.8cm of reference.north] (meta) {Ground Truth Metadata};
\node[block, right=of reference.east] (extractor) {Extractor};
\node[block, right=of extractor] (generator) {Generator};

% paraphrase
\node[draw, thick, rounded corners=6pt, minimum width=1.8cm, minimum height=2cm,
      right=of generator, inner sep=4pt, teal!80!black] (paraphrase) {};

% Add multiple snakes inside reference (smaller, stacked)
\foreach \y in {0.8,0.4, 0.0,-0.4,-0.8}{
  \draw[teal!80!black, very thick, decorate, decoration={snake,amplitude=2pt,segment length=5pt}]
    ($(paraphrase.west)+(0.3,\y)$) -- ($(paraphrase.east)+(-0.3,\y)$);
}
\node[below=0.2cm of paraphrase] {Paraphrase};

% Arrows
\draw[arrow] (dataset) -- (reference.west);
\draw[arrow] (dataset) -- node[pos=0.4, above left=2pt and -12.5pt] {$\{0,1\}$} (meta);
\draw[arrow] (reference.east) -- (extractor);
\draw[dashedarrow] (extractor) -- (generator);
\draw[arrow] (generator) -- (paraphrase);
\draw[dashedarrow] (meta) -- (generator);


\end{tikzpicture}
}
  \caption[Two-step paraphraser]{Visual description of the two-step paraphraser.
  The \pgenerator{} is fed metadata information from the \pextractor{} or from ground truth data.}
  \label{fig:two_step_paraphraser}
\end{figure}


The translation-based paraphraser is another two-step approach in which a text is translated into another language and then back into English. 
We use French as second language.
While conceptually simple, it has been shown to produce only limited stylistic variation~\citep{zhou_paraphrase_2025}.
 
    \chapter{Experimental Setup}
\label{chap:experimental_setup}

In the following, we will outline the experimental setup for the experiments we ran.
This includes not only allocation, preprocessing and pair selection of each dataset, but also a description and motivation of the experiments carried out.


\section{Dataset}
\label{sec:dataset}

Since our method extends the original \impAppr{} proposed by \citet{koppel_determining_2014}, we first obtained the datasets used in their study to validate our implementation and reproduce their results. 
The original experiments were carried out on the \dataBlog{} and \dataStudent{} datasets, described in detail in \Cref{subsec:original_data}.
In addition to these, we incorporated two supplementary datasets, \dataPan{} and \dataGutenberg{}, presented in \Cref{subsec:additional_data}. 
Following the general description of all datasets, we outline our preprocessing pipeline in \Cref{subsec:dataset_preprocessing} and conclude with the text-pair selection procedure in \Cref{subsec:dataset_text_pair_selection}.


\subsection{Original Data}
\label{subsec:original_data}

% Blog
The \dataBlog{} corpus~\citep{blog_dataset_2006} consists of blog posts collected from \textit{blogger.com} on or before August 2004, with each blog authored by a single user.
According to the Kaggle repository~\footnote{\href{https://www.kaggle.com/datasets/rtatman/blog-authorship-corpus?resource=download}{Kaggle dataset \texttt{rtatman/blog-authorship-corpus}} (26.07.2025)}, the dataset contains \num{681288}~posts from \num{19320}~bloggers, averaging approximately 35~posts and \num{7250}~words per author.
Each record includes the following metadata: \texttt{id}, \texttt{gender}, \texttt{age}, \texttt{topic}, 
\texttt{sign} (referring to the author's zodiac sign), \texttt{date}, and \texttt{text}.

% student essays
The \dataStudent{} dataset is not publicly available due to the presence of sensitive student information. 
Access may be requested from J. W. Pennebaker, the official custodian.
The dataset comprises \num{7052}~student essays written for five assignments by a cohort of 950 university students in 2006~\citep{koppel_determining_2014}.
Its restricted availability makes it a particularly valuable testbed for evaluation, as it is highly unlikely to have been incorporated into the training data of \acp{llm}.

The assignments include (1) a stream-of-consciousness task, (2) a reflections on childhood, (3) a self-assessment of personality, (4) a thematic apperception test, and (5) four examples of four different theories.
% Most files are named solely by the author ID, whereas those from the first assignment follow the format \texttt{2006\_authorID}.
Following \citet{koppel_determining_2014}, our dataset is built from the first four assignments. 
The dataset provides metadata collected from structured columns in the \texttt{.dat} files and derived from file names, which we combined into a unified resource. 
Metadata includes year, author ID, author name, political orientation, task, sex, ethnicity, and teacher.


\subsection{Additional Data}
\label{subsec:additional_data}
To broaden the evaluation scope of the \impAppr{}, we incorporated additional datasets that both control for confounding factors such as genre and topic and consist of texts with verified, undisputed authorship. 
Both the \dataPan{} and \dataGutenberg{} datasets meet these criteria.

% PAN20: Fanfiction
The \dataPan{} corpus~\citep{bischoff_importance_2020} comprises fanfiction texts sourced from \textit{fanfiction.net}.
Each text belongs exclusively to one fandom (i.e.\ thematic category), with no crossovers between fandoms.
According to \href{https://pan.webis.de/clef20/pan20-web/author-identification.html}{the official \acs{pan} website}, 
train and test set originate from two different fanfictions.
Dataset features include \texttt{id}, \texttt{fandoms}, and \texttt{pair}, where the latter contains the paired texts.
An additional \texttt{jsonl} file provides the ground truth for each pair, specifying \texttt{id}, \texttt{same} (ground truth label for \ac{av}), and \texttt{authors}.

% Gutenberg
The \dataGutenberg{} dataset~\footnote{\url{https://www.gutenberg.org/} (26.07.2025)} contains a curated selection of literary works from Project Gutenberg, a digital library dedicated primarily to older works whose U.S. copyrights have expired.
As of this writing, the collection contains over \num{75000}~digitized and proofread e-books contributed by volunteers according to their website.
% For our experiments, we selected 19 works authored by 7 writers from the 16th to 19th centuries, the distribution of genres is given in \Cref{tab:genre_counts_gutenberg}.
For our experiments, we selected 19~works authored by 7~writers from the 16th to 19th centuries, covering nine dramas, nine fiction texts and one poetry work.
Metadata for these works was manually extracted from the \href{https://www.gutenberg.org/}{Project Gutenberg website} and Wikipedia.

% \begin{table}[]
% \centering
% \caption{Unique value count of genres of \dataGutenberg{} dataset.}
% \label{tab:genre_counts_gutenberg}
% % \resizebox{\textwidth}{!}{%
% \begin{tabular}{@{}ll@{}}
%     \toprule
% \textbf{Genre} & \textbf{Count} \\
% \midrule
% Drama          & 9              \\
% Fiction        & 9              \\
% Poetry         & 1    \\
% \bottomrule         
% \end{tabular}%
% % }
% \end{table}

% \textcolor{orange}{We augment the \dataStudent{} dataset with artificial generated texts and denote this dataset \dataArtificialStudent{}.
% It contains pairs of student essays written in response to different academic assignments. 
% The dataset includes both human-written essays and artificially generated paraphrases created by \acp{llm} instructed to simulate student writing. 
% Artificially generated essays are produced using \acp{llm} prompted to emulate an 18-year-old college freshman's voice, taking into account demographic features like sex, ethnicity, and political orientation.
% Each pair is labelled as same-author or different-author indicating whether the text was generated artificially. 
% }

\subsection{Dataset Preprocessing}
\label{subsec:dataset_preprocessing}

To control confounding factors that influence authorial style, we preprocess each dataset twice:
(1) Prior to generating the arrow dataset file and (2) before using the \impAppr{}.
This two-stage approach addresses both experimental scenarios, where all material is prepared in advance, and inference scenarios in which the \impAppr{} is applied directly to texts.
The preprocessing process was designed to meet the following requirements:
\begin{enumerate}
    \item Removal of all formatting and layout information to produce plain text
    \item Cropping texts to match the length of the shorter text in each pair
    \item Removal of texts with less than 700 words
\end{enumerate}
For a controlled evaluation environment in our \impAppr{}, we opted to work with relatively small, curated datasets rather than scaling to larger collections.  
Text-length filtering is essential to the creation of optimal testbeds for \ac{av}.
To determine the necessary preprocessing steps, we examined our datasets and analysed their respective artefacts.

Since both the \dataBlog{} and the \dataPan{} dataset originate from the Internet, some of their documents contain HTML fragments such as \texttt{< >} enclosed tags.
Upon inspection of the \dataGutenberg{} dataset, we found that certain patterns reappear due to the presences of many plays.
Based on these findings, we collected eight preprocessing steps we deemed potentially useful. 
Notably, we do not consider actor instructions (e.g. character cues) or structural elements (e.g. chapter headings) as part of authorial style.
Hence, we remove these patterns using regular expressions.
You may find the regular expression attached in \Cref{app:regex_preproc}. % of \Cref{ch:appendix}.
Note that the perceived utility of these steps should be reevaluated for other datasets.
Next, we analysed the effect of individual preprocessing steps on vocabulary size.  
We define the vocabulary of a dataset as unique tokens across all texts (regardless of their text length or split) of that dataset.
In line with \citep{koppel_determining_2014}, tokens are space-free character 4-grams.
We leave the capitalisation of tokens unchanged. 
The results are displayed in \Cref{fig:preprocesing_impact_vocab_size}.

We found that HTML-specific preprocessing steps had minimal impact on the size of vocabularies.
Similarly, the removal of artefacts specific to theatre plays had little effect on the overall vocabulary size.
The results aligned with expectations, as the vocabularies are defined over space-free 4-gram units that, due to their limited length, exhibit a high likelihood of redundancy.
Consequently, repetitive patterns are mapped on the same vocabulary item.
Once the repetitive patterns are filtered out, the vocabulary lacks their respective items, which are not many.

Based on these considerations, our preprocessing steps include removing HTML artefacts, play artefacts, newlines, converting UTF-8 to ASCII, and stripping leading and trailing whitespace.
We opted to forgo lowercasing the texts, as our preliminary analysis indicated that lowercasing had no meaningful effect on any dataset while potentially discarding deliberate authorial capitalisation choices.
Due to preprocessing, our \dataPan{} version differs from those applied elsewhere.

\begin{figure}[htbp]
    \centering
    \includesvg[width=\textwidth]{images/dataset/impact_preprocessing_steps.svg}
    \caption[Effect of preprocessing steps on vocabulary size.]{Effect of preprocessing steps on vocabulary size. The vocabulary contains unique space-free character 4-grams.
    For this experiment, there was no minimum text length.
    For \dataPan{}, train and test split were combined.}
    \label{fig:preprocesing_impact_vocab_size}
\end{figure}


\subsection{Selection of Text Pairs}
\label{subsec:dataset_text_pair_selection}

For the \dataBlog{}, \dataStudent{}, and \dataGutenberg{} datasets, we selected pairs of texts according to specific criteria to control potential confounding factors.
Only texts with a minimum length of \num{700}~words were considered eligible. 
For the \dataPan{} dataset, we retained the existing pairs in the arrow dataset, but filtered out texts with less than \num{700}~words. 
All datasets include both same-author and different-author pairs. 

For the \dataBlog{} dataset, we matched text pairs on topic, year, gender, and author age. The training set contains 80\% of the data and the test set 20\%, with distinct topics in each split.

For the \dataStudent{} dataset, following \citet{koppel_determining_2014}, we drew all text pairs from different tasks, with authors matched by sex, ethnicity, and political orientation. Pairs were split into training (70\%) and test (30\%) sets.
The test portion is larger than in the \dataBlog{} dataset because each author typically contributes only one essay per task, so including only one of four tasks in the test set would prevent pair formation.

For the \dataGutenberg{} dataset, we selected text pairs that share the same genre and century and split the authors into training (80\%) and test (20\%) sets.

% minimum length necessesary for AV/ AA
The choice of minimum text length was informed by literature research.
\citet{bevendorff_generalizing_2019}\ used text chunks of at least 700 words for an unmasking approach, while \citet{koppel_authorship_2004}\ set the minimum to \num{500}~words.
Recent work~\citep{llm_detection_av_2025} identifies \num{2500}-\num{4000}~characters to be sufficient for effective \ac{llm} detection framed as \ac{av}.

Regardless of the selection criteria, the final datasets contain only three columns: \texttt{authors}, \texttt{pair}, and \texttt{same}.
The \texttt{pair} column contains the texts of the pair as a list of strings,
the \texttt{authors} column contains the authors of the texts as a list of strings,
and the \texttt{same} column indicates whether the texts originate from the same author (\texttt{True}) or from different authors (\texttt{False}).
Descriptive statistics for all preprocessed datasets are provided in Table~\ref{tab:data_stats}.

% The \dataArtificialStudent{} dataset is split into training and test sets using a stratified approach, ensuring that all combinations of author type (human vs. \ac{llm}), pair type (same- vs. different-author), and artificial generation (True vs. False) are proportionally represented in both splits. 
% Since the \dataArtificialStudent{} dataset is artificially created, its feature differ from the other datasets. 
% Each record contains the assignment names and descriptions, the paired texts, the authors, and metadata flags indicating author sameness and artificial generation.

% The presence of short artefacts in the \dataArtificialStudent{} dataset cannot be attributed to human-authored texts, as these have been filtered to meet a minimum length.
% Hence, the artificially generated texts exhibit shorter average lengths compared to other datasets.

\begin{table}[H]
% \begin{sidewaystable}
\centering\small
\caption[Statistics of preprocessed datasets.]{Statistics of preprocessed datasets \dataBlog{}, \dataGutenberg{}, \dataPan{}, and \dataStudent{}. %, and \dataArtificialStudent{}.
$p_s$, $p_{\neg s}$ denote same-author and different-authors pairs, while $l_w$, $l_c$ denote text length in words and characters, respectively.
}
\label{tab:data_stats}
\resizebox{\textwidth}{!}{%
\begin{tabular}{@{}lrrrrrrrrr@{}}   % numbers should be right aligned, text left aligned
\toprule
dataset & \# pairs & \# authors & \# $p_s$ & \# $p_{\neg s}$ & \diameter $l_w$ ($l_c$) & max $l_w$ & $\sigma_{l_w}$ & median $l_w$ \\
\midrule
\dataBlog{}            & 11565 & 5997  & 6204 & 5361  & 6249.94 (1154.25)     & 115365 & 1493.97 & 913 \\
\dataGutenberg{}       & 12    & 7     & 6     & 6     & 437870.75 (78698.79) & 297704 & 68329.91 & 60282 \\
\dataPan{}           & 66905 & 52771 & 35616 & 31289 & 21418.76 (3914.76)   & 55413 & 512.19 & 3889 \\
\dataStudent{} & 224  & 163   & 112   & 112  & 4403.73 (851.45)     & 1520 & 138.15 & 807  \\
% \dataArtificialStudent{} & 110 & 32 & 50 & 60 & 661.39 (3581.46) & 1769 & 267.08 & 703 \\
\bottomrule
\end{tabular}%
}
\end{table}
% \end{sidewaystable}


% regardless of experimental design
\section{Evaluation Measures}
\label{sec:evaluation_measures}

In the following sections, we review state-of-the-art quantitative evaluation metrics for \ac{av} in \Cref{subsec:av_quality_measures} and for paraphrase generation in \Cref{subsec:paraphrase_evaluation}.
While human judgment is subjective, quantitive metrics are designed to be comparable and reproducible, providing a more objective basis for evaluation.


\subsection{Authorship Verification Quality Measures}
\label{subsec:av_quality_measures}

Since \ac{av} forms the core of \ac{aa}, and because every \ac{aa} task can be reduced to \ac{av}, this section focuses primarily on \ac{av}. 
Evaluation relies on standard classification metrics, each with distinct advantages and limitations.

% \paragraph{Accuracy}
% The most straightforward metric is accuracy (\Cref{eq:accuracy}), which measures the proportion of correctly classified cases across all samples. 
% While intuitive, accuracy can be misleading in scenarios with class imbalance. 

% \begin{equation}\label{eq:accuracy}
%     \text{Accuracy} = \frac{\text{\acs{tp}} + \acs{tn}}{\acs{tp} + \acs{tn} + \acs{fp} + \acs{fn}}
% \end{equation}

% \paragraph{Precision, Recall \& $\operatorname{F_{1}}$}
Precision $\operatorname{P}$ and recall $\operatorname{R}$ address the limitations of accuracy in class imbalance scenarios by focusing on the effectiveness for the positive class, that is, the correct identification of same-author pairs. 
Precision measures the proportion of positive predictions that are correct, while recall quantifies the proportion of \ac{tp} cases that are successfully detected. 
Because these metrics often trade off against each other, their harmonic mean, the $\operatorname{F_{1}}$ score, is commonly used to provide a balanced assessment of effectiveness. 
\Cref{eq:f1} shows the computation of the $\operatorname{F_{1}}$ score~\citep{neal_surveying_2018}.

\begin{equation}\label{eq:f1}
     \operatorname{F_{1}} = \frac{2\mathrm{P}  \mathrm{R}}{\mathrm{P} + \mathrm{R}}
\end{equation}

% \paragraph{$\operatorname{F_{0.5u}}$ \& $\operatorname{c@1}$}
% Beyond these basic measures, \ac{av} research has introduced metrics that explicitly account for difficult borderline cases. 
% The modified $\operatorname{F_{0.5u}}$ score penalizes non-answers by treating them as \acp{fn}~\citep{bevendorff_overview_2024}. 
% This places additional weight on correctly deciding same-author cases~\citep{weerasinghe_feature_vector_difference_2021}, thereby evaluating the ability of \ac{av} methods to abstain from hard samples~\citep{tyo_state_2022}. 
% The $\operatorname{c@1}$ metric, by contrast, was designed to reward abstention from particularly ambiguous cases. 
% It does so by granting unanswered problems partial credit, equal to the average accuracy on the remaining cases~\citep{bevendorff_overview_2024}, thus encouraging systems to remain silent when uncertain.

% However, in the present work, $\operatorname{c@1}$ and $\operatorname{F_{0.5u}}$ are not appropriate measures. 
Although we also used $\operatorname{F_{1}}$ and accuracy in our work, we opted to exclude them in this thesis to keep results simple and comparable to the original work by \citet{koppel_determining_2014}.
More advanced metrics, such as $\operatorname{F_{0.5u}}$ and $\operatorname{c@1}$, assume that models can explicitly abstain by assigning a score of exactly 0.5, a convention used in \acs{pan}'s shared tasks~\citep{tyo_state_2022,bevendorff_overview_2024,kocher_unine_2015}. 
In our setting, the model produces only the binary outcomes "same author" or "don't know". 
Since there is no natural mechanism for producing a calibrated 0.5 score, abstention cannot be meaningfully represented. 
One could introduce a second threshold to create an artificial abstention region, but this would not align with the open-set nature of \ac{av}. 
The different-author class is inherently ill-defined, as the set of possible authors is unbounded and cannot be exhaustively represented. 

% Moreover, we do not report average effectiveness scores across different recall values, since we do not need good results across different thresholds, but only high effectiveness for one threshold.
% For this reason, the focus in this thesis remains on precision, recall, and $\operatorname{F_{1}}$. 




\subsection{Paraphrase Evaluation}
\label{subsec:paraphrase_evaluation}

There is no universal definition of what constitutes a paraphrase. 
Definitions vary in degree of semantic equivalence required. 
This conceptual ambiguity makes the task of evaluation especially challenging, since different applications may prioritize different aspects such as fidelity to meaning, stylistic variation, or grammatical well-formedness.
Because of this, paraphrase evaluation must account both for syntactic diversity and for the extent to which semantic content is preserved. 

Existing approaches can broadly be grouped into automatic and human-based methods. 
Automatic measures attempt to quantify the similarity between a candidate and a reference paraphrase using algorithmic techniques. 
These methods can be further distinguished by the linguistic level at which they operate. 
Some focus on syntactic structure, while others evaluate semantic preservation~\citep{gohsen_captions_2023}. 
Human evaluation, in contrast, remains the gold standard, as it naturally incorporates all of these dimensions.
In the following, we focus on both automatic evaluation strategies. 

\input{chapter/section-05/quantitative_evaluation_metrics.tex}


% \subsubsection{Qualitative Evaluation}
% \label{subsec:qualitative_evaluation}

% Human qualitative evaluation can combine syntactic and semantic dimensions more reliable than any automatic metric proposed.
% Naturally, when being asked to evaluate the quality of a paraphrase, individuals will score syntactic difference from the reference text, the readability from the paraphrase and semantic similarity to the reference text.
% Evaluation is usually formalized via a Likert scale~\citep{gohsen_captions_2023}.


\section{Experimental Setup}
\label{sec:experimental_setup}
% for each: Question to answer, experiment design, same language (unified description), when duplicate: short description and reference to other occurence (but avoid sole references)
% questions can be specific, should be related to research question(s)

The following experiments are designed to systematically evaluate the extension of the \impAppr{}~\citep{koppel_determining_2014} with \ac{llm} generated impostors. 

We begin by reproducing the original experiments by \citet{koppel_determining_2014} to ensure comparability between our implementation and previously reported results. 
Next, we investigate the quality of paraphrases used in \imp{} generation. 
In particular, we compare our one-step paraphrasing strategy with our two-step approach based on the paraphrasing measures from \autoref{subsec:traditional_quantitative_evaluation_measures}. 
Then, we evaluate paraphrase quality across different numbers of text chunks.
We then analyse the effect of syntactic similarity between paraphrases and original texts on \imp{} scores. 
Then, we compare different \ac{av} methods including the \impAppr{} with several strategies for \imp{} generation for varying sources of input text pairs. 
This allows us to examine the relative strengths of different \ac{av} approaches for specific \ac{av} tasks.


\subsection{Exp.\ 1: Reproduction of Original Work}

To assess the validity of our extension to the traditional \impAppr{}, we first verified the correctness of our implementation. 
For this purpose, we designed two experiments, which we ran on a subset of 100~pairs from the training and test sets of the \dataBlog{} and \dataStudent{} dataset respectively. 
Half of the selected samples belong to the same-author class.

\begin{table}[h]
\centering\small
\caption{Exp.\ 1: \impAppr{} configuration.}
\label{tab:config_exp1}
\begin{tabular}{@{}rrlrrl@{}}   % numbers should be right aligned, text left aligned
\toprule
Experiment & \# Impostors & Generation & Rounds & Top $n$ & Upsample \\
\midrule
1(a) & \textit{Variable} & Fixed & 100 & \num{100000} & False \\
1(b) & 50 & \textit{Variable} & 100 & \num{100000} & False \\
\bottomrule
\end{tabular}%
\end{table}

\paragraph{Exp.\ 1(a): Varying number of \imps{}.}
The first experiment evaluates the effect of varying the number of \imps{} while setting the \imp{} generation method to \texttt{fixed}.
All other hyperparameter values are set to the default values reported by \citet{koppel_determining_2014}\ (cf.~Table~\ref{tab:config_exp1}). 
Adhering \citet{koppel_determining_2014}, we compute precision and recall scores across different thresholds.
\textcolor{orange}{For comparison, reference precision-recall points reported by \citet{koppel_determining_2014}\ are included in our visualization.} 
Based on their description, we deduced that their reported scores were obtained using the \dataBlog{} dataset.

\paragraph{Exp.\ 1(b): Varying \imp{} generation.}
The second experiment evaluates different \imp{} generation methods while keeping the number of \imps{} fixed.
Again, all other hyperparameter values are set to the default values reported by \citet{koppel_determining_2014}\ (cf.~Table~\ref{tab:config_exp1}). 
Following \citet{koppel_determining_2014}, we compare the \texttt{fixed} and \texttt{on-the-fly} \imp{} generation methods with the baseline approaches unsupervised min-max similarity, unsupervised cosine similarity, and supervised linear \ac{svm}.

As in the first experiment, precision and recall are used as the primary evaluation metrics. 
Consistent with \citet{koppel_determining_2014}, we calculate precision and recall with respect to both the same-author and different-author class, alternately treating each as the reference class.
We note that the different-author class is ill-defined, as \ac{av} constitutes a one-class classification scenario in which covering representative instances of this class is infeasible.

\subsection{Exp. 2: Comparison of different Paraphrasers}
\label{subsec:comp_paraphrasers_setup}

Next, we wanted to assess our paraphrasing approaches.
We hence designed two experiments.
The first experiment computes state-of-the-art paraphrasing measures for all paraphrasers on different datasets.
The second experiments aims to evaluate the ability of our two-step models to adhere to instructions.
We tested their proficiency extracting metadata and generating paraphrases of similar length as the reference text.

\paragraph{Exp. 2(a): Quantitative evaluation.}

We select one text from the \dataBlog{}, the \dataGutenberg{} and the \dataStudent{} dataset, respectively.
For the \dataGutenberg{} dataset, we load ground truth metadata. % Student Essay: not used, even though existent
The paraphraser configurations contain two different temperatures for two-step paraphrasers, and two different prompts for one-step paraphrasers.
We create one paraphrase for each text configuration pair.
Evaluation measures include BLEU, ROUGE1, ROUGE2, ROUGEL, ROUGELsum, METEOR, BERTScore Precision, BERTScore Recall, BERTScore F1, SBERT \ac{wms}, SBERT cosine similarity.
Based on these we also compute syntactic and semantic similarity, as well as Gohsen Delta~\citep{gohsen_captions_2023}.
We save the extremest (min, max) paraphrases per metric.
The scores are subsequently visualized via syntactic-semantic scatters, score distributions, and radar plots per paraphraser and per prompt. 

\paragraph{Exp. 2(b): Evaluation of prompt adherence.}
For the second experiment, we select five samples from the \dataBlog{}, \dataGutenberg{} and the \dataStudent{} datasets. 
Our extractor extracts the genre, the topic, and the century of each input text.
Extracted and ground truth values are lowercased and stripped from leading and trailing whitespaces. 
For single genre and topic values, we compute the cosine similarity on their respective SBERT embedding.
For the genre extraction, we split the extractors' result by comma and use the maximum similarity.
Since the ground truth topic usually consists of multiple topics separated by comma, we split them into a list and use the maximum similarity.
For century match, we processed the result of the extractor by mapping \textit{present}, \textit{current}, and \textit{now} to 21, then extracting digits and omitting the last two digits from any numbers with at least three digits and finally adding one if the original digit was not divisible by 100.
We then use the ground truth as a baseline $b$ for extracted century $a$ and compute $\frac{a}{b}$.
\textcolor{red}{TODO: examples}
While the \dataBlog{} metadata comes its csv dataset, the \dataStudent{} metadata is derived from existing information about and in the dataset, the \dataGutenberg{} metadata was completely manually curated.
Additional to the previously mentioned processing steps for century values, ground truth century date values were casted to dates.
Moreover, we obtain the relative length difference of the paraphrase and original text for every selected sample. 
This concludes in an evaluation of our paraphrasing (extractors) in terms of genre, topic, century and length similarity.

\subsection{Exp.\ 3: Paraphrasing Chunks}
\label{subsec:paraphrasing_chunks}

We designed this experiment to evaluate whether chunk-to-chunk paraphrases exhibit better control than text-to-text paraphrases, since chunks contain fewer topic changes than whole texts in theory.
We use one text from the \dataBlog{}, \dataGutenberg{}, and the \dataStudent{} dataset, respectively.


\begin{figure}[htbp]
  \centering
  \includesvg[width=\linewidth]{images/paraphrasing/experiments/chunks/setup/chunk_api_calls.svg}
  \caption[Paraphrase configuration hyperparameters.]{Breakdown of individual hyperparameters in the paraphrase configuration.
  We use one document per dataset, chunked into one to five sections and paraphrased with all nine paraphrasers in two variance inducing settings (i.e.\ prompt for one-step, temperature for two-step).
  This amounts to a total of 936 API calls. 
  }
  \label{fig:chunks_api_calls}
\end{figure}


First, texts are chunked preserving sentences.
Chunks are filled with sentences in sentence order such that each chunk roughly contains the same number of words.
Second, paraphrase configurations are defined.
Each one-step paraphraser is paired with two prompts, while each two-step paraphraser is paired with two temperatures.
Third, each chunk is paraphrased with all configurations.
These steps account for a minimum of 936 API calls for paraphrasing.
Each component of the configuration is displayed in \autoref{fig:chunks_api_calls}.
Finally, for each paraphrase, we compute \ac{bleu}, \ac{rouge}-1, \ac{rouge}-2, \ac{rouge}-L, \ac{rouge}-Lsum, METEOR, \ac{bert}\-Score Precision, \ac{bert}\-Score Recall, \ac{bert}\-Score F1, \ac{sbert} \ac{wms}, \ac{sbert} cosine similarity.
Final scores per metric for each text-configuration pair are computed by averaging the scores of its constituent text chunks.
The adequate formula is given in \autoref{eq:avg_chunks} and an example is illustrated in \autoref{fig:mean-bleu}.

\begin{equation}
    score(t) = \frac{1}{\#\text{ chunks}}\sum_{i=1}^{\#\text{ chunks}}score(c_i)\text{, for chunk }c_i \in \text{text }t
\label{eq:avg_chunks}
\end{equation}

\begin{figure}[ht]
  \centering
\resizebox{0.9\textwidth}{!}{%
\begin{tikzpicture}[line join=round,line cap=round, >=latex, font=\sffamily]

% --- Left black container with three chunks ---
\draw[black, very thick, rounded corners=6pt]
  (-0.2,3.6) rectangle (3.0,-0.4);

% three inner rounded rectangles
\foreach \y in {2.8,1.6,0.4}{
  \draw[black, thick, rounded corners=5pt] (0.15,\y+0.45) rectangle (2.65,\y-0.45);
  % squiggle inside
  \draw[black, thick, decorate, decoration={snake,amplitude=1.2pt,segment length=6pt}]
    (0.45,\y-0.15) -- (2.35,\y-0.15);
  \draw[black, thick, decorate, decoration={snake,amplitude=1.2pt,segment length=6pt}]
  (0.45,\y+0.15) -- (2.35,\y+0.15);
}

% --- Colored BLEU labels next to each chunk ---
\node[anchor=west, text=teal!80!black, scale=1.2]  at (3.6,2.8) {BLEU $=\,0.1$};
\node[anchor=west, text=orange!85!black, scale=1.2] at (3.6,1.6) {BLEU $=\,0.5$};
\node[anchor=west, text=violet, scale=1.2]         at (3.6,0.4) {BLEU $=\,0.3$};

% --- n_chunks = 3 (black) ---
\node[anchor=west, text=black, scale=1.2] at (0.0,-1.0) {$\#\text{ chunks}=3$};

% --- Arrow to the right and mean BLEU expression ---
\draw[black, very thick, ->, >=latex] (7.7,1.6) -- (9.1,1.6);

\node[anchor=west, text=black, scale=1.4] at (9.3,1.6)
  {$\varnothing\ \text{BLEU} \;=\; \displaystyle
   \frac{\textcolor{teal!80!black}{0.1}+\textcolor{orange!85!black}{0.5}+\textcolor{violet}{0.3}}{3}$};

\end{tikzpicture}
}
  \caption[Computation of the mean \ac{bleu} score over chunks.]{Computation of the mean \ac{bleu} score over three text chunks of a text.}
  \label{fig:mean-bleu}
\end{figure}


\subsection{Exp.\ 4: Impact of Syntactic Similarity on \impApprTitle{} Performance}
\label{sec:syn_sim_impact_}

We designed this experiment in order to assess whether the syntactic similarity of generated paraphrases, i.e.\ the difficulty of hard negatives, influences the effectiveness of the \impAppr{}.
We conducted this experiment on both the \dataBlog{} and \dataStudent{} datasets, selecting 15 samples each from the training and test splits. 
The detector was configured according to Table~\ref{tab:imp_syn_sim_config}.

\begin{table}[h]
\centering\small
\caption{Exp.\ 4: \impAppr{} configuration.}
\label{tab:imp_syn_sim_config}
\begin{tabular}{@{}rlrrl@{}}   % numbers should be right aligned, text left aligned
\toprule
\# Impostors & Generation & Rounds & Top $n$ & Upsample \\
\midrule
50 & LLM & 100 & \num{100000} & False \\
\bottomrule
\end{tabular}%
\end{table}

For generation, we loop through all \ac{llm}-based paraphrasers until we successfully created 50 \imps{}.
One-step paraphrasers are used with both prompts.
Predictions on the test set were obtained by thresholding the detector’s scores with a decision threshold was determined using Youden’s J statistic on the training set.
We computed the average syntactic similarity on the test set. 
Following \citet{gohsen_captions_2023}, we define average syntactic similarity $\diameter_{syn}$ as the mean of the \ac{bleu}, \ac{rouge}-1, and \ac{rouge}-L scores. 
For each input pair in the test set, we calculated
(1) the average syntactic similarity between the two texts in the pair, (2) the mean average syntactic similarity between the candidate reference text and its paraphrases, and (3) the mean average syntactic similarity between the disputed text and the paraphrases.

We further grouped samples based on (1), (2), (3) and the difference (2)–(1). 
For each group, we computed accuracy, precision, recall, and F1 score of the detector’s predictions. 
The average values for each metric in a bin are presented in a bar chart.



\subsection{Exp. 5: Comparing \acs{av} Methods in Traditional Human-Human Scenario}
\label{subsec:imp_gen}

We want to answer the question of how our \ac{llm}-based \imp{} generation performs compared to (a) traditional \imp{} generation methods in the \impAppr{}~\citep{koppel_determining_2014}, and compared to (b) \acl{sota} \ac{av} methods in the traditional \ac{av} scenario.
We thus, create 10 same- and 10 different-author pairs from the \dataStudent{}. % (llm_detection_scenarios.py)
% We thus, create 100 same- and 100 different-author pairs from the \dataStudent{} (comp_av.py)
% 1 each on and the \dataBlog{} datasets % 1 for Blog, 100 for Student Essays
It is noteworthy, that an approach predicting only one output will obtain an accuracy of $0.5$.
The \impAppr{} and \unmasking{} detector configuration are shown in \autoref{tab:exp5_imp_config} and \autoref{tab:exp5_unmasking_config}, respectively.

\begin{table}[h]
\centering\small
\caption{Exp. 5: \impAppr{} configurations.}
\label{tab:exp5_imp_config}
\begin{tabular}{@{}rlrrl@{}}   % numbers should be right aligned, text left aligned
\toprule
\# Impostors & Generation & Rounds & Top $n$ & Upsample \\
\midrule
50 & \textit{Variable} & 100 & \num{100000} & False \\
\bottomrule
\end{tabular}%
\end{table}

\begin{table}[h]
\centering\small
\caption[Exp. 5: Unmasking configurations.]{Exp. 5: Unmasking configurations. CV denotes cross-validation.}
\label{tab:exp5_unmasking_config}
\begin{tabular}{@{}rrrrl@{}}   % numbers should be right aligned, text left aligned
\toprule
\# CV Folds & \# Chunks & Rounds & Top $n$ & Upsample \\
\midrule
3 & 60 & 30 & \num{250} & False \\
\bottomrule
\end{tabular}%
\end{table}

For each impostor generation method, we computed accuracy, precision, recall, and F1 score for different thresholds. 


\subsection{Comparing \ac{av} Methods}

This experiment evaluates the performance of the \impAppr{} in comparison to established \ac{av} methods.
As baselines, we employ generalized unmasking~\citep{bevendorff_generalizing_2019} and the compression-based approach PPMD approach.
All methods share the common characteristic of operating on lower-dimensional representations of the input text pair to determine whether both texts originate from the same author.

Following the experimental setup described in \autoref{subsec:imp_gen}, we assess performance using  accuracy, precision, recall, and the F1 score. 





 % ggf experimente setup extra/ oder extra + datensätze + exp+ analyse (Tabelle)
    \chapter{Experimental Results}
\label{chap:experimental_results}

This chapter presents the experimental results. 
To begin, we reproduce original experiments to validate our approach, after which we compare different paraphrasing techniques. 
We then analyse paraphrase quality across text chunk sizes and prompts, and finally contrast traditional \ac{av} methods with the \ac{llm}-based \impAppr{}.


\section{Exp.\ 1: Reproduction of Original Work}

Our first experiment covers the reproduction of the original results~\citep{koppel_determining_2014}.
It is noteworthy, that we do not expect exact replication since we could only reimplement the approach to our best knowledge.
Both Mr. Koppel and Mr. Winter were very forthcoming when we contacted them regarding implementation details that could help to improve our implementation.
Unfortunately, the code was no longer traceable and neither author could recall the preprocessing steps.
After consultation with Mr. Winter, he assured that our preprocessing steps seem reasonable.
This, however, certainly is one of the reasons why our implementation diverges from the original work.

\paragraph{Exp.\ 1(a): Varying number of \imps{}.}

For this experiment, we generate \imps{} using the \texttt{fixed} method while varying only the number of \imps{}.
\citet{koppel_determining_2014}\ report results of this experiment exclusively on the \dataBlog{} dataset.
While we compare their findings with ours, it is important to note that direct comparability is limited due to differences in the underlying datasets, as our results are based on the \dataStudent{} corpus.
In contrast to the original study, our \impAppr{} recall–precision curves intersect, suggesting that the results reported by \citet{koppel_determining_2014}\ exhibit stronger separation.
In our experiments, the number of \imps{} exerts minimal influence on precision and recall.
Nevertheless, similar to the original findings, we observe that using 50 \imps{} yields marginally superior performance.


\begin{figure}[htbp]
    \centering
    \includesvg[width=\textwidth]{images/imposter/reproduction_koppel_figures/fig2/student_essays/student_roc_prec_recall_curve_r100_top100000_dif_n_imp.svg}
    \caption[Recall-precision curves for the various sized \imp{} set sizes.]{Recall-precision curves for the various sized \texttt{fixed} \imp{} set sizes on the \dataStudent{} dataset.}
    \label{fig:student_essays_dif_n}
\end{figure}

% \begin{figure}[htbp]
%   \centering
%   \begin{subfigure}[b]{0.48\textwidth}
%     \centering
%     \includesvg[width=\linewidth]{images/imposter/reproduction_koppel_figures/fig2/student_essays/student_roc_prec_recall_curve_r100_top100000_dif_n_imp.svg}
%     \caption{\dataBlog{} \textcolor{red}{TODO: needs to run (15.09.2025)}}
%     \label{fig:blog_dif_n}
%   \end{subfigure}
%   \hfill
%   \begin{subfigure}[b]{0.48\textwidth}
%     \centering
%     \includesvg[width=\linewidth]{images/imposter/reproduction_koppel_figures/fig2/student_essays/student_roc_prec_recall_curve_r100_top100000_dif_n_imp.svg}
%     \caption{\dataStudent{}}
%     \label{fig:student_essays_dif_n}
%   \end{subfigure}
%   \caption{Recall-precision curves for the various sized \imp{} set sizes.}
%   \label{fig:repr_diff_n_imps_fixed}
% \end{figure}


\paragraph{Exp.\ 1(b): Varying \imp{} generation.}

Adhering to \citet{koppel_determining_2014}, we compare multiple \ac{av} methods with variants of the \impAppr{} that employ different \imp{} generation strategies, evaluating them in the \ac{av} scenario with human-authored text pairs.
The \imp{} generation strategies include sampling from a \texttt{fixed} set of potential \imp{} candidates, and \texttt{on-the-fly} \imp{} generation. 
The baselines introduced by \citet{koppel_determining_2014} are \ac{svm}, and unsupervised similarity based approaches.
Each class is treated as the positive (i.e. reference) class once, with the other class serving as the negative class. 
This choice affects how the evaluation scores are computed, but not how the thresholds are fit during the training.
Although not explicitly stated, it is likely that \citet{koppel_determining_2014} have trained each approach for both reference class scenarios once.
The original work reports the results for the \dataBlog{} dataset only.

% original results
The recall–precision curves reported by \citet{koppel_determining_2014}\ indicate that \texttt{fixed} \imp{} generation outperforms \texttt{on-the-fly} \imp{} generation.
In their study, the minimum recall values for the different-author class were consistently higher (approximately $0.625$ for \texttt{fixed} and $0.58$ for \texttt{on-the-fly}) than for the same-author class (approximately $0.38$ for \texttt{fixed} and $0.22$ for \texttt{on-the-fly}).
In contrast, precision values exhibited the opposite trend where all approaches achieved higher precision in the same-author setting compared to the different-author setting.

Our interpretation of the results reported by \citet{koppel_determining_2014}\ is presented in \autoref{fig:findings_original_work}.
We argue that these findings are a consequence of the nature of \ac{av} as an open-set, one-class classification problem.
Because the different-author class lacks representative samples, it remains inherently ill-defined.
As a consequence, both \imp{} generation methods tend to yield more different-author predictions than same-author predictions, producing higher recall for the different-author class and higher precision for the same-author class.

\begin{figure}[htbp]
    \centering
    \includesvg[width=\textwidth]{images/imposter/reproduction_koppel_figures/fig2/student_essays/fig2_original_findings.svg}
    \caption[Aggregating original \impAppr{} experiment results.]{Results reported by \citet{koppel_determining_2014} suggest that the different\-author class is more difficult to model, leading to more different-author predictions and consequently higher recall for that class.}
    \label{fig:findings_original_work}
\end{figure}

% our results
Both \citet{koppel_determining_2014}\ and our results indicate that \texttt{fixed} \imp{} generation performs best in the same-author scenario.
Beyond this agreement, however, our recall–precision curves deviate from the original findings.
In the different-author scenario, all methods except \texttt{on-the-fly} generation exhibit low precision, with \texttt{fixed} \imps{} performing worst.
Low precision corresponds to a high number of \acp{fp}, i.e., many same-author pairs incorrectly classified as different authors.
This outcome suggests that training on the same-author class produces systematically poor performance in a different-author class test scenario, as the model is not optimized for it.

Opposed to the original work of \citet{koppel_determining_2014}, our \texttt{on-the-fly} \imps{} lack difficulty explaining the low precision in the same-author and the high precision in the different-author scenario.
Since we had to reduce the number of test samples for \texttt{on-the-fly} \imp{} generation due to API call limits, we suggest analysing this approaches' results with a grain of salt.
However, Mr. Winter noted that our \texttt{on-the-fly} \imp{} generation implementation seems insufficient, due to the rise of bot prevention blocking web scraping and the very limited number of API calls per month.

% \textcolor{red}{TODO: figures SVC to SVM}
\begin{figure}[htbp]
  \centering
  \begin{subfigure}[b]{0.49\textwidth}
    \centering
    \includesvg[width=\linewidth]{images/imposter/reproduction_koppel_figures/fig4/blog/blog_roc_prec_recall_curve_r100_top100000_Same_Author_dif_imp_gen.svg}
    \caption{Same-author reference class. }
    \label{fig:blog_same_author}
  \end{subfigure}
  \hfill
  \begin{subfigure}[b]{0.49\textwidth}
    \centering
    \includesvg[width=\linewidth]{images/imposter/reproduction_koppel_figures/fig4/blog/blog_roc_prec_recall_curve_r100_top100000_Different_Author_dif_imp_gen.svg}
    \caption{Different-author reference class.}
    \label{fig:blog_diff_author}
  \end{subfigure}
  \caption[Recall-precision curves for the \dataBlog{} dataset. ]{Recall-precision curves for the \dataBlog{} dataset. 
  (B) indicates the original baseline approaches from~\citep{koppel_determining_2014}.
  Due to API limit restrictions, the test set for \texttt{on-the-fly} was smaller which is visible in the respective curves.
  Classifiers were not retrained for the different-author reference class scenario explaining the poor results.}
  \label{fig:diff_imp_gen_blog}
\end{figure}


\section{Exp.\ 2: Comparison of Paraphrasers}
\label{sec:comp_paraphrases}

To evaluate the quality of generated paraphrases, we conducted two experiments. 
In Exp.\ 2(a), we assessed paraphrasing using standard quantitative metrics, while in Exp.\ 2(b), we compared the extracted information and text lengths of the generated paraphrases to the ground truth metadata and original texts, respectively.

\paragraph{Exp.\ 2(a): Quantitative evaluation.}

Paraphrasing scores were computed separately for the \dataBlog{}, \dataGutenberg{}, and \dataStudent{} datasets. 
\autoref{fig:sem_syn_blog} presents aggregated semantic and syntactic measurement scores for the \dataBlog{} dataset, while results for \dataGutenberg{} are provided in the Appendix in \autoref{sec:app_paraphrases}.

Syntactic similarity was quantified by averaging \ac{rouge}-1, \ac{rouge}-L, and BLEU scores. 
Semantic similarity was assessed using BERTScore, cosine similarity of BERT-based embeddings, and SBERT \ac{wmd}. 
It is important to note that for our purposes, syntactic diversity is desirable.
High syntactic similarity values may reflect near-identical paraphrases due to n-gram overlap. 
Semantic similarity measures the content overlap between the paraphrase and the original text, often via vector-based cosine similarity. 
% The precise interpretation of these metrics remains somewhat unclear.

\begin{figure}[htbp]
    \centering
    \includesvg[width=\textwidth]{images/paraphrasing/experiments/sem_syn_scatter/Blog_sem_syn_scatter_grouped_by_Paraphraser.svg}
    \caption[Comparison of paraphrasers on the \dataBlog{} dataset.]{Average semantic $\diameter_{sem}$ and syntactic similarity $\diameter_{syn}$ for different paraphraser on the \dataBlog{}.}
    \label{fig:sem_syn_blog}
\end{figure}


Analysis reveals two distinct clusters corresponding to one-step and two-step paraphrasers. 
Most one-step paraphrasers achieve lower syntactic and semantic similarity than two-step paraphrasers. 
The exception is the one-step paraphraser using \texttt{qwen3-32b}, which exhibits higher syntactic similarity than most two-step paraphrasers. 
The translation-based approach emerges as an outlier in terms of both syntactic and semantic similarity. 
Different prompt formulations appear to have minimal impact on these similarity scores (cf. \autoref{sec:app_paraphrases}). 
Overall, most paraphrases fall within the desired quadrant of high semantic similarity with low syntactic similarity.


\paragraph{Exp.\ 2(b): Evaluation of prompt adherence.}

Our two-step paraphrasing approach relies on extracting valid information from the source text. 
To evaluate the quality of extraction by the \pextractor{}, we compared the topic, genre, and century to the ground truth metadata for the \dataBlog{}, \dataGutenberg{}, and \dataStudent{} datasets. 
Genre and topic were evaluated in terms of semantic similarity, while century was assessed via the percentage deviation from the ground truth.

We observed that instructions for the \pextractor{} must follow the input text, as \acp{llm} tend to focus attention toward the end of the input. 
Otherwise, the \pextractor{} failed to produce the requested JSON format for long texts from the \dataGutenberg{} dataset. 
Results across datasets are summarized in \autoref{tab:extraction_eval_stats}, with $\diameter$ and $\sigma$ denoting the mean and standard deviation across the five selected samples. 
The \dataBlog{} dataset proved most challenging for genre and topic extraction. 
While differences in text length between reference and paraphrase were maximal among all datasets, the \pextractor{} performed best on the \dataGutenberg{} dataset.


\begin{table}[h]
\centering
\caption[Extraction performance and length matching for different datasets.]{Extraction performance and length matching for different datasets. Five documents per dataset were processed using the \pextractor{} to obtain genre, century, and topic.}
\label{tab:extraction_eval_stats}
\begin{tabular}{lrrrrrrrr} % numbers should be right aligned, text left aligned
\toprule
 &
  \multicolumn{2}{l}{\textbf{Genre}} &
  \multicolumn{2}{l}{\textbf{Century}} &
  \multicolumn{2}{l}{\textbf{Topic}} &
  \multicolumn{2}{l}{\textbf{Length}} \\
  \textbf{Dataset}
 &
  \textbf{\diameter} &
  \textbf{$\sigma$} &
  \textbf{\diameter} &
  \textbf{$\sigma$} &
  \textbf{\diameter} &
  \textbf{$\sigma$} &
  \textbf{\diameter} &
  \textbf{$\sigma$} \\
  \midrule
\dataBlog{}            & 0.38 & 0.06  & 0.99 & 0.02 & 0.04  & 0.05  & -0.10 & 0.73 \\
\dataGutenberg{}       & 0.58 & 0.14  & 1.00 & 0.04 & 0.3 & 0.15 & -1.00 & 0.00  \\
\dataStudent{} & 0.53 & 0.26 & 0.60 & 0.55 & 0.25 & 0.05  & 0.34 & 0.20 \\
  \bottomrule
\end{tabular}%
\end{table}

Notably, paraphrases generated in other experiments were often substantially shorter than the reference texts. 
In multiple cases, the \pgenerator{} returned placeholders such as \texttt{I’m sorry, but I can’t help with that}.


\section{Exp.\ 3: Paraphrasing Chunks}
\label{sec:results_chunks}

This experiment tests the hypothesis that dividing a text into smaller chunks improves paraphrasing effectiveness, as individual chunks typically contain fewer topics than the full text. 
We computed several paraphrasing metrics for each chunk and averaged the results over the chunks of a text.

\begin{table}[t]
\centering
\caption[Impact of the number of chunks on paraphrase measures]{Impact of the number of chunks on syntactic and semantic paraphrase measures. 
Impact is reported as the absolute change between a single-chunk paraphrase and the maximum number of chunks (i.e.\ 5). 
Bold values indicate the largest observed changes. 
Ideally, syntactic measures should be minimised, while semantic measures are maximised.}
\label{tab:impact_chunks_dataset_paraphraser}
\resizebox{\textwidth}{!}{%
\begin{tabular}{@{}llrrrr@{}} % numbers should be right aligned, text left aligned
    \toprule
\textbf{}         & \textbf{}            & \multicolumn{2}{l}{\textbf{Syntactic Measure} $\downarrow$} & \multicolumn{2}{l}{\textbf{Semantic Measure} $\uparrow$} \\
\textbf{Dataset} & \textbf{Model Type} & \textbf{\diameter}          & \textbf{$\sigma$}          & \textbf{\diameter}          & \textbf{$\sigma$}         \\
\midrule
\dataBlog{}        & one-step & 0.01  & 0.01 & -0.03 & 0.02 \\
                                & two-step & \textbf{-0.12} & 0.08 & -0.03 & 0.04 \\
\dataGutenberg{}    & one-step & 0.0   & 0.0  & -0.04 & 0.03 \\
                                & two-step & 0.0   & 0.0  & -0.01 & 0.05 \\
\dataStudent{} & one-step & 0.03  & 0.04 & 0.03  & 0.07 \\
                                & two-step & \textbf{-0.15} & 0.08 & -0.05 & 0.02 \\
                                \bottomrule
\end{tabular}%
}
\end{table}

\Cref{tab:impact_chunks_dataset_paraphraser} summarises the effect of chunking on syntactic and semantic measures. 
For two-step paraphrasers, the difference between semantic and syntactic scores increases with the number of chunks, driven by decreasing mean average syntactic similarity for the \dataBlog{} and \dataStudent{} datasets. 
\Cref{fig:abl_chunks_blog_translation} illustrates the effect for the translation-based paraphraser on the \dataBlog{} dataset. 
On the \dataGutenberg{} dataset, chunking reduces the semantic similarity of translation-based paraphrases, while other two-step paraphrasers remain largely unaffected, resulting in minimal change to the overall mean semantic similarity. 
As shown in \Cref{fig:abl_chunks_student_essays_llama}, chunking had negligible impact on one-step paraphrasing approaches. 
Additional visualisations are provided in the Appendix (cf.~\Cref{sec:app_chunks}).

Since \ac{rouge} scores are computed on individual reference-candidate pairs, the union of sentences in candidate contains only a single text per comparison. 
Consequently, \ac{rouge}-L and \ac{rouge}-Lsum yield identical results in this setting.

In summary, increasing the number of chunks decreases syntactic diversity for two-step approaches, with only minor reductions in semantic similarity for some datasets. 
Given that processing $n$ chunks with a two-step approach requires $2n$ API calls in the best case, and that chunking has no effect on one-step paraphrases, we excluded chunking from our paraphrasing pipeline.

% \begin{figure}[htbp]
%     \centering
%     \includesvg[width=\textwidth]{images/paraphrasing/experiments/chunks/setup/results/Blog/Translation_metrics_plot_category_Blog.svg}
%     \caption[Impact of the number of chunks on \dataBlog{} dataset]{
%     Average syntactic and semantic measures (shaded areas indicate standard deviation) for the translation-based paraphraser on the \dataBlog{} dataset. 
%     Increasing the number of chunks reduces syntactic similarity.    
%     }
%     \label{fig:abl_chunks_blog_translation}
% \end{figure}

% \begin{figure}[htbp]
%     \centering
%     \includesvg[width=\textwidth]{images/paraphrasing/experiments/chunks/setup/results/Student_Essays/meta-llama-3.1-8b-instruct_metrics_plot_category_Student Essays.svg}
%     \caption[Impact of the number of chunks on \dataStudent{} dataset]{Average paraphrasing measures (shaded areas indicate standard deviation) for a Llama-based paraphraser on the \dataStudent{} dataset. 
%     One-step paraphrasing is unaffected by the number of chunks.
%     }
%     \label{fig:abl_chunks_student_essays_llama}
% \end{figure}

\begin{figure}[htbp]
  \centering
  \begin{subfigure}[b]{\textwidth}
    \centering
    \includesvg[width=\textwidth]{images/paraphrasing/experiments/chunks/setup/results/Blog/Translation_metrics_plot_category_Blog.svg}
    \caption[Translation-based paraphraser on \dataBlog{}]{
    Translation-based paraphraser on the \dataBlog{} dataset.    
    }
    \label{fig:abl_chunks_blog_translation}
  \end{subfigure}
  \hfill
  \begin{subfigure}[b]{\textwidth}
    \centering
    \includesvg[width=\textwidth]{images/paraphrasing/experiments/chunks/setup/results/Student_Essays/meta-llama-3.1-8b-instruct_metrics_plot_category_Student Essays.svg}
    \caption[Llama-based paraphraser on \dataStudent{}]{Llama-based paraphraser on the \dataStudent{} dataset.
    }
    \label{fig:abl_chunks_student_essays_llama}
  \end{subfigure}
  \caption[Effect of chunking on syntactic and semantic measures]{Average syntactic and semantic similarity measures (shaded areas indicate standard deviation) for different number of chunks.
  \Cref{fig:abl_chunks_blog_translation} shows that increasing the number of chunks reduces syntactic similarity for the translation-based paraphrases, whereas \Cref{fig:abl_chunks_student_essays_llama} demonstrates that one-step paraphrasing is largely unaffected by the number of chunks.
  }
  \label{fig:abl_chunks}
\end{figure}

\section{Exp.\ 4: Comparing Prompts}%Assessing the Impact of the Prompt on Paraphrasing}
\label{sec:prompt_impact_res}

In this experiment, we investigate how different prompting strategies influence the quality of paraphrases generated by \acp{llm}. 
To this end, we measured the relative length difference between reference and paraphrase pairs across different \ac{llm}–prompt combinations. 
A subset of pairs was also manually inspected to assess semantic fidelity and readability.

Post-processing was required to remove reasoning traces present in some model outputs, particularly in generations from models such as \texttt{qwen3-32b}. 
These traces, typically delimited by \texttt{</think>}, consist of repeated fragments of the input prompt and do not contribute to the semantic content of the paraphrase. 
We excluded them to retain only the task-relevant text produced by the \ac{llm}.

After post-processing, we computed the relative length difference between each reference and its corresponding paraphrase for all model–prompt combinations. 
The distribution of these differences is presented in \Cref{fig:prompt_impact_post_processed}. 
Because our objective in \imp{} generation is to control for confounding variables, a paraphrase length close to that of the reference is interpreted as an indicator of higher paraphrase quality. 
We additionally performed a manual assessment of content quality that focused on paraphrases both extremely long and length-balanced relative to the reference.

\begin{figure}[H]
    \centering
    \includesvg[width=\textwidth]{images/prompt_impact/paraphraser_length_distribution_post_process_len_perc(qwen)_linear.svg}
    \caption[Impact of different prompts on paraphrases]{
    Box plots of relative paraphrase lengths after post-processing across different prompts.    
    The dotted gray line marks the optimal paraphrase length.
    \texttt{prompt2} consistently generates paraphrases whose lengths are more comparable to the reference than those produced by the other two prompts.
    }
    \label{fig:prompt_impact_post_processed}
\end{figure}

Our results show that the relative length difference of paraphrases strongly depends on the prompt used to instruct the \ac{llm}. 
Notably, the third prompt, i.e.\ \texttt{prompt2}, explicitly instructed models to generate paraphrases three times longer than the reference. 
While this instruction might seem extreme, it consistently generated paraphrases whose lengths were more comparable to the reference than those produced by the other two prompts, across different models, as summarised in \Cref{tab:impact_prompts_paraphrases_lengths}.

\begin{table}[h]
\centering
\caption[Impact of different prompts on paraphrase lengths]{Impact of the prompts on paraphrase lengths. 
Relative length difference is defined as $\frac{\mathrm{len}(paraphrase)}{\mathrm{len}(reference)}\times 100\%$ and denoted $d$. 
Optimal paraphrases are expected to approximate the reference length, i.e.\ $\diameter d \approx 100$. 
Subscript $pp$ indicates post-processed outputs (with reasoning traces removed). 
``Count'' denotes the number of paraphrases considered for each setting. 
For \texttt{prompt2}, only post-processed results are reported.
Bold \diameter $r_{pp}$ values are those closest to the optimal paraphrase length.
}
\label{tab:impact_prompts_paraphrases_lengths}
\resizebox{\textwidth}{!}{%
\begin{tabular}{@{}llrrrrr@{}}
\toprule
Paraphraser & Prompt  & \diameter $d$ & $\sigma d$ & \diameter $d_{pp}$ & $\sigma d_{pp}$ & Count \\
\midrule
meta-llama-3.1-8b-instruct & prompt0 & 39.93 & 52.64 & 39.93 & 52.64  & 135   \\
                            & prompt1 & 40.27  & 24.21 & 40.27  & 24.21 & 124 \\
                            & prompt2 & - & - & \textbf{98.15} & 24.97 & 639  \\
mistral-large-instruct & prompt0 & 1.89   & 1.0   & 1.89   & 1.0   & 138 \\
                        & prompt1 & 13.09  & 17.96 & 13.09  & 17.96 & 129 \\
                        & prompt2 & - & - & \textbf{79.70}  & 11.40 & 449  \\
openai-gpt-oss-120b   & prompt0 & 5.53   & 13.47 & 5.53   & 13.47 & 139 \\
                        & prompt1 & 19.21  & 25.0  & 19.21  & 25.0  & 129 \\
                        & prompt2 & - & - & \textbf{147.54} & 48.45 & 590  \\
qwen3-32b           & prompt0 & 88.36  & 70.02 & 18.68  & 24.09 & 134 \\
                        & prompt1 & 95.73  & 47.72 & 38.34  & 15.64 & 123 \\
                        & prompt2 & - & - & \textbf{81.42}  & 13.46 & 532 \\
                                \bottomrule
\end{tabular}%
}
\end{table}

Manual inspection indicated that paraphrases generated with \texttt{prompt2} exhibited only mild hallucinations and generally remained on topic. 
Moreover, paraphrases generated with \texttt{prompt2} outperformed paraphrases from other prompts in terms of semantic preservation across all \acp{llm}.

Based on these findings, subsequent experiments adopted the following design choices: 
(1) exclude paraphrases generated with \texttt{prompt0} and \texttt{prompt1}, 
(2) remove all \texttt{</think>}–delimited reasoning traces via post-processing, and 
(3) discard paraphrases shorter than $60\%$ of the reference length.


% \section{Exp. 4: Impact of Syntactic Similarity on \impApprTitle{} Performance}
\label{sec:res_syn_sim_impact}

\textcolor{red}{TODO: still runs on 23.08.2025}

% \subsection{\dataStudent{}}
% \begin{figure}[htbp]
%     \centering
%     \includesvg[width=\textwidth]{images/paraphrasing/experiments/syntactic_similarity_impact/student_essay/student_essays_syn_sim_Syntactic_Similarity_Difference_accuracy.svg}
%     \caption{.}
%     \label{fig:impact_syn_student_diff}
% \end{figure}

% \begin{figure}[htbp]
%     \centering
%     \includesvg[width=\textwidth]{images/paraphrasing/experiments/syntactic_similarity_impact/student_essay/student_essays_syn_sim_Syntactic_Similarity_of_Disputed_&_Candidate_accuracy.svg}
%     \caption{.}
%     \label{fig:impact_syn_student_disp_cand}
% \end{figure}

% \begin{figure}[htbp]
%     \centering
%     \includesvg[width=\textwidth]{images/paraphrasing/experiments/syntactic_similarity_impact/student_essay/student_essays_syn_sim_Syntactic_Similarity_of_Reference_&_Paraphrases_accuracy.svg}
%     \caption{.}
%     \label{fig:impact_syn_student_ref_paraph}
% \end{figure}

% \subsection{\dataBlog{}}
% \begin{figure}[htbp]
%     \centering
%     \includesvg[width=\textwidth]{images/paraphrasing/experiments/syntactic_similarity_impact/blog/blog_syn_sim_Syntactic_Similarity_Difference_accuracy.svg}
%     \caption{.}
%     \label{fig:impact_syn_student_diff_acc}
% \end{figure}

% \begin{figure}[htbp]
%     \centering
%     \includesvg[width=\textwidth]{images/paraphrasing/experiments/syntactic_similarity_impact/blog/blog_syn_sim_Syntactic_Similarity_Difference_precision.svg}
%     \caption{.}
%     \label{fig:impact_syn_student_diff_prec}
% \end{figure}

% \begin{figure}[htbp]
%     \centering
%     \includesvg[width=\textwidth]{images/paraphrasing/experiments/syntactic_similarity_impact/blog/blog_syn_sim_Syntactic_Similarity_Difference_f1.svg}
%     \caption{.}
%     \label{fig:impact_syn_student_diff_f1}
% \end{figure}

% \section{Exp.\ 5: Comparing \acs{av} Methods in Traditional Human-Human Scenario}
\label{sec:results_trad_av}


This experiment investigates how our \ac{llm}-based \imp{} generation performs relative to (a) traditional \imp{} generation within the \impAppr{}, and (b) \acl{sota} \ac{av} methods in a conventional human-human \ac{av} scenario. 
To this end, we construct \textcolor{orange}{10} same-author and \textcolor{orange}{10} different-author pairs from the \dataStudent{} dataset and evaluate the approaches across varying thresholds.

The \mirrorMinds{} approach assigns nearly every input pair to the same-author class, yielding a precision of approximately $0.5$. 
This behaviour is due to the nature of its paraphrases, which consist of single words. 
The discriminator interprets any candidate text as stylistically more similar than \mirrorMinds{}' minimal \imps{}, leading to a bias toward same-author predictions. 
In contrast, naive \ac{llm}-based \imp{} generation achieves the highest $F_1$ scores, although its precision remains between $0.6$ and $0.7$. 
This suggests that the generated \imps{} are still relatively easy, producing \acp{fp} in some cases.

\begin{figure}[h]
\centering
    \includesvg[width=\linewidth]{images/AV_comparison/detection_scenarios/f1/student_essays_Human-Human_threshold_f1s_curves_all_incl_baselines.svg}
  \caption[Traditional \ac{av} $F_1$ scores.]{$F_1$ scores for the same-author class across different thresholds on the \dataStudent{} dataset. 
\ac{llm}-based approaches achieve slightly better effectiveness than baselines across thresholds.}
  \label{fig:human-human_f1}
\end{figure}

$F_1$ scores generally decline for most approaches as the threshold increases. 
\ac{llm}-based methods maintain relatively stable effectiveness across thresholds due to consistently high recall values. 

Among the original baselines~\citep{koppel_determining_2014}, the fixed \imp{} generation performs best, with its text-length-informed variant showing slightly improved results. 
Both approaches exhibit high precision but low recall. 
The length-based \imp{} generation consistently achieves higher recall than the fixed variant. 
Incorporating content similarity into \imp{} selection does not significantly affect effectiveness, as both content-based and fixed \imp{} approaches yield similar outcomes. 
Recall curves across thresholds are provided in the Appendix (cf.~\autoref{sec:app_detection_scenarios}).


% % results
\section{Exp.\ 6: Comparing \acs{av} Methods in \acs{llm} Author Scenarios}
\label{sec:results_llm_av}

This experiment evaluates the performance of the \impAppr{} against established \ac{av} methods on the \dataArtificialStudent{} dataset where each input pair contains at least one \ac{llm}-generated text.  
The baseline \mirrorMinds{} performs comparatively well due to its consistently high recall.  
The recall curves across different thresholds and scenarios can be found in the Appendix in \autoref{sec:app_detection_scenarios}.  
However, \mirrorMinds{}' strength is a misconception due to its oversimplified paraphrases, which consist of single words.  
This triviality biases the discriminator toward predicting same-author, since any candidate text necessarily contains more stylistic information than the single-word \imps{}.  
As a result, this method produces a large number of \acp{fp}.


\paragraph{\ac{llm} detection}

In the \ac{llm} detection setting, the traditional baselines \ac{ppmd}, Unmasking, and the supervised \ac{svm} perform best in terms of precision.  
Nevertheless, it becomes evident that no method achieves a favourable balance between high precision or high recall since none can attain both simultaneously.  

Among the \imp{}-based approaches, random sampling variants (content, text length, and fixed) outperform naive \ac{llm}-based \imps{}, although their absolute performance remains unsatisfactory.  
\mirrorMinds{} and naive \ac{llm} \imps{} are overly simplistic, leading to consistently low precision and inflated recall, since the majority of input pairs are incorrectly classified as same-author.  
Consequently, these approaches are particularly ill-suited for detecting \acp{llm}.

\begin{figure}[h]
  \includesvg[width=\linewidth]{images/AV_comparison/detection_scenarios/precision/student_essays_LLM-Detection_threshold_precisions_curves_all_incl_baselines.svg}
\caption[\ac{llm} detection precision curves.]{Precision curves for the class same-author. 
The candidate text is \ac{llm} generated.
In these scenarios, ground truth True corresponds to the disputed text being \ac{llm} generated irrespective of the responsible model.
}
\label{fig:llm_detection_prec}
\end{figure}

\paragraph{General \ac{llm} \ac{av}}

Contrary to initial intuition, neither \mirrorMinds{} nor naive \ac{llm} approaches are suitable for this task, as their high recall results in excessive \acp{fp} and low precision.  
Although \ac{ppmd} and Unmasking achieve relatively high precision, their recall is consistently low, yielding poor overall $F_1$ performance.  
Both methods predominantly predict different-author, thereby underestimating same-author cases.  

The best-performing baseline is the supervised model, which reaches a maximum precision of approximately $0.7$ for recall $\leq 0.55$.  
However, overall performance remains unsatisfactory across all methods, particularly in terms of precision, which hovers slightly above $0.2$.


\begin{figure}[h]
  \centering
  \includesvg[width=\linewidth]{images/AV_comparison/detection_scenarios/f1/student_essays_LLM-AV_threshold_f1s_curves_all_incl_baselines.svg}
  \caption[\ac{llm} \ac{av} $F_1$ scores.]{$F_1$ scores for the class same-author. 
The candidate text is \ac{llm} generated.
In these scenarios, ground truth True corresponds to the disputed text being generated by the same \ac{llm} as the candidate text.
}
  \label{fig:llm_av_prec}
\end{figure}



\paragraph{\ac{llm}-\ac{llm} \ac{av}}

Overall $F_1$ scores for \ac{av} in the \ac{llm}-\ac{llm} scenario are higher than in the general \ac{llm} \ac{av} setting, suggesting that the presence of human-authored texts complicates attribution.  

At first glance, \mirrorMinds{} appears to perform best. 
However, this impression is misleading.  
Its trivial \imps{} inflate recall without offering genuine discriminative power.  
Interestingly, the Unmasking baseline exhibits similar recall behaviour for thresholds below $0.4$.  

Fixed, content-based, and text-length-driven \imp{} generation methods achieve precision between $0.4$ and $0.7$, yet their $F_1$ scores remain low due to poor recall, i.e.\ most pairs are incorrectly assigned to the different-author class.  
Similarly, \ac{ppmd} and the unsupervised min-max approach yield high precision but equally low recall, again resulting in poor $F_1$ values.  

The overall best-performing approach is naive \ac{llm} generation, which maintains a precision of approximately $0.6$ across thresholds.  
However, with increasing threshold its recall decreases monotonically indicating both \acp{tp} and \acp{fp} are discarded in roughly equal measure, implying that the generated \imps{} fail to meaningfully distinguish between same- and different-author pairs.


  \begin{figure}[h]
    \centering
    \includesvg[width=\linewidth]{images/AV_comparison/detection_scenarios/f1/student_essays_LLM-AV-(only-LLMs)_threshold_f1s_curves_all_incl_baselines.svg}
    \caption[\ac{llm}-\ac{llm} \ac{av} $F_1$ scores.]{$F_1$ scores for the class same-author.
  All texts are \ac{llm} generated.
  The overall $F_1$ scores for \ac{av} in \ac{llm}-\ac{llm} are better than those of the general \ac{llm} \ac{av} scenario.
  }
    \label{fig:llm-llm_f1}
  \end{figure}



\section{Exp.\ 5: Comparing \acs{av} Methods}% in Traditional Human-Human Scenario}
\label{subsec:imp_gen_res}

We evaluate precision–recall values across different thresholds for both the baseline methods and the fixed approach proposed by \citet{koppel_determining_2014}.
Our results indicate that the supervised baseline and our one-step paraphrasing approach (\textcolor{red}{Naive \ac{llm}} in \autoref{fig:sem_syn_blog}) produce identical outcomes.
An examination of the individual values confirms this observation, revealing identical precision–recall pairs where respective thresholds are different.
We attribute this effect to the relatively small sample size of only \textcolor{red}{10} text pairs.

With respect to optimizing the precision–recall trade-off, the supervised baseline and our \ac{llm}-based extension of the \impAppr{} achieve the most favorable performance at a threshold of $0.075$, and $0.32$ respectively.
When precision is prioritized over recall, however, all remaining methods outperform these two approaches, as they achieve higher recall while maintaining the same maximum precision.
Although the definition of an optimal balance between precision and recall is application-dependent, we argue that a recall of $0.4$ at perfect precision is generally less informative than jointly optimizing both metrics.

\begin{figure}[htbp]
    \centering
    \includesvg[width=\textwidth]{images/imposter/our_contribution/roc_prec_recall_curve_r100_top100000_Same_Author_dif_imp_gen.svg}
    \caption[Recall-precision curves for the \dataStudent{}.]{Recall-precision curves for the \textcolor{red}{10} samples of the \dataStudent{}. 
    One-step paraphrasers are denoted \textcolor{red}{Naive \ac{llm}}.
    (B)~indicates the original baseline approaches from~\citep{koppel_determining_2014}.
    }
    \label{fig:sem_syn_blog}
\end{figure}

These findings contrast with our initial expectation that \ac{llm}-based \imp{} generation would achieve higher precision, but indicate that our approach performs best when precision and recall are considered equally important.

    \chapter{Discussion}
\label{chap:discussion}
    \chapter{Conclusion}
\label{chap:conclusion}

reproduction not entirely possible

syntactic and semantic scores of paraphrases (+ impact?),
paraphrases and chunks,
extraction of metadata,
quality of paraphrases?

does not work for llm detection,
works better in av without any humans,





\section{Future Work}
\subsection{Perplexity}
\label{subsec:perplexity}

% Perplexity is measure to assess how surprised a language model is by a text.
Perplexity $PPL$ can be employed to compute the likelihood of a \ac{lm} generating a text.
A low perplexity indicates that the sequence aligns with model's predictions, 
while a high perplexity indicates that the sequence is unexpected or unlikely according to the model.
Perplexity is computed as follows:
\begin{equation}
    PPL = \exp\left(-\frac{1}{t}\sum_{i=1}^{t}\log P(w_i|w_{<i})\right)
\end{equation}
where $t$ is the number of words or tokens in the sequence, 
$w_i$ is the $i$-th word/ token, and $P(w_i|w_{<i})$ is the probability of the $i$-th word/ token given all previous words/ tokens in the sequence.
The exponent is the cross-entropy loss between the model's predictions and the actual sequence.
The cross-entropy can be refactored to the sum of the entropy of the model's predictions and the KL divergence of the prediction and the data.
While Python libraries such as \texttt{PyTorch} and \texttt{TensorFlow} use the natural logarithm $\log$ for perplexity calculations,
traditional information theory uses the logarithm to base 2. 
Note, that different bases differ only by a constant factor.
For sequences longer than the context window of the model, 
perplexity is computed on the windows of $n$ tokens, where $n$ is the context window size.
% strides: not good
Depending on the tokenizer, perplexity can be computed on the word or sub-word level, 
where sub-word level perplexity is often smaller due to higher likelihoods of smaller character sequences.
Since larger vocabulary lead to lower likelihoods per token, perplexity is generally higher for larger vocabularies.
% disadvantages
Due to the lack of comparability across different tokenizers or models and 
the requirement for access to the model's probabilities $P(w_i|w_{<i})$, which are often not available, 
we decided to refrain from using perplexity for \ac{llm} detection.


experiment llm detection scenario: Add llm pair of same architecture or trained on same data and see if scores influenced (related llms)



Ausblick: LLM Detection
% AV
% LLM detection using generative models
%% AA against LLMs
With the recent advances of \ac{nlg} come new challenges in text authorship.
The new technologies may be misused for fraudulent activities to scam naive or inexperienced users~\citep{uchendu_authorship_2020,bhattacharjee_fighting_2024}.
\citet{uchendu_authorship_2020} identified three authorship tasks essential for fighting fraudulent activities:
(1) Given two texts $t_1$ and $t_2$, determine whether they were produced by the same method (i.e. human author or a specific \ac{nlg} method).
(2) Given a text $t$, determine whether it was human authored or machine generated (Turing Test).
(3) Given a text $t$, find its author among $k+1$ candidates, which consists of one human and $k$ machines.
They compare classical \ac{ml} models, neural models and state-of-the-art \ac{aa} models as classifiers 
for these single- (Problem 1 and 2) and multi-class (Problem 3) tasks.
Their findings include, that as of 2020, most \ac{nlg} methods were distinguishable from human authors, 
but some \acp{llm} proved difficult to detect.
%%% compared to our work
In the following, we consider (1) \ac{av}, (2) classical \ac{llm} detection, and (3) closed-set \ac{aa}.
Our approach differs from the work of \citet{uchendu_authorship_2020} in that our candidates (i.e. \imps{}) do not include a human author (3), 
but only \acp{llm}.
Moreover, we use different classifiers originally designed for \ac{av}, rather than \ac{aa}.

There are different categories of \ac{llm} detectors-
Metric-based detectors classify based on a threshold and the inferred log-probability from the generator \ac{llm}.
Examples inlcude GLTR, Range, LogRank and DetectGPT.
Fine-tuned detectors are pretrained \acp{lm} in a supervised scenario.
Examples include the OpenAI Detector.
Watermark-based detectors add algorithmically detectable signatures into the text during generation~\citep{wang_stumbling_2024}.

%% LLM (gpt-3.5, GPT-4) as detector
\citet{bhattacharjee_fighting_2024} evaluate using an \ac{llm} as classifier for \ac{llm} detection.
They use \ac{gpt}-3.5 and \ac{gpt}-4 to classify texts as human or machine generated.
They find that \ac{gpt}-3.5 performs better when being fed simple instructions, rather than constrained prompts.
They find that \ac{gpt}-4 predicts almost exclusively \ac{ai} generated texts, 
while \ac{gpt}-3.5 predictions are more reliable (especially for actually human authored texts).
%%% compared to our work
Our work differs from theirs in that we use \acp{llm} to generate \imp{} texts specific to the candidate text, 
rather than using the publicly available dataset TuringBench with previously generated texts.

%% DetectGPT: Perturb (Mask), score, compare (unsupervised)
\citet{mitchell_detectgpt_2023} propose DetectGPT, a method that is threefold:
(1) They perturb the input text by (1.1) masking out random 2-word spans until 15 \% of the text is masked. 
Masked spans are replaced (1.2) with words from an off-the-shelf (i.e. not finetuned to target domain) \ac{llm} (e.g. T5-3B). 
These perturbations are semantically similar paraphrases of the original text.
(2) They score (in terms of log probability) each perturbed text using a scoring \ac{llm} 
(ideally their candidate \ac{llm}, but it works also with any \ac{llm}, though scores deteriorate). 
(3) The difference of the score of the original text and the average score of the perturbed texts is denoted perturbation discrepancy $d$. 
(4) Normalize $d$ by the standard deviation of the scores of the perturbed texts.
(5) Based on a threshold $\epsilon=0.1$, classify the original text as human authored or machine generated 
(formally Local Perturbation Discrepancy Gap hypothesis).
If $d$ is positive, the original text is likely machine generated.
If $d$ is near zero, i.e. $d < \epsilon$, the original text is likely human authored.
\citet{mitchell_detectgpt_2023} motivate their method by the observation that generated texts tend to occupy 
negative curvature regions of the model's log probability function (i.e. they lie on the local maximum of the manifold).
When the text is machine generated, it lies on a local maximum, 
and perturbing it will lead to lower log probabilities of perturbed texts.
When the text is human authored, it does not lie on a local maximum to begin with, 
rendering log probabilities of perturbed texts similar either bigger or smaller than the original text.
Averaging the log probabilities of perturbed human texts leads to a value that is 
close to the original text's log probability (i.e. a perturbation discrepancy $d$ near zero).
Even though, DetectGPT works best when the source (i.e. generating) \ac{llm} and the scoring \ac{llm} are the same 
(requires white-box access to the \ac{llm}), 
it works also with different \acp{llm} as surrogate for the source model when scoring (in a black-box case).
%%% compared to our work
We can not supply a white box setting, because we do know the source \ac{llm} that generated the \imp{} texts.
%%%% \imps{} and perturbations
However, this approach is similar to our approach, because perturbing texts can be seen as a 
form of \imp{} generation (especially as we use paraphrases). 
%%%% sample from the source model
Both approaches try to sample from the probability distribution of the source model either 
by using \imps{} (via prompting an \ac{llm}) or by perturbing the original text (using an \ac{llm}).
%%%% input
While the \impAppr{} is an \ac{av} task (i.e. input is a disputed and a candidate text), 
DetectGPT receives a disputed text and a candidate \ac{llm} as input.
%%%% similarity measure
While we use a similarity measure on traditional n-gram frequency vectors, 
\citet{mitchell_detectgpt_2023} require a scoring \ac{llm} to compute the perturbation discrepancy $d$.
Hence, our approach is easier in terms of computational resources and requirements.
Research building on DetectGPt finds that the whitebox approach is vulnerable to unknown models, especiall GPT-3.5-Turbo~\citep{Wu_ODD_challenges_2025}.

%% LLM rewrite LLM texts less than human texts (no AA, but edit distance hypothesis)
RAIDAR~\citep{mao_raidar_2024} builds upon the invariance property of \acp{llm}, 
which states that prompting an \ac{llm} to rewrite a machine generated text will introduce little change.
They motivate this by the observation that (different) autoregressive models produce similar patterns and thus, 
consider texts generated by (different) \acp{llm} as high quality that do not require rewriting.
Change is measured by the edit distance between the original text and the rewritten text. 
\citet{mao_raidar_2024} propose using an edit distance based on the Levenshtein distance or \ac{bow} representations.
RAIDAR operates on character level rather than using deep neural network features, and it does not require the original generating model for classification. 
RAIDAR fails to detect \ac{llm} generated texts in out-of-distribution scenarios (i.e. different domains than training), 
or when \ac{llm} were explicitly instructed to produce text prone to heavy \ac{llm} modification when being asked to rewrite the text \citep{li_learning_2025}.
Based on RAIDAR (\citep{mao_raidar_2024}), \citet{li_learning_2025} propose fine-tuning an \ac{llm} to rewrite human authored texts more than machine generated text.
Classification is carried out by comparing the edit distance of the original text and the rewritten text to a threshold.
\citet{li_learning_2025} admit that their approach is slow in inference time, 
since a candidate text has to be rewritten multiple times (about 200 different prompts) to obtain a reliable score.
\citet{mao_raidar_2024} find that the quality of perturbation based models (i.e. rewriting) for \ac{llm} detection correlates with the perturbation model size.
\citet{mitchell_detectgpt_2023} find a negative correlation (\textcolor{red}{TODO: chapter 2 vorletzter Absatz}) between the size of the perturbation model and the performance of DetectGPT.
%%% compared to our work
%%%% generation of texts during inference
Both approaches are similar to our work in that they use \acp{llm} to generate texts during inference.
We do not fine-tune an \ac{llm} for paraphrasing but use off-the-shelf models (like RAIDAR).
%%%% similarity measure
All these approaches compute the similarity of the original text and the generated text.
However, we do not use edit distance (i.e. Levenshtein distance) as similarity measure.
%%%% limitations
This approach is unable to detect which \ac{llm} generated the text.

%% LLMDet: Proxy to perplexity (problem: requires access to the LLM to build the dictionary)
Perplexity is a reliable statistical metric for attributing texts to \acp{llm}~\citep{zhang_llmdet_2023}.
Unfortunately, perplexity requires access to \acp{llm}' parameters (i.e., white-box detection).
\citet{wu_llmdet_2023} propose LLMDet, a method that uses a proxy to perplexity, 
where a dictionary of frequent n-gram (frequent among $n$ randomly prompted generated texts per \ac{llm}) 
next token probabilities is pre-computed (i.e. requiring access to the \ac{llm}), 
and is subsequently used during inference to approximate perplexity by replacing $x_{<i}$ in $p(x_i | x_{<i})$ with an n-gram.
Since the construction of the dictionary requires access to the \ac{llm}, LLMDet requires contribution of the closed-source model owners.
The disputed text is tokenized and the proxy perplexity is calculated for each model and thus, constructing a proxy perplexity vector.
This vector is input to a trained classifier.
%%% compared to our work
Proxy perplexity could be used as a baseline for our approach, though it requires access to the \ac{llm} and is thus not applicable in our case.

%% Mirror Minds: extract query, genrate two paraphrases, compare & classify via threshold (very similar to our work)
\citet{baradia_mirror_2025} propose (1) extracting a query from the disputed text, which captures the essence of the text, 
(2) generating two paraphrases of the original text using the query as input prompt to two \acp{llm}, 
and (3) comparing the paraphrases to the original text via the BLEU and the METEOR score.
Both score capture syntactic similarity, even though \citet{baradia_mirror_2025} argue they also capture semantic similarity.
They use the maximum across the two models per similarity measure as a final score pair.
Classification of the resemblance to \ac{ai} generated content requires a threshold.
%%% compared to our work
%%%% same approach
This approach is similar to our approach in that it uses \acp{llm} to generate paraphrases of the original text.
Moreover, it compares the original text to the generated paraphrases as in a \ac{aa} problem. % rather AI detection?????
%%%% similarity measure
We do not use BLEU or METEOR as similarity measure, nor do we compare directly on paraphrase-level (i.e. BLEU calculates n-gram overlap) 
but construct our own frequency based n-gram vectors input vector similarity metrics.
%%%% they discard information, solve another problem
However, this approach discards the information which \ac{llm} produced the most similar paraphrase. 
While our goal is to solve an \ac{aa} problem (i.e. multiclass classification), 
\citet{baradia_mirror_2025} solve a binary classification problem (i.e. human vs. \ac{ai} generated text).

    % \chapter{Notes}
    % % notes
    % % using input to avoid pagebreaks

\section{\acs{pan}}
\label{sec:pan}

The \ac{pan} workshop series accommodates shared tasks % like kaggel competitions
on authorship analysis, computational ethics, and the originality of writing \cite{ayele_overview_2024}.
\ac{pan}'s goal is to contribute to reproducible research in the fields of \ac{ir} and \ac{nlp}.
Submissions are submitted to the submission software \tira{}.
Among others, the workshop includes the \textit{Generative \ac{ai} Authorship Verification} task, which focuses on the detection of \ac{ai}-generated text.

\section{\acs{pan} \ac{av}}
\label{sec:pan_authorship_verification}

\citet{ayele_overview_2024} provide \autoref{tab:hierarchy_authorship_verification_problems}, i.e., 
the decomposition of \ac{av} into multiple subtasks. 
They order the subtasks in terms of their complexity.
The first task on the one hand, is considered the easiest, since we know that out of two text, one is guaranteed to be human-generated while the other one is \ac{llm}-generated.
The last task on the other hand, is denoted as the most difficult, since we do not know whether the text is human- or \ac{llm}-generated.


\begin{table}[tbp]
    \centering
    \caption{Hierarchy of \ac{av} problems \citep{ayele_overview_2024,bevendorff_overview_2024} from easiest (1) to most difficult (7), 
    where A, B corresponds to human-authored text and M denotes \ac{llm}-generated text.}
    \label{tab:hierarchy_authorship_verification_problems}
    \resizebox{\textwidth}{!}{%
    \begin{tabular}{lll}
        \toprule
    \rowcolor[HTML]{EFEFEF} 
    \textbf{Difficulty} & \textbf{Input/ Task} & \textbf{Possible Assignment Patterns} \\  \midrule
    1 & \{?,?\} & \{A,M\} \\ 
    2 & \{?,?\} & \{A,M\}, \{A,A\} \\
    3 & \{?,?\} & \{A,M\}, \{M,M\} \\
    4 & \{?,?\} & \{A,M\}, \{A,A\}, \{M,M\} \\
    5 & \{?,?\} & \{A,M\}, \{A,A\}, \{A,B\} \\
    6 & \{?,?\} & \{A,M\}, \{A,A\}, \{A,B\}, \{M,M\} \\
    7 & ? & A, M \\ \bottomrule
    \end{tabular}%
    }       
\end{table}

\citet{boenninghoff_o2d2_2021} state that \ac{pan} moved from a cross-topic closed-set \ac{av} task 
to cross-topic open-set \ac{av} task over three years.

\citet{boenninghoff_o2d2_2021} formally define the task:
For two documents $D_1$ and $D_2$, ground-truth hypothesis $H_a$ for $a \in \{0,1\}$ 
where $a=1$ means that $D_1$ and $D_2$ are written by the same author and 
$a=0$ means that they are not written by the same author, find $f: \{D_1, D_2\} \rightarrow p \in [0,1]$.
The estimated label $\hat{a}$ is then determined by the threshold $\theta$ applied to $p$:
$$\hat{a}=\left\{ \begin{matrix}
1\text{ if }p > 0.5 \\
0\text{ if }p < 0.5 \\
non-response\text{ if }p\text{ close to }0.5
\end{matrix}\right\}$$
These rules lead to three hypotheses:
\begin{itemize}
    \item $H_0$: $D_1$ and $D_2$ are written by different authors
    \item $H_1$: $D_1$ and $D_2$ are written by the same author
    \item $H_2$: Undecidable, trial does not suffice to establish authorship
\end{itemize}
% acquisition
\citet{ayele_overview_nodate} collected articles of major 2021 US news from Google News.
They chose this time period since it predates the release of GPT-3.5.
As a consequence, they claim that their dataset is most likely human-authored.
% artificial texts
Next, they used GPT-4-Turbo to create bullet-point summaries of the articles. 
Based on these summaries, 15 \acp{llm}-generated newspaper articles.
% split
The authors split the dataset into training and test set.
% robustness
In order to test the submissions' robustness, \citet{ayele_overview_nodate} generated 65 variants of the newspapers in the test split.
These variants include:
\begin{enumerate}
    \item Change English to German texts
    \item Replace 15\% of the characters
    \item Shuffling test case pairs to break topic coherence \todo{macht es das nicht leichter?}
    \item Contrastive decoding instead of top-k/top-p sampling
    \item Cropping texts to 35 words
    \item Using the prompt from a previous Kaggle competition on \ac{llm} detection to generate more faithful paraphrases of the original articles
\end{enumerate}
\section{\acs{pan} evaluation}
\label{sec:pan_evaluation}

\citet{ayele_overview_nodate} evaluate the participants submissions averaging the datasets' evaluation measures.
They claim the following measures are established in authorship verification:
\begin{itemize}
    \item \ac{roc-auc}
    \item BRIER
    \item C@1
    \item $F_1$
    \item $F_{0.5u}$
\end{itemize}

    % \section{Definitions}
\label{sec:definitions}


% \begin{definition}
%     []
% \end{definition}

\begin{definition}
    [Text]
    A sequence of tokens or characters grouped into sentences \cite{elmanarelbouanani_authorship_2014}.
\end{definition}

\begin{definition}
    [Token]
    A token can be a word, a number or a punctuation mark \cite{elmanarelbouanani_authorship_2014}.
\end{definition}

\begin{definition}
    [Monograph]
    Single author document \cite{bevendorff_smauc_2023}.
\end{definition}

\begin{definition}
    [Multi-author (i.e. Collaborative) publication]
    Document with multiple authors \cite{bevendorff_smauc_2023}.
\end{definition}

\begin{definition}
    [Author profiling]
    Task of inferring an extensive set of (sensitive) personal information.
    This includes sociolinguistic attributes like age, gender, occupation, education, socio-economic status, cultural background, 
    language familiarity and mental health issues 
    \cite{emmery_adversarial_2021,stamatatos_survey_2009,elmanarelbouanani_authorship_2014}.
    The task is also referred to as author characterization \cite{stamatatos_survey_2009,elmanarelbouanani_authorship_2014}.
\end{definition}

\begin{definition}
    [Stylometry]
    Liguistic research area, which refers to the (statistical) analysis of authorial/ literally style \cite{elmanarelbouanani_authorship_2014,neal_surveying_2018}.
    Stylometry assumes that style is quantifiably measurable for evaluation of distinctive qualities and 
    that features, such as subconscious syntactic idiosyncrasies are sufficient in defining an author's unique style \cite{neal_surveying_2018}.
    The construction of models for the quantification of writing style, text complexity, and grading level assessment.
    Stylometric features include lexical, syntactic and structural features \cite{stein_intrinsic_2011}.
    In other words, stylometry is the statistical analysis of literary style between one writer or genre and another \cite{tyo_state_2022}.
    Research includes five subtasks \cite{neal_surveying_2018}:
    \begin{itemize}
        \item \ac{aa}
        \item \ac{av}
        \item Author profiling
        \item Stylochronometry
        \item adversarial stylometry
    \end{itemize}
\end{definition}

\begin{definition}
    [Stylochronometry]
    The study and detection of changes in authorial style over time \cite{neal_surveying_2018}.
\end{definition}

\begin{definition}
    [Stylistics]
    The study of stylometric features \cite{elmanarelbouanani_authorship_2014,abbasi_writeprints_2008}.
\end{definition}

\begin{definition}
    [Author writing style]
    Among others, syntactic structure of sentences in a document \cite{jafariakinabad_self_supervised_2022}.
    Opposed to other text categorization tasks, features do not include content \cite{koppel_authorship_2004}.
\end{definition}

\citet{elmanarelbouanani_authorship_2014} claim there are four (five \cite{abbasi_writeprints_2008,neal_surveying_2018}) types of writing style/ stylistic features:
\begin{itemize}
    \item Lexical features \citet{elmanarelbouanani_authorship_2014,abbasi_writeprints_2008,neal_surveying_2018}
    \item Syntactic features \citet{elmanarelbouanani_authorship_2014,abbasi_writeprints_2008,neal_surveying_2018}
    \item Structural features \citet{elmanarelbouanani_authorship_2014,abbasi_writeprints_2008,neal_surveying_2018}
    \item Content/Domain-specific features \citet{elmanarelbouanani_authorship_2014,abbasi_writeprints_2008,neal_surveying_2018}
    \item Idiosyncratic features \citet{abbasi_writeprints_2008}
    \item Semantic features \cite{neal_surveying_2018}
\end{itemize}

\begin{definition}
    [Online stylometric analysis]
    The analysis of authors style in online texts \cite{abbasi_writeprints_2008}.
    \citet{abbasi_writeprints_2008} define online texts as any textual documents that may be found in an online setting, 
    including computer-mediated communication, non-literary electronic documents (e.g., student essays, mews, articles, etc.), and program code.
\end{definition}

\begin{definition}
    [Style markers]
    Taxonomies of features to quantify the writing style \cite{stamatatos_survey_2009}.
\end{definition}

\begin{definition}
    [Stylistic features]
    Features that are the attributes or writing-style markers that are the most effective discriminators of authorship \cite{abbasi_writeprints_2008}.
\end{definition}

\begin{definition}
    [Lexical features]
    Lexical features are common features used in stylometry.
    They are character- (i.e. character (n-gram) frequency, ...) 
    or word-based (i.e. average word~\cite{stein_intrinsic_2011}, sentence length~\cite{stein_intrinsic_2011,abbasi_writeprints_2008}, 
    line length~\cite{abbasi_writeprints_2008}, word length distribution~\cite{abbasi_writeprints_2008}, 
    vocabulary richness~\cite{abbasi_writeprints_2008,neal_surveying_2018} ...) features. 
    These features consider text as a mere sequence of word-tokens or characters, respectively \cite{stamatatos_survey_2009}.
    Word features are more complex than character features \cite{stamatatos_survey_2009}.
    Character features are language-independent, not highly affected by noise (compared to word features), 
    and do not require natural language processors \cite{neal_surveying_2018}.
\end{definition}

\begin{definition}
    [Syntactic features]
    Syntactic features include function words, punctuation, and \ac{pos} tag $n$-grams \cite{abbasi_writeprints_2008}.
    They are language-dependent and require natural language processors \cite{neal_surveying_2018}.
\end{definition}

\begin{definition}
    [Semantic features]
    Semantic features capture meaning behind words, phrases, and sentences, such as through analysis of synonyms and semantic dependencies \cite{neal_surveying_2018}.
\end{definition}

\begin{definition}
    [Structural features]
    Structural features include text organization, layout, file extensions, font, sizes, colours, 
    use of braces and comments (for analysing computer programs) \cite{abbasi_writeprints_2008,neal_surveying_2018}.
\end{definition}

\begin{definition}
    [Content-specific features]
    Content-specific features include important keywords and phrases on certain topics such as word $n$-grams \cite{abbasi_writeprints_2008}.
    Domain-specific features include ratios of quoted words and external links, number of paragraphs, 
    and paragraphs average length for the news article domain \cite{potthast_stylometric_2018}
\end{definition}

\begin{definition}
    [Idiosyncratic features]
    Idiosyncratic features include misspellings, grammatical mistakes, and other usage anomalies \cite{abbasi_writeprints_2008,neal_surveying_2018}.
    Such features are extracted using spelling and grammar checking tools and dictionaries \cite{abbasi_writeprints_2008}.
\end{definition}

\begin{definition}
    [Feature-set type]
    There are two types of feature sets \cite{abbasi_writeprints_2008,neal_surveying_2018}:
    \begin{itemize}
        \item Author-group-level where one set of features is applied across all authors.
        \item Individual-author-level where each author has a unique set of features (e.g., 10 authors = 10 feature sets; 5000 most frequent character $n$-grams per author).
    \end{itemize}
    Individual-author-level features are effective for feature categories with potentially large feature spaces, such as $n$-grams or misspellings \cite{abbasi_writeprints_2008}.
    However, standard machine learning techniques typically require a fixed feature set for all authors \cite{abbasi_writeprints_2008}.
    Traditional single-author-group-level feature sets include \acp{svm} and \ac{pca} \cite{abbasi_writeprints_2008}.
\end{definition}

\begin{definition}
    [Static features]
    Static features include context-free categories such as function words, 
    word-length distributions, vocabulary richness measures, etc. \cite{abbasi_writeprints_2008}.
\end{definition}

\begin{definition}
    [Dynamic features]
    Dynamic features are context-dependent attributes and include $n$-grams and misspelled words \cite{abbasi_writeprints_2008}.
\end{definition}

\begin{definition}
    [Stability]
    Stability refers to how often a feature changes across authors and documents for a constant topic.
    \citet{abbasi_writeprints_2008} state that nouns are more stable than function words and thus, 
    function words are better stylistic discriminators than nouns 
    since using function words involves making choices between sets of synonyms.
\end{definition}

\begin{definition}
    [Adversarial Stylometry]
    Attack models that automatically infer a variety of potentially sensitive author information \cite{emmery_adversarial_2021} 
    via alteration of one's style \cite{neal_surveying_2018}.
    These attacks are not to be confused with adversarial learning \cite{emmery_adversarial_2021}.
    There are three forms of adversarial stylometry \cite{neal_surveying_2018}:
    \begin{itemize}
        \item Imitation: Writing to closely match the style of another author.
        \item Translation: Machine translating the language of a document to another language and back to the original language one or multiple times.
        \item Obfuscation: Paraphrasing such that the meaning of the text is preserved, but the style is changed.
    \end{itemize}
\end{definition}

\begin{definition}
    [Closed world]
    In the realm of plagiarism detection, closed world refers to the assumption 
    that a reference collection $D$ of documents, 
    that are supposed to be compared to the possibly plagiarized text, is given \cite{stein_intrinsic_2011}.
    In the realm of \ac{av} and \ac{aa} texts in the test set are assumed to be written by one of the authors in the training set \cite{boenninghoff_o2d2_2021,neal_surveying_2018}.
\end{definition}

\begin{definition}
    [Plagiarism]
    In the context of texts, plagiarism is the usage of another author's information, language, 
    or writing without properly acknowledging the original source \cite{stein_intrinsic_2011}.
    \citet{elmanarelbouanani_authorship_2014} define plagiarism as the complete or partial replication 
    of a piece of work with or without permission of the original author.
\end{definition}

\begin{definition}
    [Plagiarism detection]
    The task of identifying plagiarized text \cite{stein_intrinsic_2011}, i.e. finding similarities between two texts \cite{stamatatos_survey_2009}.
    Plagiarism detection uses similarity detection, determining whether multiple pieces of work were produced by a single author 
    without necessarly identifying the author \cite{elmanarelbouanani_authorship_2014}.
\end{definition}

\begin{definition}
    [Intrinsic plagiarism detection]
    This task can be understood as a more general form of \ac{av}.
    By analysing undeclared changes in writing style, potential plagiarism can be detected.
    Opposed to \ac{av}, where the decision is made based on the whole text, 
    intrinsic plagiarism detects plagiarism on a section level \cite{stein_intrinsic_2011}.
    Intrinsic analysis does not use any information on authorship from external sources \cite{zangerle_overview_2024}.
\end{definition}

\begin{definition}
    [Authorship analysing]
    This problem devises into two \textcolor{orange}{(i.e. first two acc. to \cite{stein_intrinsic_2011}, all acc. to \cite{stamatatos_survey_2009})} subtasks \cite{stein_intrinsic_2011}:
    \begin{itemize}
        \item \ac{aa} \cite{stein_intrinsic_2011}
        \item \ac{av} \cite{stein_intrinsic_2011,stamatatos_survey_2009}
        \item Plagiarism detection \cite{stamatatos_survey_2009}
        \item Author profiling \cite{stamatatos_survey_2009}
        \item Detection of stylistic inconsistencies (i.e. in collaborative writing) \cite{stamatatos_survey_2009}
    \end{itemize}
\end{definition}

\begin{definition}
    [\ac{aa}]   % authorship attribution
    The task of determining the author of a text based on textual features 
    given a (canonical) set of candidate authors with undisputed writing samples 
    \cite{stein_intrinsic_2011,koppel_authorship_2004,stamatatos_survey_2009,tyo_state_2022,bischoff_importance_2020,barlas_cross_domain_2020,altakrori_topic_2021,bevendorff_divergence_based_2020,elmanarelbouanani_authorship_2014,abbasi_writeprints_2008,llm_detection_av_2025,neal_surveying_2018}.
    The decision is made based on stylistic traits rather than the content of the document \cite{neal_surveying_2018}.
    In terms of machine learning, this is a multiclass, single-label text categorization task 
    \cite{stamatatos_survey_2009,koppel_authorship_2004,elmanarelbouanani_authorship_2014} 
    or text classification task \cite{elmanarelbouanani_authorship_2014}.
    The task is also referred to as author(ship) identification \cite{stamatatos_survey_2009,elmanarelbouanani_authorship_2014}.
    \citet{barlas_cross_domain_2020} express the \ac{aa} task as a tuple $(A,K,U)$, 
    where $A$ is the set of authors, $K=\underset{a\in A}{\cup}K_a$ is the set of known texts and $U$ is the set of unknown texts.
    If closed-set \ac{aa}: Each text $d \in U$ is attributed to exactly one author $a \in A$.
    If cross-topic(/-genre) \ac{aa}: The topic(/genre) of documents in $d \in U$ is distinct 
    with respect to the topics(/genres) found in $K$ \cite{barlas_cross_domain_2020}. 
    \citet{llm_detection_av_2025,neal_surveying_2018} state that \ac{aa} with few (<20) candidate authors is typically highly effective, 
    even if only short writing samples are available \cite{llm_detection_av_2025}/ 
    where each author has training samples of at least 1000 words \cite{neal_surveying_2018}.
    Many candidate authors impede the \ac{aa} problem.
    Models that abstain in uncertain many candidates scenarios still achieve good results \cite{llm_detection_av_2025}.
    $k$-attribution/ ranking (i.e. relaxed \ac{aa}, where the classifier output the top $k$ authors ranked by their probability of being the author of the text), 
    cross-domain/ cross-genre \ac{aa} (i.e., identify author $M$ of document written in domain $A$, while only having documents from $m$ in domain $B$ and thus, 
    features of domain $B$ must be applicable to domain $A$), 
    and source code \ac{aa} 
    are forms of \ac{aa} \cite{neal_surveying_2018}.
\end{definition}

\citet{elmanarelbouanani_authorship_2014} describe the workflow of \ac{aa} as follows:
\begin{enumerate}
    \item Data cleaning
    \item Feature extraction
    \item Normalization
    \item Converting each text into a feature vector, where author is the class label
    \item Split the dataset into training and test set
\end{enumerate}
Common classifiers include \ac{svm}, decision trees, and \acp{nn} \cite{elmanarelbouanani_authorship_2014}.

\begin{definition}
    [\ac{av}]   % authorship verification
    Given a set of writing samples of author $A$ and a text $t$,    % tyo_state_2022: only one wrinting sample
    the task is to determine whether $t$ was written by $A$ \cite{stein_intrinsic_2011,stamatatos_survey_2009,koppel_authorship_2011,tyo_state_2022,kocher_unine_2015,koppel_authorship_2004}.
    This task can also be formulated as whether two texts $t_1$ and $t_2$ are written by the same author 
    \cite{bevendorff_generalizing_2019,bevendorff_divergence_based_2020,embarcadero_ruiz_graph_based_2022,rivera_soto_learning_2021,ordonez_will_2020,futrzynski_pairwise_2021,weerasinghe_feature_vector_difference_2021,llm_detection_av_2025}.
    % Gespräch Martin Potthast 19.05.2025: problem formulation 2 is less common and in the context of very sparse (metadata) information
    Related research areas include \cite{stein_intrinsic_2011}:
    \begin{itemize}
        \item stylometry
        \item outlier analysis and meta learning
        \item symbolic knowledge representation, i.e. \todo{knownledge representation, deduction, heuristic inference}
    \end{itemize}
    \citet{tyo_state_2022} state that \ac{av} is the fundamental problem of \ac{aa} \cite{tyo_state_2022}, 
    where there is only one candidate author \cite{barlas_cross_domain_2020}.
    \citet{llm_detection_av_2025,koppel_authorship_2004} state that \ac{av} is a more general and difficult (than \ac{aa}) one-class classification problem.
    Hence, the disputed document is compared only to documents from the candidate, 
    ignoring intermediate negatives (all human texts not written by the known author) 
    \cite{llm_detection_av_2025,neal_surveying_2018,koppel_authorship_2004}.
    It is impossible to assemble an exhaustive, or even representative samples of the non-target class \cite{koppel_authorship_2004}.
    \ac{av} is different from usual one-class classification problems, in which there is a lack of negative examples (opposed to non-representative ones), 
    and in which the text to attribute is not necessarily long (opposed to \citet{koppel_authorship_2004}'s paper).
    \citet{neal_surveying_2018} consider \ac{av} a binary classification problem (i.e. \textit{same-author} or \textit{different-author}), 
    where \textit{different-author} renders \ac{av} as an open-set problem.
    However, \citet{neal_surveying_2018} consider framing \ac{av} as a one-class classification problem as a common approach (cf. \cite{llm_detection_av_2025}).
    \citet{elmanarelbouanani_authorship_2014} consider \ac{av} a similarity detection task.
    \citet{neal_surveying_2018} also mention the many-candidates method where \ac{av} is framed as \ac{aa} via creating a set of imposters.
    Use cases include plagiarism detection, moderation of user-generated content, historical \ac{aa}, and forensic analysis \cite{rivera_soto_learning_2021}.
\end{definition}

\begin{definition}
    [Similarity detection]
    The task of comparing anonymous texts against other anonymous texts to assess the degree of similarity in terms of stylistic characteristics \cite{abbasi_writeprints_2008,neal_surveying_2018}.
    In the context of \ac{av}, the task is to determine whether two texts are produced by the same person without knowing the real author of the document \cite{elmanarelbouanani_authorship_2014}.
\end{definition}

\begin{definition}
    [Message-level analysis]
    The analysis attempts to categorize individual texts (e.g., whether an email was authored by an email address) \cite{abbasi_writeprints_2008}.
\end{definition}

\begin{definition}
    [Identity-level analysis]
    The analysis attempts to classify individuals belonging to a particular entity 
    (e.g., whether different email addresses belong to the same entity).
    Identity-level analysis' categorization is based on all texts written by that identity.
    Hence, there are larger text samples than for message-level analysis, facilitating the task.
    If the task is framed as classification/ID identification, the disputed identity is assigned to the identity with the highest similarity among the known identities.
    If the task is framed as similarity detection task, all identities with a similarity score above a certain threshold are grouped together and 
    considered to belong to the same entity \cite{abbasi_writeprints_2008}.
\end{definition}

\begin{table}[]
    \centering
    \caption{Building blocks for \ac{av} from \cite{stein_intrinsic_2011}.}
    \label{tab:authorship_verification_blocks}
    \resizebox{\textwidth}{!}{%
    \begin{tabular}{|l|lll|l|}
    \hline
    \rowcolor[HTML]{EFEFEF} 
    Pre-analysis & \multicolumn{3}{l|}{\cellcolor[HTML]{EFEFEF}Modeling and classifier methods} & Post-processing \\ \hline
    \rowcolor[HTML]{EFEFEF} 
    Impurity assessment & \multicolumn{1}{l|}{\cellcolor[HTML]{EFEFEF}Decomposition strategy} & \multicolumn{1}{l|}{\cellcolor[HTML]{EFEFEF}Style model construction} & Outlier identification & Outlier post-processing \\ \hline
    Document length analysis & \multicolumn{1}{l|}{Uniform length} & \multicolumn{1}{l|}{Lexical character features} & One-class density estimation & Heuristric voting \\
    Genre Analysis & \multicolumn{1}{l|}{Structural boundaries} & \multicolumn{1}{l|}{Lexical word features} & One-class boundary estimation & Citation analysis \\
    Analysis of issuing institution & \multicolumn{1}{l|}{Text element boundaries} & \multicolumn{1}{l|}{Syntactical features} & One-class reconstruction & Human inspection \\
     & \multicolumn{1}{l|}{Topical boundaries} & \multicolumn{1}{l|}{Structural features} & Two-class discrimination & Unmasking \\
     & \multicolumn{1}{l|}{Stylistic boundaries} & \multicolumn{1}{l|}{Language modeling} &  & Qsum \\
     & \multicolumn{1}{l|}{} & \multicolumn{1}{l|}{} &  & Batch means
    \end{tabular}%
    }
\end{table}
Post-processing to avoid false positives, c.f. \citet{stein_intrinsic_2011} for approaches.

\begin{definition}
    [Meta learning]
    \todo{Based on learning successes and failures, the system learns to learn. }
    Approaches include:
    \begin{itemize}
        \item Unmasking: Measurement of reconstruction errors starting from a good reconstruction and iteratively impairing the reconstruction. % example on page 9 of Benno's paper
        \item Qsum heuristic: Compares the growth rates of two cumulative sums over a sequence of sentences. The sums are calculated via the deviations from the mean sentence length and the deviations of the function words.
        \item Batch means: \todo{For a series of values the variance development of the sample mean is measured while the sample size is successively increased.}
    \end{itemize}
\end{definition}

\begin{definition}
    [Unmasking]
    The idea of this meta learning approach is that with progressively omitting more and more frequent words/ 
    most discriminating features, 
    topic specific words are excluded and thus, leaving only writing style specific words \cite{stein_intrinsic_2011}.
    After several iterations, remaining features which are not powerful enough to discriminate two documents indicate that 
    these documents originate from the same author \cite{stein_intrinsic_2011,tyo_state_2022,bevendorff_divergence_based_2020,koppel_authorship_2004}.
    In other words: 
    Two texts are probably written by different authors if the differences between are robust to changes in the underlying feature set used to represent the documents.
    Differences can be measured using instance-based (meta) classification via cross-validation accuracy 
    \cite{koppel_authorship_2011,bevendorff_generalizing_2019,bevendorff_divergence_based_2020,potthast_stylometric_2018,koppel_authorship_2004}, 
    creating a performance degradation curve \cite{tyo_state_2022,koppel_authorship_2004}.
    An \ac{svm} is trained to classify the degradation curve to determine whether two text originated from the same author 
    \cite{tyo_state_2022,bevendorff_generalizing_2019,koppel_authorship_2004}.
    Cf. \cite{bevendorff_divergence_based_2020} Chapt. 2 for a detailed algorithm.
    Steep decrease in the curve indicates that the two texts are similar, and thus, 
    written by the same authors \cite{potthast_stylometric_2018,koppel_authorship_2004}.
    Provided that the unseen text is very large, this method can handle small open candidate sets \cite{koppel_authorship_2011}.
    % koppel_determining_2014, pg. 1 + bevendorff_generalizing_2019 chap. 3.1 incl. algo: based on text chunks of length >= 500 words each
    \citet{koppel_determining_2014,bevendorff_generalizing_2019} claim that effective unmasking requires input documents to be large 
    (i.e. > 10000 words~\cite{koppel_determining_2014}, book-length~\cite{bevendorff_generalizing_2019}, 
    $\geq$ 5000 words (500 words per chunk) \cite{bevendorff_divergence_based_2020}).
    Otherwise the training set becomes too sparse and no descriptive curves can be generated 
    \cite{bevendorff_generalizing_2019,bevendorff_divergence_based_2020}.
\end{definition}

\begin{definition}
    [Stop words vs. function words]
    Function words are the most common words (articles, prepositions, pronouns, etc.) 
    like "while", "upon", "though", "were", "your" \cite{stamatatos_survey_2009,elmanarelbouanani_authorship_2014}.
    They are typically regarded as context-free (and therefore less influenced by topic and genre), 
    while revealing social and personal aspects of our lives \cite{neal_surveying_2018}.
    Most function words are stop words, but not all stop words are function words \cite{stein_intrinsic_2011}.
    \citet{elmanarelbouanani_authorship_2014} state that researchers use between 150 and 675 function words as features.
    \citet{abbasi_writeprints_2008} state that function words are highly effective discriminators of authorship, since 
    the usage variations of such words are strong reflection of stylistic choices.
\end{definition}

\begin{definition}
    [One-class classification]
    A classification problem where the classifier is trained on samples of a single class.
    If counterexamples, i.e. so-called outliers, are available, they are usually not considered to be representative of \textit{non-target class}.
    Hence, the classifier has to learn the concept of the target class in the absence of discriminating features 
    \cite{stein_intrinsic_2011,koppel_authorship_2004}.
    Examples of one-class classification are intrinsic plagiarism analysis and \ac{av}.
    Approaches to one-class classification fall into the following categories \cite{stein_intrinsic_2011}:
    \begin{itemize}
        \item One-class density estimation, e.g., Naive Bayes
        \item One-class boundary estimation
        \item One-class reconstruction
    \end{itemize}
\end{definition}

\begin{definition}
    [Open-set classification]
    The true author is not necessarily included in the set of candidate authors \cite{stamatatos_survey_2009,barlas_cross_domain_2020,neal_surveying_2018}.
    It is a generalization of the closed-set classification problem allowing for an unknown author using a threshold for similarity \cite{neal_surveying_2018}.
\end{definition}

\begin{definition}
    [Closed-set classification]
    The true author is one necessarily one of the candidate authors \cite{stamatatos_survey_2009,koppel_authorship_2011,barlas_cross_domain_2020,boenninghoff_o2d2_2021,neal_surveying_2018}.
    In other words: The set of all possible author classes is known a priori.
    Hence, closed-set problems can use supervised or unsupervised classification techniques \cite{abbasi_writeprints_2008}.
\end{definition}

\begin{definition}
    [Supervised techniques]
    Supervised techniques for stylometric analysis require (author-)class labels for categorization.
    Examples include \acp{svm}, \acp{nn}, decision trees, and linear discriminant analysis.
    \acp{svm} are very common in authorship analysis due to their robustness \cite{abbasi_writeprints_2008}.
\end{definition}

\begin{definition}
    [Unsupervised techniques]
    Unsupervised techniques make categorizations with no prior knowledge of author classes.
    Examples include \ac{pca} and cluster analysis.
    \ac{pca} has been used in previous authorship studies due to its ability to 
    capture essential variance across large number of features in a reduced dimensionality \cite{abbasi_writeprints_2008}.
\end{definition}

\begin{definition}
    [Covariate shift]
    The distribution of neural stylometric features changes between training and test set due to, for instance, topic variability \cite{boenninghoff_o2d2_2021}.
\end{definition}

\begin{definition}
    [n-gram]
    $n$ contiguous words also known as word collocations. \todo{cite, reference?\cite{koppel_authorship_2011}?}
    n-grams are no stylometric features \cite{altakrori_topic_2021}.
    % Quelle Martin Potthast Gespräch 19.05.2025:
    Tri-grams are commonly used in stylistic analysis, due to their ability to capture inflections, % Flexion/ Beugung in Deutsch
    morphemes, %  smallest meaningful constituents within a linguistic expression and particularly within a word
    and other syntactic structures for Germanic languages.
    Character-level $n$-gram features capture the frequency of $n$ consecutive characters in a text \cite{neal_surveying_2018}.
    The optimal $n$ is language dependent \cite{neal_surveying_2018}.
\end{definition}

\begin{definition}
    [Space free n-gram]
    Removing spaces from the $n$-gram reduces the number of $n$-grams.
    \citet{koppel_authorship_2011} use these definitions:
    \begin{enumerate}
        \item a string of $n$ characters that not include spaces
        \item a string of less than $n$ characters that is surrounded by spaces
    \end{enumerate}
\end{definition}

\begin{definition}
    [Domain shift]
    Systematic statistical differences between the training and test sets \cite{tyo_state_2022}.
    These differences include:
    \begin{itemize}
        \item datasets are not identically distributed
        \item test set contains novel topics $\times_t$
        \item test set contains novel authors $\times_a$
        \item test set contains novel genres $\times_g$
    \end{itemize}
\end{definition}

\begin{definition}
    [Topic-confusion]
    In this setting, all topics appear in both training and test set. 
    However, the topics of the texts for each author changes in the test set, 
    i.e. author-topic configuration is switched between training and testing.
    For example, in the training set, the author $A_1$ writes about topic $T_1$, author $A_2$ writes about topic $T_2$ 
    and in the test set, 
    the author $A_1$ writes about topic $T_2$ while author $A_2$ writes about topic $T_1$ \cite{tyo_state_2022,altakrori_topic_2021}.
    Intuitive, the more a feature is influenced by the topic of document to identify its author, 
    the more confusion it will be to the classifier when the topic-author combination is switched, which will lead to performance deterioration \cite{altakrori_topic_2021}.
\end{definition}

\begin{definition}
    [text distortion]
    This domain-adversial method substitutes out-of-vocaublary items with asterisks $*$ \cite{tyo_state_2022}.
    Its goal is to reduce domain-specific information \cite{bischoff_importance_2020}.
    Distortion algorithms include \cite{bischoff_importance_2020}:
    \begin{itemize}
        \item Replacing tokens with multiple asterisks
        \item Replacing tokens with single asterisks
        \item Retaining only exterior characters of words in a dictionary
        \item Retaining the last two characters
    \end{itemize}
\end{definition}

\begin{definition}
    [Imposter method]
    This method extends the ngram-unmasking method, i.e. iteratively omitting most influencely features (repeated feature subsampling \cite{koppel_determining_2014})
    from a trained classifier and classifying the accuracy drop.
    It takes score of how often an author is predicted after each feature-elimination step.
    The final prediction is made based on this score \cite{tyo_state_2022}.
\end{definition}


\begin{definition}
    [Hard Negative Mining]
    This method updates the model during training only with the most difficult examples in each batch.
    In the \ac{aa} context, difficult is defined as the most similar two texts from different authors, 
    which makes the decision the most difficult.
    \citet{tyo_state_2022} claim that the \ac{av} setting is strictly easier since 
    it most compare to only a single text.
    Due to the fact, that the most difficult example is model-dependent, \ac{av} problems can be made harder 
    but they can not exist of exactly the hardest negatives.
\end{definition}

\begin{definition}
    [Domain]
    The domain include topic, genre, register, idiolect, time period etc. \cite{bischoff_importance_2020}.
\end{definition}
  
\begin{definition}
    [Domain variables]
    These include topic, genre and language \cite{bischoff_importance_2020}.
\end{definition}

\begin{definition}
    [Style transfer]
    Translation or rather paraphrasing a text from a source style to a desired target style.
    Two major problems include the lack of large-scale parallel training data (i.e., texts written in both styles), 
    and the lack of reliable evaluation metric (i.e. assessment by humans) \cite{bischoff_importance_2020}.
\end{definition}

\begin{definition}
    [Author obfuscation]
    Task of paraphrasing a text to render an author's style imperceptible.
    Usually, another text from the author is used as a reference for style similarity \cite{bischoff_importance_2020}.
    In other words: It is the adversarial task of preventing successful verification by altering the text's style so that 
    it no longer resembles the original author's style \cite{bevendorff_divergence_based_2020}.
\end{definition}

\begin{definition}
    [within-domain]
    Experiments with P=Q.
    Hence, it is necessary to ensure all texts are mutually from the same domain \cite{bischoff_importance_2020}.
    \begin{table}[]
        \centering
        \caption{Typical scheme $S_1$ for \ac{aa} problem instances, where A, B, are authors and P, Q domains and 
        the vertical mapping denotes which author has written in which domain. 
        For training, texts from A and B take turn; for testing, previously unseen texts from A and B are used \cite{bischoff_importance_2020}.}
        \label{tab:within_domain_aa}
        \begin{tabular}{|l|ll|ll|}
        \hline
        \textbf{Scheme $S_1$} & \multicolumn{2}{l|}{\textbf{training}} & \multicolumn{2}{l|}{\textbf{testing}} \\ \hline
        \textbf{authors} & \multicolumn{1}{l|}{A} & B & \multicolumn{1}{l|}{A} & B \\ \hline
        \textbf{domains} & \multicolumn{1}{l|}{P} & Q & \multicolumn{1}{l|}{P} & Q \\ \hline
        \end{tabular}%
    \end{table}
\end{definition}

\begin{definition}
    [Domain swapping]
    Experiments with P$\neq$Q \cite{bischoff_importance_2020}.
    \begin{table}[]
        \centering
        \caption{Domain-swapping scheme $S_2$ for \ac{aa} problem instances, where A, B, are authors and P, Q domains and 
        the vertical mapping denotes which author has written in which domain. 
        For training, texts from A and B take turn; for testing, previously unseen texts from A and B are used \cite{bischoff_importance_2020}.}
        \label{tab:within_domain_aa}
        \begin{tabular}{|l|ll|ll|}
        \hline
        \textbf{Scheme $S_2$} & \multicolumn{2}{l|}{\textbf{training}} & \multicolumn{2}{l|}{\textbf{testing}} \\ \hline
        \textbf{authors} & \multicolumn{1}{l|}{A} & B & \multicolumn{1}{l|}{A} & B \\ \hline
        \textbf{domains} & \multicolumn{1}{l|}{P} & Q & \multicolumn{1}{l|}{Q} & P \\ \hline
        \end{tabular}%
    \end{table}
    There are two kinds of domain swapping:
    \begin{itemize}
        \item \textbf{Zero-knowledge swapping}: Maximizes the potential for confusion during training, 
        since the models never see an author in writing in the other author's respective fandom.
        This approach aggravates adversarial training, since it needs domain knowledge to be effective.
        \item \textbf{High-imbalance swapping}: Imbalance is swapped between the training and test set. 
        This is an approximation of the zero-knowledge swapping, while still allowing adversarial learning.
    \end{itemize}
\end{definition}

\begin{definition}
    [train-test-validation split]
    \citet{bischoff_importance_2020,altakrori_topic_2021,boenninghoff_o2d2_2021} train their model on a selection of the dataset (i.e. training set), 
    optimize the model's hyperparameters on a second disjoint selection of the dataset (i.e. validation set),
    and evaluate the model on a third disjoint selection of the dataset (i.e. test set).
    \citet{bischoff_importance_2020} ensure that there is no data leakage between the training, validation and test sets 
    (i.e. prevent parts of one fanfiction being in more than one of the data splits).
    \citet{altakrori_topic_2021} ensure the classifier to be trained has no access to any information about the setup 
    (topic confusion: group configuration or topic labels).
\end{definition}

\begin{definition}
    [Cross-domain]
    Texts of known authorship (training set) differ from texts of disputed authorship (test set) 
    in topic (i.e. cross-topic) or genre (i.e. cross-genre) 
    \cite{barlas_cross_domain_2020}.
\end{definition}

\begin{definition}
    [Cross-topic]
    New, unseen topics are used in the testing phase \cite{altakrori_topic_2021}.
\end{definition}

% Please add the following required packages to your document preamble:
% \usepackage{graphicx}
\begin{table}[]
    \centering
    \caption{\ac{aa} scenarios with author $i$ is shortened with $A_i$ \cite{altakrori_topic_2021}.}
    \label{tab:aa_same_topic}
    \begin{tabular}{|l|l|l|}
    \hline
    \textbf{} & \textbf{Train} & \textbf{Test} \\ \hline
    \textbf{Topic $T_1$} & $A_1, A_2$ & $A_1, A_2$ \\ \hline
    \textbf{Topic $T_2$} & $A_1, A_2$ & $A_1, A_2$ \\ \hline
    \end{tabular}%
\end{table}

% Please add the following required packages to your document preamble:
% \usepackage{graphicx}
\begin{table}[]
    \centering
    \caption{\ac{aa} scenarios with author $i$ is shortened with $A_i$ \cite{altakrori_topic_2021}.}
    \label{tab:aa_cross_topic}
    \begin{tabular}{|l|l|l|}
    \hline
    \textbf{} & \textbf{Train} & \textbf{Test} \\ \hline
    \textbf{Topic $T_1$} & $A_1, A_2$ &  \\ \hline
    \textbf{Topic $T_2$} &  & $A_1, A_2$ \\ \hline
    \end{tabular}%
\end{table}

% Please add the following required packages to your document preamble:
% \usepackage{graphicx}
\begin{table}[]
    \centering
    \caption{\ac{aa} scenarios with author $i$ is shortened with $A_i$ \cite{altakrori_topic_2021}.}
    \label{tab:aa_topic_confusion}
    \begin{tabular}{|l|l|l|}
    \hline
    \textbf{} & \textbf{Train} & \textbf{Test} \\ \hline
    \textbf{Topic $T_1$} & $A_1$ & $A_2$ \\ \hline
    \textbf{Topic $T_2$} & $A_2$ & $A_1$ \\ \hline
    \end{tabular}%
\end{table}

\begin{definition}
    [Data preprocessing]
    Typical stylometry subtask for normalization and noise reduction.
    Examples include tokenization, stemming, tagging, removing non-alphabetic characters and spaces, and converting uppercase letters to lowercase \cite{neal_surveying_2018}.,
\end{definition}

\begin{definition}
    [Tokenization]
    Splitting a stream of text into words, phrases, etc. \cite{neal_surveying_2018}.
\end{definition}

\begin{definition}
    [Stemming]
    Only retaining the root or base form of a word \cite{neal_surveying_2018}.
\end{definition}

\begin{definition}
    [Tagging]
    Replacing words with their grammatical type \cite{neal_surveying_2018}.
\end{definition}

\begin{definition}
    [\ac{bow}]
    The \ac{bow} approach generally refers to lexical-level features as it represents a document as a bag (or collection) of words, 
    discarding context, grammar, and word order \cite{neal_surveying_2018}.
\end{definition}

\begin{definition}
    [Latent Dirichlet Allocation (LDA)]
    LDA is a three-level Bayesian technique for modelling a collection over a set of topics.
    LDA is a probabilistic model where each topic is determined by word distributions \cite{neal_surveying_2018}.
\end{definition}

\begin{definition}
    [Clustering]
    Clustering is an unsupervised machine-learning procedure, where the algorithm derives a natural separation of the feature space 
    that may or may not correlate with the class labels \cite{neal_surveying_2018}.
\end{definition}

\begin{definition}
    [Paraphrase divergence]
    This means that when a question is phrased in s slightly different but semantically similar way, 
    \ac{llm} may output a wrong response despite being able to answer to the original question correctly \cite{fu_learning_2024}.
\end{definition}

\begin{definition}
    [Paraphrases]
    Texts that convey identical meanings while using different words or (sentence) structures \cite{fu_learning_2024,zhou_paraphrase_2021,palivela_optimization_2021}.
\end{definition}

\begin{definition}
    [Encoder]
    The main purpose of an encoder is to extract the semantic information for the decoder \cite{zhou_paraphrase_2021}.
\end{definition}

\begin{definition}
    [Attention mechanism]
    Attention allows the model to focus on particular words/phrases in the input sequence when generating the output sequence.
    First, a weight (i.e. importance) for each token in the source sequence in each timestep is computed.
    Then, both the text input and the context vector with weights is provided to the decoder \cite{zhou_paraphrase_2021}.
\end{definition}

\begin{definition}
    [\acl{rl}]
    \ac{rl} trains agents to take actions in an environment to maximize a cumulative reward \cite{zhou_paraphrase_2021}.
\end{definition}

\begin{definition}
    [GANs]
    Generative Adversarial Networks (GANs) consist of a generator and a discriminator.
    The generator creates realistic data instances that match the real distribution, 
    while the discriminator distinguishes which instance are generated and which are real \cite{zhou_paraphrase_2021}.
\end{definition}

\begin{definition}
    [Zero-Shot]
    Zero-Shot capabilities enable models, e.g. \acp{llm}, to perform well on unseen tasks \cite{master_thesis_paraphrasing_2024}.
\end{definition}

\begin{definition}
    [Internal validity]
    Internal validity is the extent to which the study results can be attributed 
    to the manipulations of the independent variable rather than other factors \cite{master_thesis_paraphrasing_2024}.
\end{definition}

\begin{definition}
    [External validity]
    External validity is the extent to which the study results can be generalized 
    to other contexts, settings, or populations \cite{master_thesis_paraphrasing_2024}.
\end{definition}

\begin{definition}
    [Subtasks of Paraphrasing]
    There are two sub-tasks of paraphrasing \cite{palivela_optimization_2021}:
    \begin{itemize}
        \item \ac{pg}: The task of generating fluent and semantically similar paraphrases for a given sentence. 
        It can be solved by using simple lexical features and word ordering or restructuring methods or 
        by using templates extracted from WikiAnswers repositories \cite{palivela_optimization_2021}.
        Formally, for a sentence $S_1=\{w_1, \cdot, w_n\}$, generate one or more candidate sentences $S_2=\{w_1, \cdot, w_m\}, \cdot$.
        Sentence lengths may vary. \cite{palivela_optimization_2021}.
        \item \ac{pi}: The task of determining whether two sentences are paraphrases of each other.
        \ac{pi} ca be viewed as a discriminative task. The system output can be a probability (1 for paraphrase, 0 for non-paraphrase) 
        or a semantic score which can be normalized \cite{palivela_optimization_2021}.
        Formally, for a sentence pair $(S_1, S_2)$, find a target 1 or 0. Sentence lengths may vary. \cite{palivela_optimization_2021}.
    \end{itemize}
\end{definition}

\begin{definition}
    [Transfer learning]
    Transfer learning is a machine learning technique where a model trained on a data-rich task is 
    reused as the starting point (i.e. finetuning) for a model on a second task (i.e. downstream task) \cite{palivela_optimization_2021}.
\end{definition}

    % Opposed to traditional vector space models, language models consider the context of a word when computing its embedding \cite{emmery_adversarial_2021}.


% BERT
\ac{bert} is a language model trained through masked language modelling and next-sentence prediction \cite{emmery_adversarial_2021}.
Hence, \citet{emmery_adversarial_2021}'s approach to substituting certain words in a text is straightforward:
The word to be substituted is masked, and the model predict the top-$k$ most likely words.
According to this approach two potential shortcomings arise:
\begin{itemize}
    \item The predicted word can be semantically inconsistent to the original.
    \item A semantic shift could occur since the model considers predicted words prior to the current word at each position.
\end{itemize}
\citet{emmery_adversarial_2021} propose mitigating these shortcomings by using dropout to zero part of the weights of the internal embedding of the target/ original word.
They assume that the top-$k$ candidate words are semantically more similar to the original word than the masked suggestions.
    % An excessive amount of electronic texts is available on the internet.
Text categories include e-mail messages, online forum messages, blog posts and source code.
% obstacles
However, this data is prone to be noisy, short and from multiple candidate authors \cite{stamatatos_survey_2009}.
\section{Stylometric features}
\label{sec:stylometric_features}

Refer to \autoref{sec:definitions} for the definitions of the terms used in this section.

\begin{table}[]
    \centering
    \caption{Types of stylometric features with their computation tools and 
    resources where brackets indicate optional tools from \citep{stamatatos_survey_2009}.}
    \label{tabstylometric_features_tools}
    \resizebox{\textwidth}{!}{%
    \begin{tabular}{|l|l|l|}
    \hline
    \rowcolor[HTML]{EFEFEF} 
    \textbf{Category} & \textbf{Features} & \textbf{Required tools and resources} \\ \hline
    Lexical & Token-based (word/ sentence length, ...) & Tokenizer, (Sentence splitter) \\
     & Vocabulary richness & Tokenizer \\
     & Word frequencies & Tokenizer, (Stemmer, Lemmatizer) \\
     & Word n-grams & Tokenizer \\
     & Errors & Tokenizer, Orthographic spell checker \\
    Character & Character types (letters, digits, ...) & Character dictionary \\
     & Character n-grams (fixed length) & - \\
     & Character n-grams (variable length) & Feature selector \\
     & Compression methods & Text compression tool \\
    Syntactic & Part-of-Speech (POS) & Tokenizer, Sentence splitter, POS tagger \\
     & Chunks & Tokenizer, Sentence splitter, (POS tagger), Text chunker \\
     & Sentence and phrase structure & Tokenizer, Sentence splitter, POS tagger, Text chunker, Partial parser \\
     & Rewrite rule frequencies & Tokenizer, Sentence splitter, POS tagger, Text chunker, Full parser \\
     & Errors & Tokenizer, Sentence splitter, Syntactic spell checker \\
    Semantic & Synonyms & Tokenizer, (POS tagger), Thesaurus \\
     & Semantic dependencies & Tokenizer, Sentence splitter, POS tagger, Text chunker, Partial parser, Semantic Parser \\
     & Functional & Tokenizer, Sentence splitter, POS tagger, Specialized dictionaries \\
    Application-specific & Structural & HTML parser, Specialized parsers \\
     & Content-specific & Tokenizer, (Stemmer, Lemmatizer), Specialized dictionaries \\
     & Language-specific & Tokenizer, (Stemmer, Lemmatizer), Specialized dictionaries
    \end{tabular}%
    }
\end{table}
\citet{bevendorff_overview_2024} list lexical diversity, average sentence length, average word length, 
the number of grammatical errors, sentiment tendency, repetition rate, and stop word ratio as 
stylometric and linguistic features.

% all information from stamatatos_survey_2009
% lexical features
Lexical features consider sentences grouped sequences of tokens, i.e. words, numbers or punctuation marks \citep{bevendorff_overview_2024}.
They are used to learn about preferred use of characters and words of an author \citep{elmanarelbouanani_authorship_2014}.
% more lexical feature examples at elmanarelbouanani_authorship_2014 Ch. 3.1
% length counts
Simple attempts of author attribution include sentence lengths counts, and word length counts.
These approaches work straightforward assuming that the tokenizer is able to identify the tokens correctly \citep{bevendorff_overview_2024,elmanarelbouanani_authorship_2014}.
However, for some language such as Chinese, the tokenization is not trivial.
% vocabulary richness
Vocabulary richness functions attempt to quantify the diversity of vocabulary used in a text.
Approaches include $\frac{\text{size of vocabulary}}{\text{total number of tokens}}$ \citep{elmanarelbouanani_authorship_2014,bevendorff_overview_2024},
Yule's K measure \citep{elmanarelbouanani_authorship_2014},
the number of hapax legomena (words that occur only once in a text) \citep{elmanarelbouanani_authorship_2014,bevendorff_overview_2024,weerasinghe_feature_vector_difference_2021}, 
or the number of hapax dislegomena (words that occur twice in a text) \citep{elmanarelbouanani_authorship_2014,weerasinghe_feature_vector_difference_2021}.
The vocabulary size depends on the text length (quick increase in the beginning, slowly increasing later).
To mitigate this effect, researchers have proposed methods to introduce stability into these metrics \citep{elmanarelbouanani_authorship_2014,bevendorff_overview_2024}. 
According to \citet{stamatatos_survey_2009}, these proposed methods are questionable and should not be used alone.
\citet{neal_surveying_2018} list some of the most common vocabulary richness measures:
\begin{itemize}
    \item Zipf's law models a linear relationship between the number of vocabulary items appearing $r$ times in a document.
    \item Yule's K measure assumes that the occurrence of a word is based on chance and can be modelled according to a Poisson distribution.
    \item Yule's I measure $= \frac{M_1 M_1}{M_1 M_2}$, where $M_1$ is the number of words in a document and $M_2$ is the sum of weighted word forms with a certain frequency. A larger result indicates a richer vocabulary.
    \item The Burrows method considers large sets of high-frequency function words per 1000 words and then applies \ac{pca}. This method is considered a standard method in authorship attribution.
    \item Hapax legomena counts the number of words to appear one, while happax dislegomena counts the number of words to appear two times in a document.
    \item Type/token ratio is the ratio of all tokens (or words) to unique tokens (referred to as types).
\end{itemize}
% frequent words
\citet{elmanarelbouanani_authorship_2014} state that the most frequent words in a text is a lexical feature set.
% BoW
Texts can also be represented by vectors of word frequencies, i.e. the \ac{bow} model \citep{bevendorff_overview_2024}.
This representation disregards contextual, i.e., word-order, information.
While topic-based classification usually excludes so-called function words, 
i.e. most common words, since they do not carry any semantic information and are topic-independent, 
style-based text classification includes them.
In particular, function words are considered among the best features to discriminate between authors 
since they capture pure stylistic choices of authors across different topics.
Hence, style-based text classification requires much lower dimensionality (i.e., a few hundred words) 
in comparison to topic-based text classification (i.e., several thousand words).
Function words can be closed class words, i.e., articles, prepositions etc., 
or open-class words, i.e., nouns, adjectives, verbs.
Closed class words are the first dozen frequent words and 
open class words are predominantly present in the next frequent words.
% n-grams
To overcome the lack of contextual information in the \ac{bow} model, 
n-grams can be used.
However, the classification accuracy does not always increase with the usage of n-grams instead of individual word features.
Moreover, the dimensionality of the problem increases, representations become sparse and possibly content-specific information rather than stylistic information is captured.
% Errors
Similar to manual human attribution, using error measures can be used to identify the author of a text.
Spelling (i.e., letter omissions, insertions) and formatting errors (i.e., all caps words) can be used as features 
to capture the idiosyncrasies of an author's style \citep{elmanarelbouanani_authorship_2014,bevendorff_overview_2024}.
Unfortunately, according to \citep{stamatatos_survey_2009}, the availability of accurate spell checkers for many natural languages is problematic.


% character features
Character feature consider texts a sequence of characters.
Simple approaches include alphabetic character count, digit character count, uppercase/ lowercase character count, letter frequencies and punctuation marks count.
% more character feature examples at elmanarelbouanani_authorship_2014 Ch. 3.1
\citet{stamatatos_survey_2009} claim these features are available for any natural language and prove useful.
% n-grams
They also mention frequencies of $n$-grams on character level.
This approach is robust to noise, captures lexical and contextual information, grammatical errors, use of punctuation and capitalization.
It performs better on oriental languages than token-based approaches.
However, the dimensionality of the problem increases with $n$-grams compared to tokens.
Moreover, the choice of $n$ is language-dependent since certain natural language (e.g., German) tend to have longer words than others (e.g., English).
Large $n$ on the one hand, capture lexical and contextual information but may also capture content-specific information and increases the dimensionality of the representation.
Small $n$ on the other hand, may be able to represent subwords but are less adequate to capture contextual information.
% compression-based features
The idea of compression based approaches is to use the compression model acquired from one text to compress another text. 
If both texts are from the same author, the bit-wise size of the compressed file will be relatively low \citep{stamatatos_survey_2009,neal_surveying_2018}.
Since compression models usually describe characteristics of the texts based on the repetition of character sequences, 
they can be considered as character-based features. \todo{Is this still the case? Paper is from 2009}


% syntactic features
\citet{elmanarelbouanani_authorship_2014} defines syntactic features as the patterns used to form and structure sentences.
Examples include punctuation words and function words \citep{elmanarelbouanani_authorship_2014}.
Assuming that author's have a syntactic fingerprint, these approaches are considered more reliable than mere lexical information \citep{bevendorff_overview_2024}.
Unfortunately, syntactic features are language-dependent and require \ac{nlp} tools, for instance, a parser.
% Parser
Once the text has been parsed, metrics such as noun phrase counts and verb phrase counts can be computed.
% POS
Another approach is to use \ac{pos} tags on word-token or $n$-gram for frequencies.
% errors
If spell checkers are available and of sufficient quality, syntactic errors can be used as features.
% etc.

% structural features
Structural features are based on the structure of the text, i.e., the way it is organized.
They include the way sentences are organized within paragraphs, and paragraphs are organized within documents \citep{elmanarelbouanani_authorship_2014}.


% semantic features
Since high-level tasks are prone to errors and noise, semantic analysis is not as widely used as syntactic analysis.
\todo{Is the quality still too bad? Paper is from 2009}


% application-specific features
Based on the applications, structural measures include the use of greetings, farewells, types of signatures, indentation, paragraph length 
or HTML specific metrics.
Usually, in the context of stylometry, content specific features are not used.
If all documents belong to the same thematic area, content-based information may reveal some authorial choices. 
One can compare content-specific word frequencies, even though it remains unclear how to select such words for a given text domain.


% profile-based

% instance-based

% hybrid
    % \section{Real World Scenario for \ac{aa}}
\label{sec:real_world_scenario_authorship_attribution}

\citet{koppel_authorship_2011} state three aspects of \ac{aa} in real world scenarios:
\begin{itemize}
    \item Thousands of known candidate authors
    \item Real author may not be in the candidate set
    \item Texts of candidate authors or the real author, i.e. the one to attribute, may be very limited
\end{itemize}
\citet{koppel_authorship_2011} propose an approach for authorship attribution.
They randomly select a fixed number of features multiple times and collect the top matches of the unseen text and the candidate author profiles.
The score $\sigma$ of an author $A$ is the proportion of times the unseen text was attributed to $A$.
If the score $\sigma$ is above a certain threshold, the author is considered to be the true author of the text.
The author with the highest score $\sigma$ among the candidate authors is selected as the predicted author.
According to \citet{koppel_authorship_2011}, the method is works reasonably well on large (open and closed) candidate sets.

% shortcomings
In case of a small open set of candidate authors, the method is may falsely attribute the unseen text to one of the candidate authors.
\citet{koppel_authorship_2011} define coverage as the percentage of unseen texts, i.e. snippets, 
for which the score $\sigma$ is obtained.
They define the probability of the predicted author being the true author as 
$$\frac{p * H * P}{p * + (1-p) * E}$$
where $p$ is the prior probability of the true author being in the candidate set (defined by user),
$P$ is the precision (i.e. the proportion of true positives among the predicted authors),
$E$ is the propobility of a candidate author achieving a score $\sigma$ if the true author is not among the candidate authors.
    % \section{Metrics}
\label{sec:metrics}

\citet{chen_web_2008} use precision, recall (also: sensitivity or \acl{tp} rate \cite{palivela_optimization_2021}) 
and F-measure to describe the performance:
$$precision = \frac{TP}{TP + FP}$$
$$recall = \frac{TP}{TP + FN}$$
$$F-measure = \frac{2 \cdot precision \cdot recall}{precision + recall}$$~\cite{chen_web_2008,abbasi_writeprints_2008}
where $TP$ is the number of \aclp{tp}, $FP$ is the number of \aclp{fp} 
and $FN$ is the number of \aclp{fn}.
    % \section{Domain Shift}
\label{sec:domain_shift}

\cite{tyo_state_2022}
\section{\ac{pan} \ac{aa} \& \ac{av} metrics}
\label{sec:pan_aa_av_metrics}

The \todo{F0.5u, C@1, and Brier Score metrics} are used to evaluate the ability of \ac{av} methods 
to abstain from hard samples \cite{tyo_state_2022}.
For each sample, a score $\in [0, 1]$ is assigned to the sample.
A score of exactly 0.5 means the model abstains from the sample \cite{tyo_state_2022,bevendorff_overview_2024,kocher_unine_2015}.

The AUC metric is used to evaluate the ability of methods to rank predictions.
No threshold is required.
\ac{pan} ignores any abstained samples when calculating the AUC metric \cite{tyo_state_2022}.

% != PAN
\citet{tyo_state_2022} chose to adopt the macro-averaged accuracy metric, so-called macro-accuracy, for \ac{aa}, 
and AUC \ac{av} tasks.
\section{Deep learning methods}
\label{sec:deep_learning_methods}

Deep learning methods such as recurrent neural networks, transformers, word embeddings, and byte-pair encoding \citep{tyo_state_2022}.
In the \citet{tyo_state_2022}' survey, they include work from 2020 stating CNNs are no longer competitive.

\citet{tyo_state_2022} claim that since \ac{av} methods are binary classification problems (i.e. same author or not), 
opposed to \ac{aa} problems, the datasets have very large number of words per class. 
Hence, \ac{av} formulations are more effective for training deep learning methods.
\citet{tyo_state_2022} claim that the number of documents per author and the number of words per document is 
a better indicator of the complexity of the dataset than the number of authors, documents or words.
Many authors with small number of documents per authors is considered hard.
In terms of \ac{aa} dataset complexity, the imbalance, i.e. standard deviation of the number of documents per author, 
is the best indicator of the dataset complexity.

\section{Prior work by feature extraction}
\label{sec:prior_work_by_feat_extraction}

\citet{tyo_state_2022} provide a comprehensive overview over prior work in the realm of \ac{aa} and \ac{av} methods.
They organise their survey by the feature extraction methods used in the respective work.
\todo{Add visualization similar to Fig.1 in \citep{tyo_state_2022}}

% 1. feature based
% 1.1. n-grams
In the context of feature based features the present ngrams.
In most cases, n-grams features are \ac{bow} features, i.e. counting occurrences of n-grams.
In some cases, n-grams are transformed using CNNs \cite{tyo_state_2022}.
For \ac{aa}, \citet{tyo_state_2022} claim, that ngrams are state-of-the-art if there are less than 50000 words per author.
% in this context: unmasking & Impostor method, LDA, sentence syntax trees

% 1.2. summary statistics
Summary statistics, including distribution of word lengths, hapax-legomena, \todo{Maas' $a^2$, Herdan's $V_m$}, 
can be used for \ac{av} tasks: 
Calculate the difference of these feature vectors between to texts and train a logistic regression classifier 
to predict whether the texts were written by the same author.
\citet{tyo_state_2022} claim that this performs well despite its simplicity.

% 1.3. Co-occurrance graphs: cf.~tyo_state_2022 pg. 5

% 2. embedding based
Another category of feature extraction methods are embedding based methods.
% 2.1. char embeddings
\citet{tyo_state_2022} first present character embeddings.
Starting with RNN models shared across multiple authors with individual heads, 
further presenting CNNs and % CNN for AA: char better than word embeddings.
compression-based methods such as ZIP, \ac{rar} (i.e.\ a variant of \ac{ppm}), \ac{ppm}, 
to build representations comparable via distance metrics.
For \ac{aa}, \citet{tyo_state_2022} claim, that \ac{ppm} is a low performer and scales poorly to large datasets.
\citet{bevendorff_divergence_based_2020} quote someone else, who recommend compression-based cosine (CBC) calculated on text pairs after compression with the PPMD algorithm.
They explain that natural language allows very good for compression ratios due to its predictability (English has an entropy of at most 1.75 bits per character).
PPMD uses finite-order Markov language models for compression, which work well for predicting characters in a sentence but are sensitive to increased entropy stemming from obfuscation.
% sind sie dann noch geeignet? Wenn sie direkt altered Texte erkennen, wäre das doof.

% 2.2. word embeddings
In the context of word embeddings, \citet{tyo_state_2022} present the following methods:
\begin{itemize}
    \item Fasttext word embeddings concatenated with CNN embeddings as input to (Siamese) nested BiLSTMs trained with contrastive loss
    \item GloVe-like embeddings for sentences via parse-tree as input to Siamese BiLSTMs trained with contrastive loss
    \item POS tages along with word embeddings
\end{itemize}

% 2.3 Transformers
\citet{tyo_state_2022} also present research in diverse language model architectures, including RNN, BERT and GPT2.
For \ac{aa}, \citet{tyo_state_2022} claim, that BERT is state-of-the-art method if the dataset contains over 100000 words per author.

% 3. combination
\citet{tyo_state_2022} state that some researches combine transformers with summary statistics.


\todo{Partial Matching (PPM)= text compression; byte-pair encoding (modern \ac{aa} method)}
page 3 in \citep{tyo_state_2022}: table about \ac{aa} and \ac{av} datasets with their respective characteristics

    % \section{\ac{av} as One-Class categorization}
\label{sec:av_one_class}

\citet{koppel_authorship_2004} frame \ac{av} as a one-class categorization problem.

They claim research had shown that linear separators work well for text categorization.
Linear models include Naive Bayes (linear separator for two classes), 
Winnow and Exponential Gradient and linear \acp{svm} \cite{koppel_authorship_2004}. 

\subsection{\ac{aa} framework}
\citet{koppel_authorship_2004} state that the following framework solve a number of real world \ac{aa} problems:
\begin{enumerate}
    \item Construction of appropriate feature vectors
    \item Construction of a distinguishing model via a learning algorithm
    \item Assessment of effectiveness of methods using k-fold cross-validation or bootstrapping
\end{enumerate}

\subsection{Chunks}
\citet{koppel_authorship_2004} propose chunking texts such that each chunk is of approximately equal length, 
and at least 500 words without breaking paragraphs. 

\subsection{Definitions}
For author $A$ and book $X$, \citet{koppel_authorship_2004} define the following:
If $A$ is not the author of $X$, $A_X$ is the set of all works by author $A$.
If $A$ is the author of $X$, $A_X$ is the set of all works by author $A$ except $X$.
A pair of $A_X$ and $X$ is called \emph{same-author} if X was authored by $A$.
A pair of $A_X$ and $X$ is called \emph{different-author} if $X$ was not authored by $A$.

\subsection{Initial feature set}
The initial feature set consists of the 250 words with the highest average (over $X$ and $X_A$) frequency \cite{koppel_authorship_2004}.

\subsection{Features for meta-classifier}
\citet{koppel_authorship_2004} propose the following features for the meta-classifier 
(where $i$ is the number of elimination steps):
\begin{itemize}
    \item accuracy after $i$ elimination steps
    \item accuracy difference between round $i$ and round $i+1$
    \item accuracy difference between round $i$ and round $i+2$
    \item $i^{th}$ highest accuracy drop in one iteration
    \item $i^{th}$ highest accuracy drop in two iterations
\end{itemize}

The vectors are grouped by \emph{same-author} and \emph{different-author} pairs and thus, 
used to train a meta-learning scheme.

\subsection{Negative examples for Elimination method}
\citet{koppel_authorship_2004} state that negative examples are neither exhaustive nor representative.
They propose using words of several authors $A_1, ..., A_n$ roughly filling the same profile as candidate $A$ 
in terms of geography, chronology, culture and genre.
$A_1, ..., A_n$ are said to collectively represent class not-$A$.

\subsection{Elimination method}

The elimination method is only used to overrule positive predictions.
Hence, it can eliminate \acp{fp}.
One can frame it as a filter which is applied after or before unmasking.

% training
\citet{koppel_authorship_2004} learn a model for $A$ and against not-$A$, 
and multiple models for $A_i$ and against not-$A_i$.
% inference
Then, $X$ is tested against all of these models.
$A(X)$ is the percentage of examples of $X$ classed as $A$ rather than not-$A$ 
(i.e., $A_i(X)$ analogously).
If $A(X)$ is not larger than all $A_i(X)$, $A$ is not the author of $X$.
If $A(X)$ is larger than all $A_i(X)$, conclude nothing.

    % \section{Problem hierarchy}
\label{sec:problem_hierarchy}

% koppel_determining_2014 use "identification" for attribution/ AA
\todo{change identification to attribution for consistency in terminology}

% AV -> open-set
\ac{av} is an open-set problem, meaning that the author of an anonymous document 
may or may be not be part of the set of candidate authors.

% AA -> closed-set
\ac{aa} is a closed-set problem, meaning that the author of an anonymous document
is part of the set of candidate authors.
For each candidate author, writing samples are available.
The task is to determine the author of the anonymous document from the set of candidate authors.

% reduction: closed-set AA -> open-set AV
\citet{koppel_determining_2014} state that all closed-set \ac{aa} problems are reducible to the \ac{av} problem.
The reverse is not true.
To reduce the \ac{aa} problem to the \ac{av} problem, we solve a \ac{av} problem, i.e.\ if text was written by a candidate author, 
for each of the respective candidates.
Ideally, we receive one positive answer for the correct candidate author and negative answers for all other candidates.

% complexity
\citet{koppel_determining_2014} explain that the \ac{av} problem is more complex than the \ac{aa} problem.
They claim that the ability to solve a closed-set \ac{aa} problem does not imply the ability to solve an open-set \ac{av} problem.

% open-set identification/ AA = many candidates problem
\citet{koppel_determining_2014} define the many-candidates problem, or the so-called open-set identification problem:
Given a large set of candidate authors, determine which, if any, of them wrote a given anonymous document.
According to \citet{koppel_determining_2014}, the many-candidates problem can be solved reasonably well: \autoref{lst:many_candidate_algo}.

\begin{algorithm}
    \caption{Author Identification via Random Feature Subsets}
    \label{lst:many_candidate_algo}
    \begin{algorithmic}[1]
        \Procedure{GetAuthor}{$X$, $features$, $candidate\_authors$, $k$, $\sigma^*$}
            \State Initialize count vector $M[c] \gets 0$ for all $c \in candidate\_authors$

            \For{$i = 1$ to $k$} \Comment{Number of iterations $k$}
                \State $F_i \gets \text{RandomSubset}(features)$ \Comment{Set of features $features$}
                \For{$c \in candidate\_authors$}
                    \State Compute similarity: $s_c \gets \text{Similarity}(X, Y_c|_{F_i})$ \Comment{Input text $X$}
                \EndFor
                \State $c^* \gets \arg\max_{c} (s_c)$
                \State $M[c^*] \gets M[c^*] + 1$  \Comment{Increment match count for best match}
            \EndFor

            \For{$c \in candidate\_authors$}
                \State $\text{Score}(c) \gets \frac{M[c]}{k}$ \Comment{Proportion of times $c$ was top match}
            \EndFor

            \State $c^{\text{final}} \gets \arg\max_{c} \text{Score}(c)$
            \If{$\text{Score}(c^{\text{final}}) > \sigma^*$} \Comment{Threshold $\sigma^*$}
                \State \Return $c^{\text{final}}$
            \Else
                \State \Return $\text{Don't know}$
            \EndIf
        \EndProcedure
    \end{algorithmic}
\end{algorithm}

% AV -> many-candidates problem
To reduce the \ac{av} problem to the many-candidates problem, we need to generate a large set of imposter candidates.
% open-set identification/ AA -> open-set AV => equivalence
\citet{koppel_determining_2014} state that the many-candidates problem is equivalent to the open-set \ac{av} problem.
% closed-set AA/ identification -> many-candidates problem/ open-set AV/ identification
The closed-set identification is reducible to the many-candidates problem.

\begin{figure}[htbp]
    \centering
    \includesvg{notes/Koppel_imposter_2014/problem_hierarchy_complexity}
    \caption{Problem hierarchy and complexity.}
    \label{fig:problem_hierarchy}
\end{figure}
\section{Impostor method}
\label{sec:impostor_method}

The problem to solve is the \ac{av} problem:
For two documents $X, Y$ determine if they were written by the same author.
As displayed in \Cref{fig:problem_reduction}, \citet{koppel_determining_2014} propose reducing the \ac{av} problem to the many-candidates problem.

\begin{figure}[htbp]
    \centering
    \includesvg{notes/Koppel_imposter_2014/reduction_closed_set_AV_to_open_set_AA}
    \caption{Reducing the \ac{av} problem to the many-candidates problem.}
    \label{fig:problem_reduction}
\end{figure}

This method produces a set of \textit{impostor} documents.
Then, \citet{koppel_determining_2014} determine whether $X$ is sufficiently more similar to $Y$ than to the impostor documents.
There are multiple important settings:
\begin{itemize}
    \item Proper methods to select the impostor documents.
    \item Proper methods to measure the similarity between documents.
\end{itemize}
Similar to unmasking, \citet{koppel_determining_2014} repeatedly select random subsets of features that serve as the basis for comparing documents.
If a documents $Y$ is more similar to document $X$ than any other document for many feature subsets, 
it is likely that $X$ and $Y$ are by the same author.
% FIXME: Cref doesn't know lst 
The algorithmic approach is displayed in \Cref{alg:impostor_algo}.
\citet{koppel_determining_2014} state that k=100 iterations are sufficient.
The threshold $\sigma^*$ varies the recall-precision tradeoff.
\citet{koppel_determining_2014} claim that the imposter method obtains strong results even for documents with no more than 500 words.
Moreover, they highlight the similarity between the impostor method and an ensemble of classifiers learning different subset of features.
% TODO: two paragraphs below conflicting? cf.~koppel_determining_2014 pg. 181, 182
Furthermore, the performance of the method improves as the number of candidate author diminishes.
In an open-set scenario, fewer candidate authors makes the problem more difficult for the impostor method 
since one author being consistently more similar to the text than the others candidates across multiple feature subsets is less likely in large sets of competing candidate authors.
A greater number of \imps{} reduces the number of \acp{fp} and \acp{fn}.
% cf.~koppel_determining_2014 pg. 182, but conflicting with above:
% Small sets of candidate authors are more likely to generate \acp{fp} 
% while large sets may generate \acp{fn} and thus, creating a tradeoff.

\begin{algorithm}
    \caption{\imps{} Method for Author Verification}
    \label{alg:impostor_algo}
    \begin{algorithmic}[1]
    \Procedure{IsSameAuthor}{$X$, $Y$, $\sigma^*$}
        \Comment{$X$, $Y$: input documents; $\sigma^*$: decision threshold}
    
        \State Initialize $scores = \{\}$ 
    
        \ForAll{$d \in \{X, Y\}$}  
            \State $d' \gets \{X, Y\} \setminus \{d\}$ 
            \Comment{Disputed document $d'$}
            \State $\mathcal{I}_d \gets \text{GenerateImpostors}(d)$ 
            \Comment{$m$ impostor documents for candidate $d$}
    
            \State $scores[d] \gets 0$ 
    
            \For{$i = 1$ to $100$} 
                \Comment{100 random feature subsets}
                \State $F_i \gets \text{RandomSubset}(\text{Features})$ 
                \State $S \gets \{ \text{Similarity}(d', I_j \mid F_i) : I_j \in \mathcal{I}_d \cup \{d\} \}$ 
                \State $j^* \gets \arg\max_j S_j$ 
                \If{$I_{j^*} = d$} \Comment{Count times $d=$ top match}
                    \State $scores[d] \gets scores[d] + 1$ 

                \EndIf
            \EndFor
        \EndFor
    
        \State \Return $\left( \frac{scores[X] + scores[Y]}{2} > \sigma^* \right)$ 
        \Comment{Return \textbf{True} if average $> \sigma^*$}
    \EndProcedure
    \end{algorithmic}
    \end{algorithm}
    
% Tradeoff/ Risks
If we want to generate \imps{} for a document $Y$ and compare it to a document $X$, 
the \imps{} should be similar (e.g. in terms of genre) to $Y$ rather than $X$.
Otherwise, the $Y$ would never be the top match and thus, producing \acp{fn}.
Unconvincing impostor documents would lead to \acp{fp} due to the lack of similarity between $X$ and the impostor documents making $Y$ consistently the top match.

% Choice of the impostor documents 
First the $m$ most similar impostor (in terms of min-max similarity) documents are selected.
Then, $n$ random impostor documents are selected from the $m$ impostor documents.
\citet{koppel_determining_2014} found that using a selection of $n$ \imps{} rather than all $m$ impostor documents produces better results.
The approach is insensitive to $m,n$.
\citet{koppel_determining_2014} propose three options of generating the impostor documents.
\begin{itemize}
    \item Fixed
    \item On-the-fly
    \item Blogs
\end{itemize}

Fixed impostor documents can be aggregated results of random (English) Google queries.
They have not special relation to the document pair in question.

% sicher, non-reference???
On-the-fly impostor documents are generated by randomly selecting medium-frequency words from the document $Y$ (i.e.\ the \textcolor{red}{non-reference document} in \Cref{alg:impostor_algo}).
The words are then used to query Google and aggregate the top results of the respective queries. 
Hence, the impostor documents are similar to the document $Y$ in terms of content.

When using blogs as impostor documents, the impostor documents are selected from other bloggers. 
Hence, the impostor documents are similar to both documents $X,Y$ in terms of assuming $X,Y$ share a genre.
According to \citet{koppel_determining_2014}, this method produces the best results.

\citet{koppel_determining_2014} claim that similar \imps{} to the document $Y$ reduce the number of impostor documents needed to achieve a certain level of performance.
They also state that the search engine is used if no information about the input documents is available.
\section{Imposter dataset}
\label{sec:imposter_dataset}

\citet{koppel_determining_2014} construct a corpus of 500 documents pairs $<X,Y>$ of blog posts.
Half of them are by the same author, half by different authors.
No single author is represented more than once.
The task is to determine whether the two documents are by the same author.
Since the model is not supplied any labelled samples of any author, the method is considered unsupervised.
\section{Baseline methods}
\label{sec:impostor_baseline_methods}

\citet{koppel_determining_2014} compare the impostor (cf. \autoref{sec:impostor_method}) method to two baseline methods:
\begin{itemize}
    \item similarity detection
    \item supervised baseline
\end{itemize}


\subsection{Similarity detection}
\label{sec:imp_similarity_detection}

For similarity detection, $X,Y$ are considered to be authored by the same author if they are sufficiently similar, 
i.e. their similarity exceeds a threshold.
They represent a text $X$ as a tf-idf vector $\overrightarrow{X}= (x_1, \cdots , x_n)$ 
of space-free 4-grams (cf. \autoref{sec:definitions}).
They select the 100000 most frequent 4-grams from the training set.
As similarity measures, they compare the cosine similarity and the min-max measure.

$$sim(X,Y)=cosine(\overrightarrow{X},\overrightarrow{Y})=\frac{\overrightarrow{X}*\overrightarrow{Y}}{\left\| \overrightarrow{X} \right\|*\left\| \overrightarrow{Y} \right\|}$$
$$sim(X,Y)=minmax(\overrightarrow{X},\overrightarrow{Y})=\frac{\sum_{i=1}^{n}min(x_i,y_i)}{\sum_{i=1}^{n}max(x_i,y_i)}$$

The optimal threshold on \citet{koppel_determining_2014} development set depended on the similarity measure.


\subsection{Supervised baseline}
\label{sec:imp_supervised_baseline}

\citet{koppel_determining_2014} use a disjoint labelled training set consisting of 1000 pairs $<X,Y>$.
They assign labels \textit{same-author} and \textit{different-author} 
to the difference vectors $diff(X,Y)= (\left| x_1-y_1 \right|, \cdots , \left| x_n-y_n \right|)$ 
depending on whether $X,Y$ have the same author or not.
They train a linear \ac{svm} as a classifier on the difference vectors.

\section{Metrics in the impostor method paper}
\label{sec:impostor_metrics}

\citet{koppel_determining_2014} define precision as the proportion of correct attributions among all test snippets 
for which some attribution is given by the algorithm.
They denote recall the proportion of test snippets for which an attribution is given by the algorithm and is correct.
    % \section{Generalized Unmasking Approach}
\label{sec:generalized_unmasking_approach}

\citet{bevendorff_generalizing_2019} prioritize precision, 
i.e. high confidence verification of authorship and thus, rejecting low confidence predictions.
They propose an \todo{open-source, customizable unmasking framework}. % cf. bevendorff_generalizing_2019 for url

\citet{bevendorff_generalizing_2019} propose creating chunks by oversampling words in a bootstrap aggregating manner, 
Each text is a pool of words, from which words are sampled without replacement.
The pool is replenished if it is exhausted before the chunk has sufficiently many words.
Since the random sampling of unmasking features introduces variance, unmasking is performed multiple times and the curves are averaged.
The algorithm is displayed in \autoref{alg:generalized_unmasking}.

% prediction
Another linear \ac{svm} classifier is trained on the accuracy curve, its central-difference gradients (first- and second order), and its gradients sorted by magnitude.
This classifier is used to predict the whether the texts originate from the same author.

\begin{algorithm}
    \caption{Generalized Unmasking Algorithm}
    \label{alg:generalized_unmasking}
    \begin{algorithmic}[1]
    \Procedure{Unmasking}{$A$, $B$}
        \Comment{$A$, $B$: input documents}
    
        \State $\mathcal{C}_A \gets \text{RandomChunks}(A, 30, 700)$ \Comment{30 chunks, 700 words each}
        \State $\mathcal{C}_B \gets \text{RandomChunks}(B, 30, 700)$
        \State $\mathcal{F} \gets \text{TopFreqWords}(A, B, 250)$
        \State $\mathcal{C} \gets \mathcal{C}_A \cup \mathcal{C}_B$

        
        \While{$|\mathcal{F}| \geq 0$}
        \State $a \gets \text{CVAcc}(\mathcal{C}_A, \mathcal{C}_B, \mathcal{F}, linSVM)$ \Comment{Append $10$-fold cross-validation accuracy}
        \State $\mathcal{F} \gets \mathcal{F} \setminus \mathcal{F}_{\text{top}}^{\pm}$ \Comment{Remove top $5$ most significant positive and negative features}
    
        \EndWhile
    
        \State \Return List of recorded accuracies $a$
    \EndProcedure
    \end{algorithmic}
\end{algorithm}
    
\section{Generalised Unmasking Metrics}
\label{sec:generalized_unmasking_metrics}

\citet{bevendorff_generalizing_2019} focus on precision and sacrifice recall.
Hence, they do not use standard two-class accuracy.
Opposed to \ac{pan} \citet{bevendorff_generalizing_2019} choose to not use c@1 since it is designed for 
binary classification with equal weights for both classes.
They argue that they are primarily interested in the positive class, i.e.\ the same author.
Therefore, they choose the $F_0.5u$ measure where non-answers are treated as \acp{fn} as a special case of $F_{0.5}$:
$$\frac{(1+0.5^2)n_{tp}}{(1+0.5^2)n_{tp}+0.5^2(n_{fn}+n_{u})+n_{fp}}$$
$n_{tp}$ denoting the number of true positives, $n_{fp}$ the number of false positives, 
$n_{fn}$ the number of false negatives, and $n_{u}$ the number of unanswered problems.
$\beta=0.5$ is used to weight precision higher than recall.


\citet{bevendorff_bias_2019} compare (binary) decision quality of two approaches, where one is unmasking, 
using the \todo{McNemar test}.
    % \section{Voigt-Kampff}
\label{sec:voigt_kampff}
\newcommand{\voigtkampff}{Voigt-Kampff}

% problem definition
\citet{bevendorff_overview_2024} present results from the \voigtkampff{} challenge at 
\ac{pan} and \ac{eloquent} at \ac{clef} 2024.
The task is to determine whether a text was \ac{ai} generated or written by a human.
The authors want to establish a baseline and thus, choose to formulate the task in its simplest form:
\begin{quote}
    \textit{Given two texts, one authored by a human, one by a machine: pick out the human.}
\end{quote}
As in its fictional inspiration, 
\textit{Blade Runner} where an officer uses the \voigtkampff{} machine to test whether a subject is a replicant, 
the challenge is presented in a builder-breaker format:
\begin{itemize}
    \item \textbf{Builder at \ac{pan}}: Participants are asked to build a system that can detect 
    whether a text was written by a human or an \ac{ai}.
    \item \textbf{Breaker at \ac{eloquent}}: Participants are asked to build a system that can 
    generate texts that are indistinguishable from human-written texts.
\end{itemize}
Participants' submissions compete in an adversarial setting.

% similarity to Authorship identification
\citet{bevendorff_overview_2024} note the similarity to \ac{aa} and \ac{av} tasks, 
where the \ac{ai} is considered an author that exhibits identifiable characteristics.
They want prior research of authorship tasks in this \ac{ai} detection task to be considered.
Baselines for the \ac{ai} detection task include two \ac{av} systems, 
where each text is split in half and comparing them under the assumption that \ac{llm} texts 
are more self-similar than human texts.
The baselines are a compression model (\todo{PPMd Compression-based Cosine}) and 
short-text authorship unmasking.
\citet{bevendorff_overview_2024} also assume that \ac{llm}-generated texts show 
weaker coherence and textual entailment between sentences.

% machine/ AI
In this context, \ac{ai} refers to a language model.
They recommended using the following prompt to instruct the \ac{llm} to generate a text 
based on the bulletpoint summary of a news article generated as described in \cite{pan_dataset_authorship_verification}:
\begin{quote}
    \textit{Write a text of about 500 words which covers the following items:}
\end{quote}
% 9 different language models
The language models used in the challenge are:
\begin{itemize}
    \item \textbf{Poro} is an open-source, freely available multilingual decoder-only model.
    \item \textbf{Mistral}is an open-source, freely available model and trained for conversational data.
    \item and more cf. pg.8 \cite{bevendorff_overview_2024}
\end{itemize}
\citet{bevendorff_overview_2024} manually reviewed the generated texts and removed obvious artefacts 
(i.e., typical \ac{llm} chat phrases, excessive bullet lists, etc.).
They found that \ac{llm} texts failed to incorporate the quotes in the text and count the number of words.
The majority of open-source models were found to be less prone to these artefacts than closed-source \acp{llm}.
Finally, \citet{bevendorff_overview_2024} truncated the human texts to the same length as the \ac{llm} texts by 
fitting a \todo{log-normal distribution} to the \ac{llm} text lengths and drawing from the distribution for each human text.
The drawn samples were used as starting point and texts were truncated after the nearest paragraph ending in a window of at most 200 characters.

% test case
\citet{bevendorff_overview_2024} chose to hold back a variant of Gemini Pro model on the training set to test 
how robust submitted solutions are against the unseen \ac{llm} models.

% model output
\citet{bevendorff_overview_2024} chose to adapt the \ac{pan} \ac{av} scoring scheme.
During inference, the models received pairs of human and \ac{llm} texts.
For each sample, a score $\in [0, 1]$ is assigned to the sample.
For a score $<0.5$, the model predicts the left text to be human,
and for a score $>0.5$, the model predicts the right text to be human.
A score of exactly 0.5 means the model abstains from the sample.

% model metrics
They defined the effectiveness of the systems, i.e. models, as the arithmetic mean of the metrics defined in \autoref{sec:pan_evaluation}, 
where each metric is $\in [0, 1]$.
They correct all metrics by discounting half a standard deviation, estimated on each dataset individually, 
from the system's scores \todo{with $n-1$ degrees of freedom}.
This penalizes unstable systems with widely varying scores on the individual dataset variants, 
and promotes systems more robust to the dataset variants' alterations, including text obfuscation and modification.
Finally, they computed the macro-averaged scores across all datasets since they consider all datasets evenly important 
(even those with fewer number of examples).
With regard to the leader board, i.e. ranking of the approaches, they computed a \todo{mean score correlation coefficient (Kendall's $\tau$)}.

% dataset metrics
They defined the difficulty of the datasets as the inverse mean effectiveness score of the detection systems.


% source: Ben
micro/ macro in the context of multi-class confusion matrix (bc precision/ accuracy/ recall for binary classification)
micro average score = same as binary, but sums over class indices in both nominator and denominator, e.g.
$micro\_precision = \frac{\sum_{i=1}^n TP_i}{\sum_{i=1}^n (TP_i + FP_i)}$
macro average score = average of the class-wise scores, e.g.
$macro\_precision = \frac{1}{n} \sum_{i=1}^n \frac{TP_i}{TP_i + FP_i}$
where $n$ is the number of classes.
    % \section{Suppressing Domain Style}

% potential introduction to style chapter
\citet{bischoff_importance_2020} assume that each author has a unique style, unconsciously encoded in their writing.
This style depends on the author's personal traits, customs an author adopts due to genre, register, type, and topic.
These concepts are too vague to be efficiently operationalized.
The goal is to discover a set of style markers more likely to be determined by the author's personality than by domain customs.

They claim that features frequent function words and word length have a high correlation with topic. \todo{????}

% char 3-gram
\citet{bischoff_importance_2020} analyse the robustness of character trigrams as a feature for \ac{aa}.
They find that the character trigrams feature set is not robust in a cross-topic setting, but across two genres.

% this approach
\citet{bischoff_importance_2020} propose an approach where they train with respect to the author labels, 
and adversarially train on the texts with respect to their domain labels.
They claim that this results in discriminative features for the task of \ac{aa} and in 
indiscriminative for the text domain differences.

% Domain swapping
\citet{bischoff_importance_2020} quantify the influence of domains rather than author style on the performance of models by 
computing the difference $\Delta(S_1,S_2)$ of the model performances in both the traditional scheme $S_1$ 
and the domain-swapped scheme $S_2$ (cf. \autoref{sec:definitions}).

% min len
\citet{bischoff_importance_2020} claim that the minimum sufficient length to measure author style is 500 words.
    % \section{Topic Confusion}
\label{sec:topic_confusion}

% error
\citet{altakrori_topic_2021} propose topic confusion.
It is a task specifically designed to evaluate \ac{aa} models' effectiveness.
The task splits the error into two parts:
\begin{itemize}
    \item \textbf{Models' confusion}: The model is confused about the topic of the text.
    \item \textbf{Features' inability}: The features are unable to capture the authors' writing style.
\end{itemize}

% dependency of author and topic
\citet{altakrori_topic_2021} present an example to explain the dependency of author and topic:
Consider a topic defined by a unique word distribution.
An author selects a subset of words from the topic due to the limited length of documents.
Since the authors choose wordings, i.e. synonyms, the dependency of the topic on the author varies from one author to another.
Based on this, they model the attribution process as follows:

\begin{equation}
    \label{eq:topic_confusion_model_world}
    P(A,T,D) = P(A) P(T \mid A)P(D\mid A,T)
\end{equation}

\begin{equation}
    \label{eq:topic_confusion_attribution_process}
    P(A=a \mid D) \propto \sum_{T}^{t}\left[ P(A=a) P(T=t \mid A = a)P(D\mid T=t,A=a) \right]
\end{equation}

% topic confusion
Opposed to traditional cross-topic \ac{aa} tasks, 
which do not shed any light on whether the error arose from the topic or the features themselves, 
\citet{altakrori_topic_2021}'s topic confusion task offers insights in the error sources and 
whether features are indicative of writing style or topic.
% data constraints
The task requires writing samples written by $N$ authors on $T$ topics where $N \geq 4$ and $T \geq 3$ and 
each author has written approximately the same number of samples on each topic.
% procedure
First, authors are divided into two groups.
During training, the model receive only topic $A$ for authors in group one and topic $B$ for authors in group two.
Topic $A$ and topic $B$ are chosen randomly.
During testing, the group-topic configuration is flipped.
Unused writing samples are used for the validation set.

% metrics
\citet{altakrori_topic_2021} define three metrics:
\begin{itemize}
    \item \textbf{Correct (\%)}: Percentage of correctly classified samples.
    \item \textbf{Same-group error (\%)}: Percentage of samples that were attributed to the wrong author but within the same group 
    as the correct author.
    \item \textbf{Cross-group error (\%)}: Percentage of samples that were attributed to the wrong author and to the author group 
    that does not contain the correct author.
\end{itemize}

% explanation
Topic invariant features capture the writing style of the author.
Hence, the model should perform well on the test set.
Features that capture topic rather than writing style lead to a model classifying according to topic, 
resulting in cross-group errors.
Other error sources result in same-group errors.

\subsection{Common classifiers and features}
\label{sec:topic_confusion_classifiers}
% common models
Common options of classifiers are \acp{svm}, \ac{nb} and decision trees.
\citet{altakrori_topic_2021} attach their hyperparameter configuration for a \ac{svm} classifier to their paper.
They also state n-grams are a common approach to represent documents in \ac{aa} tasks.
Commonly, tokenization for n-grams is done either on the word level or on the character level.
\citet{altakrori_topic_2021} also include \ac{pos}-level n-grams in their feature set 
which are considered an essential indication of style.
If large pre-trained language models are used with the comparably small training datasets, embedding layers are frozen and 
only the classification layer is trained.

\subsection{Features}
\label{sec:topic_confusion_features}
% features
% stylometric features
Based on empirical evidence (one paper, one dataset), stylometric features are rather topic-invariant \citep{altakrori_topic_2021}.
% POS
According to \citet{altakrori_topic_2021}, \ac{pos} tags capture stylistic variations in language grammar between authors.
% n-grams
Character-level n-grams are favourable over word-level n-grams in terms of cross-group error.
Hence, character-level n-grams are more topic-invariant than word-level n-grams.

\subsection{Masking}
\label{sec:topic_confusion_masking}
% masking
\citet{altakrori_topic_2021} also include a masking technique to replace every character in masked words with (*) and all digits by (\#).
Low frequency words in the British National Corpus (BNC) are masked.
After masking, the original document structure is recreated, based on which the n-gram features are extracted.

% Imbalanced dataset
\subsection{Imbalanced Dataset}
\label{sec:topic_confusion_imbalanced_dataset}
\citet{altakrori_topic_2021} state that proper metrics for imbalanced datasets include 
weighted accuracy, precision, recall and F-Score.


\subsection{\ac{ap} model}
\label{sec:topic_confusion_ap_model}
\citet{altakrori_topic_2021} train a separate neural language model for each author.
Each model is called an \ac{ap} model.
Embedding layers are intialized with pre-trained equivalences.

\subsection{How \ac{ap} model works}
\label{sec:topic_confusion_ap_model_works}
For a document of disputed authorship, the average perplexity of all \ac{ap} models is calculated.
The perplexity scores are normalized using a normalization vector $\begin{pmatrix}
    n_0
     \\\vdots 
     \\n_k
    
    \end{pmatrix}$
where $n_i$ is the average perplexity of author $A_i$ on the normalization corpus.
\citet{altakrori_topic_2021} propose two normalization corpora:
\begin{itemize}
    \item \textbf{Training set}
    \item \textbf{Testing set without labels} (unrealistic scenario, since often only one document is available during inference)
\end{itemize}
The lowest perplexity score is assigned the author of the document.

\subsection{Shortcomings of \acp{lm}}
\label{sec:topic_confusion_shortcomings_lm}
\citet{altakrori_topic_2021} suspect that due to the nature of \acp{lm}, where words of similar meaning, i.e. "color" and "colour", 
are mapped to a similar vector, they do not work well in \ac{aa} tasks.
In \ac{aa} tasks, language system differences are highly relevant since they reveal the author's identity.  
    % \section{Heuristic Author Obfuscation}
\label{sec:heuristic_author_obfuscation}

\citet{bevendorff_divergence_based_2020} cast the task of author obfuscation as a heuristic search problem.
They look for a cost-minimum sequence of tailored paraphrasing operations for a significant increase in the distance 
of the paraphrased text to other texts from the author.
Estimated text quality reduction is used as a cost function.

% trigram
\subsection{Trigram}
In the realm of authorship analysis, character 3-grams are considered effective \cite{bevendorff_divergence_based_2020}, 
even though it is not systematically proven.
\citet{bevendorff_divergence_based_2020} claim that current (up to 2020) \ac{av} algorithms analyse (implicitly/explicitly) 
character 3-grams.
According to \citet{bevendorff_divergence_based_2020}, without loss of generality, since character 3-gram are the lowest-level feature that 
still captures word boundaries and morphology, % Lehre über Worte
directly influence higher-level features, such as word n-grams and \ac{pos} features.
This holds true for most European languages \cite{bevendorff_divergence_based_2020}.
3-grams encode vocabulary, morphology, and punctuation.

% distance measure
\subsection{Distance Measure}
% KL divergence
The \ac{kld} is a character-based style distance measure.
It acts as a feature- and task agnostic % skeptisch
information-theoretic divergence measure.
Moreover, it can be used as a simple and computationally feasible stopping criterion for an obfuscation process.
Furthermore, based on \ac{kld}, a normalization criterion for obfuscating texts of different lengths can be derived.
Finally, \ac{kld}'s derivative can serve as a selection criterion for parts of text that 
will yield the highest obfuscation gain if changed \cite{bevendorff_divergence_based_2020}.
This could be used for a greedy algorithm.
However, \citet{bevendorff_divergence_based_2020} claim greedy strategies can be reverse engineered.
$$KLD(P\mid \mid Q) = \sum_{i}^{}P[i] log\frac{P[i]}{Q[i]}$$
where $P,Q$ are the discrete probability distribution corresponding to the relative frequencies of character 3-grams in the 
to-be-obfuscated text and the known texts respectively.
For true probability distributions, the \ac{kld} is always non-negative.
However, the \ac{kld} is not symmetric, i.e. $KLD(P\mid \mid Q) \neq KLD(Q\mid \mid P)$.
Moreover, it is only defined for distributions $P,Q$ where $Q[i] = 0$ implies $P[i] = 0$ (which yields zero summands).
% Jensen-Shannon divergence
To overcome \ac{kld}'s shortcomings, 
\citet{bevendorff_divergence_based_2020} use the $JS_\Delta$ metric, based on the \ac{jsd}:
$$JSD(P\mid \mid Q) = \frac{KLD(P\mid \mid M)+KLD(Q\mid \mid M)}{2}$$
where $M = \frac{P + Q}{2}$ is the average of the two distributions.
The $JS_\Delta$ metric is defined as:
$$JS_\Delta(P,Q) = \sqrt{2\cdot JSD(P\mid \mid Q)}$$

% writeprints
\begin{definition}
    [Writeprints]
    \label{def:writeprints}
    A set of over twenty lexical, syntactic, and structural text feature types \cite{bevendorff_divergence_based_2020}.
\end{definition}

% task
\subsection{Task}
After \ac{pan}'s author obfuscation tasks from 2016 to 2018 received either poor rule-based approaches or such that 
produced unreadable texts, they chose to refine the task to paraphrase the text.

\begin{definition}
    [Paraphrase]
    The meaning should stay the same and the text should be still readable \cite{bevendorff_divergence_based_2020}.
\end{definition}

% proposed approach
\subsection{Proposed Approach}
\citet{bevendorff_divergence_based_2020} propose obfuscation by reduction.
First, they rank 3-grams based on their influence on $JS_\Delta$ via their partial \ac{kld} derivative.
They remove one occurrence of the most influential 3-gram from the to-be-obfuscated text via naively omitting it without replacement, 
targeted paraphrasing (semantically equivalent text passage without the 3-gram).
The problem is defined as potentially infinite space with possible (text) states, 
in which each state is reachable from one or multiple nodes in a graph spanned over the entire space by operators.
Edges are labelled with the cost of the operator.
The task is to find a minimum-cost path from a starting node to a node that satisfies a pre-determined goal condition 
(i.e. sufficient obfuscation).
Applying an operator has highly non-linear effects on the text quality and may restrict the set of applicable operators in the same text.
They define the informed heuristic search A*.

The verifier most resilient against this specific reductive obfuscation (obfuscating only n-grams that are already rare for maximum effect) 
was based on an imposter approach on most frequent words.

% dataset
\subsection{Dataset}
The dataset used for the task is the Webis \acl{av} corpus 2019.
    % \section{Authorship Analysis Survey}
\label{sec:authorship_analysis_survey}

% features
\citet{elmanarelbouanani_authorship_2014} claim that there is no feature set optimized and applicable to all people and to all domains.

% metrics
\subsection{Metrics}
\citet{elmanarelbouanani_authorship_2014} state that the common metrics for evaluating the performance of a particular are:
\begin{itemize}
    \item $Accuracy = \frac{TP + TN}{TP + TN + FP + FN}$ \citep{elmanarelbouanani_authorship_2014,neal_surveying_2018} 
    measures the percentage of classified correctly over all test cases \citep{neal_surveying_2018}.

    \item $Precision = \frac{TP}{TP + FP}$ \citep{elmanarelbouanani_authorship_2014,neal_surveying_2018} 
    measures how often a system gets positive classification correctly \citep{neal_surveying_2018}.

    \item $Recall = \frac{TP}{TP + FN}$ \citep{elmanarelbouanani_authorship_2014,neal_surveying_2018} 
    measures how often a system correctly classifies positive samples when it encounters them \citep{neal_surveying_2018}.
\end{itemize}


% distance measures
\subsection{Distance Measures}
\citet{elmanarelbouanani_authorship_2014} state that the most common distance measures are:
\begin{itemize}
    \item Delta measure
    \item Chi-Square distance
    \item \ac{kld}
\end{itemize}
They claim the Delta measure outperforms the other measures.
    % \section{Cross-Domain Pre-Trained Language Models}
\label{sec:cross_domain_pre_trained_LM}

\subsection{Features}
% topic invariant features
\citep{barlas_cross_domain_2020} claim function words or \ac{pos} n-grams are topic invariant features.
% efficient in AA
\citep{barlas_cross_domain_2020} state that character n-grams associated with word affixes and punctuation marks 
are most efficient in cross-topic \ac{aa} tasks.
% pre-processing
It is possible to pre-process texts to mask topic-related information while keeping the text structure including 
function words and punctuation marks.

\subsection{Dataset splits}
\label{sec:dataset_splits}

\citet{barlas_cross_domain_2020} propose Leave-One-Topic-Out cross-validation splits for cross-topic \ac{aa} tasks:
All texts on a specific topic (within a certain genre) are included in the test set,
while the remaining texts (in that genre) are used for training.
Mean classification accuracy is reported over all topics.

Analogously, \citet{barlas_cross_domain_2020} propose Leave-One-Genre-Out cross-validation splits for cross-genre \ac{aa} tasks:
All texts of a specific genre (within a certain topic) are included in the test set,
while the remaining texts (in that topic) are used for training.
Mean classification accuracy is reported over all genres.

\subsection{Normalization corpus} 
\label{sec:normalization_corpus}

In \ac{aa} tasks, it is important for the normalization corpus to have 
exactly the same properties with documents of unknown authorship.
This is not feasible in practice \citep{barlas_cross_domain_2020}.
    % \section{SMAuC dataset}
\label{sec:smauc_dataset}

SMAuC is a comprehensive, metadata-rich dataset of scientific papers across multiple fields tailored to scientific authorship analysis \citep{bevendorff_smauc_2023}.

\citet{bevendorff_smauc_2023} caraterize different paper types by character count:
\begin{itemize}
    \item Abstract Papers: < one page
    \item Short Papers: one to two pages
    \item Essay-length Papers: up to 10 pages
    \item Long Papers: up to 50 pages
    \item Books or dissertations: > 50 pages
\end{itemize}
    % \section{Feature Vector Difference}
\label{sec:feature_vector_difference}

\citet{weerasinghe_feature_vector_difference_2021}'s goal is to create a topic agnostic \ac{av} model that works well on the open-world setting 
where authors and topics in test set are not in the training set. They propose a new feature vector difference method that is based on the \ac{av} model of \citet{stamatatos_2020} and show that it outperforms the state-of-the-art methods in the open-world setting.
The input to the model is a feature vector $X$ encoding the two documents $D_i,D_j$ 
and the target variable $Y$ indicating whether the two documents are written by the same author or not. 
Hence, they model the problem as a binary classification task.

% nltk library
\subsection{\ac{nltk} Library usage}
They use \ac{nltk}'s \texttt{casual\_tokenize} method, which uses their \texttt{TweetTokenizer} to tokenize the documents.
For \ac{pos} tagging, the use \ac{nltk}'s Perceptron Tagger.
In order to generate a \todo{partial parse tree} (i.e.\ \ac{pos} chunks), they train a Maxent (Maximum Entropy) classifier.
The parse trees are used to extract so-called syntactic dependency-based n-grams of \ac{pos} tags, i.e, stylometric features.

\subsection{Sklearn Library usage}
They use \ac{sklearn}'s \texttt{TFIDFVectorizer} to create the \ac{tf-idf} vectors for their features (cf.~below).
They set \texttt{min\_df=0.1} to ignore terms that have less than a 10 \% document frequency.

\citet{weerasinghe_feature_vector_difference_2021} used a Stochastic Gradient descent training algorithm with logarithmic loss function, 
which results in a logistic regression classifier (by sklearn's  \texttt{SGDClassifier}) since the complete feature matrix cannot be stored in-memory.

\subsection{Features}
\textcolor{brown}{cf.~Table 1, \citep{weerasinghe_feature_vector_difference_2021} Chapter 2.3 for urls}
\citet{weerasinghe_feature_vector_difference_2021} used some features from Writeprints feature set.
Character n-grams are \ac{tfidf} values for character n-grams of size 1 to 3.
They claim $n>3$ is not useful for \ac{av} because it is affected by topic similarities.
\ac{pos}-Tag n-grams are \ac{tfidf} values for \ac{pos}-tag 3-grams.
Special characters are \ac{tfidf} values for 31 pre-defined special characters.
Frequency of function words (where \citep{weerasinghe_feature_vector_difference_2021} linked a stop word website!) 
are frequencies for 851 common English words.
They also computed the average number of characters per word. %, the average number of words per sentence,
Moreover, they included the distribution of word-lengths (fraction of tokens of length $l \in [1,10]$).
They employed Python's \texttt{textComplexity} library to compute a variety of vocabulary richness measures 
(cf.~page 4, \citep{weerasinghe_feature_vector_difference_2021}).
The \ac{pos}- tags chunks are the \ac{tfidf} values for the second level \ac{pos}-tag chunks (e.g., 'NP', 'VP', 'IN').
The \ac{pos}-tag construction are \ac{tfidf} values of each noun phrase, verb phrase, and prepositional phrase expansion.
They also define the stop-word and \ac{pos}-tag hybrid tri-grams to capture the syntactic information about the word order, 
inspired by text distortion:
In order to prevent topic related biases, all non-function words are replaced with the \ac{pos}-tag of the word.
Based on this, \ac{tfidf} values are computed for tri-grams.
The \ac{pos}-tag ratio calculated the portion of all \ac{pos}-tags in the \todo{Penn Treebank \ac{pos} Tag} collection.
They also include unique spelling, where the fraction of words present in the document that belong to each of the following dictionaries: 
Commonly misspelled English words, common typos when communicating online, common errors with determiner, 
British spelling of English words and popular online abbreviations.

\subsection{Pipeline}
Feature extractors were fit on training data.
Data was standarized (i.e., mean=0, std=1).
An absolute vector difference was computed for each feature vector pair and standarized again.
This standarized vector difference was used as input to the classifier.




    % \section{UniNE CLEF 2015 \ac{av} Imposter}
\label{sec:UniNE_CLEF2015_AV_imposter}

The idea of \cite{kocher_unine_2015} is to apply a distance measure to the candidate author and the disputed text and 
compare the distance to those between the disputed text and a set of imposters.
Hence, they need a sample text of the candidate author and texts from the imposters.

% features
\subsection{Features}
% \label{sec:features}

\citet{kocher_unine_2015} use the $k$ most frequent terms of the disputed texts where isolated words and punctuation symbols are considered terms.
The state a reasonable range for $k$ is 200 to 300.


\subsection{Distance measure}


\subsection{Set of Imposters}

\citet{kocher_unine_2015} claim \ac{aa} is not more difficult than \ac{av} 
due to the fact that in \ac{av} tasks one has to compare the disputed text with a set of all representative imposter texts.
The authors claim it is difficult to determine whether one has included all imposters, 
i.e., other writers having a similar style to the candidate author.

\subsection{\tira{}}
\label{sec:tira}

The evaluation of shared \ac{pan} CLEF tasks is done via the \tira{} platform.
\tira{} is an automated tool for deployment and evaluation of the software submitted.
During runs of the submitted software, no data leakage to the participants is possible since \tira{} encapsulates the software.
\citet{kocher_unine_2015} claim that \tira{} offers a fair evaluation of the tome needed to produce an answer.
However, system incompatibilities may occur \cite{kocher_unine_2015}.


    % \section{Universal Authorship Representations using \acp{nn}}
\label{sec:universal_authorship_representations}

\citet{rivera_soto_learning_2021} present the trade-off of neural methods:
Neural methods learn relevant features and thus, obviate the need of manual feature design, but 
at the same time these features are not explicitly identified.
Hence, the domain influence on the learned features is not clear.
Moreover, neural methods require large training datasets.
They found that neural methods are not universal.

% zero-shot transfer
\subsection{Zero-Shot Transfer}
\citet{rivera_soto_learning_2021} define zero-shot transfer as training a model on a specific domain and 
evaluating it on a held-out evaluation set from other domains. % chapter 4.1. defintion ot entirely clear

\subsection{Domain agnostic data augmentation}
\citet{rivera_soto_learning_2021} propose a simple domain agnostic data augmentation method where 
random subwords are omitted (i.e. dropout) to mask domain-specific features.
They found that it is difficult to find a suitable threshold for frequency based dropout and thus, 
resolved with random dropout.
    % \newcommand{\writeprints}{Writeprints}
\section{\writeprints{}}
\label{sec:writeprints}

\citet{abbasi_writeprints_2008} propose the \writeprints{} technique for \ac{aa} and similarity detection.
\writeprints{} is an unsupervised method. % Karhunen-Loeve is supervised?!
It is a Karhunen-Loeve transform-based technique that uses a sliding window and 
pattern disruption algorithm with individual author-level feature sets.

\writeprints{} is constructed for each author using the author's key features, 
i.e.\ individual-author-level (cf.~\autoref{sec:definitions}) feature sets.
The algorithm consists of two parts as outlined in \autoref{alg:writeprints}:
% part 1: creation
Via applying Karhunen-Loeve transforms with a sliding window, patterns are created.
These \writeprints{} patterns project usage variance into a lower-dimension space for each author feature.
Each lower-dimensional point reflects a single window instance.
% part 2: pattern disruption
Features an author never uses are treated as pattern disruptors (i.e. red flags), 
which means that the occurrence of these features in the disputed text decreases the similarity score 
between the author and the disputed text.

\begin{algorithm}
\caption{Writeprints Steps}
\label{alg:writeprints}
\begin{algorithmic}[1]
    \Procedure{Writeprints}{$X$, $authors$, $feature\_sets$}

        \State \textbf{Step 1: Creation of \writeprints{} Patterns}
        \ForAll{individual-level features with occurrence frequency > 0}    \Comment{Individual per author}
            \State Extract feature vectors for each sliding window instance \Comment{sliding window size $L=1500$ characters (i.e. roughly 250 words) \& jump interval $J=L$ characters (i.e. no overlap)}
            \State Derive basis matrix (set of eigenvectors) from feature usage covariance matrix using Karhunen-Loeve transforms \Comment{number basis vectors via Kaiser-Guttmann stopping rule}
            \State Compute \writeprints{} patterns (= Coordinates/ Principal Components for each window instance) by multiplying window feature vectors with basis
        \EndFor
       

        \State \textbf{Step 2: Pattern Disruption}  
        \ForAll{author features with occurrence frequency = 0} \Comment{From set of features of all authors}
            \State Compute feature disruption value as product of information gain, synonymy usage, and disruption constant $K$
            \State Append features' disruption values to basis matrix
            \State Apply disruptor based on pattern orientations
          
        \EndFor

        \State Repeat step 1-2 for each identity
    \EndProcedure
\end{algorithmic}
\end{algorithm}



\subsection{Karhunen-Loeve}
\label{sec:karhunen-loeve}

Karhunen-Loeve transforms are a supervised form of \ac{pca} that allows for inclusion of class information in the transform process.
It consists of a dimensionality reduction technique where the transformation is done by deriving the basis matrix (set of eigenvectors) and 
then projecting the feature usage matrix into a lower-dimension space.
While \ac{pca} captures the variance across a set of authors (interclass variance) using a single feature set and basis matrix, 
Karhunen-Loeve can be applied to each author (intraclass variance) by only considering the author's feature set and basis matrix.
Thus, Karhunen-Loeve can be used as an individual-level (cf.~\autoref{sec:definitions}) similarity detection technique 
(more about that in \citep{abbasi_writeprints_2008} Chap. 2 last paragraph).


\subsection{Feature set dimension}
\label{sec:feature-set-dimension}

\citet{abbasi_writeprints_2008} define feature set breadth as the number of feature categories and 
the feature set depth as the number of features.

\subsection{Features}

First, the \citet{abbasi_writeprints_2008} filter out all message signatures in order remove obvious identifies.
Then, they extract two sets of features: Baseline feature set consisting of 327 static author-group-level features and an 
extended feature set consisting of static and dynamic (author-group-level and individual-identity level) features.
They provide a table with quantities and description of features both of the baseline and extended feature set.
\citet{abbasi_writeprints_2008} choose their features using the information gain heuristic $IG$.
The information gain of a feature $j$ across a set of classes $c$ is derived as $IG(c,j)=H(c)-H(c|j)$, 
where $H(c)$ is the overall entropy across author classes and $H(c|j)$ is the conditional entropy for feature $j$.

For \ac{pos} tagging, they used the Arizona noun-phrase extractor, which uses the Penn Treebank tag set.
    % \section{Hyperpartisan Fake News Styles}
\label{sec:hyperpartisan_fake_news_styles}

\citet{potthast_stylometric_2018} presented using the unmasking approach, traditionally used for distinguishing author styles, 
to distinguish between genres (i.e., hyperpartisan and non-hyperpartisan news articles). 

There are different categories of fake news detection mechanisms, such as knownledge-based (by relating to known facts), 
context-based (by analysing news spread in social media), and style-based (by analysing writing style).

\citet{potthast_stylometric_2018} discard hardly represented features, such as word tokens that occur in less than $2.5 \%$ of the documents, 
and n-gram features that occur in less than $10 \%$ of the documents.
Otherwise, the model could overfit and could not generalize well to unseen data.

To avoid biases from datasets, they balance the training set by oversampling.

They use WEKA's random forest implementation.
Their code is publicly available (cf. \citep{potthast_stylometric_2018}). 

Satire is form of fake news, but its purpose is to entertain.

\subsection{Topic features for satire classification}
\label{sec:topic_features_for_sarcasm_classification}

\citet{potthast_stylometric_2018} argue that topic features are not appropriate for satire classification, 
since topics of satire change along the topics of news and thus, a classifier with topic features does not generalize.

\subsection{Generalizing unmasking to genre detection}
\label{sec:generalizing_unmasking_to_genre_detection}

\citet{potthast_stylometric_2018} omit the first step of the unmasking approach, i.e., chunking texts to create multiple text instances for the curve.
Instead, texts of possibly different genres but in the same genre are considered as belonging to the same class.
The documents of two genres are used as input to the classifier.
As usual, steeper decreases in classification error curves indicate higher style similarity of the two input documents.
    % \subsection{\acp{llm}}
\label{sec:llms}

There are two types of \acp{llm}:
\begin{itemize}
    \item \textbf{Autoregressive/Causal} \acp{llm}
    \item \textbf{Masked} \acp{llm}
\end{itemize}
Autoregressive \acp{llm} predict the next token based on the previous tokens in a sentence. 
Formally, the model samples a token from the distribution $p(x_i | x_1 ... x_{t-1})$ based on a pre-defined sampling strategy.
The text is formed one token at the time.
Possible sampling strategies include greedy sampling, top-k sampling, and nucleus sampling.
The latter two are most commonly used.
The GPT family of \acp{llm} are autoregressive \acp{llm}.
Autoregressive \acp{llm} are good at \ac{nlg} tasks \citep{bhattacharjee_fighting_2024}.

Masked \acp{llm} predict a masked tokens based on the surrounding context.
$k\%$ of the tokens are masked using special \texttt{[MASK]} tokens.
They have Bidirectional Attention mechanisms, which makes them better at NLU (Natural Language Understanding) tasks.
The BERT family of \acp{llm} are masked \acp{llm} \citep{bhattacharjee_fighting_2024}.
    % \section{\ac{llm} detection as an \ac{av} task}
\label{sec:llm_detection_av}

\citet{llm_detection_av_2025} claim that opposed to common belief, the task of detecting \ac{llm} generated text is not an \ac{aa} task,
i.e., a closed-set binary classification where both classes are sufficiently discriminative, 
but an \ac{av} task, i.e., an open-set one-class classification problem. 

\subsection{\ac{llm} vs human text}

% differences
\citet{llm_detection_av_2025} claim that it is unclear how exactly \ac{llm} generated text differs from human text.
Research so far has observed, that \ac{llm} generated text lacks lexical diversity, 
overuses certain adjectives ("commendable", "innovative", "meticulous", etc.) and produces longer, more complex sentences.
Moreover, \acp{llm} possess stylistic fingerprints and memorize patterns from the training data.
% lengths
While \citet{llm_detection_av_2025} found the average word length of human texts across datasets is 5.1 $\pm$ 0.1 charcters, 
which matches the excepted value for English, 
words in \ac{llm} generated text are significantly longer (>5.4 characters).
Text length is genre-dependent, however, human texts have generally longtailed distributions, 
while \ac{llm} generated texts have narrow stopping windows. 
% entropy
According to \citet{llm_detection_av_2025}, apart from the newest closed-source models, 
\acp{llm} struggle to produce sufficiently many new words and characters to match the entropy of human text.
Strategies for higher entropy include:
\begin{itemize}
    \item being more human-like
    \item penalizing repetitive sampling
    \item writing nonsensical gibberish
\end{itemize}
Character n-grams entropy alone is a weak discriminator, 
but can give an upper bound for minimum non-zero text length requirements for separation (below limit texts are indistinguishable) and 
establishes a baseline how human text is distributed \cite{llm_detection_av_2025}.

% similarities
However, due to Nucleus sampling (i.e. top-p sampling: Sampling next token from subset of $p$ most probable tokens) 
among other advances, generated text adheres better to human text characteristics.
% artifacts is AE, artefacts is BE
Newer \acp{llm} are able to produce fewer pathological language artefacts \cite{llm_detection_av_2025}.

\subsection{\ac{llm} detectors}

As \acp{llm} advanced, basic heuristics applied by human detectors no longer suffice.
However, experienced ChatGPT users can still detect generated text due to their prior experience.

\subsection{Assumptions}

% future of LLMs
According to \citet{llm_detection_av_2025}, one core assumption as of today is that 
language distributions of human-authored and machine-generated text are sufficiently distinct due to training data or technology.
\citet{llm_detection_av_2025} argue that this assumption is not true, since 
(1) the distance of the distributions depends on the feature space used, which is not specified, and
(2) human text is assumed to be uniform across all authors.
Instead, \citet{llm_detection_av_2025} argue that \acp{llm} will become more human-like and thus, 
\ac{llm} detection will increasingly resemble a human authorship classification task.

% dataset requirements
Currently, benchmarks should be designed to be diverse and include many (low quality) documents in order to make the benchmark more robust.
\citet{llm_detection_av_2025} argue that these datasets are not representative given the one-class nature of the problem.
Consequently, these datasets will not find good detectors and increase the Type-II error rate.

% scores
\citet{llm_detection_av_2025} argue that maintaining a low \ac{fpr} is more important than high accuracy.
This idea is core of the \ac{roc} analysis.
However, low \ac{fpr} and high accuracy are both possible to achieve if non-answers are a valid third option.
Abstaining from answering can be achieved by using a meta learner that can abstain from answering if the confidence is low, 
or to use \ac{svm} hyperplane distances to calibrate precision thresholds.

\subsection{Evaluation metrics}
Metrics also include c@1.
\citet{llm_detection_av_2025} argue that the reduction from the \ac{fpr}-\ac{tpr} curve of \ac{roc} to a single \ac{roc-auc} number 
comes with information loss due to the absence of a fixed threshold and trade-off.
Moreover, \ac{roc}'s \ac{fpr} and \ac{tpr} are independent of class prevalence, which is desirable.
However, in highly imbalanced class scenarios \ac{roc} can be misleading (overly optimistic or pessimistic).

\subsection{Dataset \& \acp{llm} for \ac{llm} detection} 
\citet{llm_detection_av_2025} propose to use the \ac{pan}'24, Human Detectors, Ghostbuster, RAID, MAGE, M4 dataset for \ac{llm} detection.
The datasets are preprocessed by removing extremely short texts, texts with adversarial attacks (i.e. obfuscation), 
low-quality texts created by older models, prompt artefacts and the peer review genre.
\acp{llm} used for \ac{llm} detection are \texttt{GPT-3.x}, \texttt{GPT-4}, \texttt{LLaMA2}, \texttt{Mistral}, \texttt{GPT-4o}, 
\texttt{GPT-4o-mini}, and \texttt{OpenAI o1}.

    % \section{Paraphrasing}
\label{sec:paraphrasing}


\citet{kurt_pehlivanoglu_comparative_2024} intentionally decided to use synthetic data to meet the study's specific goals.
According to them, synthetic data allows for a controlled and reproducible setup, ensuring all reference sentences are consistently generated.
They define dataset quality beyond mere reference-paraphrase sentence pairs, but also include multiple evaluation metrics.
Their code is available on \href{https://github.com/massyakur/ParaGPT}{GitHub}.

\subsection{Approaches to Paraphrasing}

Traditional approaches include thesaurus-based methods, where paraphrases are generated by 
replacing words with their synonyms from a thesaurus \citep{zhou_paraphrase_2021}.
Hence, this is called word-level paraphrasing.
First, all synonyms for the words to be replaced are retrieved from a thesaurus 
(or alignment tables based on IBM model, or WordNet).
Then, the optimal candidate is selected based on the context of the sentence. 
Although simple, this approach lacks diversity \citep{zhou_paraphrase_2021}.

\citet{fu_learning_2024} propose PEARL, a black-box paraphrasing model to meet \ac{llm} expression style.

UPRISE is a universal prompt auto-retrieval method that tunes a lightweight prompt retriever based on contrastive learning \citep{fu_learning_2024}.
There is also an official recommended template for manual prompt construction \citep{fu_learning_2024}.

\citet{zhou_paraphrase_2021}'s survey includes ParaNMT (originated around 2018), 
a dataset which was automatically generated by using 
back-translation to translate the non-English side of a large Czech-English parallel corpus.
This approach is a traditional approach, motivated by paraphrasing being a special case of statistical machine translation (SMT), 
i.e.\ monolingual translation. 
The goal is to find the best paraphrase $\hat{t}$ of a 
text in the source side $s$ to a text in the target side $t$: 
$ \hat{t} = argmax_{t \in t*} p(s|t)p(t) $.

\citet{zhou_paraphrase_2021} talk about neural methods to paraphrase generation.
They state that encoder encodes the source texts into a contextualized vector representation, 
along with a list of vector representations capturing the semantics of each word and context.
The decoder will generate paraphrases based on the vectors given by the encoder.
There is greedy decoding, where the words with the highest probability across the vocabulary are selected, 
and there is beam search, where the top $k$ paths are identified.
Since both greedy decoding and beam search are not specialized on paraphrase generation, but rather generic text generation, 
there exist methods to improve the quality of paraphrases.
One such method is to use blocking the words from the source text.
Other improvements based on encoder-decoder architectures include
\begin{itemize}
    \item \textbf{Copy mechanism}: The model can copy words from the source text to the target text (to counter the effect of rare and out-of-vocabulary words).
    \item \textbf{Attention mechanism}: The model can focus on different parts of the source text when generating each word in the target text.
    \item \textbf{\ac{rl}}: The model is trained to maximize a reward function that measures the quality of the generated paraphrases 
    (rather than minimize a loss that might not be aligned with metric used to evaluate paraphrase generation quality). 
    Discriminators of Generative Adversarial Networks (GANs) with policy gradient act like reward function in \ac{rl}.
\end{itemize}

Another neural method for paraphrase generation is to use \acp{vae} \citep{zhou_paraphrase_2021}.
The encoder learns a latent representation $z ~ N(\mu, \sigma)$.
The decoder generates realistic outputs conditioned on the latent representation $z$.
The learning objective is to reconstruct the original input from the latent representation $z$.
Paraphrase patterns are encoded into the latent representation $z$, where multiple paraphrase 
patterns and related words/ phrases are grouped under the same latent assignment.
Every time we sample a latent representation $z$, we get a different paraphrase pattern.

\citet{zhou_paraphrase_2021} also distinguish explicit and implicit syntax control:
Explicit syntax control methods encode the syntax tree of the source text into a list of vector representations 
and feed them into decoder at each timestep when decoding \citep{zhou_paraphrase_2021,palivela_optimization_2021}.
Implicit syntax control methods do not explicitly encode the syntax tree, 
but learns distributions over syntax information by \ac{vae}. 
The latent syntax variable is sampled from the learned distribution and will be fed into the decoder at each decoding step.
Implicit method do not require exemplar sentences like exemplar methods \citep{zhou_paraphrase_2021}.
\citet{palivela_optimization_2021} second major category of \ac{pg} is using pre-trained language models finetuning \acp{llm}.

\citet{palivela_optimization_2021,kurt_pehlivanoglu_comparative_2024} describe the 
\ac{t5} model as an algorithm that aims to convert every language problem into a text-to-text format.
\ac{t5} was trained on a mix of labeled (Colossal Clean Crawled Corpus) and unlabeled data \citep{palivela_optimization_2021}.
Its pre-training was done on data-rich tasks before fine-tuning on downstream tasks \citep{kurt_pehlivanoglu_comparative_2024}.

\citet{master_thesis_paraphrasing_2024} compare open-source \acp{llm} for paraphrasing.
They also provide descriptions of the models, including their architecture.
The models include Gemma (which they found to add asterics around the output), LLaMA (diverse version, where LLaMA3 significantly outperformed all other models), 
Mistral, and Phi3, Solar, and Starling (which they found to produce unclear outputs, due to excessive explanation of the model's actions).
Their first prompting strategy is \texttt{Paraphrase the following user story and output only paraphrased version:\\
\{user\_story\}}.
They tried including the stylometry metric and how they wanted to alter the metric: 
\texttt{Based on the following instruction: \{option\} \{stylometry\_metric\}.\\
Paraphrase the following user story and output only paraphrased version:\\
\{user\_story\}}.
Options include increase, decrease, or don't change.
The styloemtric metric is one of 23 options they defined.
They did not know to what extent the \ac{llm} was familiar with the stylometry metric. 
Hence, they also tried to include a definition of the metric in the prompt:
\texttt{\{stylometry metric definition\}\\Based on the following instruction:\\
\{option\} \{stylometry\_metric\} \\
Paraphrase the following user story and output only paraphrased version: \{user\_story\}}.
\citet{master_thesis_paraphrasing_2024} found that \acp{llm} struggle to adhere instructions with regard to specific stylometry metrics, 
since it generally did not enhance the performance of the models.
They also finetuned the \acp{llm} via the Unsloth library (pg. 21 \citep{master_thesis_paraphrasing_2024}).
Their experiments where run on Google Colab (pg. 28).

\citet{krishna_paraphrasing_2023} built \href{https://huggingface.co/kalpeshk2011/dipper-paraphraser-xxl}{DIPPER} (NeurIPS 2023, not accepted).
 

\citet{kurt_pehlivanoglu_comparative_2024} use ChatGPT to generate paraphrases with the prompt:
\texttt{paraphrase:\{sentence\}}.
They have a table of existing datasets (tab. 1, pg. 4) with genre specified, including MRPC a news dataset.
They use more specific prompts, such as:
\begin{itemize}
    \item \texttt{Generate sentences of kind simple}
    \item \texttt{Generate sentences of kind simple, compound, complex and compound-complex}
    \item \texttt{Generate compound-complex sentences}
    \item \texttt{Generate sentences with a complex conditional clause}
\end{itemize}
Other models they us include GPT-3 (Api using the Text-Davinci-003 model with temperature$=0.7$ and top\_p$=1$), 
and a pre-trained \ac{t5}-based model known as 'prithivida/parrot\_paraphraser\_on\_\ac{t5}' on Huggingface (fine-tuned for paraphrasing tasks).
Since some \ac{t5} paraphrases were the same as the input sentence, they used \textcolor{red}{beam search} technique to rephrase these identical sentences.
Even though, \ac{t5} was the best in terms of BERTScore, 
\citet{kurt_pehlivanoglu_comparative_2024} found that it had the lowest fluency (i.e.\ grammatical scceptability) measured by \ac{t5}-CoLA.

The Parrot Paraphraser is a rule-based \ac{t5}-based paraphrase generation framework whose 
rules balance adequacy (semantic preservation), fluency (text smoothness), and diversity (lexical/syntactic variation) \citep{zhou_paraphrase_2025}.

\citet{hassanipour_ability_2024} use the following prompts to generate paraphrases:
\begin{itemize}
    \item \texttt{Paraphrase the text}
    \item \texttt{Rephrase the text}
    \item \texttt{Reduce the plagiarism of the text}
    \item \texttt{Rephrase it in a way that conveys the same meaning using different words and sentence structure}
    \item \texttt{Reword this text using different language}
\end{itemize}
They found no significant difference between the prompts in terms of flagged plagiarism.
They propose that the prompts are too brief, too similar to each other, or that ChatGPT understands the intention of the user when reading any of these prompts.
They found that generated texts based on single paragraph were less prone to be flagged as plagiarized than those based on fragmented text, i.e.\ multiple paragraphs
(maybe the context and coherence improves ChatGPT's understanding of the text).
Multiple iterations if paraphrasing achieved modest improvements in reducing plagiarism, indicating that ChatGPT has the ability to learn.

\citet{zhou_paraphrase_2025} state that prompts should aligned with the \ac{llm} preference (of prompts).

Few-shot prompting for paraphrases is a technique where \acp{llm} are tasked to paraphrase text using a few samples of the desired paraphrase transformation \citep{zhou_paraphrase_2025}.

High-quality paraphrases are texts with high semantic similarity and high lexical and syntactic diversity to the original text \citep{gohsen_captions_2023}.


\subsection{Paraphrase divergence}
\citet{fu_learning_2024} state that paraphrase divergence (cf.~\autoref{sec:definitions}) can be explained as language models 
not only learn knowledge but also expression patterns associated with the knowledge from a corpus during pre-training.
Ideally, prompt learning should be independent of questions, but in reality, prompts are task- or domain-specific.
Hence, preferences for a certain format within a particular task or domain learnt during training persist in the model.
In other words, \acp{llm} may exhibit different preferences for various semantics.


\subsection{Metrics and Evaluation}
There is manual (by humans) evaluation and automatic evaluation for paraphrase generation \citep{fu_learning_2024,zhou_paraphrase_2021}.
According to \citet{zhou_paraphrase_2021}, automatic evaluation metrics mainly focus on the n-gram overlaps instead of meaning, 
and hence, human evaluation is more accurate and has a higher quality.
In the following, we focus on automatic evaluation.

There is syntactic and semantic evaluation of paraphrases \citep{gohsen_captions_2023}.
Metrics for syntactic evaluation include BLEU, ROUGE-1, ROUGE-L, 
while metrics for semantic similarity include BERTScore, 
cosine similarity of dense vector representations derived from a BERT-based sentence transformer, 
and Word Mover Distance \citep{gohsen_captions_2023}.
The Word Mover Distance computes the minimum amount of distance that embedded words of a text need to travel 
to reach the embedded words of another text \citep{gohsen_captions_2023}.
\citet{gohsen_captions_2023} normalized all metrics and averaged the semantic and syntactic scores separately.

\bluert{} is machine evaluation metric for paraphrase generation.
\citet{fu_learning_2024} use \bluert{} to filter out incorrect paraphrases (i.e.\ using a threshold $\theta$).

BLEU (Bilingual Evaluation Understudy \citep{palivela_optimization_2021,zhou_paraphrase_2025,papineni_bleu_2001}) (2002) 
was developed for machine translation \citep{zhou_paraphrase_2021,papineni_bleu_2001}.
It can take values from 0 to 1 \citep{papineni_bleu_2001}.
BLEU is a precision measure \citep{kurt_pehlivanoglu_comparative_2024,papineni_bleu_2001}.
It counts the matching n-grams (unigrams) in the generated/candidate text that appear in any of the gold/ reference texts \citep{palivela_optimization_2021,papineni_bleu_2001}, 
and then divides them by the total number of n-grams (unigrams) in the candidate text \citep{papineni_bleu_2001}.
Since candidates consisting only of high-probability n-grams (e.g.\ "the") would receive a high score without deserving it, 
\citet{papineni_bleu_2001} introduced a clipping mechanism to limit the count of n-grams in the candidate text to the maximum count of that n-gram in any of the reference texts.
The clipped n-grams occurences are added up and divided by the total number of unclipped n-grams in the candidate text \citep{papineni_bleu_2001}.
\citet{papineni_bleu_2001} state that unigrams are used to test adequacy, while longer n-grams are used to test fluency.
BLEU's basic unit of evaluation is a sentence. 
In order to compute the BLEU score from \autoref{eq:notes_bleu} for more than one sentence, one (1) computes the n-grams matches sentence by sentence, 
then (2) add the clipped n-grams matches across all sentences, 
and finally (3) divides the total clipped n-grams matches by the total number of unclipped n-grams in all candidate sentences \citep{papineni_bleu_2001}.
\begin{equation}
    p_n = \frac{\sum_{\mathcal{C} \in \left\{ Candidates \right\}}\sum_{n-gram \in\mathcal{C}}Count_{clip}(n-gram)}{\sum_{\mathcal{C'} \in \left\{ Candidates \right\}}\sum_{n-gram' \in\mathcal{C'}}Count(n-gram')}
\label{eq:notes_bleu}
\end{equation}
BLEU combines the scores for different n-grams (separately computed) using the average logarithm with uniform weights, 
which is equivalent to using the geometric mean of the scores \citep{papineni_bleu_2001,banerjee_METEOR_2005}.
\citet{gohsen_captions_2023} use up to 4-grams.
BLEU automatically penalizes n-grams appearing in the candidate text but not in the reference text, as well as n-grams appearing more often in the candidate than in the reference text \citep{papineni_bleu_2001}.
According to \citet{papineni_bleu_2001}, they need to add a brevity penalty to the BLEU score to enforce proper length of the candidate text. 
For multiple sentences, they (1) add the best match (among the reference texts) length for each candidate sentence, and (2) divide this sum $r$ by the total length of all candidate sentences $c$. 
Hence, the brevity penalty $BP$ is defined as follows in \autoref{eq:notes_bleu_brevity_penalty}:
\begin{equation}
    BP = \begin{cases}
        1 & \text{if } c > r \\
        e^{1 - \frac{r}{c}} & \text{if } c \leq r
    \end{cases}
\label{eq:notes_bleu_brevity_penalty}
\end{equation}
Combining all these, the final BLEU score is computed as follows in \autoref{eq:notes_bleu_final}:
\begin{equation}
    \text{BLEU} = BP \cdot \exp\left(\sum_{n=1}^{N} w_n \cdot \log p_n\right)
\label{eq:notes_bleu_final}
\end{equation}
They cannot use recall for length-related problems here, because BLEU uses multiple reference texts, which may have different lengths \citep{papineni_bleu_2001,banerjee_METEOR_2005}.
If the generated candidate is significantly shorter than the reference text, the brevity penalty $BP$ is applied.
A BLEU score approaching 1 signifies the candidate matches one reference almost exactly \citep{papineni_bleu_2001}, 
and thus, limited syntactic diversity (i.e.\ inadequate paraphrase) \citep{kurt_pehlivanoglu_comparative_2024}.
Note that more reference texts lead to higher BLEU scores \citep{papineni_bleu_2001}.
Unigrams are token-wise and bi-grams are word-pairs \citet{palivela_optimization_2021}.
According to \citet{zhou_paraphrase_2021}'s survey, BLEU is the most frequently used metric for paraphrase generation.
BLEU is unable to measure semantic equivalents \citep{kurt_pehlivanoglu_comparative_2024,zhou_paraphrase_2021} 
when applied to low-resource languages \citep{zhou_paraphrase_2021}.
Moreover, BLEU fails to capture good paraphrases that are not similar to the reference text \citep{zhou_paraphrase_2021}.
\citet{kurt_pehlivanoglu_comparative_2024} found that BLEU tends to overestimate the quality of paraphrases.
\citet{zhou_paraphrase_2021} suggest combining BLEU with human evaluation to overcome its limitations.

GLEU (Google-BLEU) (ranges from 0 to 1 \citep{kurt_pehlivanoglu_comparative_2024}) is a variant of BLEU that was developed to be closer to human judgement, and to 
overcome BLEU's drawback of per sentence reward objective \citep{palivela_optimization_2021}.
GLEU computes n-gram precisions (overlaps \citep{kurt_pehlivanoglu_comparative_2024}) overgold/reference paraphrases 
and weighs n-grams by their change from the source text \citep{palivela_optimization_2021}.
GLEU assesses the fluency, order of n-grams, structural and semantic accuracy 
and penalizes shorter average m-gram lengths in the generated text compared to the reference \citep{kurt_pehlivanoglu_comparative_2024}.
Lower GLEU scores indicate greater diversity \citep{kurt_pehlivanoglu_comparative_2024}.

METEOR (Metric for Evaluation of Translation with Explicit Ordering \citep{palivela_optimization_2021,banerjee_METEOR_2005}) 
(ranges from 0 to 1 \citep{kurt_pehlivanoglu_comparative_2024}) (2014) aims to address BLEU's shortcomings.
First a mapping, so-called aligment, between the unigrams in the candidate text and the reference texts is created \citep{banerjee_METEOR_2005}.
Each unigram has zero or one match.
This aligment is created incrementally in repeating two steps:
(1) List all possible unigram mappings derived from different modules (i.e.\ exact matches, Porter stemmed matches, synonym matches), 
and (2) select the largest subset of unigram mappings that constitute a valid alignment (matches obtained from different modules are treated the same).
(3) Choose the subset with the largest cardinality and if there are multiple, choose the one with the fewest unigram mapping crosses \citep{banerjee_METEOR_2005}.
METEOR computes a weighted F-score 
(unigram-precision, unigram-recall \citep{kurt_pehlivanoglu_comparative_2024,banerjee_METEOR_2005} 
and a measure of fragmentation \citep{banerjee_METEOR_2005,kurt_pehlivanoglu_comparative_2024})
with a penality function whenever an incorrect word is encountered \citep{palivela_optimization_2021} as displayed in \autoref{eq:notes_meteor}.
\begin{equation}
    METEOR = F_{mean} = \frac{10 \cdot P \cdot R}{R + 9P} \cdot (1 - Penalty)
\label{eq:notes_meteor}
\end{equation}
The penality is designed to reduce the $F_{mean}$ score to $50\%$ if there are no bigram or longer matches \citep{banerjee_METEOR_2005}.
It has better correlation with human judgement at the sentence/segment level than BLEU \citep{zhou_paraphrase_2021}, 
because it not only consists of simple n-gram matching but also including synonymy and stemming \citep{kurt_pehlivanoglu_comparative_2024}.

ROUGE (Recall-Oriented Understudy for Gisting Evaluation \citep{palivela_optimization_2021,lin_rouge_2004}) 
(ranges from 0 to 1 \citep{kurt_pehlivanoglu_comparative_2024}) (2004) 
is a recall-based metric developed for text summarisation \citep{zhou_paraphrase_2021,palivela_optimization_2021,kurt_pehlivanoglu_comparative_2024,lin_rouge_2004}.
ROUGE can focus on the word variations and diversity.
It has multiple versions, the most popular ones include 
ROUGE-N (computing the n-gram recall) \citep{zhou_paraphrase_2021,palivela_optimization_2021,kurt_pehlivanoglu_comparative_2024}, 
ROUGE-L (computing the longest common subsequence) \citep{zhou_paraphrase_2021,palivela_optimization_2021,kurt_pehlivanoglu_comparative_2024}, 
ROUGE-W (Weighted longest common subsequence) \citep{palivela_optimization_2021}, 
ROUGE-S (skip-bigram co-occurrence statistics) \citep{palivela_optimization_2021}.
ROUGE-1 computes the recall by analysing the matching unigrams between the generated paraphrase and the reference paraphrase \citep{palivela_optimization_2021,kurt_pehlivanoglu_comparative_2024}.
% ROUGE-N
ROUGE-N is an n-gram recall between the candidate text and the reference texts \citep{lin_rouge_2004} as displayed in \autoref{eq:notes_rouge_n}.
\begin{equation}
    ROUGE-N = \frac{\sum_{\mathcal{S} \in \left\{ References \right\}}\sum_{n-gram \in\mathcal{S}}Count_{match}(n-gram)}{\sum_{\mathcal{S'} \in \left\{ References \right\}}\sum_{n-gram' \in\mathcal{S'}}Count(n-gram')}
\label{eq:notes_rouge_n}
\end{equation}
$Count_{match}(n-gram)$ is the maximum number of n-grams co-occuring in the candidate text and the set of reference texts \citep{lin_rouge_2004}.
The nominator sums over all references and thus, gives more weight to matching n-grams that occur in multiple references (i.e.\ a consensus between references) \citep{lin_rouge_2004}.
Refer to \citet{lin_rouge_2004} for more details on the work with multiple references (I do not understand that, because I thought we already use multiple).
% ROUGE-L
For ROUGE-L, the intuition is that the longer the longest common subsequence (LCS) between the candidate and reference texts, the more similar they are \citep{lin_rouge_2004}.
For a candidate $Y$ of length $n$ and a reference $X$ of length $m$, the ROUGE-L score is defined as follows in \autoref{eq:notes_rouge_l}:
\begin{equation}
    ROUGE-L = F_{lcs} = \frac{(1 + \beta^2)R_{lcs}P_{lcs}}{R_{lcs} + \beta^2 P_{lcs}}
\label{eq:notes_rouge_l}
\end{equation}
where $R_{lcs} = \frac{LCS(X,Y)}{m}$ and $P_{lcs} = \frac{LCS(X,Y)}{n}$ \citep{lin_rouge_2004}.
ROUGE-L requires in-sequence matches that reflect the sentence level word order as n-grams \citep{lin_rouge_2004}.
Moreover, no predefined $n$ is necessary, because ROUGE-L includes the longest in-sequence common n-grams \citep{lin_rouge_2004}.
However, ROUGE-L does not include shorter sequences or alternative LCSes in the final score \citep{lin_rouge_2004}.
% ROUGE-S
A skip-bigram is any pair of words in their sentence order, allowing for arbitrary gaps \citep{lin_rouge_2004}.
ROUGE-S measures the overlap of skip-bigrams between the candidate text and the reference texts \citep{lin_rouge_2004}.
Hence, if the candidate text is the reverse of the reference text, the ROUGE-S score is 0 even though it is not as bad as completely unrelated candidates \citep{lin_rouge_2004}.
ROUGE-SU extends ROUGE-S with unigrams to solve this issue \citep{lin_rouge_2004}.
% ROUGE generally
\citet{kurt_pehlivanoglu_comparative_2024} claim that ROUGE may not be adequate to assess semantic similarity and fluency.
Lower ROUGE scores indicate greater diversity \citep{kurt_pehlivanoglu_comparative_2024}.

TER (2006) was developed for machine translation \citep{zhou_paraphrase_2021}.
It computes the number of edits required to change the translation until it matches the reference translation.
It ranges from 0 (i.e.\ no edits needed) to 1 (i.e.\ all words need to be changed).

\citet{fu_learning_2024} describe the Gini Coefficient as a measure of inequality in a distribution, ranging from 0 (i.e.\ even distribution across categories) to 1.

\citet{master_thesis_paraphrasing_2024} include a table (tab. 3.1, pg. 18) with numerous stylometric metric including readbility, vocabulary richness, and word/character counts.

\citet{palivela_optimization_2021} state that accuracy, precision, recall and F1-score are suitable for \ac{pi}.
For \ac{pg}, they suggest ROUGE, BLEU, GLEU, WER (Word Error Rate), and METEOR as suitable metrics.
WER is the number of substitutions (replacements of words), insertions (adding words) and deletions (removing words) 
divided by the total number of words in the reference text \citep{palivela_optimization_2021}.

\citet{kurt_pehlivanoglu_comparative_2024} additionally use T5-STSB.
The metric is based on \ac{t5} model adapted to the Semantic Textual Similarity Benchmark (STSB).
It evaluates semantic equivalence by assigning a similarity score from 0 (no similarity) to 5 (complete equivalence) \citep{kurt_pehlivanoglu_comparative_2024}.

BERTScore calculates the cosine similarity between the contextual embeddings of the reference and generated texts. 
Hence, is assesses semantic equivalence and correlates well with human judgement \citep{kurt_pehlivanoglu_comparative_2024}.
First, token vector representations are computed for both the reference and generated texts using a pre-trained BERT model \citep{hanna_fine_grained_2021}.
Let reference $z$ and candidate $\hat{z}$ be the vector representations of the reference and candidate texts, respectively.
Then, the BERTScore precision, recall and $F_1$ score is computed as follows in \autoref{eq:notes_bert_p}, \autoref{eq:notes_bert_r}, and \autoref{eq:notes_bert_f1}, respectively:
\begin{equation}
    P_{BERT} = \frac{1}{|\hat{z}|} \sum_{\hat{z}_j \in \hat{z}} \max_{z_j \in z} z_i\top \hat{z}_j
\label{eq:notes_bert_p}
\end{equation}
\begin{equation}
    R_{BERT} = \frac{1}{|z|} \sum_{z_j \in z} \max_{\hat{z}_j \in \hat{z}} z_i\top \hat{z}_j
\label{eq:notes_bert_r}
\end{equation}
\begin{equation}
    F_1 = \frac{2 P_{BERT} R_{BERT}}{P_{BERT} + R_{BERT}} 
\label{eq:notes_bert_f1}
\end{equation}
Since $F_1 \in \left[-1,1\right]$ it can be rescaled to $[0,1]$ by modifying the precision and recall calculation 
to $\hat{P}_{BERT} = \frac{P_{BERT} - a}{1 - a}$ ($R_{BERT}$ analoguous), where $a$ is the empirical lower bound on the BERTScore \citep{hanna_fine_grained_2021}.
The BERTScore has difficulties on datasets with lexically similar (i.e.\ lexical overlap of content words) incorrect candidates 
opposed to lexically different more correct candidates \citep{hanna_fine_grained_2021}.


\ac{t5}-CoLA metric (ranges from 0 to 5 \citep{kurt_pehlivanoglu_comparative_2024}) utilizes the Corpus of Linguistic Acceptability (CoLA) to evaluate the grammatical correctness of sentences and thus, 
contributes linguistic evaluation \citep{kurt_pehlivanoglu_comparative_2024}.

Generally, \citet{kurt_pehlivanoglu_comparative_2024} order the metrics by their contribution area:
\begin{itemize}
    \item semantic: BERTScore, STSB, METEOR
    \item Fluency: CoLA
    \item Diversity: ROUGE1/2/L, BLEU, GLEU
\end{itemize}

\citet{krishna_paraphrasing_2023} compute the lexical diversity using unigram token overlap and call it F1 score.
As a semantic similarity score, they use the ACL Antology 2022 published \href{https://aclanthology.org/2022.emnlp-demos.38.pdf}{P-SP}.

According to \citet{gohsen_task_oriented_2024}, there are two perspectives to paraphrasing: 
Lexical (i.e.\ changes at word level) and syntactic (i.e.\ changes at syntactic level).
Paraphrase types can be classified into surface and semantic level. Finer levels are outlined in \citep{gohsen_task_oriented_2024}.

\textcolor{red}{Do we want semantically similar paraphrases even though our task is style transfer belonging to semantically equivalent paraphrasing?}

Popular paraphrase categories include \citep{fu_learning_2024}:
\begin{itemize}
    \item Top-level classification perspective: 
        \begin{itemize}
            \item Lexicon-based changes
            \item Morphology-based changes
            \item others
        \end{itemize}
    \item Second-level classification perspective:
        \begin{itemize}
            \item Change of format
            \item Semantic-based
            \item Change of order
        \end{itemize}
\end{itemize}
\citet{zhou_paraphrase_2025} (pg. 3, tab.1, and examples afterwards) define a topology of paraphrase types:
\begin{itemize}
    \item Morphology based: inflection changes (e.g.\ singular to plural), derivation changes (e.g.\ adjective to verb), functional word substitution (e.g.\ this to that).
    \item Lexicon based: Same polarity substitution (e.g.\ synonym), opposite polarity substitution (e.g.\ antonym), converse substitution (e.g.\ opposite view point), spelling changes, synthetic/ analytic substitution, relational substitution
    \item Syntax based: Negation switching (i.e.\ other negation), diathesis alternation (e.g.\ change position of verb), etc.
    \item Discourse based: Indirect/direct substitutions, sentence modality changes, punctuation changes, etc.
    \item Other changes: Change of order, change of format, etc.
\end{itemize}

\citet{fu_learning_2024} give three prompt tips:
\begin{enumerate}
    \item Make question/ prompt as clear as possible even if some restrictive requirements may seem unnecessary.
    \item Place significant details, including restrictive elements such as time, place, manner, reason, purpose and conditions, at the beginning of the prompt.
    \item Pay attention to spelling, i.e.\ proper nouns, title, honorifics, abbreviations, acronyms and observe capitalization.
\end{enumerate}

\citet{zhou_paraphrase_2021} claim it is difficult to control the style of generated paraphrases.

Another approach to paraphrasing is back-translation, which may limit diversity \citep{zhou_paraphrase_2025}.


\subsection{\ac{llm} training}

\citet{master_thesis_paraphrasing_2024} describes the training of \acp{llm} as a three-stage development process:
Starting with pre-training on a vast dataset to acquire comprehensive knowledge of language and the world, 
followed by supervised finetuning for specific downstream tasks and finally,
human preference alignment training to mimic human behaviour.

\subsection{Preprocessing of Dataset}

\citet{palivela_optimization_2021} propose the following transformation steps for the dataset:
\begin{enumerate}
    \item Remove sentence pairs having more than 60$\%$ unigram, bigram or trigram overlap 
    (to discourage model from copying input sentence and increases diversity of paraphrases).
    \item Remove sentence pairs having very little semantic similarity by using Sentence-BERT 
    (to force model to generate semantically similar paraphrases).
    \item If dataset contains non-paraphrase sentence pairs, remove them.   
\end{enumerate}

\citet{gohsen_captions_2023} remove exact or near-duplicates from their paraphrase dataset.
Near-duplicates differ only in punctuation, capitalization or whitespaces \citep{gohsen_captions_2023}.

\subsection{Plagiarism Detection}
\citet{hassanipour_ability_2024} analyse texts for plagiarism using the \textcolor{orange}{iThenticate} tool.
It is really \href{https://www.ithenticate.com/pricing}{expensive}.
Alternatives to detectors of \ac{ai}-generated text include watermarking and statistical outlier detection 
(\ac{ai}-generated text's artefacts such as irregularities in entropy and perplexity) \citep{krishna_paraphrasing_2023}.
    \section*{Acknowledgements}

The authors gratefully acknowledge the computing time granted by the KISSKI project. 
The calculations for this research were conducted with computing resources under the project \textcolor{red}{<ID of your project>}.

We also thank J.~W.~Pennebaker for granting access to the original data used by \citet{koppel_determining_2014}, as well as Moshe Koppel and Mr.~Winter for their valuable and forthcoming communication, which informed the development of this work.
    \appendix
\chapter{First Appendix}
\label{ch:appendix}


\section{Extractor prompts used for Two-Step Paraphrasing}
\label{app:extractor_prompts}
% bullet point
\begin{quote}
    \textit{Summarize the text above in five to six short bullet points. Respond ONLY with a JSON object in the following format: $\{$"bullet\_points":"<list of bullet points>","tone":"<tone>","time\_period":<time\_period>,"language\_register":<register>,"target\_audience":"<target\_audience>","genre":"<genre>"$\}$. Do not use direct quotes.}
\end{quote}

% task
\begin{quote}
    \textit{Act as the author of the text above. From that perspective, infer your role or identity, the topic being addressed, and the purpose or instruction behind writing the text. Combine these elements into a concise task prompt that you would give to an LLM to reproduce the text. Respond ONLY with a JSON object in the following format: \{"task":"<task>","tone":"<tone>","time\_period":<time\_period>,"language\_register":<register>,"target\_audience":"<target\_audience>","genre":"<genre>"$\}$. Do not use direct quotes.}
\end{quote}

% topic
\begin{quote}
    \textit{Extract the topic, tone, time period, register, target audience, and genre from the text above. Respond ONLY with a JSON object in the following format: \{"topic":"<topic>","tone":"<tone>","time\_period":<time\_period>,"language\_register":<register>,"target\_audience":"<target\_audience>","genre":"<genre>"$\}$. Do not use direct quotes.}
\end{quote}

% title
\begin{quote}
    \textit{Find a concise title for the text, extract the tone, time period, register, target audience and genre from the text above. Respond ONLY with a JSON object in the following format: \{"title":"<title>","tone":"<tone>","time\_period":<time\_period>,"language\_register":<register>,"target\_audience":"<target\_audience>","genre":"<genre>"$\}$. Do not use direct quotes.}
\end{quote}

\section{Generator prompts used for Two-Step Paraphrasing}
\label{app:generator_prompts}
% bullet point
\begin{minted}{python}
prompt = "Write a text which covers the following items:\n" 
    + "\n".join(f"- {bp}" for bp in bullet_points)
\end{minted}

% task
\begin{minted}{python}
generator_prompt = "Write a text of about {l} words with a {tone} tone, a {genre} genre, in the {register} register for the target audience of {target_audience} and in the {time_period} time period, covering the following task:\n{task}. Do not use asterisks. Only output the text without any additional commentary.".format(
            l=len(text.split()),
            tone=tone,
            genre=genre,
            task=task,
            time_period=time_period,
            register=register,
            target_audience=target_audience,
        )
\end{minted}

% topic
\begin{minted}{python}
generator_prompt = "Write a text of about {l} words with a {topic} topic, {tone} tone, a {genre} genre, in the {register} register for the target audience of {target_audience} and in the {time_period} time period. Do not use asterisks. Only output the text without any additional commentary.".format(
            l=len(text.split()),
            tone=tone,
            genre=genre,
            topic=topic,
            time_period=time_period,
            register=register,
            target_audience=target_audience,
        )
\end{minted}

% title
\begin{minted}{python}
generator_prompt = "Write a text of about {l} words with a {title} title, {tone} tone, a {genre} genre, in the {register} register for the target audience of {target_audience} and in the {time_period} time period. Do not use asterisks. Only output the text without any additional commentary.".format(
            l=len(text.split()),
            tone=tone,
            genre=genre,
            title=title,
            time_period=time_period,
            register=register,
            target_audience=target_audience,
        )
\end{minted}

\section{Preprocessing Regular Expressions}
\label{app:regex_preproc}

\textcolor{red}{TODO}

\section{Exp. 2: Comparison of Different Paraphrasers}
\label{sec:app_paraphrases}

\begin{figure}[H]
    \centering
    \includesvg[width=0.9\textwidth]{images/paraphrasing/experiments/radar/Blog_paraphrasing_metrics_grouped_by_Paraphraser_radar_chart.svg}
    \caption{Radar chart of syntactic and semantic paraphrase evaluation measures different paraphrasers on the \dataBlog{} dataset.}
    \label{fig:radar_blog}
\end{figure}


\begin{figure}[H]
    \centering
    \includesvg[width=0.9\textwidth]{images/paraphrasing/experiments/sem_syn_scatter/Gutenberg_sem_syn_scatter_grouped_by_Paraphraser.svg}
    \caption{Average semantic and syntactic similarity for different paraphraser on the \dataGutenberg{}.}
    \label{fig:sem_syn_gutenberg}
\end{figure}

\begin{figure}[H]
    \centering
    \includesvg[width=0.9\textwidth]{images/paraphrasing/experiments/radar/Gutenberg_paraphrasing_metrics_grouped_by_Paraphraser_radar_chart.svg}
    \caption{Radar chart of syntactic and semantic paraphrase evaluation measures different paraphrasers on the \dataGutenberg{} dataset.}
    \label{fig:radar_gutenberg}
\end{figure}


% appendix
\section{Exp. 6: Comparing \ac{av} Methods in different scenarios}
\label{sec:app_detection_scenarios}

% \begin{figure}[htbp]
% \centering
%     \includesvg[width=\linewidth]{}
%   \caption{Accuracy curves for the class same-author across different threshold on artificially augmented \dataStudent{} dataset. 
%   }
%   \label{fig:human-human_acc}
% \end{figure}


\begin{figure}[htbp]
  \centering
  \begin{subfigure}[b]{0.52\textwidth}
    \centering
    \includesvg[width=\linewidth]{images/AV_comparison/detection_scenarios/recall/student_essays_Human-Human_threshold_recalls_curves_all_incl_baselines.svg}
    \caption{Human-Human}
    \label{fig:detec_scen_human-human_recall}
  \end{subfigure}
  \hfill
  \begin{subfigure}[b]{0.52\textwidth}
    \centering
    \includesvg[width=\linewidth]{images/AV_comparison/detection_scenarios/recall/student_essays_Human-LLM_threshold_recalls_curves_all_incl_baselines.svg}
    \caption{Human-\ac{llm}}
    \label{fig:detec_scen_human-llm_recall}
  \end{subfigure}
  \hfill
  \begin{subfigure}[b]{0.52\textwidth}
    \centering
    \includesvg[width=\linewidth]{images/AV_comparison/detection_scenarios/recall/student_essays_LLM-LLM_same_threshold_recalls_curves_all_incl_baselines.svg}
    \caption{\ac{llm}-\ac{llm}}
    \label{fig:detec_scen_llm-llm_recall}
    \end{subfigure}
  \caption{Recall curves for the class same-author across different threshold on artificially augmented \dataStudent{} dataset.}
  \label{fig:detec_scen_recall}
\end{figure}


\section{Libraries used}
\label{app:libraries}

\textcolor{red}{TODO}
e.g. sentence tokens: \texttt{nltk}'s \texttt{sent\_tokenize}

    % Die nächsten zwei Zeilen sind optional, sie sorgen dafür dass alles nach dem Inhalt wieder mit römischen Zahlen nummeriert wird.
    \pagenumbering{roman}
    \addtocounter{page}{10} % Dies ist die Anzahl der Seiten vor der Einleitung, muss möglicherweise angepasst werden, wenn das Inhaltsverzeichnis mehrere Seiten umfasst.

    \nocite{*}
    \bibliographystyle{plainnat} % requires package natbib. An alternative is apalike
    \bibliography{
        bibliography/author_identification
    }

    % \chapter*{Eidesstattliche Erklärung}

\markboth{Eidesstattliche Erklärung}{Eidesstattliche Erklärung}

Hiermit erkläre ich, \thesisauthorname, dass ich die vorliegende Arbeit mit dem Titel "\thesistitle" selbstständig 
und nur mit den nach der Prüfungsordnung der Universität Kassel zulässigen Hilfsmitteln angefertigt habe.
Die verwendete Literatur ist im Literaturverzeichnis angegeben.
Wörtlich oder sinngemäß übernommene Inhalte habe ich als solche kenntlich gemacht.

\vspace{1cm}

Kassel, \today

\begin{flushright}
  \underline{\hspace{7cm}} \\
  \thesisauthorname
\end{flushright}

\end{document}
